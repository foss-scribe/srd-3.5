%&pdfLaTeX
% !TEX encoding = UTF-8 Unicode
\documentclass{article}
\usepackage{ifxetex}
\ifxetex
\usepackage{fontspec}
\setmainfont[Mapping=tex-text]{STIXGeneral}
\else
\usepackage[T1]{fontenc}
\usepackage[utf8]{inputenc}
\fi
\usepackage{textcomp}

\usepackage{amssymb}
\usepackage{fancyhdr}
\renewcommand{\headrulewidth}{0pt}
\renewcommand{\footrulewidth}{0pt}

\begin{document}

This material is Open Game Content, and is licensed for public use under the terms 
of the Open Game License v1.0a.

\section*{{\LARGE{}RACES}}

\vspace{12pt}
FAVORED CLASS

A character's favored class doesn't count against him or her when determining experience 
point penalties for multiclassing. 

\vspace{12pt}
RACE AND LANGUAGES

All characters know how to speak Common. A dwarf, elf, gnome, half-elf, half-orc, 
or halfling also speaks a racial language, as appropriate. A character who has 
an Intelligence bonus at 1st level speaks other languages as well, one extra language 
per point of Intelligence bonus as a starting character. 

Literacy: Any character except a barbarian can read and write all the languages 
he or she speaks.

Class-Related Languages: Clerics, druids, and wizards can choose certain languages 
as bonus languages even if they're not on the lists found in the race descriptions. 
These class-related languages are as follows:

\textit{Cleric: }Abyssal, Celestial, Infernal.

\textit{Druid: }Sylvan.

\textit{Wizard: }Draconic.

\vspace{12pt}
SMALL CHARACTERS

A Small character gets a +1 size bonus to Armor Class, a +1 size bonus on attack 
rolls, and a +4 size bonus on Hide checks. A Small character's carrying capacity 
is three-quarters of that of a Medium character.

A Small character generally moves about two-thirds as fast as a Medium character.

A Small character must use smaller weapons than a Medium character.

\vspace{12pt}
{\LARGE{}HUMANS• }

\parindent=3pt
Medium: As Medium creatures, humans have no special bonuses or penalties due to 
their size. • 

Human base land speed is 30 feet. • 

\parindent=7pt
1 extra feat at 1st level.• 

\parindent=3pt
4 extra skill points at 1st level and 1 extra skill point at each additional level. 
• 

\parindent=7pt
Automatic Language: Common. Bonus Languages: Any (other than secret languages, 
such as Druidic). See the Speak Language skill. • 

\parindent=3pt
Favored Class: Any. When determining whether a multiclass human takes an experience 
point penalty, his or her highest-level class does not count.

\vspace{12pt}
\parindent=0pt
{\LARGE{}DWARVES• }

\parindent=3pt
+2 Constitution, -2 Charisma.• 

Medium: As Medium creatures, dwarves have no special bonuses or penalties due to 
their size.• 

\parindent=7pt
Dwarf base land speed is 20 feet. However, dwarves can move at this speed even 
when wearing medium or heavy armor or when carrying a medium or heavy load (unlike 
other creatures, whose speed is reduced in such situations).• 

\parindent=3pt
Darkvision: Dwarves can see in the dark up to 60 feet. Darkvision is black and 
white only, but it is otherwise like normal sight, and dwarves can function just 
fine with no light at all.• 

Stonecunning: This ability grants a dwarf a +2 racial bonus on Search checks to 
notice unusual stonework, such as sliding walls, stonework traps, new construction 
(even when built to match the old), unsafe stone surfaces, shaky stone ceilings, 
and the like. Something that isn't stone but that is disguised as stone also counts 
as unusual stonework. A dwarf who merely comes within 10 feet of unusual stonework 
can make a Search check as if he were actively searching, and a dwarf can use the 
Search skill to find stonework traps as a rogue can. A dwarf can also intuit depth, 
sensing his approximate depth underground as naturally as a human can sense which 
way is up. • 

\parindent=7pt
Weapon Familiarity: Dwarves may treat dwarven waraxes and dwarven urgroshes as 
martial weapons, rather than exotic weapons.• 

\parindent=3pt
Stability: A dwarf gains a +4 bonus on ability checks made to resist being bull 
rushed or tripped when standing on the ground (but not when climbing, flying, riding, 
or otherwise not standing firmly on the ground).• 

+2 racial bonus on saving throws against poison.• 

\parindent=7pt
+2 racial bonus on saving throws against spells and spell-like effects.• 

\parindent=3pt
+1 racial bonus on attack rolls against orcs and goblinoids.• 

+4 dodge bonus to Armor Class against monsters of the giant type. Any time a creature 
loses its Dexterity bonus (if any) to Armor Class, such as when it's caught flat-footed, 
it loses its dodge bonus, too. • 

\parindent=7pt
+2 racial bonus on Appraise checks that are related to stone or metal items.• 

\parindent=3pt
+2 racial bonus on Craft checks that are related to stone or metal.• 

Automatic Languages: Common and Dwarven. Bonus Languages: Giant, Gnome, Goblin, 
Orc, Terran, and Undercommon.• 

\parindent=7pt
Favored Class: Fighter. A multiclass dwarf 's fighter class does not count when 
determining whether he takes an experience point penalty for multiclassing

\vspace{12pt}
\parindent=0pt
{\LARGE{}ELVES• }

\parindent=3pt
+2 Dexterity, -2 Constitution.• 

Medium: As Medium creatures, elves have no special bonuses or penalties due to 
their size.• 

\parindent=7pt
Elf base land speed is 30 feet.• 

\parindent=3pt
Immunity to magic sleep effects, and a +2 racial saving throw bonus against enchantment 
spells or effects.• 

Low-Light Vision: An elf can see twice as far as a human in starlight, moonlight, 
torchlight, and similar conditions of poor illumination. She retains the ability 
to distinguish color and detail under these conditions.• 

\parindent=7pt
Weapon Proficiency: Elves receive the Martial Weapon Proficiency feats for the 
longsword, rapier, longbow (including composite longbow), and shortbow (including 
composite shortbow) as bonus feats.• 

\parindent=3pt
+2 racial bonus on Listen, Search, and Spot checks. An elf who merely passes within 
5 feet of a secret or concealed door is entitled to a Search check to notice it 
as if she were actively looking for it.• 

Automatic Languages: Common and Elven. Bonus Languages: Draconic, Gnoll, Gnome, 
Goblin, Orc, and Sylvan.• 

\parindent=7pt
Favored Class: Wizard. A multiclass elf 's wizard class does not count when determining 
whether she takes an experience point penalty for multiclassing.

\vspace{12pt}
\parindent=0pt
{\LARGE{}GNOMES• }

\parindent=3pt
+2 Constitution, -2 Strength.• 

Small: As a Small creature, a gnome gains a +1 size bonus to Armor Class, a +1 
size bonus on attack rolls, and a +4 size bonus on Hide checks, but he uses smaller 
weapons than humans use, and his lifting and carrying limits are three-quarters 
of those of a Medium character.• 

\parindent=7pt
Gnome base land speed is 20 feet.• 

\parindent=3pt
Low-Light Vision: A gnome can see twice as far as a human in starlight, moonlight, 
torchlight, and similar conditions of poor illumination. He retains the ability 
to distinguish color and detail under these conditions.• 

Weapon Familiarity: Gnomes may treat gnome hooked hammers as martial weapons rather 
than exotic weapons.• 

\parindent=7pt
+2 racial bonus on saving throws against illusions.• 

\parindent=3pt
Add +1 to the Difficulty Class for all saving throws against illusion spells cast 
by gnomes. This adjustment stacks with those from similar effects.• 

+1 racial bonus on attack rolls against kobolds and goblinoids.• 

\parindent=7pt
+4 dodge bonus to Armor Class against monsters of the giant type. Any time a creature 
loses its Dexterity bonus (if any) to Armor Class, such as when it's caught flat-footed, 
it loses its dodge bonus, too.• 

\parindent=3pt
+2 racial bonus on Listen checks.• 

+2 racial bonus on Craft (alchemy) checks.• 

\parindent=7pt
Automatic Languages: Common and Gnome. Bonus Languages: Draconic, Dwarven, Elven, 
Giant, Goblin, and Orc. In addition, a gnome can speak with a burrowing mammal 
(a badger, fox, rabbit, or the like, see below). This ability is innate to gnomes. 
See the \textit{speak with animals }spell description.• 

\parindent=3pt
Spell-Like Abilities: 1/day---\textit{speak with animals }(burrowing mammal only, 
duration 1 minute). A gnome with a Charisma score of at least 10 also has the following 
spell-like abilities: 1/day---\textit{dancing lights, ghost sound, prestidigitation. 
}Caster level 1st; save DC 10 + gnome's Cha modifier + spell level.• 

Favored Class: Bard. A multiclass gnome's bard class does not count when determining 
whether he takes an experience point penalty.

\vspace{12pt}
\parindent=0pt
{\LARGE{}HALF-ELVES• }

\parindent=3pt
Medium: As Medium creatures, half-elves have no special bonuses or penalties due 
to their size.• 

Half-elf base land speed is 30 feet.• 

\parindent=7pt
Immunity to \textit{sleep }spells and similar magical effects, and a +2 racial 
bonus on saving throws against enchantment spells or effects.• 

\parindent=3pt
Low-Light Vision: A half-elf can see twice as far as a human in starlight, moonlight, 
torchlight, and similar conditions of poor illumination. She retains the ability 
to distinguish color and detail under these conditions.• 

+1 racial bonus on Listen, Search, and Spot checks.• 

\parindent=7pt
+2 racial bonus on Diplomacy and Gather Information checks.• 

\parindent=3pt
Elven Blood: For all effects related to race, a half-elf is considered an elf.• 

Automatic Languages: Common and Elven. Bonus Languages: Any (other than secret 
languages, such as Druidic).• 

\parindent=7pt
Favored Class: Any. When determining whether a multiclass half-elf takes an experience 
point penalty, her highest-level class does not count.

\vspace{12pt}
\parindent=0pt
{\LARGE{}HALF-ORCS• }

\parindent=3pt
+2 Strength, -2 Intelligence, -2 Charisma.

\parindent=0pt
A half-orc's starting Intelligence score is always at least 3. If this adjustment 
would lower the character's score to 1 or 2, his score is nevertheless 3.• 

\parindent=3pt
Medium: As Medium creatures, half-orcs have no special bonuses or penalties due 
to their size.• 

Half-orc base land speed is 30 feet.• 

\parindent=7pt
Darkvision: Half-orcs (and orcs) can see in the dark up to 60 feet. Darkvision 
is black and white only, but it is otherwise like normal sight, and half-orcs can 
function just fine with no light at all.• 

\parindent=3pt
Orc Blood: For all effects related to race, a half-orc is considered an orc.• 

Automatic Languages: Common and Orc. Bonus Languages: Draconic, Giant, Gnoll, Goblin, 
and Abyssal.• 

\parindent=7pt
Favored Class: Barbarian. A multiclass half-orc's barbarian class does not count 
when determining whether he takes an experience point penalty.

\vspace{12pt}
\subsection*{{\LARGE{}HALFLINGS• }}

\parindent=3pt
+2 Dexterity, -2 Strength.• 

Small: As a Small creature, a halfling gains a +1 size bonus to Armor Class, a 
+1 size bonus on attack rolls, and a +4 size bonus on Hide checks, but she uses 
smaller weapons than humans use, and her lifting and carrying limits are three-quarters 
of those of a Medium character.• 

\parindent=7pt
Halfling base land speed is 20 feet.• 

\parindent=3pt
+2 racial bonus on Climb, Jump, and Move Silently checks.• 

+1 racial bonus on all saving throws.• 

+2 morale bonus on saving throws against fear: This bonus stacks with the halfling's 
+1 bonus on saving throws in general.• 

\parindent=7pt
+1 racial bonus on attack rolls with thrown weapons and slings.• 

\parindent=3pt
+2 racial bonus on Listen checks.• 

Automatic Languages: Common and Halfling. Bonus Languages: Dwarven, Elven, Gnome, 
Goblin, and Orc.• 

\parindent=7pt
Favored Class: Rogue. A multiclass halfling's rogue class does not count when determining 
whether she takes an experience point penalty for multiclassing.

\newpage

\end{document}
