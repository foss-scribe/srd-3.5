%&pdfLaTeX
% !TEX encoding = UTF-8 Unicode
\documentclass{article}
\usepackage{ifxetex}
\ifxetex
\usepackage{fontspec}
\setmainfont[Mapping=tex-text]{STIXGeneral}
\else
\usepackage[T1]{fontenc}
\usepackage[utf8]{inputenc}
\fi
\usepackage{textcomp}

\usepackage{array}
\usepackage{amssymb}
\usepackage{fancyhdr}
\renewcommand{\headrulewidth}{0pt}
\renewcommand{\footrulewidth}{0pt}

\newcommand{\tab}{\hspace{5mm}}

\begin{document}

This material is Open Game Content, and is licensed for public use under the terms 
of the Open Game License v1.0a.

\section*{{\LARGE{}CLASSES I}}

\vspace{12pt}
{\LARGE{}BARBARIAN}

\textbf{Alignment:} Any nonlawful.

\textbf{Hit Die:} d12.

\vspace{12pt}
\textbf{Class Skills}

The barbarian's class skills (and the key ability for each skill) are Climb (Str), 
Craft (Int), Handle Animal (Cha), Intimidate (Cha), Jump (Str), Listen (Wis), Ride 
(Dex), Survival (Wis), and Swim (Str).

\textbf{Skill Points at 1st Level:} (4 + Int modifier) x$ $4.

\textbf{Skill Points at Each Additional Level:} 4 + Int modifier.

\vspace{12pt}
\begin{tabular}{|>{\raggedright}p{20pt}|>{\raggedright}p{52pt}|>{\raggedright}p{28pt}|>{\raggedright}p{25pt}|>{\raggedright}p{28pt}|>{\raggedright}p{123pt}|}
\hline
\multicolumn{6}{|p{278pt}|}{
T\textbf{able: The Barbarian}}\tabularnewline
\hline
L\textbf{evel } & B\textbf{ase Attack Bonus } & F\textbf{ort Save } & R\textbf{ef 
Save } & W\textbf{ill Save } & S\textbf{pecial}\tabularnewline
\hline
1st  & +1  & +2  & +0  & +0  & Fast movement, illiteracy, rage 1/day\tabularnewline
\hline
2nd & +2 & +3 & +0 & +0 & Uncanny dodge\tabularnewline
\hline
3rd & +3 & +3 & +1 & +1 & Trap sense +1\tabularnewline
\hline
4th & +4 & +4 & +1 & +1 & Rage 2/day\tabularnewline
\hline
5th & +5 & +4 & +1 & +1 & Improved uncanny dodge\tabularnewline
\hline
6th & +6/+1 & +5 & +2 & +2 & Trap sense +2\tabularnewline
\hline
7th & +7/+2 & +5 & +2 & +2 & Damage reduction 1/---\tabularnewline
\hline
8th & +8/+3 & +6 & +2 & +2 & Rage 3/day\tabularnewline
\hline
9th & +9/+4 & +6 & +3 & +3 & Trap sense +3\tabularnewline
\hline
10th & +10/+5 & +7 & +3 & +3 & Damage reduction 2/---\tabularnewline
\hline
11th & +11/+6/+1 & +7 & +3 & +3 & Greater rage\tabularnewline
\hline
12th & +12/+7/+2 & +8 & +4 & +4 & Rage 4/day, trap sense +4\tabularnewline
\hline
13th & +13/+8/+3 & +8 & +4 & +4 & Damage reduction 3/---\tabularnewline
\hline
14th & +14/+9/+4 & +9 & +4 & +4 & Indomitable will\tabularnewline
\hline
15th & +15/+10/+5 & +9 & +5 & +5 & Trap sense +5\tabularnewline
\hline
16th & +16/+11/+6/+1 & +10 & +5 & +5 & Damage reduction 4/---, rage 5/day\tabularnewline
\hline
17th & +17/+12/+7/+2 & +10 & +5 & +5 & Tireless rage\tabularnewline
\hline
18th & +18/+13/+8/+3 & +11 & +6 & +6 & Trap sense +6\tabularnewline
\hline
19th & +19/+14/+9/+4 & +11 & +6 & +6 & Damage reduction 5/---\tabularnewline
\hline
20th & +20/+15/+10/+5 & +12 & +6 & +6 & Mighty rage, rage 6/day\tabularnewline
\hline
\end{tabular}

\vspace{12pt}
\subsection*{\textbf{Class Features}}

All of the following are class features of the barbarian.

\textbf{Weapon and Armor Proficiency:} A barbarian is proficient with all simple 
and martial weapons, light armor, medium armor, and shields (except tower shields).

\textbf{Fast Movement (Ex):} A barbarian's land speed is faster than the norm for 
his race by +10 feet. This benefit applies only when he is wearing no armor, light 
armor, or medium armor and not carrying a heavy load. Apply this bonus before modifying 
the barbarian's speed because of any load carried or armor worn.

\textbf{Illiteracy:} Barbarians are the only characters who do not automatically 
know how to read and write. A barbarian may spend 2 skill points to gain the ability 
to read and write all languages he is able to speak.

A barbarian who gains a level in any other class automatically gains literacy. 
Any other character who gains a barbarian level does not lose the literacy he or 
she already had.

\textbf{Rage (Ex):} A barbarian can fly into a rage a certain number of times per 
day. In a rage, a barbarian temporarily gains a +4 bonus to Strength, a +4 bonus 
to Constitution, and a +2 morale bonus on Will saves, but he takes a -2 penalty 
to Armor Class. The increase in Constitution increases the barbarian's hit points 
by 2 points per level, but these hit points go away at the end of the rage when 
his Constitution score drops back to normal. (These extra hit points are not lost 
first the way temporary hit points are.) While raging, a barbarian cannot use any 
Charisma-, Dexterity-, or Intelligence-based skills (except for Balance, Escape 
Artist, Intimidate, and Ride), the Concentration skill, or any abilities that require 
patience or concentration, nor can he cast spells or activate magic items that 
require a command word, a spell trigger (such as a wand), or spell completion (such 
as a scroll) to function. He can use any feat he has except Combat Expertise, item 
creation feats, and metamagic feats. A fit of rage lasts for a number of rounds 
equal to 3 + the character's (newly improved) Constitution modifier. A barbarian 
may prematurely end his rage. At the end of the rage, the barbarian loses the rage 
modifiers and restrictions and becomes fatigued (-2 penalty to Strength, -2 penalty 
to Dexterity, can't charge or run) for the duration of the current encounter (unless 
he is a 17th-level barbarian, at which point this limitation no longer applies; 
see below).

A barbarian can fly into a rage only once per encounter. At 1st level he can use 
his rage ability once per day. At 4th level and every four levels thereafter, he 
can use it one additional time per day (to a maximum of six times per day at 20th 
level). Entering a rage takes no time itself, but a barbarian can do it only during 
his action, not in response to someone else's action.

\textbf{Uncanny Dodge (Ex): }At 2nd level, a barbarian retains his Dexterity bonus 
to AC (if any) even if he is caught flat-footed or struck by an invisible attacker. 
However, he still loses his Dexterity bonus to AC if immobilized. If a barbarian 
already has uncanny dodge from a different class, he automatically gains improved 
uncanny dodge (see below) instead.

\textbf{Trap Sense (Ex):} Starting at 3rd level, a barbarian gains a +1 bonus on 
Reflex saves made to avoid traps and a +1 dodge bonus to AC against attacks made 
by traps. These bonuses rise by +1 every three barbarian levels thereafter (6th, 
9th, 12th, 15th, and 18th level). Trap sense bonuses gained from multiple classes 
stack.

\textbf{Improved Uncanny Dodge (Ex):} At 5th level and higher, a barbarian can 
no longer be flanked. This defense denies a rogue the ability to sneak attack the 
barbarian by flanking him, unless the attacker has at least four more rogue levels 
than the target has barbarian levels. If a character already has uncanny dodge 
(see above) from a second class, the character automatically gains improved uncanny 
dodge instead, and the levels from the classes that grant uncanny dodge stack to 
determine the minimum level a rogue must be to flank the character.

\textbf{Damage Reduction (Ex):} At 7th level, a barbarian gains Damage Reduction. 
Subtract 1 from the damage the barbarian takes each time he is dealt damage from 
a weapon or a natural attack. At 10th level, and every three barbarian levels thereafter 
(13th, 16th, and 19th level), this damage reduction rises by 1 point. Damage reduction 
can reduce damage to 0 but not below 0.

\textbf{Greater Rage (Ex):} At 11th level, a barbarian's bonuses to Strength and 
Constitution during his rage each increase to +6, and his morale bonus on Will 
saves increases to +3. The penalty to AC remains at -2.

\textbf{Indomitable Will (Ex):} While in a rage, a barbarian of 14th level or higher 
gains a +4 bonus on Will saves to resist enchantment spells. This bonus stacks 
with all other modifiers, including the morale bonus on Will saves he also receives 
during his rage.

\textbf{Tireless Rage (Ex):} At 17th level and higher, a barbarian no longer becomes 
fatigued at the end of his rage.

\textbf{Mighty Rage (Ex):} At 20th level, a barbarian's bonuses to Strength and 
Constitution during his rage each increase to +8, and his morale bonus on Will 
saves increases to +4. The penalty to AC remains at -2.

\vspace{12pt}
\subsection*{\textbf{Ex-Barbarians}}

A barbarian who becomes lawful loses the ability to rage and cannot gain more levels 
as a barbarian. He retains all the other benefits of the class (damage reduction, 
fast movement, trap sense, and uncanny dodge).

\vspace{12pt}
{\LARGE{}BARD}

\textbf{Alignment:} Any nonlawful.

\textbf{Hit Die:} d6.

\vspace{12pt}
\subsubsection*{\textbf{Class Skills}}

The bard's class skills (and the key ability for each skill) are Appraise (Int), 
Balance (Dex), Bluff (Cha), Climb (Str), Concentration (Con), Craft (Int), Decipher 
Script (Int), Diplomacy (Cha), Disguise (Cha), Escape Artist (Dex), Gather Information 
(Cha), Hide (Dex), Jump (Str), Knowledge (all skills, taken individually) (Int), 
Listen (Wis), Move Silently (Dex), Perform (Cha), Profession (Wis), Sense Motive 
(Wis), Sleight of Hand (Dex), Speak Language (n/a), Spellcraft (Int), Swim (Str), 
Tumble (Dex), and Use Magic Device (Cha).

\textbf{Skill Points at 1st Level:} (6 + Int modifier) x4.

\textbf{Skill Points at Each Additional Level:} 6 + Int modifier.

\vspace{12pt}
\begin{tabular}{|>{\raggedright}p{14pt}|>{\raggedright}p{28pt}|>{\raggedright}p{10pt}|>{\raggedright}p{10pt}|>{\raggedright}p{17pt}|>{\raggedright}p{61pt}|>{\raggedright}p{3pt}|>{\raggedright}p{8pt}|>{\raggedright}p{8pt}|>{\raggedright}p{8pt}|>{\raggedright}p{8pt}|>{\raggedright}p{8pt}|>{\raggedright}p{8pt}|}
\hline
\multicolumn{13}{|p{194pt}|}{
T\textbf{able: The Bard}}\tabularnewline
\hline
 &  &  &  &  & -------- & \multicolumn{7}{p{51pt}|}{\centering 
 \textbf{Spells per Day ----------}}\tabularnewline
\hline
L\textbf{evel } & B\textbf{ase Attack Bonus } & F\textbf{ort Save } & R\textbf{ef 
Save } & W\textbf{ill Save } & \subsubsection*{S\textbf{pecial }} & \centering 0 & 1\textbf{st} & 2\textbf{nd} & 3\textbf{rd} & 4\textbf{th} & 5\textbf{th} & 6\textbf{th}\tabularnewline
\hline
1st & +0 & \centering +0 & +2 & +2 & Bardic music, bardic knowledge, countersong, 
\textit{fascinate, }inspire courage +1 & \centering 2--- & --- & --- & --- & --- & --- & \tabularnewline
\hline
2nd & +1 & \centering +0 & +3 & +3 &  & \centering 3 & 0--- & --- & --- & --- & --- & \tabularnewline
\hline
3rd & +2 & \centering +1 & +3 & +3 & Inspire competence & \centering 3 & 1--- & --- & --- & --- & --- & \tabularnewline
\hline
4th & +3 & \centering +1 & +4 & +4 &  & \centering 3 & 2 & 0--- & --- & --- & --- & \tabularnewline
\hline
5th & +3 & \centering +1 & +4 & +4 &  & \centering 3 & 3 & 1--- & --- & --- & --- & \tabularnewline
\hline
6th & +4 & \centering +2 & +5 & +5 & S\textit{uggestion} & \centering 3 & 3 & 2--- & --- & --- & --- & \tabularnewline
\hline
7th & +5 & \centering +2 & +5 & +5 &  & \centering 3 & 3 & 2 & 0--- & --- & --- & \tabularnewline
\hline
8th & +6/+1 & \centering +2 & +6 & +6 & Inspire courage +2 & \centering 3 & 3 & 3 & 1--- & --- & --- & \tabularnewline
\hline
9th & +6/+1 & \centering +3 & +6 & +6 & Inspire greatness & \centering 3 & 3 & 3 & 2--- & --- & --- & \tabularnewline
\hline
10th & +7/+2 & \centering +3 & +7 & +7 &  & \centering 3 & 3 & 3 & 2 & 0--- & --- & \tabularnewline
\hline
11th & +8/+3 & \centering +3 & +7 & +7 &  & \centering 3 & 3 & 3 & 3 & 1--- & --- & \tabularnewline
\hline
12th & +9/+4 & \centering +4 & +8 & +8 & S\textit{ong of freedom} & \centering 3 & 3 & 3 & 3 & 2--- & --- & \tabularnewline
\hline
13th & +9/+4 & \centering +4 & +8 & +8 &  & \centering 3 & 3 & 3 & 3 & 2 & 0--- & \tabularnewline
\hline
14th & +10/+5 & \centering +4 & +9 & +9 & Inspire courage +3 & \centering 4 & 3 & 3 & 3 & 3 & 1--- & \tabularnewline
\hline
15th & +11/+6/+1 & \centering +5 & +9 & +9 & Inspire heroics & \centering 4 & 4 & 3 & 3 & 3 & 2--- & \tabularnewline
\hline
16th & +12/+7/+2 & \centering +5 & +10 & +10 &  & \centering 4 & 4 & 4 & 3 & 3 & 2 & 0\tabularnewline
\hline
17th & +12/+7/+2 & \centering +5 & +10 & +10 &  & \centering 4 & 4 & 4 & 4 & 3 & 3 & 1\tabularnewline
\hline
18th & +13/+8/+3 & \centering +6 & +11 & +11 & M\textit{ass suggestion} & \centering 4 & 4 & 4 & 4 & 4 & 3 & 2\tabularnewline
\hline
19th & +14/+9/+4 & \centering +6 & +11 & +11 &  & \centering 4 & 4 & 4 & 4 & 4 & 4 & 3\tabularnewline
\hline
20th & +15/+10/+5 & \centering +6 & +12 & +12 & Inspire courage +4 & \centering 4 & 4 & 4 & 4 & 4 & 4 & 4\tabularnewline
\hline
\end{tabular}

\vspace{12pt}
\begin{tabular}{|>{\raggedright}p{24pt}|>{\raggedright}p{6pt}|>{\raggedright}p{15pt}|>{\raggedright}p{19pt}|>{\raggedright}p{15pt}|>{\raggedright}p{15pt}|>{\raggedright}p{17pt}|>{\raggedright}p{21pt}|}
\hline
\multicolumn{8}{|p{134pt}|}{
T\textbf{able: Bard Spells Known}}\tabularnewline
\hline
--------------- & \multicolumn{7}{p{109pt}|}{\centering 
 \textbf{Spells Known ---------------}}\tabularnewline
\hline
L\textbf{evel} & 0 & 1\textbf{st} & 2\textbf{nd} & 3\textbf{rd} & 4\textbf{th} & 5\textbf{th} & 6\textbf{th}\tabularnewline
\hline
1st & 4--- & --- & --- & --- & --- & --- & \tabularnewline
\hline
2nd & 5 & 2\textsuperscript{1}--- & --- & --- & --- & --- & \tabularnewline
\hline
3rd & 6 & 3--- & --- & --- & --- & --- & \tabularnewline
\hline
4th & 6 & 3 & 2\textsuperscript{1}--- & --- & --- & --- & \tabularnewline
\hline
5th & 6 & 4 & 3--- & --- & --- & --- & \tabularnewline
\hline
6th & 6 & 4 & 3--- & --- & --- & --- & \tabularnewline
\hline
7th & 6 & 4 & 4 & 2\textsuperscript{1}--- & --- & --- & \tabularnewline
\hline
8th & 6 & 4 & 4 & 3--- & --- & --- & \tabularnewline
\hline
9th & 6 & 4 & 4 & 3--- & --- & --- & \tabularnewline
\hline
10th & 6 & 4 & 4 & 4 & 2\textsuperscript{1}--- & --- & \tabularnewline
\hline
11th & 6 & 4 & 4 & 4 & 3--- & --- & \tabularnewline
\hline
12th & 6 & 4 & 4 & 4 & 3--- & --- & \tabularnewline
\hline
13th & 6 & 4 & 4 & 4 & 4 & 2\textsuperscript{1}--- & \tabularnewline
\hline
14th & 6 & 4 & 4 & 4 & 4 & 3--- & \tabularnewline
\hline
15th & 6 & 4 & 4 & 4 & 4 & 3--- & \tabularnewline
\hline
16th & 6 & 5 & 4 & 4 & 4 & 4 & 2\textsuperscript{1}\tabularnewline
\hline
17th & 6 & 5 & 5 & 4 & 4 & 4 & 3\tabularnewline
\hline
18th & 6 & 5 & 5 & 5 & 4 & 4 & 3\tabularnewline
\hline
19th & 6 & 5 & 5 & 5 & 5 & 4 & 4\tabularnewline
\hline
20th & 6 & 5 & 5 & 5 & 5 & 5 & 4\tabularnewline
\hline
\multicolumn{8}{|p{134pt}|}{
1 Provided the bard has a high enough Charisma score to have a bonus spell of this 
level.}\tabularnewline
\hline
\end{tabular}

\vspace{12pt}
\subsubsection*{\textbf{Class Features}}

All of the following are class features of the bard.

\textbf{Weapon and Armor Proficiency:} A bard is proficient with all simple weapons, 
plus the longsword, rapier, sap, short sword, shortbow, and whip. Bards are proficient 
with light armor and shields (except tower shields). A bard can cast bard spells 
while wearing light armor without incurring the normal arcane spell failure chance. 
However, like any other arcane spellcaster, a bard wearing medium or heavy armor 
or using a shield incurs a chance of arcane spell failure if the spell in question 
has a somatic component (most do). A multiclass bard still incurs the normal arcane 
spell failure chance for arcane spells received from other classes.

\textbf{Spells:} A bard casts arcane spells, which are drawn from the bard spell 
list. He can cast any spell he knows without preparing it ahead of time. Every 
bard spell has a verbal component (singing, reciting, or music). To learn or cast 
a spell, a bard must have a Charisma score equal to at least 10 + the spell. The 
Difficulty Class for a saving throw against a bard's spell is 10 + the spell level 
+ the bard's Charisma modifier.

Like other spellcasters, a bard can cast only a certain number of spells of each 
spell level per day. His base daily spell allotment is given on Table: The Bard. 
In addition, he receives bonus spells per day if he has a high Charisma score. 
When Table: Bard Spells Known indicates that the bard gets 0 spells per day of 
a given spell level, he gains only the bonus spells he would be entitled to based 
on his Charisma score for that spell level.

The bard's selection of spells is extremely limited. A bard begins play knowing 
four 0-level spells of your choice. At most new bard levels, he gains one or more 
new spells, as indicated on Table: Bard Spells Known. (Unlike spells per day, the 
number of spells a bard knows is not affected by his Charisma score; the numbers 
on Table: Bard Spells Known are fixed.)

Upon reaching 5th level, and at every third bard level after that (8th, 11th, and 
so on), a bard can choose to learn a new spell in place of one he already knows. 
In effect, the bard ``loses'' the old spell in exchange for the new one. The new 
spell's level must be the same as that of the spell being exchanged, and it must 
be at least two levels lower than the highest-level bard spell the bard can cast. 
A bard may swap only a single spell at any given level, and must choose whether 
or not to swap the spell at the same time that he gains new spells known for the 
level.

As noted above, a bard need not prepare his spells in advance. He can cast any 
spell he knows at any time, assuming he has not yet used up his allotment of spells 
per day for the spell's level. 

\textbf{Bardic Knowledge:} A bard may make a special bardic knowledge check with 
a bonus equal to his bard level + his Intelligence modifier to see whether he knows 
some relevant information about local notable people, legendary items, or noteworthy 
places. (If the bard has 5 or more ranks in Knowledge (history), he gains a +2 
bonus on this check.)

A successful bardic knowledge check will not reveal the powers of a magic item 
but may give a hint as to its general function. A bard may not take 10 or take 
20 on this check; this sort of knowledge is essentially random. 

\vspace{12pt}
\begin{tabular}{|>{\raggedright}p{11pt}|>{\raggedright}p{314pt}|}
\hline
\subsubsection*{D\textbf{C}} & \subsubsection*{T\textbf{ype of Knowledge}}\tabularnewline
\hline
10  & Common, known by at least a substantial minority drinking; common legends 
of the local population.\tabularnewline
\hline
20  & Uncommon but available, known by only a few people legends.\tabularnewline
\hline
25  & Obscure, known by few, hard to come by.\tabularnewline
\hline
30  & Extremely obscure, known by very few, possibly forgotten by most who once 
knew it, possibly known only by those who don't understand the significance of 
the knowledge.\tabularnewline
\hline
\end{tabular}

\vspace{12pt}
\textbf{Bardic Music:} Once per day per bard level, a bard can use his song or 
poetics to produce magical effects on those around him (usually including himself, 
if desired). While these abilities fall under the category of bardic music and 
the descriptions discuss singing or playing instruments, they can all be activated 
by reciting poetry, chanting, singing lyrical songs, singing melodies, whistling, 
playing an instrument, or playing an instrument in combination with some spoken 
performance. Each ability requires both a minimum bard level and a minimum number 
of ranks in the Perform skill to qualify; if a bard does not have the required 
number of ranks in at least one Perform skill, he does not gain the bardic music 
ability until he acquires the needed ranks.

Starting a bardic music effect is a standard action. Some bardic music abilities 
require concentration, which means the bard must take a standard action each round 
to maintain the ability. Even while using bardic music that doesn't require concentration, 
a bard cannot cast spells, activate magic items by spell completion (such as scrolls), 
or activate magic items by magic word (such as wands). Just as for casting a spell 
with a verbal component, a deaf bard has a 20\% chance to fail when attempting 
to use bardic music. If he fails, the attempt still counts against his daily limit.

\textit{Countersong (Su): }A bard with 3 or more ranks in a Perform skill can use 
his music or poetics to counter magical effects that depend on sound (but not spells 
that simply have verbal components). Each round of the countersong, he makes a 
Perform check. Any creature within 30 feet of the bard (including the bard himself 
) that is affected by a sonic or language-dependent magical attack may use the 
bard's Perform check result in place of its saving throw if, after the saving throw 
is rolled, the Perform check result proves to be higher. If a creature within range 
of the countersong is already under the effect of a noninstantaneous sonic or language-dependent 
magical attack, it gains another saving throw against the effect each round it 
hears the countersong, but it must use the bard's Perform check result for the 
save. Countersong has no effect against effects that don't allow saves. The bard 
may keep up the countersong for 10 rounds.

\textit{Fascinate (Sp): }A bard with 3 or more ranks in a Perform skill can use 
his music or poetics to cause one or more creatures to become fascinated with him. 
Each creature to be fascinated must be within 90 feet, able to see and hear the 
bard, and able to pay attention to him. The bard must also be able to see the creature. 
The distraction of a nearby combat or other dangers prevents the ability from working. 
For every three levels a bard attains beyond 1st, he can target one additional 
creature with a single use of this ability.

To use the ability, a bard makes a Perform check. His check result is the DC for 
each affected creature's Will save against the effect. If a creature's saving throw 
succeeds, the bard cannot attempt to fascinate that creature again for 24 hours. 
If its saving throw fails, the creature sits quietly and listens to the song, taking 
no other actions, for as long as the bard continues to play and concentrate (up 
to a maximum of 1 round per bard level). While fascinated, a target takes a -4 
penalty on skill checks made as reactions, such as Listen and Spot checks. Any 
potential threat requires the bard to make another Perform check and allows the 
creature a new saving throw against a DC equal to the new Perform check result.

Any obvious threat, such as someone drawing a weapon, casting a spell, or aiming 
a ranged weapon at the target, automatically breaks the effect. \textit{Fascinate 
}is an enchantment (compulsion), mind-affecting ability.

\textit{Inspire Courage (Su): }A bard with 3 or more ranks in a Perform skill can 
use song or poetics to inspire courage in his allies (including himself ), bolstering 
them against fear and improving their combat abilities. To be affected, an ally 
must be able to hear the bard sing. The effect lasts for as long as the ally hears 
the bard sing and for 5 rounds thereafter. An affected ally receives a +1 morale 
bonus on saving throws against charm and fear effects and a +1 morale bonus on 
attack and weapon damage rolls. At 8th level, and every six bard levels thereafter, 
this bonus increases by 1 (+2 at 8th, +3 at 14th, and +4 at 20th). Inspire courage 
is a mind-affecting ability.

\textit{Inspire Competence (Su): }A bard of 3rd level or higher with 6 or more 
ranks in a Perform skill can use his music or poetics to help an ally succeed at 
a task. The ally must be within 30 feet and able to see and hear the bard. The 
bard must also be able to see the ally.

The ally gets a +2 competence bonus on skill checks with a particular skill as 
long as he or she continues to hear the bard's music. Certain uses of this ability 
are infeasible. The effect lasts as long as the bard concentrates, up to a maximum 
of 2 minutes. A bard can't inspire competence in himself. Inspire competence is 
a mind-affecting ability.

\textit{Suggestion (Sp): }A bard of 6th level or higher with 9 or more ranks in 
a Perform skill can make a \textit{suggestion }(as the spell) to a creature that 
he has already fascinated (see above). Using this ability does not break the bard's 
concentration on the \textit{fascinate }effect, nor does it allow a second saving 
throw against the \textit{fascinate }effect.

Making a \textit{suggestion }doesn't count against a bard's daily limit on bardic 
music performances. A Will saving throw (DC 10 + 1/2 bard's level + bard's Cha 
modifier) negates the effect. This ability affects only a single creature (but 
see \textit{mass suggestion}, below). \textit{Suggestion }is an enchantment (compulsion), 
mind-affecting, language dependent ability.

\textit{Inspire Greatness (Su): }A bard of 9th level or higher with 12 or more 
ranks in a Perform skill can use music or poetics to inspire greatness in himself 
or a single willing ally within 30 feet, granting him or her extra fighting capability. 
For every three levels a bard attains beyond 9th, he can target one additional 
ally with a single use of this ability (two at 12th level, three at 15th, four 
at 18th). To inspire greatness, a bard must sing and an ally must hear him sing. 
The effect lasts for as long as the ally hears the bard sing and for 5 rounds thereafter. 
A creature inspired with greatness gains 2 bonus Hit Dice (d10s), the commensurate 
number of temporary hit points (apply the target's Constitution modifier, if any, 
to these bonus Hit Dice), a +2 competence bonus on attack rolls, and a +1 competence 
bonus on Fortitude saves. The bonus Hit Dice count as regular Hit Dice for determining 
the effect of spells that are Hit Dice dependant. Inspire greatness is a mind-affecting 
ability.

\textit{Song of Freedom (Sp): }A bard of 12th level or higher with 15 or more ranks 
in a Perform skill can use music or poetics to create an effect equivalent to the 
\textit{break enchantment }spell (caster level equals the character's bard level). 
Using this ability requires 1 minute of uninterrupted concentration and music, 
and it functions on a single target within 30 feet. A bard can't use \textit{song 
of freedom }on himself.

\textit{Inspire Heroics (Su): }A bard of 15th level or higher with 18 or more ranks 
in a Perform skill can use music or poetics to inspire tremendous heroism in himself 
or a single willing ally within 30 feet. For every three bard levels the character 
attains beyond 15th, he can inspire heroics in one additional creature. To inspire 
heroics, a bard must sing and an ally must hear the bard sing for a full round. 
A creature so inspired gains a +4 morale bonus on saving throws and a +4 dodge 
bonus to AC. The effect lasts for as long as the ally hears the bard sing and for 
up to 5 rounds thereafter. Inspire heroics is a mind-affecting ability.

\textit{Mass Suggestion (Sp): }This ability functions like \textit{suggestion, 
}above, except that a bard of 18th level or higher with 21 or more ranks in a Perform 
skill can make the \textit{suggestion }simultaneously to any number of creatures 
that he has already fascinated (see above). \textit{Mass suggestion }is an enchantment 
(compulsion), mind-affecting, language-dependent ability.

\vspace{12pt}
\subsubsection*{\textbf{Ex-Bards}}

A bard who becomes lawful in alignment cannot progress in levels as a bard, though 
he retains all his bard abilities.

\vspace{12pt}
{\LARGE{}CLERIC}

\textbf{Alignment:} A cleric's alignment must be within one step of his deity's 
(that is, it may be one step away on either the lawful-chaotic axis or the good-evil 
axis, but not both). A cleric may not be neutral unless his deity's alignment is 
also neutral.

\textbf{Hit Die:} d8.

\vspace{12pt}
\subsubsection*{\textbf{Class Skills}}

The cleric's class skills (and the key ability for each skill) are Concentration 
(Con), Craft (Int), Diplomacy (Cha), Heal (Wis), Knowledge (arcana) (Int), Knowledge 
(history) (Int), Knowledge (religion) (Int), Knowledge (the planes) (Int), Profession 
(Wis), and Spellcraft (Int). 

\textbf{Domains and Class Skills:} A cleric who chooses the Animal or Plant domain 
adds Knowledge (nature) (Int) to the cleric class skills listed above. A cleric 
who chooses the Knowledge domain adds all Knowledge (Int) skills to the list. A 
cleric who chooses the Travel domain adds Survival (Wis) to the list. A cleric 
who chooses the Trickery domain adds Bluff (Cha), Disguise (Cha), and Hide (Dex) 
to the list. See Deity, Domains, and Domain Spells, below, for more information.

\textbf{Skill Points at 1st Level:} (2 + Int modifier) x$ $4.

\textbf{Skill Points at Each Additional Level:} 2 + Int modifier.

\vspace{12pt}
\subsubsection*{}

  \begin{tabular}{|>{\raggedright}p{13pt}|>{\raggedright}p{25pt}|>{\raggedright}p{11pt}|>{\raggedright}p{10pt}|>{\raggedright}p{10pt}|>{\raggedright}p{19pt}|>{\raggedright}p{1pt}|>{\raggedright}p{7pt}|>{\raggedright}p{7pt}|>{\raggedright}p{7pt}|>{\raggedright}p{7pt}|>{\raggedright}p{7pt}|>{\raggedright}p{7pt}|>{\raggedright}p{7pt}|>{\raggedright}p{7pt}|>{\raggedright}p{7pt}|}
\hline
\multicolumn{16}{|p{158pt}|}{\textbf{Table: The Cleric}}\tabularnewline
\hline
 &  &  &  &  & ------------------------ & \multicolumn{10}{p{67pt}|}{
 \textbf{Spells per Day}\textsuperscript{\textbf{1}}\textbf{ ----------------------}}\tabularnewline
\hline
L\textbf{evel} & B\textbf{ase Attack Bonus} & F\textbf{ort Save} & R\textbf{ef 
Save} & W\textbf{ill Save} & \centering S\textbf{pecial} & 0 & 1\textbf{st} & 2\textbf{nd} & 3\textbf{rd} & 4\textbf{th} & 5\textbf{th} & 6\textbf{th} & 7\textbf{th} & 8\textbf{th} & 9\textbf{th}\tabularnewline
\hline
1st & +0 & \centering +2 & +0 & +2 & Turn or rebuke undead & \centering 3 & 1+1--- & --- & --- & --- & --- & --- & --- & --- & \tabularnewline
\hline
2nd & +1 & \centering +3 & +0 & +3 &  & \centering 4 & 2+1--- & --- & --- & --- & --- & --- & --- & --- & \tabularnewline
\hline
3rd & +2 & \centering +3 & +1 & +3 &  & \centering 4 & 2+1 & 1+1--- & --- & --- & --- & --- & --- & --- & \tabularnewline
\hline
4th & +3 & \centering +4 & +1 & +4 &  & \centering 5 & 3+1 & 2+1--- & --- & --- & --- & --- & --- & --- & \tabularnewline
\hline
5th & +3 & \centering +4 & +1 & +4 &  & \centering 5 & 3+1 & 2+1 & 1+1--- & --- & --- & --- & --- & --- & \tabularnewline
\hline
6th & +4 & \centering +5 & +2 & +5 &  & \centering 5 & 3+1 & 3+1 & 2+1--- & --- & --- & --- & --- & --- & \tabularnewline
\hline
7th & +5 & \centering +5 & +2 & +5 &  & \centering 6 & 4+1 & 3+1 & 2+1 & 1+1--- & --- & --- & --- & --- & \tabularnewline
\hline
8th & +6/+1 & \centering +6 & +2 & +6 &  & \centering 6 & 4+1 & 3+1 & 3+1 & 2+1--- & --- & --- & --- & --- & \tabularnewline
\hline
9th & +6/+1 & \centering +6 & +3 & +6 &  & \centering 6 & 4+1 & 4+1 & 3+1 & 2+1 & 1+1--- & --- & --- & --- & \tabularnewline
\hline
10th & +7/+2 & \centering +7 & +3 & +7 &  & \centering 6 & 4+1 & 4+1 & 3+1 & 3+1 & 2+1--- & --- & --- & --- & \tabularnewline
\hline
11th & +8/+3 & \centering +7 & +3 & +7 &  & \centering 6 & 5+1 & 4+1 & 4+1 & 3+1 & 2+1 & 1+1--- & --- & --- & \tabularnewline
\hline
12th & +9/+4 & \centering +8 & +4 & +8 &  & \centering 6 & 5+1 & 4+1 & 4+1 & 3+1 & 3+1 & 2+1--- & --- & --- & \tabularnewline
\hline
13th & +9/+4 & \centering +8 & +4 & +8 &  & \centering 6 & 5+1 & 5+1 & 4+1 & 4+1 & 3+1 & 2+1 & 1+1--- & --- & \tabularnewline
\hline
14th & +10/+5 & \centering +9 & +4 & +9 &  & \centering 6 & 5+1 & 5+1 & 4+1 & 4+1 & 3+1 & 3+1 & 2+1--- & --- & \tabularnewline
\hline
15th & +11/+6/+1 & \centering +9 & +5 & +9 &  & \centering 6 & 5+1 & 5+1 & 5+1 & 4+1 & 4+1 & 3+1 & 2+1 & 1+1--- & \tabularnewline
\hline
16th & +12/+7/+2 & \centering +10 & +5 & +10 &  & \centering 6 & 5+1 & 5+1 & 5+1 & 4+1 & 4+1 & 3+1 & 3+1 & 2+1--- & \tabularnewline
\hline
17th & +12/+7/+2 & \centering +10 & +5 & +10 &  & \centering 6 & 5+1 & 5+1 & 5+1 & 5+1 & 4+1 & 4+1 & 3+1 & 2+1 & 1+1\tabularnewline
\hline
18th & +13/+8/+3 & \centering +11 & +6 & +11 &  & \centering 6 & 5+1 & 5+1 & 5+1 & 5+1 & 4+1 & 4+1 & 3+1 & 3+1 & 2+1\tabularnewline
\hline
19th & +14/+9/+4 & \centering +11 & +6 & +11 &  & \centering 6 & 5+1 & 5+1 & 5+1 & 5+1 & 5+1 & 4+1 & 4+1 & 3+1 & 3+1\tabularnewline
\hline
20th & +15/+10/+5 & \centering +12 & +6 & +12 &  & \centering 6 & 5+1 & 5+1 & 5+1 & 5+1 & 5+1 & 4+1 & 4+1 & 4+1 & 4+1\tabularnewline
\hline
\multicolumn{16}{|p{158pt}|}{
1 In addition to the stated number of spells per day for 1st- through 9th-level 
spells, a cleric gets a domain spell for each spell level, starting at 1st.\linebreak{}
The ``+1'' in the entries on this table represents that spell. Domain spells are 
in addition to any bonus spells the cleric may receive for having a high Wisdom 
score.}\tabularnewline
\hline
\end{tabular}

\vspace{12pt}
\subsubsection*{\textbf{Class Features}}

All of the following are class features of the cleric.

\textbf{Weapon and Armor Proficiency:} Clerics are proficient with all simple weapons, 
with all types of armor (light, medium, and heavy), and with shields (except tower 
shields).

A cleric who chooses the War domain receives the Weapon Focus feat related to his 
deity's weapon as a bonus feat. He also receives the appropriate Martial Weapon 
Proficiency feat as a bonus feat, if the weapon falls into that category.

\textbf{Aura (Ex):} A cleric of a chaotic, evil, good, or lawful deity has a particularly 
powerful aura corresponding to the deity's alignment (see the \textit{detect evil 
}spell for details). Clerics who don't worship a specific deity but choose the 
Chaotic, Evil, Good, or Lawful domain have a similarly powerful aura of the corresponding 
alignment.

\textbf{Spells:} A cleric casts divine spells, which are drawn from the cleric 
spell list. However, his alignment may restrict him from casting certain spells 
opposed to his moral or ethical beliefs; see Chaotic, Evil, Good, and Lawful Spells, 
below. A cleric must choose and prepare his spells in advance (see below).

To prepare or cast a spell, a cleric must have a Wisdom score equal to at least 
10 + the spell level. The Difficulty Class for a saving throw against a cleric's 
spell is 10 + the spell level + the cleric's Wisdom modifier.

Like other spellcasters, a cleric can cast only a certain number of spells of each 
spell level per day. His base daily spell allotment is given on Table: The Cleric. 
In addition, he receives bonus spells per day if he has a high Wisdom score. A 
cleric also gets one domain spell of each spell level he can cast, starting at 
1st level. When a cleric prepares a spell in a domain spell slot, it must come 
from one of his two domains (see Deities, Domains, and Domain Spells, below).

Clerics meditate or pray for their spells. Each cleric must choose a time at which 
he must spend 1 hour each day in quiet contemplation or supplication to regain 
his daily allotment of spells. Time spent resting has no effect on whether a cleric 
can prepare spells. A cleric may prepare and cast any spell on the cleric spell 
list, provided that he can cast spells of that level, but he must choose which 
spells to prepare during his daily meditation.

\textbf{Deity, Domains, and Domain Spells:} A cleric's deity influences his alignment, 
what magic he can perform, his values, and how others see him. A cleric chooses 
two domains from among those belonging to his deity. A cleric can select an alignment 
domain (Chaos, Evil, Good, or Law) only if his alignment matches that domain.

If a cleric is not devoted to a particular deity, he still selects two domains 
to represent his spiritual inclinations and abilities. The restriction on alignment 
domains still applies.

Each domain gives the cleric access to a domain spell at each spell level he can 
cast, from 1st on up, as well as a granted power. The cleric gets the granted powers 
of both the domains selected.

With access to two domain spells at a given spell level, a cleric prepares one 
or the other each day in his domain spell slot. If a domain spell is not on the 
cleric spell list, a cleric can prepare it only in his domain spell slot.

\textbf{Spontaneous Casting:} A good cleric (or a neutral cleric of a good deity) 
can channel stored spell energy into healing spells that the cleric did not prepare 
ahead of time. The cleric can ``lose'' any prepared spell that is not a domain 
spell in order to cast any \textit{cure }spell of the same spell level or lower 
(a \textit{cure }spell is any spell with ``cure'' in its name). 

An evil cleric (or a neutral cleric of an evil deity), can't convert prepared spells 
to \textit{cure }spells but can convert them to \textit{inflict }spells (an \textit{inflict 
}spell is one with ``inflict'' in its name).

A cleric who is neither good nor evil and whose deity is neither good nor evil 
can convert spells to either \textit{cure }spells or \textit{inflict }spells (player's 
choice). Once the player makes this choice, it cannot be reversed. This choice 
also determines whether the cleric turns or commands undead (see below).

\textbf{Chaotic, Evil, Good, and Lawful Spells:} A cleric can't cast spells of 
an alignment opposed to his own or his deity's (if he has one). Spells associated 
with particular alignments are indicated by the chaos, evil, good, and law descriptors 
in their spell descriptions.

\textbf{Turn or Rebuke Undead (Su):} Any cleric, regardless of alignment, has the 
power to affect undead creatures by channeling the power of his faith through his 
holy (or unholy) symbol (see Turn or Rebuke Undead).

A good cleric (or a neutral cleric who worships a good deity) can turn or destroy 
undead creatures. An evil cleric (or a neutral cleric who worships an evil deity) 
instead rebukes or commands such creatures. A neutral cleric of a neutral deity 
must choose whether his turning ability functions as that of a good cleric or an 
evil cleric. Once this choice is made, it cannot be reversed. This decision also 
determines whether the cleric can cast spontaneous \textit{cure }or \textit{inflict 
}spells (see above).

A cleric may attempt to turn undead a number of times per day equal to 3 + his 
Charisma modifier. A cleric with 5 or more ranks in Knowledge (religion) gets a 
+2 bonus on turning checks against undead.

\textbf{Bonus Languages:} A cleric's bonus language options include Celestial, 
Abyssal, and Infernal (the languages of good, chaotic evil, and lawful evil outsiders, 
respectively). These choices are in addition to the bonus languages available to 
the character because of his race.

\vspace{12pt}
\subsubsection*{\textbf{Ex-Clerics}}

A cleric who grossly violates the code of conduct required by his god loses all 
spells and class features, except for armor and shield proficiencies and proficiency 
with simple weapons. He cannot thereafter gain levels as a cleric of that god until 
he atones (see the \textit{atonement }spell description).

\vspace{12pt}
{\LARGE{}DRUID}

\textbf{Alignment:} Neutral good, lawful neutral, neutral, chaotic neutral, or 
neutral evil.

\textbf{Hit Die:} d8.

\vspace{12pt}
\subsubsection*{\textbf{Class Skills}}

The druid's class skills (and the key ability for each skill) are Concentration 
(Con), Craft (Int), Diplomacy (Cha), Handle Animal (Cha), Heal (Wis), Knowledge 
(nature) (Int), Listen (Wis), Profession (Wis), Ride (Dex), Spellcraft (Int), Spot 
(Wis), Survival (Wis), and Swim (Str). 

\textbf{Skill Points at 1st Level:} (4 + Int modifier) x$ $4.

\textbf{Skill Points at Each Additional Level: }4 + Int modifier.

\vspace{12pt}
\subsubsection*{}
\begin{tabular}{|>{\raggedright}p{13pt}|>{\raggedright}p{25pt}|>{\raggedright}p{13pt}|>{\raggedright}p{9pt}|>{\raggedright}p{9pt}|>{\raggedright}p{28pt}|>{\raggedright}p{1pt}|>{\raggedright}p{5pt}|>{\raggedright}p{7pt}|>{\raggedright}p{6pt}|>{\raggedright}p{6pt}|>{\raggedright}p{6pt}|>{\raggedright}p{6pt}|>{\raggedright}p{6pt}|>{\raggedright}p{6pt}|>{\raggedright}p{6pt}|}
\hline
\multicolumn{16}{|p{158pt}|}{
T\textbf{able: The Druid}}\tabularnewline
\hline
 &  &  &  &  & ------------------------ & \multicolumn{10}{p{57pt}|}{
 \textbf{Spells per Day ----------------------}}\tabularnewline
\hline
L\textbf{evel} & B\textbf{ase Attack Bonus} & F\textbf{ort Save} & R\textbf{ef 
Save} & W\textbf{ill Save} & S\textbf{pecial} & 0 & 1\textbf{st} & 2\textbf{nd} & 3\textbf{rd} & 4\textbf{th} & 5\textbf{th} & 6\textbf{th} & 7\textbf{th} & 8\textbf{th} & 9\textbf{th}\tabularnewline
\hline
1st & +0 & +2 & +0 & +2 & Animal companion, nature sense, wild empathy & \centering 3 & 1--- & --- & --- & --- & --- & --- & --- & --- & \tabularnewline
\hline
2nd & +1 & +3 & +0 & +3 & Woodland stride & \centering 4 & 2--- & --- & --- & --- & --- & --- & --- & --- & \tabularnewline
\hline
3rd & +2 & +3 & +1 & +3 & Trackless step & \centering 4 & 2 & 1--- & --- & --- & --- & --- & --- & --- & \tabularnewline
\hline
4th & +3 & +4 & +1 & +4 & Resist nature's lure & \centering 5 & 3 & 2--- & --- & --- & --- & --- & --- & --- & \tabularnewline
\hline
5th & +3 & +4 & +1 & +4 & Wild shape (1/day) & \centering 5 & 3 & 2 & 1--- & --- & --- & --- & --- & --- & \tabularnewline
\hline
6th & +4 & +5 & +2 & +5 & Wild shape (2/day) & \centering 5 & 3 & 3 & 2--- & --- & --- & --- & --- & --- & \tabularnewline
\hline
7th & +5 & +5 & +2 & +5 & Wild shape (3/day) & \centering 6 & 4 & 3 & 2 & 1--- & --- & --- & --- & --- & \tabularnewline
\hline
8th & +6/+1 & +6 & +2 & +6 & Wild shape (Large) & \centering 6 & 4 & 3 & 3 & 2--- & --- & --- & --- & --- & \tabularnewline
\hline
9th & +6/+1 & +6 & +3 & +6 & Venom immunity & \centering 6 & 4 & 4 & 3 & 2 & 1--- & --- & --- & --- & \tabularnewline
\hline
10th & +7/+2 & +7 & +3 & +7 & Wild shape (4/day) & \centering 6 & 4 & 4 & 3 & 3 & 2--- & --- & --- & --- & \tabularnewline
\hline
11th & +8/+3 & +7 & +3 & +7 & Wild shape (Tiny) & \centering 6 & 5 & 4 & 4 & 3 & 2 & 1--- & --- & --- & \tabularnewline
\hline
12th & +9/+4 & +8 & +4 & +8 & Wild shape (plant) & \centering 6 & 5 & 4 & 4 & 3 & 3 & 2--- & --- & --- & \tabularnewline
\hline
13th & +9/+4 & +8 & +4 & +8 & A thousand faces & \centering 6 & 5 & 5 & 4 & 4 & 3 & 2 & 1--- & --- & \tabularnewline
\hline
14th & +10/+5 & +9 & +4 & +9 & Wild shape (5/day) & \centering 6 & 5 & 5 & 4 & 4 & 3 & 3 & 2--- & --- & \tabularnewline
\hline
15th & +11/+6/+1 & +9 & +5 & +9  & Timeless body, wild shape (Huge) & \centering 6 & 5 & 5 & 5 & 4 & 4 & 3 & 2 & 1--- & \tabularnewline
\hline
16th & +12/+7/+2 & +10 & +5 & +10 & Wild shape (elemental 1/day) & \centering 6 & 5 & 5 & 5 & 4 & 4 & 3 & 3 & 2--- & \tabularnewline
\hline
17th & +12/+7/+2 & +10 & +5 & +10 &  & \centering 6 & 5 & 5 & 5 & 5 & 4 & 4 & 3 & 2 & 1\tabularnewline
\hline
18th & +13/+8/+3 & +11 & +6 & +11 & Wild shape (6/day, elemental 2/day) & \centering 6 & 5 & 5 & 5 & 5 & 4 & 4 & 3 & 3 & 2\tabularnewline
\hline
19th & +14/+9/+4 & +11 & +6 & +11 &  & \centering 6 & 5 & 5 & 5 & 5 & 5 & 4 & 4 & 3 & 3\tabularnewline
\hline
20th & +15/+10/+5 & +12 & +6 & +12 & Wild shape (elemental 3/day, Huge elemental) & \centering 6 & 5 & 5 & 5 & 5 & 5 & 4 & 4 & 4 & 4\tabularnewline
\hline
\end{tabular}

\vspace{12pt}
\subsubsection*{\textbf{Class Features}}

All of the following are class features of the druid.

\textbf{Weapon and Armor Proficiency:} Druids are proficient with the following 
weapons: club, dagger, dart, quarterstaff, scimitar, sickle, shortspear, sling, 
and spear. They are also proficient with all natural attacks (claw, bite, and so 
forth) of any form they assume with wild shape (see below).

Druids are proficient with light and medium armor but are prohibited from wearing 
metal armor; thus, they may wear only padded, leather, or hide armor. (A druid 
may also wear wooden armor that has been altered by the \textit{ironwood }spell 
so that it functions as though it were steel. See the \textit{ironwood }spell description) 
Druids are proficient with shields (except tower shields) but must use only wooden 
ones.

A druid who wears prohibited armor or carries a prohibited shield is unable to 
cast druid spells or use any of her supernatural or spell-like class abilities 
while doing so and for 24 hours thereafter.

\textbf{Spells:} A druid casts divine spells, which are drawn from the druid spell 
list. Her alignment may restrict her from casting certain spells opposed to her 
moral or ethical beliefs; see Chaotic, Evil, Good, and Lawful Spells, below. A 
druid must choose and prepare her spells in advance (see below).

To prepare or cast a spell, the druid must have a Wisdom score equal to at least 
10 + the spell level. The Difficulty Class for a saving throw against a druid's 
spell is 10 + the spell level + the druid's Wisdom modifier.

Like other spellcasters, a druid can cast only a certain number of spells of each 
spell level per day. Her base daily spell allotment is given on Table: The Druid. 
In addition, she receives bonus spells per day if she has a high Wisdom score. 
She does not have access to any domain spells or granted powers, as a cleric does.

A druid prepares and casts spells the way a cleric does, though she cannot lose 
a prepared spell to cast a \textit{cure }spell in its place (but see Spontaneous 
Casting, below). A druid may prepare and cast any spell on the druid spell list, 
provided that she can cast spells of that level, but she must choose which spells 
to prepare during her daily meditation.

\textbf{Spontaneous Casting:} A druid can channel stored spell energy into summoning 
spells that she hasn't prepared ahead of time. She can ``lose'' a prepared spell 
in order to cast any \textit{summon nature's ally }spell of the same level or lower.\textbf{ 
Chaotic, Evil, Good, and Lawful Spells:} A druid can't cast spells of an alignment 
opposed to her own or her deity's (if she has one). Spells associated with particular 
alignments are indicated by the chaos, evil, good, and law descriptors in their 
spell descriptions.

\textbf{Bonus Languages:} A druid's bonus language options include Sylvan, the 
language of woodland creatures. This choice is in addition to the bonus languages 
available to the character because of her race.

A druid also knows Druidic, a secret language known only to druids, which she learns 
upon becoming a 1st-level druid. Druidic is a free language for a druid; that is, 
she knows it in addition to her regular allotment of languages and it doesn't take 
up a language slot. Druids are forbidden to teach this language to nondruids.

Druidic has its own alphabet.

\textbf{Animal Companion (Ex):} A druid may begin play with an animal companion 
selected from the following list: badger, camel, dire rat, dog, riding dog, eagle, 
hawk, horse (light or heavy), owl, pony, snake (Small or Medium viper), or wolf. 
If the campaign takes place wholly or partly in an aquatic environment, the following 
creatures are also available: crocodile, porpoise, Medium shark, and squid. This 
animal is a loyal companion that accompanies the druid on her adventures as appropriate 
for its kind.

A 1st-level druid's companion is completely typical for its kind except as noted 
below. As a druid advances in level, the animal's power increases as shown on the 
table. If a druid releases her companion from service, she may gain a new one by 
performing a ceremony requiring 24 uninterrupted hours of prayer. This ceremony 
can also replace an animal companion that has perished.

A druid of 4th level or higher may select from alternative lists of animals (see 
below). Should she select an animal companion from one of these alternative lists, 
the creature gains abilities as if the character's druid level were lower than 
it actually is. Subtract the value indicated in the appropriate list header from 
the character's druid level and compare the result with the druid level entry on 
the table to determine the animal companion's powers. (If this adjustment would 
reduce the druid's effective level to 0 or lower, she can't have that animal as 
a companion.) 

\textbf{Nature Sense (Ex):} A druid gains a +2 bonus on Knowledge (nature) and 
Survival checks.

\textbf{Wild Empathy (Ex):} A druid can improve the attitude of an animal. This 
ability functions just like a Diplomacy check made to improve the attitude of a 
person. The druid rolls 1d20 and adds her druid level and her Charisma modifier 
to determine the wild empathy check result.

The typical domestic animal has a starting attitude of indifferent, while wild 
animals are usually unfriendly.

To use wild empathy, the druid and the animal must be able to study each other, 
which means that they must be within 30 feet of one another under normal conditions. 
Generally, influencing an animal in this way takes 1 minute but, as with influencing 
people, it might take more or less time.

A druid can also use this ability to influence a magical beast with an Intelligence 
score of 1 or 2, but she takes a -4 penalty on the check.

\textbf{Woodland Stride (Ex):} Starting at 2nd level, a druid may move through 
any sort of undergrowth (such as natural thorns, briars, overgrown areas, and similar 
terrain) at her normal speed and without taking damage or suffering any other impairment. 
However, thorns, briars, and overgrown areas that have been magically manipulated 
to impede motion still affect her.

\textbf{Trackless Step (Ex):} Starting at 3rd level, a druid leaves no trail in 
natural surroundings and cannot be tracked. She may choose to leave a trail if 
so desired.

\textbf{Resist Nature's Lure (Ex):} Starting at 4th level, a druid gains a +4 bonus 
on saving throws against the spell-like abilities of fey.

\textbf{Wild Shape (Su):} At 5th level, a druid gains the ability to turn herself 
into any Small or Medium animal and back again once per day. Her options for new 
forms include all creatures with the animal type. This ability functions like the 
\textit{polymorph }spell, except as noted here. The effect lasts for 1 hour per 
druid level, or until she changes back. Changing form (to animal or back) is a 
standard action and doesn't provoke an attack of opportunity.

The form chosen must be that of an animal the druid is familiar with. 

A druid loses her ability to speak while in animal form because she is limited 
to the sounds that a normal, untrained animal can make, but she can communicate 
normally with other animals of the same general grouping as her new form. (The 
normal sound a wild parrot makes is a squawk, so changing to this form does not 
permit speech.)

A druid can use this ability more times per day at 6th, 7th, 10th, 14th, and 18th 
level, as noted on Table: The Druid\textit{. }In addition, she gains the ability 
to take the shape of a Large animal at 8th level, a Tiny animal at 11th level, 
and a Huge animal at 15th level.

The new form's Hit Dice can't exceed the character's druid level.

At 12th level, a druid becomes able to use wild shape to change into a plant creature 
with the same size restrictions as for animal forms. (A druid can't use this ability 
to take the form of a plant that isn't a creature.)

At 16th level, a druid becomes able to use wild shape to change into a Small, Medium, 
or Large elemental (air, earth, fire, or water) once per day. These elemental forms 
are in addition to her normal wild shape usage. In addition to the normal effects 
of wild shape, the druid gains all the elemental's extraordinary, supernatural, 
and spell-like abilities. She also gains the elemental's feats for as long as she 
maintains the wild shape, but she retains her own creature type.

At 18th level, a druid becomes able to assume elemental form twice per day, and 
at 20th level she can do so three times per day. At 20th level, a druid may use 
this wild shape ability to change into a Huge elemental.

\textbf{Venom Immunity (Ex):} At 9th level, a druid gains immunity to all poisons. 

\textbf{A Thousand Faces (Su):} At 13th level, a druid gains the ability to change 
her appearance at will, as if using the \textit{alter self }spell, but only while 
in her normal form.

\textbf{Timeless Body (Ex):} After attaining 15th level, a druid no longer takes 
ability score penalties for aging and cannot be magically aged. Any penalties she 
may have already incurred, however, remain in place.

Bonuses still accrue, and the druid still dies of old age when her time is up.

\vspace{12pt}
\subsubsection*{\textbf{Ex-Druids}}

A druid who ceases to revere nature, changes to a prohibited alignment, or teaches 
the Druidic language to a nondruid loses all spells and druid abilities (including 
her animal companion, but not including weapon, armor, and shield proficiencies). 
She cannot thereafter gain levels as a druid until she atones (see the \textit{atonement 
}spell description).

\vspace{12pt}
THE DRUID'S ANIMAL COMPANION

A druid's animal companion is different from a normal animal of its kind in many 
ways. A druid's animal companion is superior to a normal animal of its kind and 
has special powers, as described below.

\vspace{12pt}
\begin{tabular}{|>{\raggedright}p{38pt}|>{\raggedright}p{33pt}|>{\raggedright}p{66pt}|>{\raggedright}p{41pt}|>{\raggedright}p{43pt}|>{\raggedright}p{55pt}|}
\hline
C\textbf{lass Level } & B\textbf{onus HD } & N\textbf{atural Armor Adj. } & S\textbf{tr/Dex 
Adj.} & B\textbf{onus Tricks } & S\textbf{pecial}\tabularnewline
\hline
1st-2nd  & \centering +0 & +0 & +0 & 1 & Link, share spells\tabularnewline
\hline
3rd-5th  & \centering +2 & +2 & +1 & 2 & Evasion\tabularnewline
\hline
6th-8th  & \centering +4 & +4 & +2 & 3 & Devotion\tabularnewline
\hline
9th-11th  & \centering +6 & +6 & +3 & 4 & Multiattack\tabularnewline
\hline
12th-14th  & \centering +8 & +8 & +4 & 5 & \tabularnewline
\hline
15th-17th  & \centering +10 & +10 & +5 & 6 & Improved evasion\tabularnewline
\hline
18th-20th  & \centering +12 & +12 & +6 & 7 & \tabularnewline
\hline
\end{tabular}

\textbf{Animal Companion Basics: }Use the base statistics for a creature of the 
companion's kind,\textit{ }but make the following changes.

\textit{Class Level: }The character's druid level. The druid's class levels stack 
with levels of any other classes that are entitled to an animal companion for the 
purpose of determining the companion's abilities and the alternative lists available 
to the character.

\textit{Bonus HD: }Extra eight-sided (d8) Hit Dice, each of which gains a Constitution 
modifier, as normal. Remember that extra Hit Dice improve the animal companion's 
base attack and base save bonuses. An animal companion's base attack bonus is the 
same as that of a druid of a level equal to the animal's HD. An animal companion 
has good Fortitude and Reflex saves (treat it as a character whose level equals 
the animal's HD). An animal companion gains additional skill points and feats for 
bonus HD as normal for advancing a monster's Hit Dice.

\textit{Natural Armor Adj.: }The number noted here is an improvement to the animal 
companion's existing natural armor bonus.

\textit{Str/Dex Adj.: }Add this value to the animal companion's Strength and Dexterity 
scores.

\textit{Bonus Tricks: }The value given in this column is the total number of ``bonus'' 
tricks that the animal knows in addition to any that the druid might choose to 
teach it (see the Handle Animal skill). These bonus tricks don't require any training 
time or Handle Animal checks, and they don't count against the normal limit of 
tricks known by the animal. The druid selects these bonus tricks, and once selected, 
they can't be changed.

\textit{Link (Ex): }A druid can handle her animal companion as a free action, or 
push it as a move action, even if she doesn't have any ranks in the Handle Animal 
skill. The druid gains a +4 circumstance bonus on all wild empathy checks and Handle 
Animal checks made regarding an animal companion.

\textit{Share Spells (Ex): }At the druid's option, she may have any spell (but 
not any spell-like ability) she casts upon herself also affect her animal companion. 
The animal companion must be within 5 feet of her at the time of casting to receive 
the benefit. If the spell or effect has a duration other than instantaneous, it 
stops affecting the animal companion if the companion moves farther than 5 feet 
away and will not affect the animal again, even if it returns to the druid before 
the duration expires. 

Additionally, the druid may cast a spell with a target of ``You'' on her animal 
companion (as a touch range spell) instead of on herself. A druid and her animal 
companion can share spells even if the spells normally do not affect creatures 
of the companion's type (animal).

\textit{Evasion (Ex): }If an animal companion is subjected to an attack that normally 
allows a Reflex saving throw for half damage, it takes no damage if it makes a 
successful saving throw.

\textit{Devotion (Ex): }An animal companion gains a +4 morale bonus on Will saves 
against enchantment spells and effects.

\textit{Multiattack: }An animal companion gains Multiattack as a bonus feat if 
it has three or more natural attacks and does not already have that feat. If it 
does not have the requisite three or more natural attacks, the animal companion 
instead gains a second attack with its primary natural weapon, albeit at a -5 penalty.

\textit{Improved Evasion (Ex): }When subjected to an attack that normally allows 
a Reflex saving throw for half damage, an animal companion takes no damage if it 
makes a successful saving throw and only half damage if the saving throw fails.

\vspace{12pt}
ALTERNATIVE ANIMAL COMPANIONS

A druid of sufficiently high level can select her animal companion from one of 
the following lists, applying the indicated adjustment to the druid's level (in 
parentheses) for purposes of determining the companion's characteristics and special 
abilities.

\vspace{12pt}
\textbf{4th Level or Higher (Level -3)}

Ape (animal) 

Bear, black (animal) 

Bison (animal)

Boar (animal) 

Cheetah (animal) 

Crocodile (animal)\textsuperscript{\textbf{1}}

Dire badger 

Dire bat

Dire weasel

Leopard (animal)

Lizard, monitor (animal)

Shark, Large\textsuperscript{\textbf{1}}\textbf{ }(animal)

Snake, constrictor (animal)

Snake, Large viper (animal)

Wolverine (animal)

\vspace{12pt}
\textbf{7th Level or Higher (Level -6)}

Bear, brown (animal) 

Dire wolverine

Crocodile, giant (animal) 

Deinonychus (dinosaur) 

Dire ape 

Dire boar 

Dire wolf 

Elasmosaurus\textsuperscript{\textbf{1}}\textbf{ }(dinosaur)

Lion (animal)

Rhinoceros (animal)

Snake, Huge viper (animal)

Tiger (animal)

\vspace{12pt}
\textbf{10th Level or Higher (Level -9)}

Bear, polar (animal) 

Dire lion 

Megaraptor (dinosaur) 

Shark, Huge\textsuperscript{\textbf{1}}\textbf{ }(animal)

Snake, giant constrictor (animal)

Whale, orca\textsuperscript{\textbf{1}}\textbf{ }(animal)

\vspace{12pt}
\textbf{13th Level or Higher (Level -12)}

Dire bear 

Elephant (animal)

Octopus, giant\textsuperscript{\textbf{1}}\textbf{ }(animal)

\vspace{12pt}
\textbf{16th Level or Higher (Level -15)}

Dire shark\textsuperscript{\textbf{1}}\textbf{ }

Dire tiger 

Squid, giant\textsuperscript{\textbf{1}}\textbf{ }(animal)

Triceratops (dinosaur)

Tyrannosaurus (dinosaur)

\vspace{12pt}
\textsuperscript{1} Available only in an aquatic environment.

\vspace{12pt}
{\LARGE{}FIGHTER}

\textbf{Alignment:} Any.

\textbf{Hit Die:} d10.

\vspace{12pt}
\subsubsection*{\textbf{Class Skills}}

The fighter's class skills (and the key ability for each skill) are Climb (Str), 
Craft (Int), Handle Animal (Cha), Intimidate (Cha), Jump (Str), Ride (Dex), and 
Swim (Str).

\textbf{Skill Points at 1st Level:} (2 + Int modifier) x$ $4.

\textbf{Skill Points at Each Additional Level:} 2 + Int modifier.

\vspace{12pt}
\begin{tabular}{|>{\raggedright}p{24pt}|>{\raggedright}p{66pt}|>{\raggedright}p{36pt}|>{\raggedright}p{32pt}|>{\raggedright}p{35pt}|>{\raggedright}p{44pt}|}
\hline
\multicolumn{6}{|p{239pt}|}{
\subsection*{T\textbf{able: The Fighter}}}\tabularnewline
\hline
L\textbf{evel} & B\textbf{ase Attack Bonus} & F\textbf{ort Save} & R\textbf{ef 
Save} & W\textbf{ill Save} & S\textbf{pecial}\tabularnewline
\hline
1st & +1 & +2 & +0 & +0 & Bonus feat\tabularnewline
\hline
2nd & +2 & +3 & +0 & +0 & Bonus feat\tabularnewline
\hline
3rd & +3 & +3 & +1 & +1 & \tabularnewline
\hline
4th & +4 & +4 & +1 & +1 & Bonus feat\tabularnewline
\hline
5th & +5 & +4 & +1 & +1 & \tabularnewline
\hline
6th & +6/+1 & +5 & +2 & +2 & Bonus feat\tabularnewline
\hline
7th & +7/+2 & +5 & +2 & +2 & \tabularnewline
\hline
8th & +8/+3 & +6 & +2 & +2 & Bonus feat\tabularnewline
\hline
9th & +9/+4 & +6 & +3 & +3 & \tabularnewline
\hline
10th & +10/+5 & +7 & +3 & +3 & Bonus feat\tabularnewline
\hline
11th & +11/+6/+1 & +7 & +3 & +3 & \tabularnewline
\hline
12th & +12/+7/+2 & +8 & +4 & +4 & Bonus feat\tabularnewline
\hline
13th & +13/+8/+3 & +8 & +4 & +4 & \tabularnewline
\hline
14th & +14/+9/+4 & +9 & +4 & +4 & Bonus feat\tabularnewline
\hline
15th & +15/+10/+5 & +9 & +5 & +5 & \tabularnewline
\hline
16th & +16/+11/+6/+1 & +10 & +5 & +5 & Bonus feat\tabularnewline
\hline
17th & +17/+12/+7/+2 & +10 & +5 & +5 & \tabularnewline
\hline
18th & +18/+13/+8/+3 & +11 & +6 & +6 & Bonus feat\tabularnewline
\hline
19th & +19/+14/+9/+4 & +11 & +6 & +6 & \tabularnewline
\hline
20th & +20/+15/+10/+5 & +12 & +6 & +6 & Bonus feat\tabularnewline
\hline
\end{tabular}

\vspace{12pt}
\subsubsection*{\textbf{Class Features}}

All of the following are class features of the fighter.

\textbf{Weapon and Armor Proficiency: }A fighter is proficient with all simple 
and martial weapons and with all armor (heavy, medium, and light) and shields (including 
tower shields).

\textbf{Bonus Feats:} At 1st level, a fighter gets a bonus combat-oriented feat 
in addition to the feat that any 1st-level character gets and the bonus feat granted 
to a human character. The fighter gains an additional bonus feat at 2nd level and 
every two fighter levels thereafter (4th, 6th, 8th, 10th, 12th, 14th, 16th, 18th, 
and 20th). These bonus feats must be drawn from the feats noted as fighter bonus 
feats. A fighter must still meet all prerequisites for a bonus feat, including 
ability score and base attack bonus minimums.

These bonus feats are in addition to the feat that a character of any class gets 
from advancing levels. A fighter is not limited to the list of fighter bonus feats 
when choosing these feats.

\vspace{12pt}
{\LARGE{}MONK}

\textbf{Alignment:} Any lawful.

\textbf{Hit Die:} d8.

\vspace{12pt}
\subsubsection*{\textbf{Class Skills}}

The monk's class skills (and the key ability for each skill) are Balance (Dex), 
Climb (Str), Concentration (Con), Craft (Int), Diplomacy (Cha), Escape Artist (Dex), 
Hide (Dex), Jump (Str), Knowledge (arcana) (Int), Knowledge (religion) (Int), Listen 
(Wis), Move Silently (Dex), Perform (Cha), Profession (Wis), Sense Motive (Wis), 
Spot (Wis), Swim (Str), and Tumble (Dex).

\textbf{Skill Points at 1st Level:} (4 + Int modifier) x$ $4.

\textbf{Skill Points at Each Additional Level:} 4 + Int modifier.

\vspace{12pt}
\begin{tabular}{|>{\raggedright}p{27pt}|>{\raggedright}p{28pt}|>{\raggedright}p{12pt}|>{\raggedright}p{12pt}|>{\raggedright}p{10pt}|>{\raggedright}p{51pt}|>{\raggedright}p{47pt}|>{\raggedright}p{23pt}|>{\raggedright}p{15pt}|>{\raggedright}p{28pt}|}
\hline
\multicolumn{10}{|p{256pt}|}{
T\textbf{able: The Monk}}\tabularnewline
\hline
L\textbf{evel} & B\textbf{ase Attack Bonus} & F\textbf{ort Save} & R\textbf{ef 
Save} & W\textbf{ill Save} & S\textbf{pecial} & F\textbf{lurry of Blows Attack 
Bonus} & U\textbf{narmed Damage}\textsuperscript{\textbf{1}} & A\textbf{C Bonus} & U\textbf{narmored 
 Speed Bonus}\tabularnewline
\hline
1st & +0 & +2 & +2 & +2 & Bonus feat, flurry of blows, unarmed strike- & 2/-2 & 1d6 & +0 & +0 
ft.\tabularnewline
\hline
2nd & +1 & +3 & +3 & +3 & Bonus feat, evasion- & 1/-1 & 1d6 & +0 & +0 ft.\tabularnewline
\hline
3rd & +2 & +3 & +3 & +3 & Still mind & +0/+0 & 1d6 & +0 & +10 ft.\tabularnewline
\hline
4th & +3 & +4 & +4 & +4 & K\textit{i }strike (magic), \linebreak{}
slow fall 20 ft. & +1/+1 & 1d8 & +0 & +10 ft.\tabularnewline
\hline
5th & +3 & +4 & +4 & +4 & Purity of body & +2/+2 & 1d8 & +1 & +10 ft.\tabularnewline
\hline
6th & +4 & +5 & +5 & +5 & Bonus feat, \linebreak{}
slow fall 30 ft. & +3/+3 & 1d8 & +1 & +20 ft.\tabularnewline
\hline
7th & +5 & +5 & +5 & +5 & Wholeness of body & +4/+4 & 1d8 & +1 & +20 ft.\tabularnewline
\hline
8th & +6/+1 & +6 & +6 & +6 & Slow fall 40 ft. & +5/+5/+0 & 1d10 & +1 & +20 ft.\tabularnewline
\hline
9th & +6/+1 & +6 & +6 & +6 & Improved evasion & +6/+6/+1 & 1d10 & +1 & +30 ft.\tabularnewline
\hline
10th & +7/+2 & +7 & +7 & +7 & K\textit{i }strike (lawful), \linebreak{}
slow fall 50 ft. & +7/+7/+2 & 1d10 & +2 & +30 ft.\tabularnewline
\hline
11th & +8/+3 & +7 & +7 & +7 & Diamond body, \linebreak{}
greater flurry & +8/+8/+8/+3 & 1d10 & +2 & +30 ft.\tabularnewline
\hline
12th & +9/+4 & +8 & +8 & +8 & Abundant step, \linebreak{}
slow fall 60 ft. & +9/+9/+9/+4 & 2d6 & +2 & +40 ft.\tabularnewline
\hline
13th & +9/+4 & +8 & +8 & +8 & Diamond soul & +9/+9/+9/+4 & 2d6 & +2 & +40 ft.\tabularnewline
\hline
14th & +10/+5 & +9 & +9 & +9 & Slow fall 70 ft. & +10/+10/+10/+5 & 2d6 & +2 & +40 
ft.\tabularnewline
\hline
15th & +11/+6/+1 & +9 & +9 & +9 & Quivering palm & +11/+11/+11/+6/+1 & 2d6 & +3 & +50 
ft.\tabularnewline
\hline
16th & +12/+7/+2 & +10 & +10 & +10 & K\textit{i }strike (adamantine), slow fall 
80 ft. & +12/+12/+12/+7/+2 & 2d8 & +3 & +50 ft.\tabularnewline
\hline
17th & +12/+7/+2 & +10 & +10 & +10 & Timeless body, tongue of the sun and moon & +12/+12/+12/+7/+2 & 2d8 & +3 & +50 
ft.\tabularnewline
\hline
18th & +13/+8/+3 & +11 & +11 & +11 & Slow fall 90 ft. & +13/+13/+13/+8/+3 & 2d8 & +3 & +60 
ft.\tabularnewline
\hline
19th & +14/+9/+4 & +11 & +11 & +11 & Empty body & +14/+14/+14/+9/+4 & 2d8 & +3 & +60 
ft.\tabularnewline
\hline
20th & +15/+10/+5 & +12 & +12 & +12 & Perfect self, \linebreak{}
slow fall any distance & +15/+15/+15/+10/+5 & 2d10 & +4 & +60 ft.\tabularnewline
\hline
\multicolumn{10}{|p{256pt}|}{
1 The value shown is for Medium monks. See Table: Small or Large Monk Unarmed Damage 
for Small or Large monks.}\tabularnewline
\hline
\end{tabular}

\vspace{12pt}
\subsubsection*{\textbf{Class Features}}

All of the following are class features of the monk.

\textbf{Weapon and Armor Proficiency: }Monks are proficient with club, crossbow 
(light or heavy), dagger, handaxe, javelin, kama, nunchaku, quarterstaff, sai, 
shuriken, siangham, and sling.

Monks are not proficient with any armor or shields

When wearing armor, using a shield, or carrying a medium or heavy load, a monk 
loses her AC bonus, as well as her fast movement and flurry of blows abilities.

\textbf{AC Bonus (Ex):} When unarmored and unencumbered, the monk adds her Wisdom 
bonus (if any) to her AC. In addition, a monk gains a +1 bonus to AC at 5th level. 
This bonus increases by 1 for every five monk levels thereafter (+2 at 10th, +3 
at 15th, and +4 at 20th level).

These bonuses to AC apply even against touch attacks or when the monk is flat-footed. 
She loses these bonuses when she is immobilized or helpless, when she wears any 
armor, when she carries a shield, or when she carries a medium or heavy load.

\textbf{Flurry of Blows (Ex):} When unarmored, a monk may strike with a flurry 
of blows at the expense of accuracy. When doing so, she may make one extra attack 
in a round at her highest base attack bonus, but this attack takes a -2 penalty, 
as does each other attack made that round. The resulting modified base attack bonuses 
are shown in the Flurry of Blows Attack Bonus column on Table: The Monk. This penalty 
applies for 1 round, so it also affects attacks of opportunity the monk might make 
before her next action. When a monk reaches 5th level, the penalty lessens to -1, 
and at 9th level it disappears. A monk must use a full attack action to strike 
with a flurry of blows.

When using flurry of blows, a monk may attack only with unarmed strikes or with 
special monk weapons (kama, nunchaku, quarterstaff, sai, shuriken, and siangham). 
She may attack with unarmed strikes and special monk weapons interchangeably as 
desired. When using weapons as part of a flurry of blows, a monk applies her Strength 
bonus (not Str bonus x $ $1-1/2 or x$ $1/2) to her damage rolls for all successful 
attacks, whether she wields a weapon in one or both hands. The monk can't use any 
weapon other than a special monk weapon as part of a flurry of blows.

In the case of the quarterstaff, each end counts as a separate weapon for the purpose 
of using the flurry of blows ability. Even though the quarterstaff requires two 
hands to use, a monk may still intersperse unarmed strikes with quarterstaff strikes, 
assuming that she has enough attacks in her flurry of blows routine to do so. 

When a monk reaches 11th level, her flurry of blows ability improves. In addition 
to the standard single extra attack she gets from flurry of blows, she gets a second 
extra attack at her full base attack bonus.

\textbf{Unarmed Strike:} At 1st level, a monk gains Improved Unarmed Strike as 
a bonus feat. A monk's attacks may be with either fist interchangeably or even 
from elbows, knees, and feet. This means that a monk may even make unarmed strikes 
with her hands full. There is no such thing as an off-hand attack for a monk striking 
unarmed. A monk may thus apply her full Strength bonus on damage rolls for all 
her unarmed strikes.

Usually a monk's unarmed strikes deal lethal damage, but she can choose to deal 
nonlethal damage instead with no penalty on her attack roll. She has the same choice 
to deal lethal or nonlethal damage while grappling.

A monk's unarmed strike is treated both as a manufactured weapon and a natural 
weapon for the purpose of spells and effects that enhance or improve either manufactured 
weapons or natural weapons.

A monk also deals more damage with her unarmed strikes than a normal person would, 
as shown on Table: The Monk. The unarmed damage on Table: The Monk is for Medium 
monks. A Small monk deals less damage than the amount given there with her unarmed 
attacks, while a Large monk deals more damage; see Table: Small or Large Monk Unarmed 
Damage.

\vspace{12pt}
\begin{tabular}{|>{\raggedright}p{47pt}|>{\raggedright}p{64pt}|>{\raggedright}p{69pt}|}
\hline
\multicolumn{3}{|p{180pt}|}{
\subsection*{T\textbf{able: Small or Large Monk Unarmed Damage}}}\tabularnewline
\hline
\centering L\textbf{evel} & D\textbf{amage }\linebreak{}
\textbf{(Small Monk)} & D\textbf{amage}\linebreak{}
\textbf{(Large Monk)}\tabularnewline
\hline
1st-3rd  & \centering 1d4 & 1d8\tabularnewline
\hline
4th-7th  & \centering 1d6 & 2d6\tabularnewline
\hline
8th-11th  & \centering 1d8 & 2d8\tabularnewline
\hline
12th-15th  & \centering 1d10 & 3d6\tabularnewline
\hline
16th-19th  & \centering 2d6 & 3d8\tabularnewline
\hline
20th  & \centering 2d8 & 4d8\tabularnewline
\hline
\end{tabular}

\vspace{12pt}
\textbf{Bonus Feat:} At 1st level, a monk may select either Improved Grapple or 
Stunning Fist as a bonus feat. At 2nd level, she may select either Combat Reflexes 
or Deflect Arrows as a bonus feat. At 6th level, she may select either Improved 
Disarm or Improved Trip as a bonus feat. A monk need not have any of the prerequisites 
normally required for these feats to select them.

\textbf{Evasion (Ex):} At 2nd level or higher if a monk makes a successful Reflex 
saving throw against an attack that normally deals half damage on a successful 
save, she instead takes no damage. Evasion can be used only if a monk is wearing 
light armor or no armor. A helpless monk does not gain the benefit of evasion.

\textbf{Fast Movement (Ex):} At 3rd level, a monk gains an enhancement bonus to 
her speed, as shown on Table: The Monk. A monk in armor or carrying a medium or 
heavy load loses this extra speed.

\textbf{Still Mind (Ex):} A monk of 3rd level or higher gains a +2 bonus on saving 
throws against spells and effects from the school of enchantment.

\textit{\textbf{Ki }}\textbf{Strike (Su):} At 4th level, a monk's unarmed attacks 
are empowered with \textit{ki. }Her unarmed attacks are treated as magic weapons 
for the purpose of dealing damage to creatures with damage reduction. \textit{Ki 
}strike improves with the character's monk level. At 10th level, her unarmed attacks 
are also treated as lawful weapons for the purpose of dealing damage to creatures 
with damage reduction. At 16th level, her unarmed attacks are treated as adamantine 
weapons for the purpose of dealing damage to creatures with damage reduction and 
bypassing hardness.

\textbf{Slow Fall (Ex): }At 4th level or higher, a monk within arm's reach of a 
wall can use it to slow her descent. When first using this ability, she takes damage 
as if the fall were 20 feet shorter than it actually is. The monk's ability to 
slow her fall (that is, to reduce the effective distance of the fall when next 
to a wall) improves with her monk level until at 20th level she can use a nearby 
wall to slow her descent and fall any distance without harm.

\textbf{Purity of Body (Ex):} At 5th level, a monk gains immunity to all diseases 
except for supernatural and magical diseases.

\textbf{Wholeness of Body (Su):} At 7th level or higher, a monk can heal her own 
wounds. She can heal a number of hit points of damage equal to twice her current 
monk level each day, and she can spread this healing out among several uses.

\textbf{Improved Evasion (Ex):} At 9th level, a monk's evasion ability improves. 
She still takes no damage on a successful Reflex saving throw against attacks, 
but henceforth she takes only half damage on a failed save. A helpless monk does 
not gain the benefit of improved evasion.

\textbf{Diamond Body (Su):} At 11th level, a monk gains immunity to poisons of 
all kinds.

\textbf{Abundant Step (Su):} At 12th level or higher, a monk can slip magically 
between spaces, as if using the spell \textit{dimension door, }once per day. Her 
caster level for this effect is one-half her monk level (rounded down).

\textbf{Diamond Soul (Ex):} At 13th level, a monk gains spell resistance equal 
to her current monk level + 10. In order to affect the monk with a spell, a spellcaster 
must get a result on a caster level check (1d20 + caster level) that equals or 
exceeds the monk's spell resistance.

\textbf{Quivering Palm (Su):} Starting at 15th level, a monk can set up vibrations 
within the body of another creature that can thereafter be fatal if the monk so 
desires. She can use this quivering palm attack once a week, and she must announce 
her intent before making her attack roll. Constructs, oozes, plants, undead, incorporeal 
creatures, and creatures immune to critical hits cannot be affected. Otherwise, 
if the monk strikes successfully and the target takes damage from the blow, the 
quivering palm attack succeeds. Thereafter the monk can try to slay the victim 
at any later time, as long as the attempt is made within a number of days equal 
to her monk level. To make such an attempt, the monk merely wills the target to 
die (a free action), and unless the target makes a Fortitude saving throw (DC 10 
+ 1/2 the monk's level + the monk's Wis modifier), it dies. If the saving throw 
is successful, the target is no longer in danger from that particular quivering 
palm attack, but it may still be affected by another one at a later time.

\textbf{Timeless Body (Ex):} Upon attaining 17th level, a monk no longer takes 
penalties to her ability scores for aging and cannot be magically aged. Any such 
penalties that she has already taken, however, remain in place. Bonuses still accrue, 
and the monk still dies of old age when her time is up.

\textbf{Tongue of the Sun and Moon (Ex): }A monk of 17th level or higher can speak 
with any living creature.

\textbf{Empty Body (Su):} At 19th level, a monk gains the ability to assume an 
ethereal state for 1 round per monk level per day, as though using the spell \textit{etherealness}. 
She may go ethereal on a number of different occasions during any single day, as 
long as the total number of rounds spent in an ethereal state does not exceed her 
monk level.

\textbf{Perfect Self:} At 20th level, a monk becomes a magical creature. She is 
forevermore treated as an outsider rather than as a humanoid (or whatever the monk's 
creature type was) for the purpose of spells and magical effects. Additionally, 
the monk gains damage reduction 10/magic, which allows her to ignore the first 
10 points of damage from any attack made by a nonmagical weapon or by any natural 
attack made by a creature that doesn't have similar damage reduction. Unlike other 
outsiders, the monk can still be brought back from the dead as if she were a member 
of her previous creature type.

\vspace{12pt}
\subsubsection*{\textbf{Ex-Monks}}

A monk who becomes nonlawful cannot gain new levels as a monk but retains all monk 
abilities.

Like a member of any other class, a monk may be a multiclass character, but multiclass 
monks face a special restriction. A monk who gains a new class or (if already multiclass) 
raises another class by a level may never again raise her monk level, though she 
retains all her monk abilities.

\newpage

\end{document}
