%&pdfLaTeX
% !TEX encoding = UTF-8 Unicode
\documentclass{article}
\usepackage{ifxetex}
\ifxetex
\usepackage{fontspec}
\setmainfont[Mapping=tex-text]{STIXGeneral}
\else
\usepackage[T1]{fontenc}
\usepackage[utf8]{inputenc}
\fi
\usepackage{textcomp}

\usepackage{array}
\usepackage{amssymb}
\usepackage{fancyhdr}
\renewcommand{\headrulewidth}{0pt}
\renewcommand{\footrulewidth}{0pt}

\begin{document}

This material is Open Game Content, and is licensed for public use under the terms 
of the Open Game License v1.0a.

{\LARGE{}CLASSES II}

\vspace{12pt}
{\LARGE{}PALADIN}

\textbf{Alignment: }Lawful good.

\textbf{Hit Die:} d10.

\vspace{12pt}
\subsubsection*{\textbf{Class Skills}}

The paladin's class skills (and the key ability for each skill) are Concentration 
(Con), Craft (Int), Diplomacy (Cha), Handle Animal (Cha), Heal (Wis), Knowledge 
(nobility and royalty) (Int), Knowledge (religion) (Int), Profession (Wis), Ride 
(Dex), and Sense Motive (Wis).

\textbf{Skill Points at 1st Level:} (2 + Int modifier) x4.

\textbf{Skill Points at Each Additional Level:} 2 + Int modifier.

\vspace{12pt}
\begin{tabular}{|>{\raggedright}p{17pt}|>{\raggedright}p{43pt}|>{\raggedright}p{13pt}|>{\raggedright}p{14pt}|>{\raggedright}p{15pt}|>{\raggedright}p{80pt}|>{\raggedright}p{10pt}|>{\raggedright}p{12pt}|>{\raggedright}p{11pt}|>{\raggedright}p{11pt}|}
\hline
\multicolumn{10}{|p{230pt}|}{T\textbf{able: The Paladin}}\tabularnewline
\hline
 &  &  &  &  & --- & \multicolumn{4}{p{45pt}|}{ \textbf{Spells per Day ---}}\tabularnewline
\hline
L\textbf{evel} & B\textbf{ase Attack Bonus} & F\textbf{ort Save} & R\textbf{ef 
Save} & W\textbf{ill Save} & \subsection*{S\textbf{pecial}} & 1\textbf{st} & 2\textbf{nd} & 3\textbf{rd} & 4\textbf{th}\tabularnewline
\hline
1st & +1 & +2 & +0 & +0 & Aura of good\textit{, detect evil}, \linebreak{}
smite evil 1/day--- & --- & --- & --- & \tabularnewline
\hline
2nd & +2 & +3 & +0 & +0 & Divine grace, lay on hands--- & --- & --- & --- & \tabularnewline
\hline
3rd & +3 & +3 & +1 & +1 & Aura of courage, divine health--- & --- & --- & --- & \tabularnewline
\hline
4th & +4 & +4 & +1 & +1 & Turn undead & 0--- & --- & --- & \tabularnewline
\hline
5th & +5 & +4 & +1 & +1 & Smite evil 2/day, special mount & 0--- & --- & --- & \tabularnewline
\hline
6th & +6/+1 & +5 & +2 & +2 & R\textit{emove disease }1/week & 1--- & --- & --- & \tabularnewline
\hline
7th & +7/+2 & +5 & +2 & +2 &  & 1--- & --- & --- & \tabularnewline
\hline
8th & +8/+3 & +6 & +2 & +2 &  & 1 & 0--- & --- & \tabularnewline
\hline
9th & +9/+4 & +6 & +3 & +3 & R\textit{emove disease }2/week & 1 & 0--- & --- & \tabularnewline
\hline
10th & +10/+5 & +7 & +3 & +3 & Smite evil 3/day & 1 & 1--- & --- & \tabularnewline
\hline
11th & +11/+6/+1 & +7 & +3 & +3 &  & 1 & 1 & 0--- & \tabularnewline
\hline
12th & +12/+7/+2 & +8 & +4 & +4 & R\textit{emove disease }3/week & 1 & 1 & 1--- & \tabularnewline
\hline
13th & +13/+8/+3 & +8 & +4 & +4 &  & 1 & 1 & 1--- & \tabularnewline
\hline
14th & +14/+9/+4 & +9 & +4 & +4 &  & 2 & 1 & 1 & 0\tabularnewline
\hline
15th & +15/+10/+5 & +9 & +5 & +5 & R\textit{emove disease }4/week, \linebreak{}
smite evil 4/day & 2 & 1 & 1 & 1\tabularnewline
\hline
16th & +16/+11/+6/+1 & +10 & +5 & +5 &  & 2 & 2 & 1 & 1\tabularnewline
\hline
17th & +17/+12/+7/+2 & +10 & +5 & +5 &  & 2 & 2 & 2 & 1\tabularnewline
\hline
18th & +18/+13/+8/+3 & +11 & +6 & +6 & R\textit{emove disease }5/week & 3 & 2 & 2 & 1\tabularnewline
\hline
19th & +19/+14/+9/+4 & +11 & +6 & +6 &  & 3 & 3 & 3 & 2\tabularnewline
\hline
20th & +20/+15/+10/+5 & +12 & +6 & +6 & Smite evil 5/day & 3 & 3 & 3 & 3\tabularnewline
\hline
\end{tabular}

\vspace{12pt}
\subsubsection*{\textbf{Class Features}}

All of the following are class features of the paladin.

\textbf{Weapon and Armor Proficiency:} Paladins are proficient with all simple 
and martial weapons, with all types of armor (heavy, medium, and light), and with 
shields (except tower shields).

\textbf{Aura of Good (Ex):} The power of a paladin's aura of good (see the \textit{detect 
good }spell) is equal to her paladin level.

\textit{\textbf{Detect Evil }}\textbf{(Sp):} At will, a paladin can use \textit{detect 
evil}, as the spell.

\textbf{Smite Evil (Su):} Once per day, a paladin may attempt to smite evil with 
one normal melee attack. She adds her Charisma bonus (if any) to her attack roll 
and deals 1 extra point of damage per paladin level. If the paladin accidentally 
smites a creature that is not evil, the smite has no effect, but the ability is 
still used up for that day.

At 5th level, and at every five levels thereafter, the paladin may smite evil one 
additional time per day, as indicated on Table: The Paladin, to a maximum of five 
times per day at 20th level.

\textbf{Divine Grace (Su):} At 2nd level, a paladin gains a bonus equal to her 
Charisma bonus (if any) on all saving throws.

\textbf{Lay on Hands (Su):} Beginning at 2nd level, a paladin with a Charisma score 
of 12 or higher can heal wounds (her own or those of others) by touch. Each day 
she can heal a total number of hit points of damage equal to her paladin level 
x her Charisma bonus. A paladin may choose to divide her healing among multiple 
recipients, and she doesn't have to use it all at once. Using lay on hands is a 
standard action.

Alternatively, a paladin can use any or all of this healing power to deal damage 
to undead creatures. Using lay on hands in this way requires a successful melee 
touch attack and doesn't provoke an attack of opportunity. The paladin decides 
how many of her daily allotment of points to use as damage after successfully touching 
an undead creature.

\textbf{Aura of Courage (Su):} Beginning at 3rd level, a paladin is immune to fear 
(magical or otherwise). Each ally within 10 feet of her gains a +4 morale bonus 
on saving throws against fear effects.

This ability functions while the paladin is conscious, but not if she is unconscious 
or dead.

\textbf{Divine Health (Ex):} At 3rd level, a paladin gains immunity to all diseases, 
including supernatural and magical diseases.

\textbf{Turn Undead (Su):}When a paladin reaches 4th level, she gains the supernatural 
ability to turn undead. She may use this ability a number of times per day equal 
to 3 + her Charisma modifier. She turns undead as a cleric of three levels lower 
would.

\textbf{Spells:} Beginning at 4th level, a paladin gains the ability to cast a 
small number of divine spells, which are drawn from the paladin spell list. A paladin 
must choose and prepare her spells in advance.

To prepare or cast a spell, a paladin must have a Wisdom score equal to at least 
10 + the spell level. The Difficulty Class for a saving throw against a paladin's 
spell is 10 + the spell level + the paladin's Wisdom modifier.

Like other spellcasters, a paladin can cast only a certain number of spells of 
each spell level per day. Her base daily spell allotment is given on Table: The 
Paladin. In addition, she receives bonus spells per day if she has a high Wisdom 
score. When Table: The Paladin indicates that the paladin gets 0 spells per day 
of a given spell level, she gains only the bonus spells she would be entitled to 
based on her Wisdom score for that spell level The paladin does not have access 
to any domain spells or granted powers, as a cleric does.

A paladin prepares and casts spells the way a cleric does, though she cannot lose 
a prepared spell to spontaneously cast a \textit{cure }spell in its place. A paladin 
may prepare and cast any spell on the paladin spell list, provided that she can 
cast spells of that level, but she must choose which spells to prepare during her 
daily meditation.

Through 3rd level, a paladin has no caster level. At 4th level and higher, her 
caster level is one-half her paladin level.

\textit{\textbf{Special Mount }}\textbf{(Sp):} Upon reaching 5th level, a paladin 
gains the service of an unusually intelligent, strong, and loyal steed to serve 
her in her crusade against evil (see below). This mount is usually a heavy warhorse 
(for a Medium paladin) or a warpony (for a Small paladin).

Once per day, as a full-round action, a paladin may magically call her mount from 
the celestial realms in which it resides. This ability is the equivalent of a spell 
of a level equal to one-third the paladin's level. The mount immediately appears 
adjacent to the paladin and remains for 2 hours per paladin level; it may be dismissed 
at any time as a free action. The mount is the same creature each time it is summoned, 
though the paladin may release a particular mount from service.

Each time the mount is called, it appears in full health, regardless of any damage 
it may have taken previously. The mount also appears wearing or carrying any gear 
it had when it was last dismissed. Calling a mount is a conjuration (calling) effect.

Should the paladin's mount die, it immediately disappears, leaving behind any equipment 
it was carrying. The paladin may not summon another mount for thirty days or until 
she gains a paladin level, whichever comes first, even if the mount is somehow 
returned from the dead. During this thirty-day period, the paladin takes a -1 penalty 
on attack and weapon damage rolls.

\textit{\textbf{Remove Disease }}\textbf{(Sp):} At 6th level, a paladin can produce 
a \textit{remove disease }effect, as the spell, once per week. She can use this 
ability one additional time per week for every three levels after 6th (twice per 
week at 9th, three times at 12th, and so forth).

\textbf{Code of Conduct:} A paladin must be of lawful good alignment and loses 
all class abilities if she ever willingly commits an evil act.

Additionally, a paladin's code requires that she respect legitimate authority, 
act with honor (not lying, not cheating, not using poison, and so forth), help 
those in need (provided they do not use the help for evil or chaotic ends), and 
punish those who harm or threaten innocents.

\textbf{Associates:} While she may adventure with characters of any good or neutral 
alignment, a paladin will never knowingly associate with evil characters, nor will 
she continue an association with someone who consistently offends her moral code. 
A paladin may accept only henchmen, followers, or cohorts who are lawful good.

\vspace{12pt}
\subsubsection*{\textbf{Ex-Paladins}}

A paladin who ceases to be lawful good, who willfully commits an evil act, or who 
grossly violates the code of conduct loses all paladin spells and abilities (including 
the service of the paladin's mount, but not weapon, armor, and shield proficiencies). 
She may not progress any farther in levels as a paladin. She regains her abilities 
and advancement potential if she atones for her violations (see the \textit{atonement 
}spell description), as appropriate.

Like a member of any other class, a paladin may be a multiclass character, but 
multiclass paladins face a special restriction. A paladin who gains a level in 
any class other than paladin may never again raise her paladin level, though she 
retains all her paladin abilities.

\vspace{12pt}
THE PALADIN'S MOUNT

The paladin's mount is superior to a normal mount of its kind and has special powers, 
as described below. The standard mount for a Medium paladin is a heavy warhorse, 
and the standard mount for a Small paladin is a warpony. Another kind of mount, 
such as a riding dog (for a halfling paladin) or a Large shark (for a paladin in 
an aquatic campaign) may be allowed as well.

A paladin's mount is treated as a magical beast, not an animal, for the purpose 
of all effects that depend on its type (though it retains an animal's HD, base 
attack bonus, saves, skill points, and feats).

\vspace{12pt}
\begin{tabular}{|>{\raggedright}p{40pt}|>{\raggedright}p{30pt}|>{\raggedright}p{56pt}|>{\raggedright}p{17pt}|>{\raggedright}p{13pt}|>{\raggedright}p{120pt}|}
\hline
P\textbf{aladin Level } & B\textbf{onus HD } & N\textbf{atural Armor Adj. } & S\textbf{tr 
Adj. } & I\textbf{nt } & S\textbf{pecial}\tabularnewline
\hline
5th-7th  & +2  & +4  & +1  & 6  & Empathic link, improved evasion, share spells, 
share saving throws\tabularnewline
\hline
8th-10th  & +4  & +6  & +2  & 7  & Improved speed\tabularnewline
\hline
11th-14th  & +6  & +8  & +3  & 8  & C\textit{ommand }creatures of its kind\tabularnewline
\hline
15th-20th  & +8  & +10  & +4  & 9  & Spell resistance\tabularnewline
\hline
\end{tabular}

\vspace{12pt}
\textbf{Paladin's Mount Basics: }Use the base statistics for a creature of the 
mount's kind,\textit{ }but make changes to take into account the attributes and 
characteristics summarized on the table and described below.

\textit{Bonus HD: }Extra eight-sided (d8) Hit Dice, each of which gains a Constitution 
modifier, as normal. Extra Hit Dice improve the mount's base attack and base save 
bonuses. A special mount's base attack bonus is equal to that of a cleric of a 
level equal to the mount's HD. A mount has good Fortitude and Reflex saves (treat 
it as a character whose level equals the animal's HD). The mount gains additional 
skill points or feats for bonus HD as normal for advancing a monster's Hit Dice.

\textit{Natural Armor Adj.: }The number on the table is an improvement to the mount's 
existing natural armor bonus.

\textit{Str Adj.: }Add this figure to the mount's Strength score.

\textit{Int: }The mount's Intelligence score.

\textit{Empathic Link (Su): }The paladin has an empathic link with her mount out 
to a distance of up to 1 mile. The paladin cannot see through the mount's eyes, 
but they can communicate empathically.

Note that even intelligent mounts see the world differently from humans, so misunderstandings 
are always possible.

Because of this empathic link, the paladin has the same connection to an item or 
place that her mount does, just as with a master and his familiar (see Familiars).

\textit{Improved Evasion (Ex): }When subjected to an attack that normally allows 
a Reflex saving throw for half damage, a mount takes no damage if it makes a successful 
saving throw and half damage if the saving throw fails.

\textit{Share Spells: }At the paladin's option, she may have any spell (but not 
any spell-like ability) she casts on herself also affect her mount. 

The mount must be within 5 feet at the time of casting to receive the benefit. 
If the spell or effect has a duration other than instantaneous, it stops affecting 
the mount if it moves farther than 5 feet away and will not affect the mount again 
even if it returns to the paladin before the duration expires. Additionally, the 
paladin may cast a spell with a target of ``You'' on her mount (as a touch range 
spell) instead of on herself. A paladin and her mount can share spells even if 
the spells normally do not affect creatures of the mount's type (magical beast).

\textit{Share Saving Throws: }For each of its saving throws, the mount uses its 
own base save bonus or the paladin's, whichever is higher. The mount applies its 
own ability modifiers to saves, and it doesn't share any other bonuses on saves 
that the master might have.

\textit{Improved Speed (Ex): }The mount's speed increases by 10 feet.

\textit{Command (Sp): }Once per day per two paladin levels of its master, a mount 
can use this ability to command other any normal animal of approximately the same 
kind as itself (for warhorses and warponies, this category includes donkeys, mules, 
and ponies), as long as the target creature has fewer Hit Dice than the mount. 
This ability functions like the \textit{command }spell, but the mount must make 
a DC 21 Concentration check to succeed if it's being ridden at the time. If the 
check fails, the ability does not work that time, but it still counts against the 
mount's daily uses. Each target may attempt a Will save (DC 10 + 1/2 paladin's 
level + paladin's Cha modifier) to negate the effect.

\textit{Spell Resistance (Ex): }A mount's spell resistance equals its master's 
paladin level + 5. To affect the mount with a spell, a spellcaster must get a result 
on a caster level check (1d20 + caster level) that equals or exceeds the mount's 
spell resistance.

\vspace{12pt}
{\LARGE{}RANGER}

\textbf{Alignment:} Any.

\textbf{Hit Die:} d8.

\vspace{12pt}
\subsubsection*{\textbf{Class Skills}}

The ranger's class skills (and the key ability for each skill) are Climb (Str), 
Concentration (Con), Craft (Int), Handle Animal (Cha), Heal (Wis), Hide (Dex), 
Jump (Str), Knowledge (dungeoneering) (Int), Knowledge (geography) (Int), Knowledge 
(nature) (Int), Listen (Wis), Move Silently (Dex), Profession (Wis), Ride (Dex), 
Search (Int), Spot (Wis), Survival (Wis), Swim (Str), and Use Rope (Dex).

\textbf{Skill Points at 1st Level:} (6 + Int modifier) x$ $4.

\textbf{Skill Points at Each Additional Level:} 6 + Int modifier.

\vspace{12pt}
\begin{tabular}{|>{\raggedright}p{19pt}|>{\raggedright}p{48pt}|>{\raggedright}p{14pt}|>{\raggedright}p{15pt}|>{\raggedright}p{13pt}|>{\raggedright}p{67pt}|>{\raggedright}p{10pt}|>{\raggedright}p{14pt}|>{\raggedright}p{13pt}|>{\raggedright}p{12pt}|}
\hline
\multicolumn{10}{|p{230pt}|}{T\textbf{able: The Ranger}}\tabularnewline
\hline
 &  &  &  &  & --- & \multicolumn{4}{p{50pt}|}{S\textbf{pells per Day---}}\tabularnewline
\hline
L\textbf{evel} & B\textbf{ase Attack Bonus} & F\textbf{ort Save} & R\textbf{ef 
Save} & W\textbf{ill Save} & S\textbf{pecial} & 1\textbf{st} & 2\textbf{nd} & 3\textbf{rd} & 4\textbf{th}\tabularnewline
\hline
1st & +1 & +2 & +2 & +0 & 1st favored enemy, Track, wild empathy--- & --- & --- & --- & \tabularnewline
\hline
2nd & +2 & +3 & +3 & +0 & Combat style--- & --- & --- & --- & \tabularnewline
\hline
3rd & +3 & +3 & +3 & +1 & Endurance--- & --- & --- & --- & \tabularnewline
\hline
4th & +4 & +4 & +4 & +1 & Animal companion & 0--- & --- & --- & \tabularnewline
\hline
5th & +5 & +4 & +4 & +1 & 2nd favored enemy & 0--- & --- & --- & \tabularnewline
\hline
6th & +6/+1 & +5 & +5 & +2 & Improved combat style & 1--- & --- & --- & \tabularnewline
\hline
7th & +7/+2 & +5 & +5 & +2 & Woodland stride & 1--- & --- & --- & \tabularnewline
\hline
8th & +8/+3 & +6 & +6 & +2 & Swift tracker & 1 & 0--- & --- & \tabularnewline
\hline
9th & +9/+4 & +6 & +6 & +3 & Evasion & 1 & 0--- & --- & \tabularnewline
\hline
10th & +10/+5 & +7 & +7 & +3 & 3rd favored enemy & 1 & 1--- & --- & \tabularnewline
\hline
11th & +11/+6/+1 & +7 & +7 & +3 & Combat style mastery & 1 & 1 & 0--- & \tabularnewline
\hline
12th & +12/+7/+2 & +8 & +8 & +4 &  & 1 & 1 & 1--- & \tabularnewline
\hline
13th & +13/+8/+3 & +8 & +8 & +4 & Camouflage & 1 & 1 & 1--- & \tabularnewline
\hline
14th & +14/+9/+4 & +9 & +9 & +4 &  & 2 & 1 & 1 & 0\tabularnewline
\hline
15th & +15/+10/+5 & +9 & +9 & +5 & 4th favored enemy & 2 & 1 & 1 & 1\tabularnewline
\hline
16th & +16/+11/+6/+1 & +10 & +10 & +5 &  & 2 & 2 & 1 & 1\tabularnewline
\hline
17th & +17/+12/+7/+2 & +10 & +10 & +5 & Hide in plain sight & 2 & 2 & 2 & 1\tabularnewline
\hline
18th & +18/+13/+8/+3 & +11 & +11 & +6 &  & 3 & 2 & 2 & 1\tabularnewline
\hline
19th & +19/+14/+9/+4 & +11 & +11 & +6 &  & 3 & 3 & 3 & 2\tabularnewline
\hline
20th & +20/+15/+10/+5 & +12 & +12 & +6 & 5th favored enemy & 3 & 3 & 3 & 3\tabularnewline
\hline
\end{tabular}

\vspace{12pt}
\subsubsection*{\textbf{Class Features}}

All of the following are class features of the ranger.

\textbf{Weapon and Armor Proficiency:} A ranger is proficient with all simple and 
martial weapons, and with light armor and shields (except tower shields).

\textbf{Favored Enemy (Ex):} At 1st level, a ranger may select a type of creature 
from among those given on Table: Ranger Favored Enemies. The ranger gains a +2 
bonus on Bluff, Listen, Sense Motive, Spot, and Survival checks when using these 
skills against creatures of this type. Likewise, he gets a +2 bonus on weapon damage 
rolls against such creatures.

At 5th level and every five levels thereafter (10th, 15th, and 20th level), the 
ranger may select an additional favored enemy from those given on the table. In 
addition, at each such interval, the bonus against any one favored enemy (including 
the one just selected, if so desired) increases by 2. 

If the ranger chooses humanoids or outsiders as a favored enemy, he must also choose 
an associated subtype, as indicated on the table. If a specific creature falls 
into more than one category of favored enemy, the ranger's bonuses do not stack; 
he simply uses whichever bonus is higher.

\vspace{12pt}
\begin{tabular}{|>{\raggedright}p{88pt}|>{\raggedright}p{84pt}|}
\hline
\multicolumn{2}{|p{172pt}|}{T\textbf{able: Ranger Favored Enemies}}\tabularnewline
\hline
T\textbf{ype (Subtype)} & T\textbf{ype (Subtype)}\tabularnewline
\hline
Aberration & Humanoid (reptilian) \tabularnewline
\hline
Animal & Magical beast \tabularnewline
\hline
Construct & Monstrous humanoid \tabularnewline
\hline
Dragon  & Ooze \tabularnewline
\hline
Elemental & Outsider (air) \tabularnewline
\hline
Fey & Outsider (chaotic) \tabularnewline
\hline
Giant & Outsider (earth) \tabularnewline
\hline
Humanoid (aquatic) & Outsider (evil) \tabularnewline
\hline
Humanoid (dwarf)  & Outsider (fire) \tabularnewline
\hline
Humanoid (elf)  & Outsider (good)\tabularnewline
\hline
Humanoid (goblinoid)  & Outsider (lawful)\tabularnewline
\hline
Humanoid (gnoll)  & Outsider (native)\tabularnewline
\hline
Humanoid (gnome) & Outsider (water) \tabularnewline
\hline
Humanoid (halfling)  & Plant \tabularnewline
\hline
Humanoid (human)  & Undead \tabularnewline
\hline
Humanoid (orc)  & Vermin \tabularnewline
\hline
\end{tabular}

\vspace{12pt}
\textbf{Track:} A ranger gains Track as a bonus feat.

\textbf{Wild Empathy (Ex): }A ranger can improve the attitude of an animal. This 
ability functions just like a Diplomacy check to improve the attitude of a person. 
The ranger rolls 1d20 and adds his ranger level and his Charisma bonus to determine 
the wild empathy check result. The typical domestic animal has a starting attitude 
of indifferent, while wild animals are usually unfriendly.

To use wild empathy, the ranger and the animal must be able to study each other, 
which means that they must be within 30 feet of one another under normal visibility 
conditions. Generally, influencing an animal in this way takes 1 minute, but, as 
with influencing people, it might take more or less time.

The ranger can also use this ability to influence a magical beast with an Intelligence 
score of 1 or 2, but he takes a -4 penalty on the check.

\textbf{Combat Style (Ex):} At 2nd level, a ranger must select one of two combat 
styles to pursue: archery or two-weapon combat. This choice affects the character's 
class features but does not restrict his selection of feats or special abilities 
in any way.

If the ranger selects archery, he is treated as having the Rapid Shot feat, even 
if he does not have the normal prerequisites for that feat.

If the ranger selects two-weapon combat, he is treated as having the Two-Weapon 
Fighting feat, even if he does not have the normal prerequisites for that feat.

The benefits of the ranger's chosen style apply only when he wears light or no 
armor. He loses all benefits of his combat style when wearing medium or heavy armor.

\textbf{Endurance:} A ranger gains Endurance as a bonus feat at 3rd level.

\textbf{Animal Companion (Ex):} At 4th level, a ranger gains an animal companion 
selected from the following list: badger, camel, dire rat, dog, riding dog, eagle, 
hawk, horse (light or heavy), owl, pony, snake (Small or Medium viper), or wolf. 
If the campaign takes place wholly or partly in an aquatic environment, the following 
creatures may be added to the ranger's list of options: crocodile, porpoise, Medium 
shark, and squid. This animal is a loyal companion that accompanies the ranger 
on his adventures as appropriate for its kind.

This ability functions like the druid ability of the same name, except that the 
ranger's effective druid level is one-half his ranger level. A ranger may select 
from the alternative lists of animal companions just as a druid can, though again 
his effective druid level is half his ranger level. Like a druid, a ranger cannot 
select an alternative animal if the choice would reduce his effective druid level 
below 1st.

\textbf{Spells:} Beginning at 4th level, a ranger gains the ability to cast a small 
number of divine spells, which are drawn from the ranger spell list. A ranger must 
choose and prepare his spells in advance (see below).

To prepare or cast a spell, a ranger must have a Wisdom score equal to at least 
10 + the spell level. The Difficulty Class for a saving throw against a ranger's 
spell is 10 + the spell level + the ranger's Wisdom modifier.

Like other spellcasters, a ranger can cast only a certain number of spells of each 
spell level per day. His base daily spell allotment is given on Table: The Ranger. 
In addition, he receives bonus spells per day if he has a high Wisdom score. When 
Table: The Ranger indicates that the ranger gets 0 spells per day of a given spell 
level, he gains only the bonus spells he would be entitled to based on his Wisdom 
score for that spell level. The ranger does not have access to any domain spells 
or granted powers, as a cleric does.

A ranger prepares and casts spells the way a cleric does, though he cannot lose 
a prepared spell to cast a \textit{cure }spell in its place. A ranger may prepare 
and cast any spell on the ranger spell list, provided that he can cast spells of 
that level, but he must choose which spells to prepare during his daily meditation.

Through 3rd level, a ranger has no caster level. At 4th level and higher, his caster 
level is one-half his ranger level.

\textbf{Improved Combat Style (Ex):} At 6th level, a ranger's aptitude in his chosen 
combat style (archery or two-weapon combat) improves. If he selected archery at 
2nd level, he is treated as having the Manyshot feat, even if he does not have 
the normal prerequisites for that feat.

If the ranger selected two-weapon combat at 2nd level, he is treated as having 
the Improved Two-Weapon Fighting feat, even if he does not have the normal prerequisites 
for that feat.

As before, the benefits of the ranger's chosen style apply only when he wears light 
or no armor. He loses all benefits of his combat style when wearing medium or heavy 
armor.

\textbf{Woodland Stride (Ex):} Starting at 7th level, a ranger may move through 
any sort of undergrowth (such as natural thorns, briars, overgrown areas, and similar 
terrain) at his normal speed and without taking damage or suffering any other impairment.

However, thorns, briars, and overgrown areas that are enchanted or magically manipulated 
to impede motion still affect him.

\textbf{Swift Tracker (Ex):} Beginning at 8th level, a ranger can move at his normal 
speed while following tracks without taking the normal -5 penalty. He takes only 
a -10 penalty (instead of the normal -20) when moving at up to twice normal speed 
while tracking.

\textbf{Evasion (Ex):} At 9th level, a ranger can avoid even magical and unusual 
attacks with great agility. If he makes a successful Reflex saving throw against 
an attack that normally deals half damage on a successful save, he instead takes 
no damage. Evasion can be used only if the ranger is wearing light armor or no 
armor. A helpless ranger does not gain the benefit of evasion.

\textbf{Combat Style Mastery (Ex):} At 11th level, a ranger's aptitude in his chosen 
combat style (archery or two-weapon combat) improves again. If he selected archery 
at 2nd level, he is treated as having the Improved Precise Shot feat, even if he 
does not have the normal prerequisites for that feat.

If the ranger selected two-weapon combat at 2nd level, he is treated as having 
the Greater Two-Weapon Fighting feat, even if he does not have the normal prerequisites 
for that feat.

As before, the benefits of the ranger's chosen style apply only when he wears light 
or no armor. He loses all benefits of his combat style when wearing medium or heavy 
armor.

\textbf{Camouflage (Ex):} A ranger of 13th level or higher can use the Hide skill 
in any sort of natural terrain, even if the terrain doesn't grant cover or concealment.

\textbf{Hide in Plain Sight (Ex):} While in any sort of natural terrain, a ranger 
of 17th level or higher can use the Hide skill even while being observed.

\vspace{12pt}
{\LARGE{}ROGUE}

\textbf{Alignment:} Any.

\textbf{Hit Die:} d6.

\vspace{12pt}
\subsubsection*{\textbf{Class Skills}}

The rogue's class skills (and the key ability for each skill) are Appraise (Int), 
Balance (Dex), Bluff (Cha), Climb (Str), Craft (Int), Decipher Script (Int), Diplomacy 
(Cha), Disable Device (Int), Disguise (Cha), Escape Artist (Dex), Forgery (Int), 
Gather Information (Cha), Hide (Dex), Intimidate (Cha), Jump (Str), Knowledge (local) 
(Int), Listen (Wis), Move Silently (Dex), Open Lock (Dex), Perform (Cha), Profession 
(Wis), Search (Int), Sense Motive (Wis), Sleight of Hand (Dex), Spot (Wis), Swim 
(Str), Tumble (Dex), Use Magic Device (Cha), and Use Rope (Dex).

\textbf{Skill Points at 1st Level:} (8 + Int modifier) x$ $4.

\textbf{Skill Points at Each Additional Level:} 8 + Int modifier.

\vspace{12pt}
\begin{tabular}{|>{\raggedright}p{25pt}|>{\raggedright}p{49pt}|>{\raggedright}p{23pt}|>{\raggedright}p{22pt}|>{\raggedright}p{23pt}|>{\raggedright}p{133pt}|}
\hline
\multicolumn{6}{|p{278pt}|}{T\textbf{able: The Rogue}}\tabularnewline
\hline
L\textbf{evel} & B\textbf{ase Attack Bonus} & F\textbf{ort }\linebreak{}
\textbf{Save} & R\textbf{ef }\linebreak{}
\textbf{Save} & W\textbf{ill }\linebreak{}
\textbf{Save} & S\textbf{pecial}\tabularnewline
\hline
1st & +0 & +0 & +2 & +0 & Sneak attack +1d6, trapfinding\tabularnewline
\hline
2nd & +1 & +0 & +3 & +0 & Evasion\tabularnewline
\hline
3rd & +2 & +1 & +3 & +1 & Sneak attack +2d6, trap sense +1\tabularnewline
\hline
4th & +3 & +1 & +4 & +1 & Uncanny dodge\tabularnewline
\hline
5th & +3 & +1 & +4 & +1 & Sneak attack +3d6\tabularnewline
\hline
6th & +4 & +2 & +5 & +2 & Trap sense +2\tabularnewline
\hline
7th & +5 & +2 & +5 & +2 & Sneak attack +4d6\tabularnewline
\hline
8th & +6/+1 & +2 & +6 & +2 & Improved uncanny dodge\tabularnewline
\hline
9th & +6/+1 & +3 & +6 & +3 & Sneak attack +5d6, trap sense +3\tabularnewline
\hline
10th & +7/+2 & +3 & +7 & +3 & Special ability\tabularnewline
\hline
11th & +8/+3 & +3 & +7 & +3 & Sneak attack +6d6\tabularnewline
\hline
12th & +9/+4 & +4 & +8 & +4 & Trap sense +4\tabularnewline
\hline
13th & +9/+4 & +4 & +8 & +4 & Sneak attack +7d6, special ability\tabularnewline
\hline
14th & +10/+5 & +4 & +9 & +4--- & \tabularnewline
\hline
15th & +11/+6/+1 & +5 & +9 & +5 & Sneak attack +8d6, trap sense +5\tabularnewline
\hline
16th & +12/+7/+2 & +5 & +10 & +5 & Special ability\tabularnewline
\hline
17th & +12/+7/+2 & +5 & +10 & +5 & Sneak attack +9d6\tabularnewline
\hline
18th & +13/+8/+3 & +6 & +11 & +6 & Trap sense +6\tabularnewline
\hline
19th & +14/+9/+4 & +6 & +11 & +6 & Sneak attack +10d6, special ability\tabularnewline
\hline
20th & +15/+10/+5 & +6 & +12 & +6--- & \tabularnewline
\hline
\end{tabular}

\vspace{12pt}
\subsubsection*{\textbf{Class Features}}

All of the following are class features of the rogue.

\textbf{Weapon and Armor Proficiency: }Rogues are proficient with all simple weapons, 
plus the hand crossbow, rapier, sap, shortbow, and short sword. Rogues are proficient 
with light armor, but not with shields.

\textbf{Sneak Attack:} If a rogue can catch an opponent when he is unable to defend 
himself effectively from her attack, she can strike a vital spot for extra damage.

The rogue's attack deals extra damage any time her target would be denied a Dexterity 
bonus to AC (whether the target actually has a Dexterity bonus or not), or when 
the rogue flanks her target. This extra damage is 1d6 at 1st level, and it increases 
by 1d6 every two rogue levels thereafter. Should the rogue score a critical hit 
with a sneak attack, this extra damage is not multiplied.

Ranged attacks can count as sneak attacks only if the target is within 30 feet.

With a sap (blackjack) or an unarmed strike, a rogue can make a sneak attack that 
deals nonlethal damage instead of lethal damage. She cannot use a weapon that deals 
lethal damage to deal nonlethal damage in a sneak attack, not even with the usual 
-4 penalty.

A rogue can sneak attack only living creatures with discernible anatomies---undead, 
constructs, oozes, plants, and incorporeal creatures lack vital areas to attack. 
Any creature that is immune to critical hits is not vulnerable to sneak attacks. 
The rogue must be able to see the target well enough to pick out a vital spot and 
must be able to reach such a spot. A rogue cannot sneak attack while striking a 
creature with concealment or striking the limbs of a creature whose vitals are 
beyond reach.

\textbf{Trapfinding:} Rogues (and only rogues) can use the Search skill to locate 
traps when the task has a Difficulty Class higher than 20. 

Finding a nonmagical trap has a DC of at least 20, or higher if it is well hidden. 
Finding a magic trap has a DC of 25 + the level of the spell used to create it.

Rogues (and only rogues) can use the Disable Device skill to disarm magic traps. 
A magic trap generally has a DC of 25 + the level of the spell used to create it.

A rogue who beats a trap's DC by 10 or more with a Disable Device check can study 
a trap, figure out how it works, and bypass it (with her party) without disarming 
it.

\textbf{Evasion (Ex): }At 2nd level and higher, a rogue can avoid even magical 
and unusual attacks with great agility. If she makes a successful Reflex saving 
throw against an attack that normally deals half damage on a successful save, she 
instead takes no damage. Evasion can be used only if the rogue is wearing light 
armor or no armor. A helpless rogue does not gain the benefit of evasion.

\textbf{Trap Sense (Ex):} At 3rd level, a rogue gains an intuitive sense that alerts 
her to danger from traps, giving her a +1 bonus on Reflex saves made to avoid traps 
and a +1 dodge bonus to AC against attacks made by traps. These bonuses rise to 
+2 when the rogue reaches 6th level, to +3 when she reaches 9th level, to +4 when 
she reaches 12th level, to +5 at 15th, and to +6 at 18th level.

Trap sense bonuses gained from multiple classes stack.

\textbf{Uncanny Dodge (Ex):} Starting at 4th level, a rogue can react to danger 
before her senses would normally allow her to do so. She retains her Dexterity 
bonus to AC (if any) even if she is caught flat-footed or struck by an invisible 
attacker. However, she still loses her Dexterity bonus to AC if immobilized.

If a rogue already has uncanny dodge from a different class she automatically gains 
improved uncanny dodge (see below) instead.

\textbf{Improved Uncanny Dodge (Ex):} A rogue of 8th level or higher can no longer 
be flanked.

This defense denies another rogue the ability to sneak attack the character by 
flanking her, unless the attacker has at least four more rogue levels than the 
target does.

If a character already has uncanny dodge (see above) from a second class, the character 
automatically gains improved uncanny dodge instead, and the levels from the classes 
that grant uncanny dodge stack to determine the minimum rogue level required to 
flank the character.

\textbf{Special Abilities:} On attaining 10th level, and at every three levels 
thereafter (13th, 16th, and 19th), a rogue gains a special ability of her choice 
from among the following options.

\textit{Crippling Strike (Ex): }A rogue with this ability can sneak attack opponents 
with such precision that her blows weaken and hamper them. An opponent damaged 
by one of her sneak attacks also takes 2 points of Strength damage. Ability points 
lost to damage return on their own at the rate of 1 point per day for each damaged 
ability.

\textit{Defensive Roll (Ex): }The rogue can roll with a potentially lethal blow 
to take less damage from it than she otherwise would. Once per day, when she would 
be reduced to 0 or fewer hit points by damage in combat (from a weapon or other 
blow, not a spell or special ability), the rogue can attempt to roll with the damage. 
To use this ability, the rogue must attempt a Reflex saving throw (DC = damage 
dealt). If the save succeeds, she takes only half damage from the blow; if it fails, 
she takes full damage. She must be aware of the attack and able to react to it 
in order to execute her defensive roll---if she is denied her Dexterity bonus to 
AC, she can't use this ability. Since this effect would not normally allow a character 
to make a Reflex save for half damage, the rogue's evasion ability does not apply 
to the defensive roll.

\textit{Improved Evasion (Ex): }This ability works like evasion, except that while 
the rogue still takes no damage on a successful Reflex saving throw against attacks 
henceforth she henceforth takes only half damage on a failed save. A helpless rogue 
does not gain the benefit of improved evasion.

\textit{Opportunist (Ex): }Once per round, the rogue can make an attack of opportunity 
against an opponent who has just been struck for damage in melee by another character. 
This attack counts as the rogue's attack of opportunity for that round. Even a 
rogue with the Combat Reflexes feat can't use the opportunist ability more than 
once per round.

\textit{Skill Mastery: }The rogue becomes so certain in the use of certain skills 
that she can use them reliably even under adverse conditions.

Upon gaining this ability, she selects a number of skills equal to 3 + her Intelligence 
modifier. When making a skill check with one of these skills, she may take 10 even 
if stress and distractions would normally prevent her from doing so. A rogue may 
gain this special ability multiple times, selecting additional skills for it to 
apply to each time.

\textit{Slippery Mind (Ex): }This ability represents the rogue's ability to wriggle 
free from magical effects that would otherwise control or compel her. If a rogue 
with slippery mind is affected by an enchantment spell or effect and fails her 
saving throw, she can attempt it again 1 round later at the same DC. She gets only 
this one extra chance to succeed on

her saving throw.

\textit{Feat: }A rogue may gain a bonus feat in place of a special ability.

\vspace{12pt}
{\LARGE{}SORCERER}

\textbf{Alignment:} Any.

\textbf{Hit Die:} d4.

\vspace{12pt}
\subsubsection*{\textbf{Class Skills}}

The sorcerer's class skills (and the key ability for each skill) are Bluff (Cha), 
Concentration (Con), Craft (Int), Knowledge (arcana) (Int), Profession (Wis), and 
Spellcraft (Int).

\textbf{Skill Points at 1st Level:} (2 + Int modifier) x 4.

\textbf{Skill Points at Each Additional Level: }2 + Int modifier.

\vspace{12pt}
\begin{tabular}{|>{\raggedright}p{15pt}|>{\raggedright}p{26pt}|>{\raggedright}p{11pt}|>{\raggedright}p{10pt}|>{\raggedright}p{11pt}|>{\raggedright}p{18pt}|>{\raggedright}p{3pt}|>{\raggedright}p{7pt}|>{\raggedright}p{6pt}|>{\raggedright}p{6pt}|>{\raggedright}p{6pt}|>{\raggedright}p{6pt}|>{\raggedright}p{6pt}|>{\raggedright}p{6pt}|>{\raggedright}p{6pt}|>{\raggedright}p{6pt}|}
\hline
\multicolumn{16}{|p{158pt}|}{T\textbf{able: The Sorcerer}}\tabularnewline
\hline
 &  &  &  &  & ----------------------------- & \multicolumn{10}{p{64pt}|}{S\textbf{pells 
per Day---------------------------}}\tabularnewline
\hline
L\textbf{evel} & B\textbf{ase Attack }\linebreak{}
\textbf{Bonus} & F\textbf{ort }\linebreak{}
\textbf{Save} & R\textbf{ef }\linebreak{}
\textbf{Save} & W\textbf{ill }\linebreak{}
\textbf{Save} & S\textbf{pecial} & 0 & 1\textbf{st} & 2\textbf{nd} & 3\textbf{rd} & 4\textbf{th} & 5\textbf{th} & 6\textbf{th} & 7\textbf{th} & 8\textbf{th} & 9\textbf{th}\tabularnewline
\hline
1st & +0 & +0 & +0 & +2 & Summon \linebreak{}
familiar & 5 & 3--- & --- & --- & --- & --- & --- & --- & --- & \tabularnewline
\hline
2nd & +1 & +0 & +0 & +3 &  & 6 & 4--- & --- & --- & --- & --- & --- & --- & --- & \tabularnewline
\hline
3rd & +1 & +1 & +1 & +3 &  & 6 & 5--- & --- & --- & --- & --- & --- & --- & --- & \tabularnewline
\hline
4th & +2 & +1 & +1 & +4 &  & 6 & 6 & 3--- & --- & --- & --- & --- & --- & --- & \tabularnewline
\hline
5th & +2 & +1 & +1 & +4 &  & 6 & 6 & 4--- & --- & --- & --- & --- & --- & --- & \tabularnewline
\hline
6th & +3 & +2 & +2 & +5 &  & 6 & 6 & 5 & 3--- & --- & --- & --- & --- & --- & \tabularnewline
\hline
7th & +3 & +2 & +2 & +5 &  & 6 & 6 & 6 & 4--- & --- & --- & --- & --- & --- & \tabularnewline
\hline
8th & +4 & +2 & +2 & +6 &  & 6 & 6 & 6 & 5 & 3--- & --- & --- & --- & --- & \tabularnewline
\hline
9th & +4 & +3 & +3 & +6 &  & 6 & 6 & 6 & 6 & 4--- & --- & --- & --- & --- & \tabularnewline
\hline
10th & +5 & +3 & +3 & +7 &  & 6 & 6 & 6 & 6 & 5 & 3--- & --- & --- & --- & \tabularnewline
\hline
11th & +5 & +3 & +3 & +7 &  & 6 & 6 & 6 & 6 & 6 & 4--- & --- & --- & --- & \tabularnewline
\hline
12th & +6/+1 & +4 & +4 & +8 &  & 6 & 6 & 6 & 6 & 6 & 5 & 3--- & --- & --- & \tabularnewline
\hline
13th & +6/+1 & +4 & +4 & +8 &  & 6 & 6 & 6 & 6 & 6 & 6 & 4--- & --- & --- & \tabularnewline
\hline
14th & +7/+2 & +4 & +4 & +9 &  & 6 & 6 & 6 & 6 & 6 & 6 & 5 & 3--- & --- & \tabularnewline
\hline
15th & +7/+2 & +5 & +5 & +9 &  & 6 & 6 & 6 & 6 & 6 & 6 & 6 & 4--- & --- & \tabularnewline
\hline
16th & +8/+3 & +5 & +5 & +10 &  & 6 & 6 & 6 & 6 & 6 & 6 & 6 & 5 & 3--- & \tabularnewline
\hline
17th & +8/+3 & +5 & +5 & +10 &  & 6 & 6 & 6 & 6 & 6 & 6 & 6 & 6 & 4--- & \tabularnewline
\hline
18th & +9/+4 & +6 & +6 & +11 &  & 6 & 6 & 6 & 6 & 6 & 6 & 6 & 6 & 5 & 3\tabularnewline
\hline
19th & +9/+4 & +6 & +6 & +11 &  & 6 & 6 & 6 & 6 & 6 & 6 & 6 & 6 & 6 & 4\tabularnewline
\hline
20th & +10/+5 & +6 & +6 & +12 &  & 6 & 6 & 6 & 6 & 6 & 6 & 6 & 6 & 6 & 6\tabularnewline
\hline
\end{tabular}

\vspace{12pt}
\begin{tabular}{|>{\raggedright}p{21pt}|>{\raggedright}p{19pt}|>{\raggedright}p{19pt}|>{\raggedright}p{19pt}|>{\raggedright}p{19pt}|>{\raggedright}p{19pt}|>{\raggedright}p{19pt}|>{\raggedright}p{19pt}|>{\raggedright}p{19pt}|>{\raggedright}p{19pt}|>{\raggedright}p{19pt}|}
\hline
\multicolumn{11}{|p{218pt}|}{T\textbf{able: Sorcerer Spells Known}}\tabularnewline
\hline
------------------------ & \multicolumn{10}{p{197pt}|}{ \textbf{Spells Known -------------------------}}\tabularnewline
\hline
L\textbf{evel} & 0 & 1\textbf{st} & 2\textbf{nd} & 3\textbf{rd} & 4\textbf{th} & 5\textbf{th} & 6\textbf{th} & 7\textbf{th} & 8\textbf{th} & 9\textbf{th}\tabularnewline
\hline
1st & 4 & 2--- & --- & --- & --- & --- & --- & --- & --- & \tabularnewline
\hline
2nd & 5 & 2--- & --- & --- & --- & --- & --- & --- & --- & \tabularnewline
\hline
3rd & 5 & 3--- & --- & --- & --- & --- & --- & --- & --- & \tabularnewline
\hline
4th & 6 & 3 & 1--- & --- & --- & --- & --- & --- & --- & \tabularnewline
\hline
5th & 6 & 4 & 2--- & --- & --- & --- & --- & --- & --- & \tabularnewline
\hline
6th & 7 & 4 & 2 & 1--- & --- & --- & --- & --- & --- & \tabularnewline
\hline
7th & 7 & 5 & 3 & 2--- & --- & --- & --- & --- & --- & \tabularnewline
\hline
8th & 8 & 5 & 3 & 2 & 1--- & --- & --- & --- & --- & \tabularnewline
\hline
9th & 8 & 5 & 4 & 3 & 2--- & --- & --- & --- & --- & \tabularnewline
\hline
10th & 9 & 5 & 4 & 3 & 2 & 1--- & --- & --- & --- & \tabularnewline
\hline
11th & 9 & 5 & 5 & 4 & 3 & 2--- & --- & --- & --- & \tabularnewline
\hline
12th & 9 & 5 & 5 & 4 & 3 & 2 & 1--- & --- & --- & \tabularnewline
\hline
13th & 9 & 5 & 5 & 4 & 4 & 3 & 2--- & --- & --- & \tabularnewline
\hline
14th & 9 & 5 & 5 & 4 & 4 & 3 & 2 & 1--- & --- & \tabularnewline
\hline
15th & 9 & 5 & 5 & 4 & 4 & 4 & 3 & 2--- & --- & \tabularnewline
\hline
16th & 9 & 5 & 5 & 4 & 4 & 4 & 3 & 2 & 1--- & \tabularnewline
\hline
17th & 9 & 5 & 5 & 4 & 4 & 4 & 3 & 3 & 2--- & \tabularnewline
\hline
18th & 9 & 5 & 5 & 4 & 4 & 4 & 3 & 3 & 2 & 1\tabularnewline
\hline
19th & 9 & 5 & 5 & 4 & 4 & 4 & 3 & 3 & 3 & 2\tabularnewline
\hline
20th & 9 & 5 & 5 & 4 & 4 & 4 & 3 & 3 & 3 & 3\tabularnewline
\hline
\end{tabular}

\vspace{12pt}
\subsubsection*{\textbf{Class Features}}

All of the following are class features of the sorcerer.

\textbf{Weapon and Armor Proficiency: }Sorcerers are proficient with all simple 
weapons. They are not proficient with any type of armor or shield. Armor of any 
type interferes with a sorcerer's gestures, which can cause his spells with somatic 
components to fail.

\textbf{Spells:} A sorcerer casts arcane spells which are drawn primarily from 
the sorcerer/wizard spell list. He can cast any spell he knows without preparing 
it ahead of time, the way a wizard or a cleric must (see below).

To learn or cast a spell, a sorcerer must have a Charisma score equal to at least 
10 + the spell level. The Difficulty Class for a saving throw against a sorcerer's 
spell is 10 + the spell level + the sorcerer's Charisma modifier.

Like other spellcasters, a sorcerer can cast only a certain number of spells of 
each spell level per day. His base daily spell allotment is given on Table: The 
Sorcerer. In addition, he receives bonus spells per day if he has a high Charisma 
score.

A sorcerer's selection of spells is extremely limited. A sorcerer begins play knowing 
four 0-level spells and two 1st-level spells of your choice. At each new sorcerer 
level, he gains one or more new spells, as indicated on Table: Sorcerer Spells 
Known. (Unlike spells per day, the number of spells a sorcerer knows is not affected 
by his Charisma score; the numbers on Table: Sorcerer Spells Known are fixed.) 
These new spells can be common spells chosen from the sorcerer/wizard spell list, 
or they can be unusual spells that the sorcerer has gained some understanding of 
by study. The sorcerer can't use this method of spell acquisition to learn spells 
at a faster rate, however.

Upon reaching 4th level, and at every even-numbered sorcerer level after that (6th, 
8th, and so on), a sorcerer can choose to learn a new spell in place of one he 
already knows. In effect, the sorcerer ``loses'' the old spell in exchange for 
the new one. The new spell's level must be the same as that of the spell being 
exchanged, and it must be at least two levels lower than the highest-level sorcerer 
spell the sorcerer can cast. A sorcerer may swap only a single spell at any given 
level, and must choose whether or not to swap the spell at the same time that he 
gains new spells known for the level.

Unlike a wizard or a cleric, a sorcerer need not prepare his spells in advance. 
He can cast any spell he knows at any time, assuming he has not yet used up his 
spells per day for that spell level. He does not have to decide ahead of time which 
spells he'll cast.

\textbf{Familiar:} A sorcerer can obtain a familiar (see below). Doing so takes 
24 hours and uses up magical materials that cost 100 gp. A familiar is a magical 
beast that resembles a small animal and is unusually tough and intelligent. The 
creature serves as a companion and servant.

The sorcerer chooses the kind of familiar he gets. As the sorcerer advances in 
level, his familiar also increases in power.

If the familiar dies or is dismissed by the sorcerer, the sorcerer must attempt 
a DC 15 Fortitude saving throw. Failure means he loses 200 experience points per 
sorcerer level; success reduces the loss to one-half that amount. However, a sorcerer's 
experience point total can never go below 0 as the result of a familiar's demise 
or dismissal. A slain or dismissed familiar cannot be replaced for a year and day. 
A slain familiar can be raised from the dead just as a character can be, and it 
does not lose a level or a Constitution point when this happy event occurs.

A character with more than one class that grants a familiar may have only one familiar 
at a time.

\vspace{12pt}
{\LARGE{}WIZARD}

\textbf{Alignment:} Any.

\textbf{Hit Die:} d4.

\vspace{12pt}
\subsubsection*{\textbf{Class Skills}}

The wizard's class skills (and the key ability for each skill) are Concentration 
(Con), Craft (Int), Decipher Script (Int), Knowledge (all skills, taken individually) 
(Int), Profession (Wis), and Spellcraft (Int). See Chapter 4: Skills for

skill descriptions.

\textbf{Skill Points at 1st Level:} (2 + Int modifier) x4.

\textbf{Skill Points at Each Additional Level:} 2 + Int modifier.

\vspace{12pt}
\begin{tabular}{|>{\raggedright}p{15pt}|>{\raggedright}p{25pt}|>{\raggedright}p{11pt}|>{\raggedright}p{10pt}|>{\raggedright}p{11pt}|>{\raggedright}p{19pt}|>{\raggedright}p{3pt}|>{\raggedright}p{7pt}|>{\raggedright}p{6pt}|>{\raggedright}p{6pt}|>{\raggedright}p{6pt}|>{\raggedright}p{6pt}|>{\raggedright}p{6pt}|>{\raggedright}p{6pt}|>{\raggedright}p{6pt}|>{\raggedright}p{6pt}|}
\hline
\multicolumn{16}{|p{158pt}|}{T\textbf{able: The Wizard}}\tabularnewline
\hline
 &  &  &  &  & ---------------------------- & \multicolumn{10}{p{64pt}|}{ \textbf{Spells 
per Day -------------------------}}\tabularnewline
\hline
L\textbf{evel} & B\textbf{ase Attack }\linebreak{}
\textbf{Bonus} & F\textbf{ort }\linebreak{}
\textbf{Save} & R\textbf{ef }\linebreak{}
\textbf{Save} & W\textbf{ill }\linebreak{}
\textbf{Save} & S\textbf{pecial} & 0 & 1\textbf{st} & 2\textbf{nd} & 3\textbf{rd} & 4\textbf{th} & 5\textbf{th} & 6\textbf{th} & 7\textbf{th} & 8\textbf{th} & 9\textbf{th}\tabularnewline
\hline
1st & +0 & +0 & +0 & +2 & Summon familiar, \linebreak{}
Scribe Scroll & 3 & 1--- & --- & --- & --- & --- & --- & --- & --- & \tabularnewline
\hline
2nd & +1 & +0 & +0 & +3 &  & 4 & 2--- & --- & --- & --- & --- & --- & --- & --- & \tabularnewline
\hline
3rd & +1 & +1 & +1 & +3 &  & 4 & 2 & 1--- & --- & --- & --- & --- & --- & --- & \tabularnewline
\hline
4th & +2 & +1 & +1 & +4 &  & 4 & 3 & 2--- & --- & --- & --- & --- & --- & --- & \tabularnewline
\hline
5th & +2 & +1 & +1 & +4 & Bonus feat & 4 & 3 & 2 & 1--- & --- & --- & --- & --- & --- & \tabularnewline
\hline
6th & +3 & +2 & +2 & +5 &  & 4 & 3 & 3 & 2--- & --- & --- & --- & --- & --- & \tabularnewline
\hline
7th & +3 & +2 & +2 & +5 &  & 4 & 4 & 3 & 2 & 1--- & --- & --- & --- & --- & \tabularnewline
\hline
8th & +4 & +2 & +2 & +6 &  & 4 & 4 & 3 & 3 & 2--- & --- & --- & --- & --- & \tabularnewline
\hline
9th & +4 & +3 & +3 & +6 &  & 4 & 4 & 4 & 3 & 2 & 1--- & --- & --- & --- & \tabularnewline
\hline
10th & +5 & +3 & +3 & +7 & Bonus feat & 4 & 4 & 4 & 3 & 3 & 2--- & --- & --- & --- & \tabularnewline
\hline
11th & +5 & +3 & +3 & +7 &  & 4 & 4 & 4 & 4 & 3 & 2 & 1--- & --- & --- & \tabularnewline
\hline
12th & +6/+1 & +4 & +4 & +8 &  & 4 & 4 & 4 & 4 & 3 & 3 & 2--- & --- & --- & \tabularnewline
\hline
13th & +6/+1 & +4 & +4 & +8 &  & 4 & 4 & 4 & 4 & 4 & 3 & 2 & 1--- & --- & \tabularnewline
\hline
14th & +7/+2 & +4 & +4 & +9 &  & 4 & 4 & 4 & 4 & 4 & 3 & 3 & 2--- & --- & \tabularnewline
\hline
15th & +7/+2 & +5 & +5 & +9 & Bonus feat & 4 & 4 & 4 & 4 & 4 & 4 & 3 & 2 & 1--- & \tabularnewline
\hline
16th & +8/+3 & +5 & +5 & +10 &  & 4 & 4 & 4 & 4 & 4 & 4 & 3 & 3 & 2--- & \tabularnewline
\hline
17th & +8/+3 & +5 & +5 & +10 &  & 4 & 4 & 4 & 4 & 4 & 4 & 4 & 3 & 2 & 1\tabularnewline
\hline
18th & +9/+4 & +6 & +6 & +11 &  & 4 & 4 & 4 & 4 & 4 & 4 & 4 & 3 & 3 & 2\tabularnewline
\hline
19th & +9/+4 & +6 & +6 & +11 &  & 4 & 4 & 4 & 4 & 4 & 4 & 4 & 4 & 3 & 3\tabularnewline
\hline
20th & +10/+5 & +6 & +6 & +12 & Bonus feat & 4 & 4 & 4 & 4 & 4 & 4 & 4 & 4 & 4 & 4\tabularnewline
\hline
\end{tabular}

\vspace{12pt}
\subsubsection*{\textbf{Class Features}}

All of the following are class features of the wizard.

\textbf{Weapon and Armor Proficiency: }Wizards are proficient with the club, dagger, 
heavy crossbow, light crossbow, and quarterstaff, but not with any type of armor 
or shield. Armor of any type interferes with a wizard's movements, which can cause 
her spells with somatic components to fail.

\textbf{Spells:} A wizard casts arcane spells which are drawn from the sorcerer/ 
wizard spell list. A wizard must choose and prepare her spells ahead of time (see 
below).

To learn, prepare, or cast a spell, the wizard must have an Intelligence score 
equal to at least 10 + the spell level. The Difficulty Class for a saving throw 
against a wizard's spell is 10 + the spell level + the wizard's Intelligence modifier.

Like other spellcasters, a wizard can cast only a certain number of spells of each 
spell level per day. Her base daily spell allotment is given on Table: The Wizard. 
In addition, she receives bonus spells per day if she has a high Intelligence score.

Unlike a bard or sorcerer, a wizard may know any number of spells. She must choose 
and prepare her spells ahead of time by getting a good night's sleep and spending 
1 hour studying her spellbook. While studying, the wizard decides which spells 
to prepare.

\textbf{Bonus Languages:} A wizard may substitute Draconic for one of the bonus 
languages available to the character because of her race.

\textbf{Familiar:} A wizard can obtain a familiar in exactly the same manner as 
a sorcerer can. See the sorcerer description and the information on Familiars below 
for details.

\textbf{Scribe Scroll:} At 1st level, a wizard gains Scribe Scroll as a bonus feat. 

\textbf{Bonus Feats:} At 5th, 10th, 15th, and 20th level, a wizard gains a bonus 
feat. At each such opportunity, she can choose a metamagic feat, an item creation 
feat, or Spell Mastery. The wizard must still meet all prerequisites for a bonus 
feat, including caster level minimums.

These bonus feats are in addition to the feat that a character of any class gets 
from advancing levels. The wizard is not limited to the categories of item creation 
feats, metamagic feats, or Spell Mastery when choosing these feats.

\textbf{Spellbooks:} A wizard must study her spellbook each day to prepare her 
spells. She cannot prepare any spell not recorded in her spellbook, except for 
\textit{read magic}, which all wizards can prepare from memory.

A wizard begins play with a spellbook containing all 0-level wizard spells (except 
those from her prohibited school or schools, if any; see School Specialization, 
below) plus three 1st-level spells of your choice. For each point of Intelligence 
bonus the wizard has, the spellbook holds one additional 1st-level spell of your 
choice. At each new wizard level, she gains two new spells of any spell level or 
levels that she can cast (based on her new wizard level) for her spellbook. At 
any time, a wizard can also add spells found in other wizards' spellbooks to her 
own.

\vspace{12pt}
SCHOOL SPECIALIZATION

A school is one of eight groupings of spells, each defined by a common theme. If 
desired, a wizard may specialize in one school of magic (see below). Specialization 
allows a wizard to cast extra spells from her chosen school, but she then never 
learns to cast spells from some other schools.

A specialist wizard can prepare one additional spell of her specialty school per 
spell level each day. She also gains a +2 bonus on Spellcraft checks to learn the 
spells of her chosen school.

The wizard must choose whether to specialize and, if she does so, choose her specialty 
at 1st level. At this time, she must also give up two other schools of magic (unless 
she chooses to specialize in divination; see below), which become her prohibited 
schools.

A wizard can never give up divination to fulfill this requirement.

Spells of the prohibited school or schools are not available to the wizard, and 
she can't even cast such spells from scrolls or fire them from wands. She may not 
change either her specialization or her prohibited schools later.

The eight schools of arcane magic are abjuration, conjuration, divination, enchantment, 
evocation, illusion, necromancy, and transmutation.

Spells that do not fall into any of these schools are called universal spells.

\textit{Abjuration: }Spells that protect, block, or banish. An abjuration specialist 
is called an abjurer.

\textit{Conjuration: }Spells that bring creatures or materials to the caster. A 
conjuration specialist is called a conjurer.

\textit{Divination: }Spells that reveal information. A divination specialist is 
called a diviner. Unlike the other specialists, a diviner must give up only one 
other school.

\textit{Enchantment: }Spells that imbue the recipient with some property or grant 
the caster power over another being. An enchantment specialist is called an enchanter.

\textit{Evocation: }Spells that manipulate energy or create something from nothing. 
An evocation specialist is called an evoker.

\textit{Illusion: }Spells that alter perception or create false images. An illusion 
specialist is called an illusionist.

\textit{Necromancy: }Spells that manipulate, create, or destroy life or life force. 
A necromancy specialist is called a necromancer.

\textit{Transmutation: }Spells that transform the recipient physically or change 
its properties in a more subtle way. A transmutation specialist is called a transmuter.

\textit{Universal: }Not a school, but a category for spells that all wizards can 
learn. A wizard cannot select universal as a specialty school or as a prohibited 
school. Only a limited number of spells fall into this category.

\vspace{12pt}
FAMILIARS

A familiar is a normal animal that gains new powers and becomes a magical beast 
when summoned to service by a sorcerer or wizard. It retains the appearance, Hit 
Dice, base attack bonus, base save bonuses, skills, and feats of the normal animal 
it once was, but it is treated as a magical beast instead of an animal for the 
purpose of any effect that depends on its type. Only a normal, unmodified animal 
may become a familiar. An animal companion cannot also function as a familiar.

A familiar also grants special abilities to its master (a sorcerer or wizard), 
as given on the table below. These special abilities apply only when the master 
and familiar are within 1 mile of each other.

Levels of different classes that are entitled to familiars stack for the purpose 
of determining any familiar abilities that depend on the master's level.

\vspace{12pt}
\begin{tabular}{|>{\raggedright}p{49pt}|>{\raggedright}p{277pt}|}
\hline
F\textbf{amiliar} & S\textbf{pecial}\tabularnewline
\hline
Bat  & Master gains a +3 bonus on Listen checks\tabularnewline
\hline
Cat  & Master gains a +3 bonus on Move Silently checks\tabularnewline
\hline
Hawk  & Master gains a +3 bonus on Spot checks in bright light\tabularnewline
\hline
Lizard  & Master gains a +3 bonus on Climb checks\tabularnewline
\hline
Owl  & Master gains a +3 bonus on Spot checks in shadows\tabularnewline
\hline
Rat  & Master gains a +2 bonus on Fortitude saves\tabularnewline
\hline
Raven\textsuperscript{1}  & Master gains a +3 bonus on Appraise checks\tabularnewline
\hline
Snake\textsuperscript{2}  & Master gains a +3 bonus on Bluff checks\tabularnewline
\hline
Toad  & Master gains +3 hit points\tabularnewline
\hline
Weasel  & Master gains a +2 bonus on Reflex saves\tabularnewline
\hline
\multicolumn{2}{|p{326pt}|}{1 A raven familiar can speak one language of its master's 
choice as a supernatural ability.}\tabularnewline
\hline
\multicolumn{2}{|p{326pt}|}{2 Tiny viper.}\tabularnewline
\hline
\end{tabular}

\vspace{12pt}
\textbf{Familiar Basics: }Use the basic statistics for a creature of the familiar's 
kind, but make the following

changes:

\textit{Hit Dice: }For the purpose of effects related to number of Hit Dice, use 
the master's character level or the familiar's normal HD total, whichever is higher.

\textit{Hit Points: }The familiar has one-half the master's total hit points (not 
including temporary hit points), rounded down, regardless of its actual Hit Dice.

\textit{Attacks: }Use the master's base attack bonus, as calculated from all his 
classes. Use the familiar's Dexterity or Strength modifier, whichever is greater, 
to get the familiar's melee attack bonus with natural weapons.

Damage equals that of a normal creature of the familiar's kind.

\textit{Saving Throws: }For each saving throw, use either the familiar's base save 
bonus (Fortitude +2, Reflex +2, Will +0) or the master's (as calculated from all 
his classes), whichever is better. The familiar uses its own ability modifiers 
to saves, and it doesn't share any of the other bonuses that the master might have 
on saves.

\textit{Skills: }For each skill in which either the master or the familiar has 
ranks, use either the normal skill ranks for an animal of that type or the master's 
skill ranks, whichever are better. In either case, the familiar uses its own ability 
modifiers. Regardless of a familiar's total skill modifiers, some skills may remain 
beyond the familiar's ability to use.

\textbf{Familiar Ability Descriptions: }All familiars have special abilities (or 
impart abilities to their masters) depending on the master's combined level in 
classes that grant familiars, as shown on the table below. The abilities given 
on the table are cumulative. 

\textit{Natural Armor Adj.: }The number noted here is an improvement to the familiar's 
existing natural armor bonus.

\textit{Int: }The familiar's Intelligence score.

\textit{Alertness (Ex): }While a familiar is within arm's reach, the master gains 
the Alertness feat.

\textit{Improved Evasion (Ex): }When subjected to an attack that normally allows 
a Reflex saving throw for half damage, a familiar takes no damage if it makes a 
successful saving throw and half damage even if the saving throw fails.

\textit{Share Spells: }At the master's option, he may have any spell (but not any 
spell-like ability) he casts on himself also affect his familiar. The familiar 
must be within 5 feet at the time of casting to receive the benefit.

If the spell or effect has a duration other than instantaneous, it stops affecting 
the familiar if it moves farther than 5 feet away and will not affect the familiar 
again even if it returns to the master before the duration expires. Additionally, 
the master may cast a spell with a target of ``You'' on his familiar (as a touch 
range spell) instead of on himself.

A master and his familiar can share spells even if the spells normally do not affect 
creatures of the familiar's type (magical beast).

\textit{Empathic Link (Su): }The master has an empathic link with his familiar 
out to a distance of up to 1 mile. The master cannot see through the familiar's 
eyes, but they can communicate empathically. Because of the limited nature of the 
link, only general emotional content can be communicated.

Because of this empathic link, the master has the same connection to an item or 
place that his familiar does.

\textit{Deliver Touch Spells (Su): }If the master is 3rd level or higher, a familiar 
can deliver touch spells for him. If the master and the familiar are in contact 
at the time the master casts a touch spell, he can designate his familiar as the 
``toucher.'' The familiar can then deliver the touch spell just as the master could. 
As usual, if the master casts another spell before the touch is delivered, the 
touch spell dissipates.

\textit{Speak with Master (Ex): }If the master is 5th level or higher, a familiar 
and the master can communicate verbally as if they were using a common language. 
Other creatures do not understand the communication without magical help.

\textit{Speak with Animals of Its Kind (Ex): }If the master is 7th level or higher, 
a familiar can communicate with animals of approximately the same kind as itself 
(including dire varieties): bats with bats, rats with rodents, cats with felines, 
hawks and owls and ravens with birds, lizards and snakes with reptiles, toads with 
amphibians, weasels with similar creatures (weasels, minks, polecats, ermines, 
skunks, wolverines, and badgers). Such communication is limited by the intelligence 
of the conversing creatures.

\textit{Spell Resistance (Ex): }If the master is 11th level or higher, a familiar 
gains spell resistance equal to the master's level + 5. To affect the familiar 
with a spell, another spellcaster must get a result on a caster level check (1d20 
+ caster level) that equals or exceeds the familiar's spell resistance.

\textit{Scry on Familiar (Sp): }If the master is 13th level or higher, he may scry 
on his familiar (as if casting the \textit{scrying }spell) once per day.

\vspace{12pt}
\begin{tabular}{|>{\raggedright}p{63pt}|>{\raggedright}p{65pt}|>{\raggedright}p{13pt}|>{\raggedright}p{160pt}|}
\hline
M\textbf{aster Class Level } & N\textbf{atural Armor Adj. } & I\textbf{nt } & S\textbf{pecial}\tabularnewline
\hline
1st-2nd  & +1  & 6  & Alertness, improved evasion, share spells, empathic link\tabularnewline
\hline
3rd-4th  & +2  & 7  & Deliver touch spells\tabularnewline
\hline
5th-6th  & +3  & 8  & Speak with master\tabularnewline
\hline
7th-8th  & +4  & 9  & Speak with animals of its kind\tabularnewline
\hline
9th-10th  & +5  & 10 --- & \tabularnewline
\hline
11th-12th  & +6  & 11  & Spell resistance\tabularnewline
\hline
13th-14th  & +7  & 12  & S\textit{cry }on familiar\tabularnewline
\hline
15th-16th  & +8  & 13 --- & \tabularnewline
\hline
17th-18th  & +9  & 14 --- & \tabularnewline
\hline
19th-20th  & +10  & 15 --- & \tabularnewline
\hline
\end{tabular}

\vspace{12pt}
ARCANE SPELLS AND ARMOR

Wizards and sorcerers do not know how to wear armor effectively.

If desired, they can wear armor anyway (though they'll be clumsy in it), or they 
can gain training in the proper use of armor (with the various Armor Proficiency 
feats---light, medium, and heavy---and the Shield Proficiency feat), or they can 
multiclass to add a class that grants them armor proficiency. Even if a wizard 
or sorcerer is wearing armor with which he or she is proficient, however, it might 
still interfere with spellcasting.

Armor restricts the complicated gestures that a wizards or sorcerer must make while 
casting any spell that has a somatic component (most do). The armor and shield 
descriptions list the arcane spell failure chance for different armors and shields.

By contrast, bards not only know how to wear light armor effectively, but they 
can also ignore the arcane spell failure chance for such armor. A bard wearing 
armor heavier than light or using any type of shield incurs the normal arcane spell 
failure chance, even if he becomes proficient with that armor.

If a spell doesn't have a somatic component, an arcane spellcaster can cast it 
with no problem while wearing armor. Such spells can also be cast even if the caster's 
hands are bound or if he or she is grappling (although Concentration checks still 
apply normally). Also, the metamagic feat Still Spell allows a spellcaster to prepare 
or cast a spell at one spell level higher than normal without the somatic component. 
This also provides a way to cast a spell while wearing armor without risking arcane 
spell failure. 

\vspace{12pt}
{\LARGE{}MULTICLASS CHARACTERS}

A character may add new classes as he or she progresses in level, thus becoming 
a multiclass character. The class abilities from a character's different classes 
combine to determine a multiclass character's overall abilities. Multiclassing 
improves a character's versatility at the expense of focus.

\vspace{12pt}
CLASS AND LEVEL FEATURES

As a general rule, the abilities of a multiclass character are the sum of the abilities 
of each of the character's classes.

\textbf{Level:} ``Character level'' is a character's total number of levels. It 
is used to determine when feats and ability score boosts are gained.

``Class level'' is a character's level in a particular class. For a character whose 
levels are all in the same class, character level and class level are the same.

\textbf{Hit Points:} A character gains hit points from each class as his or her 
class level increases, adding the new hit points to the previous total. 

\textbf{Base Attack Bonus:} Add the base attack bonuses acquired for each class 
to get the character's base attack bonus. A resulting value of +6 or higher provides 
the character with multiple attacks. 

\textbf{Saving Throws:} Add the base save bonuses for each class together.

\textbf{Skills:} If a skill is a class skill for any of a multiclass character's 
classes, then character level determines a skill's maximum rank. (The maximum rank 
for a class skill is 3 + character level.)

If a skill is not a class skill for any of a multiclass character's classes, the 
maximum rank for that skill is one-half the maximum for a class skill.

\textbf{Class Features:} A multiclass character gets all the class features of 
all his or her classes but must also suffer the consequences of the special restrictions 
of all his or her classes. (\textit{Exception: }A character who acquires the barbarian 
class does not become illiterate.) 

In the special case of turning undead, both clerics and experienced paladins have 
the same ability. If the character's paladin level is 4th or higher, her effective 
turning level is her cleric level plus her paladin level minus 3. 

In the special case of uncanny dodge, both experienced barbarians and experienced 
rogues have the same ability. When a barbarian/rogue would gain uncanny dodge a 
second time (for her second class), she instead gains improved uncanny dodge, if 
she does not already have it. Her barbarian and rogue levels stack to determine 
the rogue level an attacker needs to flank her. 

In the special case of obtaining a familiar, both wizards and sorcerers have the 
same ability. A sorcerer/wizard stacks his sorcerer and wizard levels to determine 
the familiar's natural armor, Intelligence score, and special abilities.

\textbf{Feats: }A multiclass character gains feats based on character levels, regardless 
of individual class level

\textbf{Ability Increases:} A multiclass character gains ability score increases 
based on character level, regardless of individual class level.

\textbf{Spells:} The character gains spells from all of his or her spellcasting 
classes and keeps a separate spell list for each class. If a spell's effect is 
based on the class level of the caster, the player must keep track of which class's 
spell list the character is casting the spell fr{\small{}om.}

\newpage

\end{document}
