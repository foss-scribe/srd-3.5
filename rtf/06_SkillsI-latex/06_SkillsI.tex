%&pdfLaTeX
% !TEX encoding = UTF-8 Unicode
\documentclass{article}
\usepackage{ifxetex}
\ifxetex
\usepackage{fontspec}
\setmainfont[Mapping=tex-text]{STIXGeneral}
\else
\usepackage[T1]{fontenc}
\usepackage[utf8]{inputenc}
\fi
\usepackage{textcomp}

\usepackage{array}
\usepackage{amssymb}
\usepackage{fancyhdr}
\renewcommand{\headrulewidth}{0pt}
\renewcommand{\footrulewidth}{0pt}

\begin{document}

This material is Open Game Content, and is licensed for public use under the terms 
of the Open Game License v1.0a.

{\LARGE{}SKILLS I}

\vspace{12pt}
{\LARGE{}SKILLS SUMMARY}

If you buy a class skill, your character gets 1 rank (equal to a +1 bonus on checks 
with that skill) for each skill point. If you buy other classes' skills (cross-class 
skills), you get 1/2 rank per skill point.

Your maximum rank in a class skill is your character level + 3.

Your maximum rank in a cross-class skill is one-half of this number (do not round 
up or down).

\textbf{Using Skills:} To make a skill check, roll: 1d20 + skill modifier (Skill 
modifier = skill rank + ability modifier + miscellaneous modifiers)

This roll works just like an attack roll or a saving throw--- the higher the roll, 
the better. Either you're trying to match or exceed a certain Difficulty Class 
(DC), or you're trying to beat another character's check result.

\textbf{Skill Ranks:} A character's number of ranks in a skill is based on how 
many skill points a character has invested in a skill. Many skills can be used 
even if the character has no ranks in them; doing this is called making an untrained 
skill check.

\textbf{Ability Modifier:} The ability modifier used in a skill check is the modifier 
for the skill's key ability (the ability associated with the skill's use). The 
key ability of each skill is noted in its description.

\textbf{Miscellaneous Modifiers:} Miscellaneous modifiers include racial bonuses, 
armor check penalties, and bonuses provided by feats, among others.

\vspace{12pt}
Each skill point you spend on a class skill gets you 1 rank in that skill. Class 
skills are the skills found on your character's class skill list. Each skill point 
you spend on a cross-class skill gets your character 1/2 rank in that skill. Cross-class 
skills are skills not found on your character's class skill list. (Half ranks do 
not improve your skill check, but two 1/2 ranks make 1 rank.) You can't save skill 
points to spend later.

The maximum rank in a class skill is the character's level + 3. If it's a cross-class 
skill, the maximum rank is half of that number (do not round up or down).

Regardless of whether a skill is purchased as a class skill or a cross-class skill, 
if it is a class skill for any of your classes, your maximum rank equals your total 
character level + 3.

\vspace{12pt}
{\LARGE{}USING SKILLS}

When your character uses a skill, you make a skill check to see how well he or 
she does. The higher the result of the skill check, the better. Based on the circumstances, 
your result must match or beat a particular number (a DC or the result of an opposed 
skill check) for the check to be successful. The harder the task, the higher the 
number you need to roll.

Circumstances can affect your check. A character who is free to work without distractions 
can make a careful attempt and avoid simple mistakes. A character who has lots 
of time can try over and over again, thereby assuring the best outcome. If others 
help, the character may succeed where otherwise he or she would fail.

\vspace{12pt}
SKILL CHECKS

A skill check takes into account a character's training (skill rank), natural talent 
(ability modifier), and luck (the die roll). It may also take into account his 
or her race's knack for doing certain things (racial bonus) or what armor he or 
she is wearing (armor check penalty), or a certain feat the character possesses, 
among other things. 

To make a skill check, roll 1d20 and add your character's skill modifier for that 
skill. The skill modifier incorporates the character's ranks in that skill and 
the ability modifier for that skill's key ability, plus any other miscellaneous 
modifiers that may apply, including racial bonuses and armor check penalties. The 
higher the result, the better. Unlike with attack rolls and saving throws, a natural 
roll of 20 on the d20 is not an automatic success, and a natural roll of 1 is not 
an automatic failure.

\vspace{12pt}
\textbf{Difficulty Class}

Some checks are made against a Difficulty Class (DC). The DC is a number (set using 
the skill rules as a guideline) that you must score as a result on your skill check 
in order to succeed.

\vspace{12pt}
\begin{tabular}{|>{\raggedright}p{74pt}|>{\raggedright}p{251pt}|}
\hline
\multicolumn{2}{|p{326pt}|}{T\textbf{able: Difficulty Class Examples}}\tabularnewline
\hline
D\textbf{ifficulty (DC)} & E\textbf{xample (Skill Used)}\tabularnewline
\hline
Very easy (0)  & Notice something large in plain sight (Spot)\tabularnewline
\hline
Easy (5)  & Climb a knotted rope (Climb)\tabularnewline
\hline
Average (10)  & Hear an approaching guard (Listen)\tabularnewline
\hline
Tough (15)  & Rig a wagon wheel to fall off (Disable Device)\tabularnewline
\hline
Challenging (20)  & Swim in stormy water (Swim)\tabularnewline
\hline
Formidable (25)  & Open an average lock (Open Lock)\tabularnewline
\hline
Heroic (30)  & Leap across a 30-foot chasm (Jump)\tabularnewline
\hline
Nearly impossible (40) & Track a squad of orcs across hard ground after 24 hours 
of rainfall (Survival)\tabularnewline
\hline
\end{tabular}

\vspace{12pt}
\textbf{Opposed Checks}

An opposed check is a check whose success or failure is determined by comparing 
the check result to another character's check result. In an opposed check, the 
higher result succeeds, while the lower result fails. In case of a tie, the higher 
skill modifier wins. If these scores are the same, roll again to break the tie.

\vspace{12pt}
\begin{tabular}{|>{\raggedright}p{109pt}|>{\raggedright}p{88pt}|>{\raggedright}p{116pt}|}
\hline
\multicolumn{3}{|p{314pt}|}{T\textbf{able: Example Opposed Checks}}\tabularnewline
\hline
T\textbf{ask } & S\textbf{kill (Key Ability)} & O\textbf{pposing Skill (Key Ability)}\tabularnewline
\hline
Con someone  & Bluff (Cha)  & Sense Motive (Wis)\tabularnewline
\hline
Pretend to be someone else & Disguise (Cha)  & Spot (Wis)\tabularnewline
\hline
Create a false map  & Forgery (Int)  & Forgery (Int)\tabularnewline
\hline
Hide from someone  & Hide (Dex)  & Spot (Wis)\tabularnewline
\hline
Make a bully back down  & Intimidate (Cha)  & Special\textsuperscript{1}\tabularnewline
\hline
Sneak up on someone  & Move Silently (Dex)  & Listen (Wis)\tabularnewline
\hline
Steal a coin pouch  & Sleight of Hand (Dex)  & Spot (Wis)\tabularnewline
\hline
Tie a prisoner securely  & Use Rope (Dex)  & Escape Artist (Dex)\tabularnewline
\hline
\multicolumn{3}{|p{314pt}|}{1 An Intimidate check is opposed by the target's level 
check, not a skill check. See the Intimidate skill description for more information.}\tabularnewline
\hline
\end{tabular}

\vspace{12pt}
\textbf{Trying Again}

In general, you can try a skill check again if you fail, and you can keep trying 
indefinitely. Some skills, however, have consequences of failure that must be taken 
into account. A few skills are virtually useless once a check has failed on an 
attempt to accomplish a particular task. For most skills, when a character has 
succeeded once at a given task, additional successes are meaningless.

\vspace{12pt}
\textbf{Untrained Skill Checks}

Generally, if your character attempts to use a skill he or she does not possess, 
you make a skill check as normal. The skill modifier doesn't have a skill rank 
added in because the character has no ranks in the skill. Any other applicable 
modifiers, such as the modifier for the skill's key ability, are applied to the 
check.

Many skills can be used only by someone who is trained in them.

\vspace{12pt}
\textbf{Favorable and Unfavorable Conditions}

Some situations may make a skill easier or harder to use, resulting in a bonus 
or penalty to the skill modifier for a skill check or a change to the DC of the 
skill check.

The chance of success can be altered in four ways to take into account exceptional 
circumstances.

1. Give the skill user a +2 circumstance bonus to represent conditions that improve 
performance, such as having the perfect tool for the job, getting help from another 
character (see Combining Skill Attempts), or possessing unusually accurate information. 

2. Give the skill user a -2 circumstance penalty to represent conditions that hamper 
performance, such as being forced to use improvised tools or having misleading 
information.

3. Reduce the DC by 2 to represent circumstances that make the task easier, such 
as having a friendly audience or doing work that can be subpar.

4. Increase the DC by 2 to represent circumstances that make the task harder, such 
as having an uncooperative audience or doing work that must be flawless.

Conditions that affect your character's ability to perform the skill change the 
skill modifier. Conditions that modify how well the character has to perform the 
skill to succeed change the DC. A bonus to the skill modifier and a reduction in 
the check's DC have the same result: They create a better chance of success. But 
they represent different circumstances, and sometimes that difference is important.

\vspace{12pt}
\textbf{Time and Skill Checks}

Using a skill might take a round, take no time, or take several rounds or even 
longer. Most skill uses are standard actions, move actions, or full-round actions. 
Types of actions define how long activities take to perform within the framework 
of a combat round (6 seconds) and how movement is treated with respect to the activity. 
Some skill checks are instant and represent reactions to an event, or are included 
as part of an action.

These skill checks are not actions. Other skill checks represent part of movement.

\vspace{12pt}
\textbf{Checks without Rolls}

A skill check represents an attempt to accomplish some goal, usually while under 
some sort of time pressure or distraction. Sometimes, though, a character can use 
a skill under more favorable conditions and eliminate the luck factor.

\textbf{Taking 10: }When your character is not being threatened or distracted, 
you may choose to take 10. Instead of rolling 1d20 for the skill check, calculate 
your result as if you had rolled a 10. For many routine tasks, taking 10 makes 
them automatically successful. Distractions or threats (such as combat) make it 
impossible for a character to take 10. In most cases, taking 10 is purely a safety 
measure ---you know (or expect) that an average roll will succeed but fear that 
a poor roll might fail, so you elect to settle for the average roll (a 10). Taking 
10 is especially useful in situations where a particularly high roll wouldn't help.

\textbf{Taking 20:}When you have plenty of time (generally 2 minutes for a skill 
that can normally be checked in 1 round, one full-round action, or one standard 
action), you are faced with no threats or distractions, and the skill being attempted 
carries no penalties for failure, you can take 20. In other words, eventually you 
will get a 20 on 1d20 if you roll enough times. Instead of rolling 1d20 for the 
skill check, just calculate your result as if you had rolled a 20.

Taking 20 means you are trying until you get it right, and it assumes that you 
fail many times before succeeding. Taking 20 takes twenty times as long as making 
a single check would take.

Since taking 20 assumes that the character will fail many times before succeeding, 
if you did attempt to take 20 on a skill that carries penalties for failure, your 
character would automatically incur those penalties before he or she could complete 
the task. Common ``take 20'' skills include Escape Artist, Open Lock, and Search.

\textbf{Ability Checks and Caster Level Checks:} The normal take 10 and take 20 
rules apply for ability checks. Neither rule applies to caster level checks.

\vspace{12pt}
COMBINING SKILL ATTEMPTS

When more than one character tries the same skill at the same time and for the 
same purpose, their efforts may overlap.

\vspace{12pt}
\textbf{Individual Events}

Often, several characters attempt some action and each succeeds or fails independently. 
 The result of one character's Climb check does not influence the results of other 
characters Climb check.

\vspace{12pt}
\textbf{Aid Another}

You can help another character achieve success on his or her skill check by making 
the same kind of skill check in a cooperative effort. If you roll a 10 or higher 
on your check, the character you are helping gets a +2 bonus to his or her check, 
as per the rule for favorable conditions. (You can't take 10 on a skill check to 
aid another.) In many cases, a character's help won't be beneficial, or only a 
limited number of characters can help at once. 

In cases where the skill restricts who can achieve certain results you can't aid 
another to grant a bonus to a task that your character couldn't achieve alone.

\vspace{12pt}
\textbf{Skill Synergy}

It's possible for a character to have two skills that work well together. In general, 
having 5 or more ranks in one skill gives the character a +2 bonus on skill checks 
with each of its synergistic skills, as noted in the skill description. In some 
cases, this bonus applies only to specific uses of the skill in question, and not 
to all checks. Some skills provide benefits on other checks made by a character, 
such as those checks required to use certain class features.

\vspace{12pt}
ABILITY CHECKS

Sometimes a character tries to do something to which no specific skill really applies. 
In these cases, you make an ability check. An ability check is a roll of 1d20 plus 
the appropriate ability modifier. Essentially, you're making an untrained skill 
check. 

In some cases, an action is a straight test of one's ability with no luck involved. 
Just as you wouldn't make a height check to see who is taller, you don't make a 
Strength check to see who is stronger.

\vspace{12pt}
{\LARGE{}SKILL DESCRIPTIONS}

This section describes each skill, including common uses and typical modifiers. 
Characters can sometimes use skills for purposes other than those noted here.

Here is the format for skill descriptions.

\vspace{12pt}
SKILL NAME

The skill name line includes (in addition to the name of the skill) the following 
information.

\textbf{Key Ability:} The abbreviation of the ability whose modifier applies to 
the skill check. \textit{Exception: }Speak Language has ``None'' as its key ability 
because the use of this skill does not require a check.

\textbf{Trained Only:} If this notation is included in the skill name line, you 
must have at least 1 rank in the skill to use it. If it is omitted, the skill can 
be used untrained (with a rank of 0). If any special notes apply to trained or 
untrained use, they are covered in the Untrained section (see below).

\textbf{Armor Check Penalty:} If this notation is included in the skill name line, 
an armor check penalty applies (when appropriate) to checks using this skill. If 
this entry is absent, an armor check penalty does not apply.

\vspace{12pt}
The skill name line is followed by a general description of what using the skill 
represents. After the description are a few other types of information:

\textbf{Check:} What a character (``you'' in the skill description) can do with 
a successful skill check and the check's DC.

\textbf{Action:} The type of action using the skill requires, or the amount of 
time required for a check.

\textbf{Try Again:} Any conditions that apply to successive attempts to use the 
skill successfully. If the skill doesn't allow you to attempt the same task more 
than once, or if failure carries an inherent penalty (such as with the Climb skill), 
you can't take 20. If this paragraph is omitted, the skill can be retried without 
any inherent penalty, other than the additional time required.

\textbf{Special:} Any extra facts that apply to the skill, such as special effects 
deriving from its use or bonuses that certain characters receive because of class, 
feat choices, or race.

\textbf{Synergy:} Some skills grant a bonus to the use of one or more other skills 
because of a synergistic effect. This entry, when present, indicates what bonuses 
this skill may grant or receive because of such synergies. See Table 4-5 for a 
complete list of bonuses granted by synergy between skills (or between a skill 
and a class feature).

\textbf{Restriction:} The full utility of certain skills is restricted to characters 
of certain classes or characters who possess certain feats. This entry indicates 
whether any such restrictions exist for the skill.

\textbf{Untrained:} This entry indicates what a character without at least 1 rank 
in the skill can do with it. If this entry doesn't appear, it means that the skill 
functions normally for untrained characters (if it can be used untrained) or that 
an untrained character can't attempt checks with this skill (for skills that are 
designated as ``Trained Only'').

\vspace{12pt}
APPRAISE (INT)

\textbf{Check:} You can appraise common or well-known objects with a DC 12 Appraise 
check. Failure means that you estimate the value at 50\% to 150\% (2d6+3 times 
10\%,) of its actual value.

Appraising a rare or exotic item requires a successful check against DC 15, 20, 
or higher. If the check is successful, you estimate the value correctly; failure 
means you cannot estimate the item's value.

A magnifying glass gives you a +2 circumstance bonus on Appraise checks involving 
any item that is small or highly detailed, such as a gem. A merchant's scale gives 
you a +2 circumstance bonus on Appraise checks involving any items that are valued 
by weight, including anything made of precious metals.

These bonuses stack.

\textbf{Action:} Appraising an item takes 1 minute (ten consecutive full-round 
actions).

\textbf{Try Again:} No. You cannot try again on the same object, regardless of 
success.

\textbf{Special:} A dwarf gets a +2 racial bonus on Appraise checks that are related 
to stone or metal items because dwarves are familiar with valuable items of all 
kinds (especially those made of stone or metal).

The master of a raven familiar gains a +3 bonus on Appraise checks.

A character with the Diligent feat gets a +2 bonus on Appraise checks.

\textbf{Synergy:} If you have 5 ranks in any Craft skill, you gain a +2 bonus on 
Appraise checks related to items made with that Craft skill.

\textbf{Untrained:} For common items, failure on an untrained check means no estimate. 
For rare items, success means an estimate of 50\% to 150\% (2d6+3 times 10\%).

\vspace{12pt}
BALANCE (DEX; ARMOR CHECK PENALTY)

\textbf{Check:} You can walk on a precarious surface. A successful check lets you 
move at half your speed along the surface for 1 round. A failure by 4 or less means 
you can't move for 1 round. A failure by 5 or more means you fall. The difficulty 
varies with the surface, as follows:

\vspace{12pt}
\begin{tabular}{|>{\raggedright}p{91pt}|>{\raggedright}p{54pt}|>{\raggedright}p{86pt}|>{\raggedright}p{58pt}|}
\hline
N\textbf{arrow Surface } & B\textbf{alance DC}\textsuperscript{\textbf{1}}\textbf{ 
} & D\textbf{ifficult Surface} & B\textbf{alance DC}\textsuperscript{\textbf{1}}\tabularnewline
\hline
7-12 inches wide  & 10  & Uneven flagstone  & 10\textsuperscript{2}\tabularnewline
\hline
2-6 inches wide  & 15  & Hewn stone floor  & 10\textsuperscript{2}\tabularnewline
\hline
Less than 2 inches wide  & 20  & Sloped or angled floor  & 10\textsuperscript{2}\tabularnewline
\hline
\multicolumn{4}{|p{290pt}|}{1 Add modifiers from Narrow Surface Modifiers, below, 
as appropriate.}\tabularnewline
\hline
\multicolumn{4}{|p{290pt}|}{2 Only if running or charging. Failure by 4 or less 
means the character can't run or charge, but may otherwise act normally.}\tabularnewline
\hline
\end{tabular}

\vspace{12pt}
\begin{tabular}{|>{\raggedright}p{176pt}|>{\raggedright}p{93pt}|}
\hline
N\textbf{arrow Surface Modifiers} & \tabularnewline
\hline
S\textbf{urface } & D\textbf{C Modifier}\textsuperscript{\textbf{1}}\tabularnewline
\hline
Lightly obstructed   & +2\tabularnewline
\hline
Severely obstructed  & +5\tabularnewline
\hline
Lightly slippery & +2\tabularnewline
\hline
Severely slippery  & +5\tabularnewline
\hline
Sloped or angled  & +2\tabularnewline
\hline
\multicolumn{2}{|p{269pt}|}{1 Add the appropriate modifier to the Balance DC of 
a narrow surface.}\tabularnewline
\hline
\multicolumn{2}{|p{269pt}|}{These modifiers stack.}\tabularnewline
\hline
\end{tabular}

\vspace{12pt}
\textit{Being Attacked while Balancing: }You are considered flat-footed while balancing, 
since you can't move to avoid a blow, and thus you lose your Dexterity bonus to 
AC (if any). If you have 5 or more ranks in Balance, you aren't considered flat-footed 
while balancing. If you take damage while balancing, you must make another Balance 
check against the same DC to remain standing.

\textit{Accelerated Movement: }You can try to walk across a precarious surface 
more quickly than normal. If you accept a -5 penalty, you can move your full speed 
as a move action. (Moving twice your speed in a round requires two Balance checks, 
one for each move action used.) You may also accept this penalty in order to charge 
across a precarious surface; charging requires one Balance check for each multiple 
of your speed (or fraction thereof ) that you charge.

\textbf{Action:} None. A Balance check doesn't require an action; it is made as 
part of another action or as a reaction to a situation.

\textbf{Special:} If you have the Agile feat, you get a +2 bonus on Balance checks.

\textbf{Synergy:} If you have 5 or more ranks in Tumble, you get a +2 bonus on 
Balance checks.

\vspace{12pt}
BLUFF (CHA)

\textbf{Check:} A Bluff check is opposed by the target's Sense Motive check. See 
the accompanying table for examples of different kinds of bluffs and the modifier 
to the target's Sense Motive check for each one.

Favorable and unfavorable circumstances weigh heavily on the outcome of a bluff. 
Two circumstances can weigh against you: The bluff is hard to believe, or the action 
that the target is asked to take goes against its self-interest, nature, personality, 
orders, or the like. If it's important, you can distinguish between a bluff that 
fails because the target doesn't believe it and one that fails because it just 
asks too much of the target. For instance, if the target gets a +10 bonus on its 
Sense Motive check because the bluff demands something risky, and the Sense Motive 
check succeeds by 10 or less, then the target didn't so much see through the bluff 
as prove reluctant to go along with it. A target that succeeds by 11 or more has 
seen through the bluff.

A successful Bluff check indicates that the target reacts as you wish, at least 
for a short time (usually 1 round or less) or believes something that you want 
it to believe. Bluff, however, is not a \textit{suggestion }spell. 

A bluff requires interaction between you and the target. Creatures unaware of you 
cannot be bluffed.

\textit{Feinting in Combat: }You can also use Bluff to mislead an opponent in melee 
combat (so that it can't dodge your next attack effectively). To feint, make a 
Bluff check opposed by your target's Sense Motive check, but in this case, the 
target may add its base attack bonus to the roll along with any other applicable 
modifiers.

If your Bluff check result exceeds this special Sense Motive check result, your 
target is denied its Dexterity bonus to AC (if any) for the next melee attack you 
make against it. This attack must be made on or before your next turn.

Feinting in this way against a nonhumanoid is difficult because it's harder to 
read a strange creature's body language; you take a -4 penalty on your Bluff check. 
Against a creature of animal Intelligence (1 or 2) it's even harder; you take a 
-8 penalty. Against a nonintelligent creature, it's impossible.

Feinting in combat does not provoke an attack of opportunity.

\textit{Creating a Diversion to Hide: }You can use the Bluff skill to help you 
hide. A successful Bluff check gives you the momentary diversion you need to attempt 
a Hide check while people are aware of you. This usage does not provoke an attack 
of opportunity.

\textit{Delivering a Secret Message: }You can use Bluff to get a message across 
to another character without others understanding it. The DC is 15 for simple messages, 
or 20 for complex messages, especially those that rely on getting across new information. 
Failure by 4 or less means you can't get the message across. Failure by 5 or more 
means that some false information has been implied or inferred. Anyone listening 
to the exchange can make a Sense Motive check opposed by the Bluff check you made 
to transmit in order to intercept your message (see Sense Motive).

\textbf{Action:} Varies. A Bluff check made as part of general interaction always 
takes at least 1 round (and is at least a full-round action), but it can take much 
longer if you try something elaborate. A Bluff check made to feint in combat or 
create a diversion to hide is a standard action. A Bluff check made to deliver 
a secret message doesn't take an action; it is part of normal communication.

\textbf{Try Again:} Varies. Generally, a failed Bluff check in social interaction 
makes the target too suspicious for you to try again in the same circumstances, 
but you may retry freely on Bluff checks made to feint in combat. Retries are also 
allowed when you are trying to send a message, but you may attempt such a retry 
only once per round.

Each retry carries the same chance of miscommunication.

\textbf{Special:} A ranger gains a bonus on Bluff checks when using this skill 
against a favored enemy.

The master of a snake familiar gains a +3 bonus on Bluff checks.

If you have the Persuasive feat, you get a +2 bonus on Bluff checks.

\textbf{Synergy:} If you have 5 or more ranks in Bluff, you get a +2 bonus on Diplomacy, 
Intimidate, and Sleight of Hand checks, as well as on Disguise checks made when 
you know you're being observed and you try to act in character.

\vspace{12pt}
\begin{tabular}{|>{\raggedright}p{238pt}|>{\raggedright}p{87pt}|}
\hline
\multicolumn{2}{|p{326pt}|}{B\textbf{luff Examples}}\tabularnewline
\hline
E\textbf{xample Circumstances} & S\textbf{ense Motive Modifier}\tabularnewline
\hline
The target wants to believe you.- & 5\tabularnewline
\hline
The bluff is believable and doesn't affect the target much.  & +0\tabularnewline
\hline
The bluff is a little hard to believe or puts the target at some risk.  & +5\tabularnewline
\hline
The bluff is hard to believe or puts the target at significant risk.  & +10\tabularnewline
\hline
The bluff is way out there, almost too incredible to consider. & +20\tabularnewline
\hline
\end{tabular}

\vspace{12pt}
CLIMB (STR; ARMOR CHECK PENALTY)

\textbf{Check:} With a successful Climb check, you can advance up, down, or across 
a slope, a wall, or some other steep incline (or even a ceiling with handholds) 
at one-quarter your normal speed. A slope is considered to be any incline at an 
angle measuring less than 60 degrees; a wall is any incline at an angle measuring 
60 degrees or more.

A Climb check that fails by 4 or less means that you make no progress, and one 
that fails by 5 or more means that you fall from whatever height you have already 
attained.

A climber's kit gives you a +2 circumstance bonus on Climb checks.

The DC of the check depends on the conditions of the climb. Compare the task with 
those on the following table to determine an appropriate DC.

\vspace{12pt}
\begin{tabular}{|>{\raggedright}p{18pt}|>{\raggedright}p{307pt}|}
\hline
C\textbf{limb DC } & E\textbf{xample Surface or Activity}\tabularnewline
\hline
0  & A slope too steep to walk up, or a knotted rope with a wall to brace against.\tabularnewline
\hline
5  & A rope with a wall to brace against, or a knotted rope, or a rope affected 
by the \textit{rope trick }spell. \tabularnewline
\hline
10  & A surface with ledges to hold on to and stand on, such as a very rough wall 
or a ship's rigging. \tabularnewline
\hline
15  & Any surface with adequate handholds and footholds (natural or artificial), 
such as a very rough natural rock surface or a tree, or an unknotted rope, or pulling 
yourself up when dangling by your hands. \tabularnewline
\hline
20  & An uneven surface with some narrow handholds and footholds, such as a typical 
wall in a dungeon or ruins. \tabularnewline
\hline
25  & A rough surface, such as a natural rock wall or a brick wall. \tabularnewline
\hline
25  & An overhang or ceiling with handholds but no footholds. \tabularnewline
\hline
---  & A perfectly smooth, flat, vertical surface cannot be climbed.\tabularnewline
\hline
\end{tabular}

\vspace{12pt}
\begin{tabular}{|>{\raggedright}p{52pt}|>{\raggedright}p{274pt}|}
\hline
\section*{C\textbf{limb DC Modifier}\textsuperscript{\textbf{1}}\textbf{ }} & \section*{E\textbf{xample 
Surface or Activity}}\tabularnewline
\hline
-10 &  Climbing a chimney (artificial or natural) or other location where you can 
brace against two opposite walls (reduces DC by 10).\tabularnewline
\hline
-5  & Climbing a corner where you can brace against perpendicular walls (reduces 
DC by 5).\tabularnewline
\hline
+5 &  Surface is slippery (increases DC by 5).\tabularnewline
\hline
\multicolumn{2}{|p{326pt}|}{1These modifiers are cumulative; use any that apply.}\tabularnewline
\hline
\end{tabular}

\vspace{12pt}
You need both hands free to climb, but you may cling to a wall with one hand while 
you cast a spell or take some other action that requires only one hand. While climbing, 
you can't move to avoid a blow, so you lose your Dexterity bonus to AC (if any). 
You also can't use a shield while climbing.

Any time you take damage while climbing, make a Climb check against the DC of the 
slope or wall. Failure means you fall from your current height and sustain the 
appropriate falling damage.

\textit{Accelerated Climbing: }You try to climb more quickly than normal. By accepting 
a -5 penalty, you can move half your speed (instead of one-quarter your speed).

\textit{Making Your Own Handholds and Footholds: }You can make your own handholds 
and footholds by pounding pitons into a wall. Doing so takes 1 minute per piton, 
and one piton is needed per 3 feet of distance. As with any surface that offers 
handholds and footholds, a wall with pitons in it has a DC of 15. In the same way, 
a climber with a handaxe or similar implement can cut handholds in an ice wall.

\textit{Catching Yourself When Falling: }It's practically impossible to catch yourself 
on a wall while falling. Make a Climb check (DC = wall's DC + 20) to do so. It's 
much easier to catch yourself on a slope (DC = slope's DC + 10).

\textit{Catching a Falling Character While Climbing: }If someone climbing above 
you or adjacent to you falls, you can attempt to catch the falling character if 
he or she is within your reach. Doing so requires a successful melee touch attack 
against the falling character (though he or she can voluntarily forego any Dexterity 
bonus to AC if desired). If you hit, you must immediately attempt a Climb check 
(DC = wall's DC + 10). Success indicates that you catch the falling character, 
but his or her total weight, including equipment, cannot exceed your heavy load 
limit or you automatically fall. If you fail your Climb check by 4 or less, you 
fail to stop the character's fall but don't lose your grip on the wall. If you 
fail by 5 or more, you fail to stop the character's fall and begin falling as well.

\textbf{Action:} Climbing is part of movement, so it's generally part of a move 
action (and may be combined with other types of movement in a move action). Each 
move action that includes any climbing requires a separate Climb check. Catching 
yourself or another falling character doesn't take an action.

\textbf{Special:} You can use a rope to haul a character upward (or lower a character) 
through sheer strength. You can lift double your maximum load in this manner.

A halfling has a +2 racial bonus on Climb checks because halflings are agile and 
surefooted.

The master of a lizard familiar gains a +3 bonus on Climb checks.

If you have the Athletic feat, you get a +2 bonus on Climb checks.

A creature with a climb speed has a +8 racial bonus on all Climb checks. The creature 
must make a Climb check to climb any wall or slope with a DC higher than 0, but 
it always can choose to take 10, even if rushed or threatened while climbing. If 
a creature with a climb speed chooses an accelerated climb (see above), it moves 
at double its climb speed (or at its land speed, whichever is slower) and makes 
a single Climb check at a -5 penalty. Such a creature retains its Dexterity bonus 
to Armor Class (if any) while climbing, and opponents get no special bonus to their 
attacks against it. It cannot, however, use the run action while climbing.

\textbf{Synergy:} If you have 5 or more ranks in Use Rope, you get a +2 bonus on 
Climb checks made to climb a rope, a knotted rope, or a rope-and-wall combination.

\vspace{12pt}
CONCENTRATION (CON)

\textbf{Check:} You must make a Concentration check whenever you might potentially 
be distracted (by taking damage, by harsh weather, and so on) while engaged in 
some action that requires your full attention. Such actions include casting a spell, 
concentrating on an active spell, directing a spell, using a spell-like ability, 
or using a skill that would provoke an attack of opportunity. In general, if an 
action wouldn't normally provoke an attack of opportunity, you need not make a 
Concentration check to avoid being distracted.

If the Concentration check succeeds, you may continue with the action as normal. 
If the check fails, the action automatically fails and is wasted. If you were in 
the process of casting a spell, the spell is lost. If you were concentrating on 
an active spell, the spell ends as if you had ceased concentrating on it. If you 
were directing a spell, the direction fails but the spell remains active. If you 
were using a spell-like ability, that use of the ability is lost. A skill use also 
fails, and in some cases a failed skill check may have other ramifications as well.

The table below summarizes various types of distractions that cause you to make 
a Concentration check. If the distraction occurs while you are trying to cast a 
spell, you must add the level of the spell you are trying to cast to the appropriate 
Concentration DC. If more than one type of distraction is present, make a check 
for each one; any failed Concentration check indicates that the task is not completed.

\vspace{12pt}
\begin{tabular}{|>{\raggedright}p{76pt}|>{\raggedright}p{249pt}|}
\hline
C\textbf{oncentration DC}\textsuperscript{\textbf{1}}\textbf{ } & D\textbf{istraction}\tabularnewline
\hline
10 + damage dealt  & Damaged during the action.\textsuperscript{\textbf{2}}\tabularnewline
\hline
10 + half of continuous  & Taking continuous damage during the damage last dealt 
action.\textsuperscript{\textbf{3}}\tabularnewline
\hline
Distracting spell's save DC  & Distracted by nondamaging spell.\textsuperscript{\textbf{4}}\tabularnewline
\hline
10  & Vigorous motion (on a moving mount, taking a bouncy wagon ride, in a small 
boat in rough water, belowdecks in a stormtossed ship).\tabularnewline
\hline
15  & Violent motion (on a galloping horse, taking a very rough wagon ride, in 
a small boat in rapids, on the deck of a storm-tossed ship).\tabularnewline
\hline
20  & Extraordinarily violent motion (earthquake\textit{)}.\tabularnewline
\hline
15  & Entangled.\tabularnewline
\hline
20  & Grappling or pinned. (You can cast only spells without somatic components 
for which you have any required material component in hand.)\tabularnewline
\hline
5  & Weather is a high wind carrying blinding rain or sleet.\tabularnewline
\hline
10  & Weather is wind-driven hail, dust, or debris.\tabularnewline
\hline
Distracting spell's save DC  & Weather caused by a spell, such as \textit{storm 
of vengeance.}\textsuperscript{\textbf{4}}\tabularnewline
\hline
\multicolumn{2}{|p{326pt}|}{1 If you are trying to cast, concentrate on, or direct 
a spell when the distraction occurs, add the level of the spell to the indicated 
DC.}\tabularnewline
\hline
\multicolumn{2}{|p{326pt}|}{2 Such as during the casting of a spell with a casting 
time of 1 round or more, or the execution of an activity that takes more than a 
single full-round action (such as Disable Device). Also, damage stemming from an 
attack of opportunity or readied attack made in response to the spell being cast 
(for spells with a casting time of 1 action) or the action being taken (for activities 
requiring no more than a full-round action).}\tabularnewline
\hline
\multicolumn{2}{|p{326pt}|}{3 Such as from \textit{acid arrow}.}\tabularnewline
\hline
\multicolumn{2}{|p{326pt}|}{4 If the spell allows no save, use the save DC it would 
have if it did allow a save.}\tabularnewline
\hline
\end{tabular}

\vspace{12pt}
\textbf{Action:} None. Making a Concentration check doesn't take an action; it 
is either a free action (when attempted reactively) or part of another action (when 
attempted actively).

\textbf{Try Again:} Yes, though a success doesn't cancel the effect of a previous 
failure, such as the loss of a spell you were casting or the disruption of a spell 
you were concentrating on.

\textbf{Special:} You can use Concentration to cast a spell, use a spell-like ability, 
or use a skill defensively, so as to avoid attacks of opportunity altogether. This 
doesn't apply to other actions that might provoke attacks of opportunity.

The DC of the check is 15 (plus the spell's level, if casting a spell or using 
a spell-like ability defensively). If the Concentration check succeeds, you may 
attempt the action normally without provoking any attacks of opportunity. A successful 
Concentration check still doesn't allow you to take 10 on another check if you 
are in a stressful situation; you must make the check normally. If the Concentration 
check fails, the related action also automatically fails (with any appropriate 
ramifications), and the action is wasted, just as if your concentration had been 
disrupted by a distraction. 

A character with the Combat Casting feat gets a +4 bonus on Concentration checks 
made to cast a spell or use a spell-like ability while on the defensive or while 
grappling or pinned.

\vspace{12pt}
CRAFT (INT)

Like Knowledge, Perform, and Profession, Craft is actually a number of separate 
skills. You could have several Craft skills, each with its own ranks, each purchased 
as a separate skill.

A Craft skill is specifically focused on creating something. If nothing is created 
by the endeavor, it probably falls under the heading of a Profession skill.

\textbf{Check:} You can practice your trade and make a decent living, earning about 
half your check result in gold pieces per week of dedicated work. You know how 
to use the tools of your trade, how to perform the craft's daily tasks, how to 
supervise untrained helpers, and how to handle common problems. (Untrained laborers 
and assistants earn an average of 1 silver piece per day.)

The basic function of the Craft skill, however, is to allow you to make an item 
of the appropriate type. The DC depends on the complexity of the item to be created. 
The DC, your check results, and the price of the item determine how long it takes 
to make a particular item. The item's finished price also determines the cost of 
raw materials.

In some cases, the \textit{fabricate }spell can be used to achieve the results 
of a Craft check with no actual check involved. However, you must make an appropriate 
Craft check when using the spell to make articles requiring a high degree of craftsmanship.

A successful Craft check related to woodworking in conjunction with the casting 
of the \textit{ironwood }spell enables you to make wooden items that have the strength 
of steel.

When casting the spell \textit{minor creation}, you must succeed on an appropriate 
Craft check to make a complex item.

All crafts require artisan's tools to give the best chance of success. If improvised 
tools are used, the check is made with a -2 circumstance penalty. On the other 
hand, masterwork artisan's tools provide a +2 circumstance bonus on the check.

To determine how much time and money it takes to make an item, follow these steps.

1. Find the item's price. Put the price in silver pieces (1 gp = 10 sp).

2. Find the DC from the table below.

3. Pay one-third of the item's price for the cost of raw materials.

4. Make an appropriate Craft check representing one week's work. If the check succeeds, 
multiply your check result by the DC. If the result \ensuremath{\times} the DC 
equals the price of the item in sp, then you have completed the item. (If the result 
\ensuremath{\times} the DC equals double or triple the price of the item in silver 
pieces, then you've completed the task in one-half or one-third of the time. Other 
multiples of the DC reduce the time in the same manner.) If the result \ensuremath{\times} 
the DC doesn't equal the price, then it represents the progress you've made this 
week. Record the result and make a new Craft check for the next week. Each week, 
you make more progress until your total reaches the price of the item in silver 
pieces.

If you fail a check by 4 or less, you make no progress this week.

If you fail by 5 or more, you ruin half the raw materials and have to pay half 
the original raw material cost again.

\textit{Progress by the Day: }You can make checks by the day instead of by the 
week. In this case your progress (check result \ensuremath{\times} DC) is in copper 
pieces instead of silver pieces.

\textit{Creating Masterwork Items: }You can make a masterwork item---a weapon, 
suit of armor, shield, or tool that conveys a bonus on its use through its exceptional 
craftsmanship, not through being magical. To create a masterwork item, you create 
the masterwork component as if it were a separate item in addition to the standard 
item. The masterwork component has its own price (300 gp for a weapon or 150 gp 
for a suit of armor or a shield) and a Craft DC of 20. Once both the standard component 
and the masterwork component are completed, the masterwork item is finished. \textit{Note: 
}The cost you pay for the masterwork component is one-third of the given amount, 
just as it is for the cost in raw materials.

\textit{Repairing Items: }Generally, you can repair an item by making checks against 
the same DC that it took to make the item in the first place. The cost of repairing 
an item is one-fifth of the item's price. 

\vspace{12pt}
When you use the Craft skill to make a particular sort of item, the DC for checks 
involving the creation of that item are typically as given on the following table.

\vspace{12pt}
\begin{tabular}{|>{\raggedright}p{207pt}|>{\raggedright}p{55pt}|>{\raggedright}p{51pt}|}
\hline
\subsection*{I\textbf{tem }} & \subsection*{C\textbf{raft Skill }} & \subsection*{C\textbf{raft 
DC}}\tabularnewline
\hline
Acid  & Alchemy\textsuperscript{\textbf{1}}\textbf{ } & 15\tabularnewline
\hline
Alchemist's fire, smokestick, or tindertwig  & Alchemy\textsuperscript{\textbf{1}}\textbf{ 
} & 20\tabularnewline
\hline
Antitoxin, sunrod, tanglefoot bag, or thunderstone  & Alchemy\textsuperscript{\textbf{1}}\textbf{ 
} & 25\tabularnewline
\hline
Armor or shield  & Armorsmithing  & 10 + AC bonus\tabularnewline
\hline
Longbow or shortbow  & Bowmaking  & 12\tabularnewline
\hline
Composite longbow or composite shortbow & Bowmaking  & 15\tabularnewline
\hline
Composite longbow or composite shortbow with high strength rating & Bowmaking  & 15 
+ (2 \ensuremath{\times} rating)\tabularnewline
\hline
Crossbow  & Weaponsmithing  & 15\tabularnewline
\hline
Simple melee or thrown weapon  & Weaponsmithing  & 12\tabularnewline
\hline
Martial melee or thrown weapon  & Weaponsmithing  & 15\tabularnewline
\hline
Exotic melee or thrown weapon  & Weaponsmithing  & 18\tabularnewline
\hline
Mechanical trap  & Trapmaking  & Varies\textsuperscript{\textbf{2}}\tabularnewline
\hline
Very simple item (wooden spoon)  & Varies & 5\tabularnewline
\hline
Typical item (iron pot)  & Varies  & 10\tabularnewline
\hline
High-quality item (bell) & Varies  & 15\tabularnewline
\hline
Complex or superior item (lock)  & Varies  & 20\tabularnewline
\hline
\multicolumn{3}{|p{314pt}|}{1 You must be a spellcaster to craft any of these items.}\tabularnewline
\hline
\multicolumn{3}{|p{314pt}|}{2 Traps have their own rules for construction.}\tabularnewline
\hline
\end{tabular}

\vspace{12pt}
\textbf{Action:} Does not apply. Craft checks are made by the day or week (see 
above).

\textbf{Try Again:} Yes, but each time you miss by 5 or more, you ruin half the 
raw materials and have to pay half the original raw material cost again.

\textbf{Special:} A dwarf has a +2 racial bonus on Craft checks that are related 
to stone or metal, because dwarves are especially capable with stonework and metalwork.

A gnome has a +2 racial bonus on Craft (alchemy) checks because gnomes have sensitive 
noses.

You may voluntarily add +10 to the indicated DC to craft an item. This allows you 
to create the item more quickly (since you'll be multiplying this higher DC by 
your Craft check result to determine progress). You must decide whether to increase 
the DC before you make each weekly or daily check.

To make an item using Craft (alchemy), you must have alchemical equipment and be 
a spellcaster. If you are working in a city, you can buy what you need as part 
of the raw materials cost to make the item, but alchemical equipment is difficult 
or impossible to come by in some places. Purchasing and maintaining an alchemist's 
lab grants a +2 circumstance bonus on Craft (alchemy) checks because you have the 
perfect tools for the job, but it does not affect the cost of any items made using 
the skill.

\textbf{Synergy:} If you have 5 ranks in a Craft skill, you get a +2 bonus on Appraise 
checks related to items made with that Craft skill.

\vspace{12pt}
DECIPHER SCRIPT (INT; TRAINED ONLY)

\textbf{Check:} You can decipher writing in an unfamiliar language or a message 
written in an incomplete or archaic form. The base DC is 20 for the simplest messages, 
25 for standard texts, and 30 or higher for intricate, exotic, or very old writing.

If the check succeeds, you understand the general content of a piece of writing 
about one page long (or the equivalent). If the check fails, make a DC 5 Wisdom 
check to see if you avoid drawing a false conclusion about the text. (Success means 
that you do not draw a false conclusion; failure means that you do.)

Both the Decipher Script check and (if necessary) the Wisdom check are made secretly, 
so that you can't tell whether the conclusion you draw is true or false.

\textbf{Action:} Deciphering the equivalent of a single page of script takes 1 
minute (ten consecutive full-round actions).

\textbf{Try Again:} No.

\textbf{Special:} A character with the Diligent feat gets a +2 bonus on Decipher 
Script checks.

\textbf{Synergy:} If you have 5 or more ranks in Decipher Script, you get a +2 
bonus on Use Magic Device checks involving scrolls.

\vspace{12pt}
DIPLOMACY (CHA)

\textbf{Check:} You can change the attitudes of others (nonplayer characters) with 
a successful Diplomacy check; see the Influencing NPC Attitudes sidebar, below, 
for basic DCs. In negotiations, participants roll opposed Diplomacy checks, and 
the winner gains the advantage. Opposed checks also resolve situations when two 
advocates or diplomats plead opposite cases in a hearing before a third party.

\textbf{Action:} Changing others' attitudes with Diplomacy generally takes at least 
1 full minute (10 consecutive full-round actions). In some situations, this time 
requirement may greatly increase. A rushed Diplomacy check can be made as a full-round 
action, but you take a -10 penalty on the check.

\textbf{Try Again:} Optional, but not recommended because retries usually do not 
work. Even if the initial Diplomacy check succeeds, the other character can be 
persuaded only so far, and a retry may do more harm than good. If the initial check 
fails, the other character has probably become more firmly committed to his position, 
and a retry is futile.

\textbf{Special:} A half-elf has a +2 racial bonus on Diplomacy checks.

If you have the Negotiator feat, you get a +2 bonus on Diplomacy checks.

\textbf{Synergy:} If you have 5 or more ranks in Bluff, Knowledge (nobility and 
royalty), or Sense Motive, you get a +2 bonus on Diplomacy checks.

\vspace{12pt}
INFLUENCING NPC ATTITUDES

Use the table below to determine the effectiveness of Diplomacy checks (or Charisma 
checks) made to influence the attitude of a nonplayer character, or wild empathy 
checks made to influence the attitude of an animal or magical beast.

\vspace{12pt}
\begin{tabular}{|>{\raggedright}p{62pt}|>{\raggedright}p{49pt}|>{\raggedright}p{45pt}|>{\raggedright}p{45pt}|>{\raggedright}p{44pt}|>{\raggedright}p{31pt}|}
\hline
I\textbf{nitial Attitude}--------------- & \multicolumn{5}{p{215pt}|}{ \textbf{New 
Attitude (DC to achieve)---------------}}\tabularnewline
\hline
  & H\textbf{ostile} & U\textbf{nfriendly} & I\textbf{ndifferent} & F\textbf{riendly} & H\textbf{elpful}\tabularnewline
\hline
Hostile & Less than 20 & 20 & 25 & 35 & 50\tabularnewline
\hline
Unfriendly & Less than 5 & 5 & 15 & 25 & 40\tabularnewline
\hline
Indifferent--- &  & Less than 1 & 1 & 15 & 30\tabularnewline
\hline
Friendly--- & --- &  & Less than 1 & 1 & 20\tabularnewline
\hline
Helpful--- & --- & --- &  & Less than 1 & 1\tabularnewline
\hline
\end{tabular}

\vspace{12pt}
\begin{tabular}{|>{\raggedright}p{39pt}|>{\raggedright}p{95pt}|>{\raggedright}p{179pt}|}
\hline
A\textbf{ttitude } & M\textbf{eans } & P\textbf{ossible Actions}\tabularnewline
\hline
Hostile  & Will take risks to hurt you  & Attack, interfere, berate, flee\tabularnewline
\hline
Unfriendly  & Wishes you ill  & Mislead, gossip, avoid, watch suspiciously, insult\tabularnewline
\hline
Indifferent  & Doesn't much care  & Socially expected interaction\tabularnewline
\hline
Friendly  & Wishes you well  & Chat, advise, offer limited help, advocate\tabularnewline
\hline
Helpful  & Will take risks to help you  & Protect, back up, heal, aid\tabularnewline
\hline
\end{tabular}

\vspace{12pt}
DISABLE DEVICE (INT; TRAINED ONLY)

\textbf{Check:} The Disable Device check is made secretly, so that you don't necessarily 
know whether you've succeeded.

The DC depends on how tricky the device is. Disabling (or rigging or jamming) a 
fairly simple device has a DC of 10; more intricate and complex devices have higher 
DCs.

If the check succeeds, you disable the device. If it fails by 4 or less, you have 
failed but can try again. If you fail by 5 or more, something goes wrong. If the 
device is a trap, you spring it. If you're attempting some sort of sabotage, you 
think the device is disabled, but it still works normally.

You also can rig simple devices such as saddles or wagon wheels to work normally 
for a while and then fail or fall off some time later (usually after 1d4 rounds 
or minutes of use).

\vspace{12pt}
\begin{tabular}{|>{\raggedright}p{25pt}|>{\raggedright}p{31pt}|>{\raggedright}p{64pt}|>{\raggedright}p{180pt}|}
\hline
D\textbf{evice } & T\textbf{ime } & D\textbf{isable Device DC}\textsuperscript{\textbf{1}} & E\textbf{xample}\tabularnewline
\hline
Simple & 1 round & 10 & Jam a lock\tabularnewline
\hline
Tricky & 1d4 rounds & 15 & Sabotage a wagon wheel\tabularnewline
\hline
Difficult & 2d4 rounds & 20 & Disarm a trap, reset a trap\tabularnewline
\hline
Wicked & 2d4 rounds & 25 & Disarm a complex trap, cleverly sabotage a clockwork 
device\tabularnewline
\hline
\multicolumn{4}{|p{302pt}|}{1If you attempt to leave behind no trace of your tampering, 
add 5 to the DC.}\tabularnewline
\hline
\end{tabular}

\vspace{12pt}
\textbf{Action:} The amount of time needed to make a Disable Device check depends 
on the task, as noted above. Disabling a simple device takes 1 round and is a full-round 
action. An intricate or complex device requires 1d4 or 2d4 rounds.

\textbf{Try Again:} Varies. You can retry if you have missed the check by 4 or 
less, though you must be aware that you have failed in order to try again.

\textbf{Special:} If you have the Nimble Fingers feat, you get a +2 bonus on Disable 
Device checks.

A rogue who beats a trap's DC by 10 or more can study the trap, figure out how 
it works, and bypass it (along with her companions) without disarming it.

\textbf{Restriction:} Rogues (and other characters with the trapfinding class feature) 
can disarm magic traps. A magic trap generally has a DC of 25 + the spell level 
of the magic used to create it.

The spells \textit{fire trap, glyph of warding, symbol, }and \textit{teleportation 
circle }also create traps that a rogue can disarm with a successful Disable Device 
check. \textit{Spike growth }and \textit{spike stones, }however, create magic traps 
against which Disable Device checks do not succeed. See the individual spell descriptions 
for details.

\vspace{12pt}
OTHER WAYS TO BEAT A TRAP

It's possible to ruin many traps without making a Disable Device check.

\textbf{Ranged Attack Traps: }Once a trap's location is known, the obvious way 
to ruin it is to smash the mechanism---assuming the mechanism can be accessed. 
Failing that, it's possible to plug up the holes from which the projectiles emerge. 
Doing this prevents the trap from firing unless its ammunition does enough damage 
to break through the plugs.

\textbf{Melee Attack Traps: }These devices can be thwarted by smashing the mechanism 
or blocking the weapons, as noted above. Alternatively, if a character studies 
the trap as it triggers, he might be able to time his dodges just right to avoid 
damage. A character who is doing nothing but studying a trap when it first goes 
off gains a +4 dodge bonus against its attacks if it is triggered again within 
the next minute.

\textbf{Pits: }Disabling a pit trap generally ruins only the trapdoor, making it 
an uncovered pit. Filling in the pit or building a makeshift bridge across it is 
an application of manual labor, not the Disable Device skill. Characters could 
neutralize any spikes at the bottom of a pit by attacking them---they break just 
as daggers do.

\textbf{Magic Traps: }\textit{Dispel magic }helps here. Someone who succeeds on 
a caster level check against the level of the trap's creator suppresses the trap 
for 1d4 rounds. This works only with a targeted \textit{dispel magic, }not the 
area version (see the spell description).

\vspace{12pt}
DISGUISE (CHA)

\textbf{Check:} Your Disguise check result determines how good the disguise is, 
and it is opposed by others' Spot check results. If you don't draw any attention 
to yourself, others do not get to make Spot checks. If you come to the attention 
of people who are suspicious (such as a guard who is watching commoners walking 
through a city gate), it can be assumed that such observers are taking 10 on their 
Spot checks.

You get only one Disguise check per use of the skill, even if several people are 
making Spot checks against it. The Disguise check is made secretly, so that you 
can't be sure how good the result is.

The effectiveness of your disguise depends in part on how much you're attempting 
to change your appearance.

\vspace{12pt}
\begin{tabular}{|>{\raggedright}p{155pt}|>{\raggedright}p{78pt}|}
\hline
D\textbf{isguise } & \subsubsection*{D\textbf{isguise Check Modifier}}\tabularnewline
\hline
Minor details only  & +5\tabularnewline
\hline
Disguised as different gender\textsuperscript{\textbf{1}}\textbf{ }- & 2\tabularnewline
\hline
Disguised as different race\textsuperscript{\textbf{1}}\textbf{ }- & 2\tabularnewline
\hline
Disguised as different age category\textsuperscript{\textbf{1}}- & 2\textsuperscript{\textbf{2}}\tabularnewline
\hline
\multicolumn{2}{|p{233pt}|}{1These modifiers are cumulative; use any that apply.}\tabularnewline
\hline
\multicolumn{2}{|p{233pt}|}{2Per step of difference between your actual age category 
and your disguised age category. The steps are: young (younger than adulthood), 
adulthood, middle age, old, and venerable.}\tabularnewline
\hline
\end{tabular}

\vspace{12pt}
If you are impersonating a particular individual, those who know what that person 
looks like get a bonus on their Spot checks according to the table below. Furthermore, 
they are automatically considered to be suspicious of you, so opposed checks are 
always called for.

\vspace{12pt}
\begin{tabular}{|>{\raggedright}p{82pt}|>{\raggedright}p{60pt}|}
\hline
F\textbf{amiliarity } & V\textbf{iewer's Spot Check Bonus}\tabularnewline
\hline
Recognizes on sight  & +4\tabularnewline
\hline
Friends or associates  & +6\tabularnewline
\hline
Close friends  & +8\tabularnewline
\hline
Intimate  & +10\tabularnewline
\hline
\end{tabular}

\vspace{12pt}
Usually, an individual makes a Spot check to see through your disguise immediately 
upon meeting you and each hour thereafter. If you casually meet many different 
creatures, each for a short time, check once per day or hour, using an average 
Spot modifier for the group. 

\textbf{Action:} Creating a disguise requires 1d3\ensuremath{\times}10 minutes 
of work.

\textbf{Try Again:} Yes. You may try to redo a failed disguise, but once others 
know that a disguise was attempted, they'll be more suspicious.

\textbf{Special:} Magic that alters your form, such as \textit{alter self, disguise 
self, polymorph}, or \textit{shapechange, }grants you a +10 bonus on Disguise checks 
(see the individual spell descriptions). You must succeed on a Disguise check with 
a +10 bonus to duplicate the appearance of a specific individual using the \textit{veil 
}spell. Divination magic that allows people to see through illusions (such as \textit{true 
seeing}) does not penetrate a mundane disguise, but it can negate the magical component 
of a magically enhanced one.

You must make a Disguise check when you cast a \textit{simulacrum }spell to determine 
how good the likeness is.

If you have the Deceitful feat, you get a +2 bonus on Disguise checks.

\textbf{Synergy:} If you have 5 or more ranks in Bluff, you get a +2 bonus on Disguise 
checks when you know that you're being observed and you try to act in character.

\vspace{12pt}
ESCAPE ARTIST (DEX; ARMOR CHECK PENALTY)

\textbf{Check:} The table below gives the DCs to escape various forms of restraints.

\textit{Ropes: }Your Escape Artist check is opposed by the binder's Use Rope check. 
Since it's easier to tie someone up than to escape from being tied up, the binder 
gets a +10 bonus on his or her check.

\textit{Manacles and Masterwork Manacles: }The DC for manacles is set by their 
construction.

\textit{Tight Space: }The DC noted on the table is for getting through a space 
where your head fits but your shoulders don't. If the space is long you may need 
to make multiple checks. You can't get through a space that your head does not 
fit through.

\textit{Grappler: }You can make an Escape Artist check opposed by your enemy's 
grapple check to get out of a grapple or out of a pinned condition (so that you're 
only grappling).

\vspace{12pt}
\begin{tabular}{|>{\raggedright}p{173pt}|>{\raggedright}p{132pt}|}
\hline
R\textbf{estraint } & E\textbf{scape Artist DC}\tabularnewline
\hline
Ropes Binder's  & Use Rope check at +10\tabularnewline
\hline
Net, \textit{animate rope }spell, \textit{command plants }spell, \textit{control 
plants }spell, or \textit{entangle }spell & 20\tabularnewline
\hline
Snare spell  & 23\tabularnewline
\hline
Manacles  & 30\tabularnewline
\hline
Tight space  & 30\tabularnewline
\hline
Masterwork manacles  & 35\tabularnewline
\hline
Grappler  & Grappler's grapple check result\tabularnewline
\hline
\end{tabular}

\vspace{12pt}
\textbf{Action:} Making an Escape Artist check to escape from rope bindings, manacles, 
or other restraints (except a grappler) requires 1 minute of work. Escaping from 
a net or an \textit{animate rope, command plants, control plants, }or \textit{entangle 
}spell is a full-round action. Escaping from a grapple or pin is a standard action. 
Squeezing through a tight space takes at least 1 minute, maybe longer, depending 
on how long the space is.

\textbf{Try Again:} Varies. You can make another check after a failed check if 
you're squeezing your way through a tight space, making multiple checks. If the 
situation permits, you can make additional checks, or even take 20, as long as 
you're not being actively opposed.

\textbf{Special:} If you have the Agile feat, you get a +2 bonus on Escape Artist 
checks.

\textbf{Synergy:} If you have 5 or more ranks in Escape Artist, you get a +2 bonus 
on Use Rope checks to bind someone.

If you have 5 or more ranks in Use Rope, you get a +2 bonus on Escape Artist checks 
when escaping from rope bonds.

\vspace{12pt}
FORGERY (INT)

\textbf{Check:} Forgery requires writing materials appropriate to the document 
being forged, enough light or sufficient visual acuity to see the details of what 
you're writing, wax for seals (if appropriate), and some time. To forge a document 
on which the handwriting is not specific to a person (military orders, a government 
decree, a business ledger, or the like), you need only to have seen a similar document 
before, and you gain a +8 bonus on your check. To forge a signature, you need an 
autograph of that person to copy, and you gain a +4 bonus on the check. To forge 
a longer document written in the hand of some particular person, a large sample 
of that person's handwriting is needed.

The Forgery check is made secretly, so that you're not sure how good your forgery 
is. As with Disguise, you don't even need to make a check until someone examines 
the work. Your Forgery check is opposed by the Forgery check of the person who 
examines the document to check its authenticity. The examiner gains modifiers on 
his or her check if any of the conditions on the table below exist.

\vspace{12pt}
\begin{tabular}{|>{\raggedright}p{182pt}|>{\raggedright}p{77pt}|}
\hline
C\textbf{ondition } & R\textbf{eader's Forgery Check Modifier}\tabularnewline
\hline
Type of document unknown to reader - & 2\tabularnewline
\hline
Type of document somewhat known to reader  & +0\tabularnewline
\hline
Type of document well known to reader  & +2\tabularnewline
\hline
Handwriting not known to reader - & 2\tabularnewline
\hline
Handwriting somewhat known to reader  & +0\tabularnewline
\hline
Handwriting intimately known to reader  & +2\tabularnewline
\hline
Reader only casually reviews the document - & 2\tabularnewline
\hline
\end{tabular}

\vspace{12pt}
A document that contradicts procedure, orders, or previous knowledge, or one that 
requires sacrifice on the part of the person checking the document can increase 
that character's suspicion (and thus create favorable circumstances for the checker's 
opposing Forgery check).

\textbf{Action:} Forging a very short and simple document takes about 1 minute. 
A longer or more complex document takes 1d4 minutes per page.

\textbf{Try Again:} Usually, no. A retry is never possible after a particular reader 
detects a particular forgery. But the document created by the forger might still 
fool someone else. The result of a Forgery check for a particular document must 
be used for every instance of a different reader examining the document. No reader 
can attempt to detect a particular forgery more than once; if that one opposed 
check goes in favor of the forger, then the reader can't try using his own skill 
again, even if he's suspicious about the document.

\textbf{Special:} If you have the Deceitful feat, you get a +2 bonus on Forgery 
checks.

\textbf{Restriction:} Forgery is language-dependent; thus, to forge documents and 
detect forgeries, you must be able to read and write the language in question. 
A barbarian can't learn the Forgery skill unless he has learned to read and write.

\vspace{12pt}
GATHER INFORMATION (CHA)

\textbf{Check:} An evening's time, a few gold pieces for buying drinks and making 
friends, and a DC 10 Gather Information check get you a general idea of a city's 
major news items, assuming there are no obvious reasons why the information would 
be withheld. The higher your check result, the better the information.

If you want to find out about a specific rumor, or a specific item, or obtain a 
map, or do something else along those lines, the DC for the check is 15 to 25, 
or even higher.

\textbf{Action:} A typical Gather Information check takes 1d4+1 hours.

\textbf{Try Again:} Yes, but it takes time for each check. Furthermore, you may 
draw attention to yourself if you repeatedly pursue a certain type of information.

\textbf{Special:} A half-elf has a +2 racial bonus on Gather Information checks.

If you have the Investigator feat, you get a +2 bonus on Gather Information checks.

\textbf{Synergy:} If you have 5 or more ranks in Knowledge (local), you get a +2 
bonus on Gather Information checks.

\vspace{12pt}
HANDLE ANIMAL (CHA; TRAINED ONLY)

\textbf{Check:} The DC depends on what you are trying to do.

\vspace{12pt}
\begin{tabular}{|>{\raggedright}p{156pt}|>{\raggedright}p{86pt}|}
\hline
\subsection*{T\textbf{ask }} & \subsection*{H\textbf{andle Animal DC}}\tabularnewline
\hline
Handle an animal  & 10\tabularnewline
\hline
``Push'' an animal  & 25\tabularnewline
\hline
Teach an animal a trick  & 15 or 20\textsuperscript{\textbf{1}}\tabularnewline
\hline
Train an animal for a general purpose  & 15 or 20\textsuperscript{\textbf{1}}\tabularnewline
\hline
Rear a wild animal  & 15 + HD of animal\tabularnewline
\hline
\multicolumn{2}{|p{242pt}|}{1See the specific trick or purpose below.}\tabularnewline
\hline
\end{tabular}

\vspace{12pt}
\begin{tabular}{|>{\raggedright}p{78pt}|>{\raggedright}p{20pt}|>{\raggedright}p{78pt}|>{\raggedright}p{18pt}|}
\hline
G\textbf{eneral Purpose } & D\textbf{C } & G\textbf{eneral Purpose } & D\textbf{C}\tabularnewline
\hline
Combat riding  & 20  & Hunting  & 20\tabularnewline
\hline
Fighting  & 20  & Performance  & 15\tabularnewline
\hline
Guarding  & 20  & Riding  & 15\tabularnewline
\hline
Heavy labor  & 15 & \multicolumn{2}{p{96pt}|}{}\tabularnewline
\hline
\end{tabular}

\vspace{12pt}
\textit{Handle an Animal: }This task involves commanding an animal to perform a 
task or trick that it knows. If the animal is wounded or has taken any nonlethal 
damage or ability score damage, the DC increases by 2. If your check succeeds, 
the animal performs the task or trick on its next action.

``\textit{Push'' an Animal: }To push an animal means to get it to perform a task 
or trick that it doesn't know but is physically capable of performing. This category 
also covers making an animal perform a forced march or forcing it to hustle for 
more than 1 hour between sleep cycles. If the animal is wounded or has taken any 
nonlethal damage or ability score damage, the DC increases by 2. If your check 
succeeds, the animal performs the task or trick on its next action.

\textit{Teach an Animal a Trick: }You can teach an animal a specific trick with 
one week of work and a successful Handle Animal check against the indicated DC. 
An animal with an Intelligence score of 1 can learn a maximum of three tricks, 
while an animal with an Intelligence score of 2 can learn a maximum of six tricks. 
Possible tricks (and their associated DCs) include, but are not necessarily limited 
to, the following.

Attack (DC 20): The animal attacks apparent enemies. You may point to a particular 
creature that you wish the animal to attack, and it will comply if able. Normally, 
an animal will attack only humanoids, monstrous humanoids, giants, or other animals. 
Teaching an animal to attack all creatures (including such unnatural creatures 
as undead and aberrations) counts as two tricks.

Come (DC 15): The animal comes to you, even if it normally would not do so.

Defend (DC 20): The animal defends you (or is ready to defend you if no threat 
is present), even without any command being given. Alternatively, you can command 
the animal to defend a specific other character.

Down (DC 15): The animal breaks off from combat or otherwise backs down. An animal 
that doesn't know this trick continues to fight until it must flee (due to injury, 
a fear effect, or the like) or its opponent is defeated.

Fetch (DC 15): The animal goes and gets something. If you do not point out a specific 
item, the animal fetches some random object.

Guard (DC 20): The animal stays in place and prevents others from approaching.

Heel (DC 15): The animal follows you closely, even to places where it normally 
wouldn't go.

Perform (DC 15): The animal performs a variety of simple tricks, such as sitting 
up, rolling over, roaring or barking, and so on.

Seek (DC 15): The animal moves into an area and looks around for anything that 
is obviously alive or animate.

Stay (DC 15): The animal stays in place, waiting for you to return. It does not 
challenge other creatures that come by,

though it still defends itself if it needs to.

Track (DC 20): The animal tracks the scent presented to it. (This requires the 
animal to have the scent ability)

Work (DC 15): The animal pulls or pushes a medium or heavy load.

\vspace{12pt}
\textit{Train an Animal for a Purpose: }Rather than teaching an animal individual 
tricks, you can simply train it for a general purpose. Essentially, an animal's 
purpose represents a preselected set of known tricks that fit into a common scheme, 
such as guarding or heavy labor. The animal must meet all the normal prerequisites 
for all tricks included in the training package. If the package includes more than 
three tricks, the animal must have an Intelligence score of 2.

An animal can be trained for only one general purpose, though if the creature is 
capable of learning additional tricks (above and beyond those included in its general 
purpose), it may do so. Training an animal for a purpose requires fewer checks 
than teaching individual tricks does, but no less time. 

Combat Riding (DC 20): An animal trained to bear a rider into combat knows the 
tricks attack, come, defend, down, guard, and heel. Training an animal for combat 
riding takes six weeks. You may also ``upgrade'' an animal trained for riding to 
one trained for combat riding by spending three weeks and making a successful DC 
20 Handle Animal check. The new general purpose and tricks completely replace the 
animal's previous purpose and any tricks it once knew. Warhorses and riding dogs 
are already trained to bear riders into combat, and they don't require any additional 
training for this purpose.

Fighting (DC 20): An animal trained to engage in combat knows the tricks attack, 
down, and stay. Training an animal for fighting takes three weeks.

Guarding (DC 20): An animal trained to guard knows the tricks attack, defend, down, 
and guard. Training an animal for guarding takes four weeks.

Heavy Labor (DC 15): An animal trained for heavy labor knows the tricks come and 
work. Training an animal for heavy labor takes two weeks.

Hunting (DC 20): An animal trained for hunting knows the tricks attack, down, fetch, 
heel, seek, and track. Training an animal for hunting takes six weeks.

Performance (DC 15): An animal trained for performance knows the tricks come, fetch, 
heel, perform, and stay. Training an animal for performance takes five weeks.

Riding (DC 15): An animal trained to bear a rider knows the tricks come, heel, 
and stay. Training an animal for riding takes three weeks.

\vspace{12pt}
\textit{Rear a Wild Animal: }To rear an animal means to raise a wild creature from 
infancy so that it becomes domesticated. A handler can rear as many as three creatures 
of the same kind at once.

A successfully domesticated animal can be taught tricks at the same time it's being 
raised, or it can be taught as a domesticated animal later.

\textbf{Action:} Varies. Handling an animal is a move action, while pushing an 
animal is a full-round action. (A druid or ranger can handle her animal companion 
as a free action or push it as a move action.) For tasks with specific time frames 
noted above, you must spend half this time (at the rate of 3 hours per day per 
animal being handled) working toward completion of the task before you attempt 
the Handle Animal check. If the check fails, your attempt to teach, rear, or train 
the animal fails and you need not complete the teaching, rearing, or training time. 
If the check succeeds, you must invest the remainder of the time to complete the 
teaching, rearing, or training. If the time is interrupted or the task is not followed 
through to completion, the attempt to teach, rear, or train the animal automatically 
fails.

\textbf{Try Again:} Yes, except for rearing an animal.

\textbf{Special:} You can use this skill on a creature with an Intelligence score 
of 1 or 2 that is not an animal, but the DC of any such check increases by 5. Such 
creatures have the same limit on tricks known as animals do.

A druid or ranger gains a +4 circumstance bonus on Handle Animal checks involving 
her animal companion.

In addition, a druid's or ranger's animal companion knows one or more bonus tricks, 
which don't count against the normal limit on tricks known and don't require any 
training time or Handle Animal checks to teach.

If you have the Animal Affinity feat, you get a +2 bonus on Handle Animal checks.

\textbf{Synergy:} If you have 5 or more ranks in Handle Animal, you get a +2 bonus 
on Ride checks and wild empathy checks.

\textbf{Untrained:} If you have no ranks in Handle Animal, you can use a Charisma 
check to handle and push domestic animals, but you can't teach, rear, or train 
animals. A druid or ranger with no ranks in Handle Animal can use a Charisma check 
to handle and push her animal companion, but she can't teach, rear, or train other 
nondomestic animals.

\newpage

\end{document}
