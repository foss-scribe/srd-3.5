%&pdfLaTeX
% !TEX encoding = UTF-8 Unicode
\documentclass{article}
\usepackage{ifxetex}
\ifxetex
\usepackage{fontspec}
\setmainfont[Mapping=tex-text]{STIXGeneral}
\else
\usepackage[T1]{fontenc}
\usepackage[utf8]{inputenc}
\fi
\usepackage{textcomp}

\usepackage{array}
\usepackage{amssymb}
\usepackage{fancyhdr}
\renewcommand{\headrulewidth}{0pt}
\renewcommand{\footrulewidth}{0pt}

\begin{document}

This material is Open Game Content, and is licensed for public use under the terms 
of the Open Game License v1.0a.

{\LARGE{}SKILLS II}

\vspace{12pt}
HEAL (WIS)

\textbf{Check:} The DC and effect depend on the task you attempt.

\vspace{12pt}
\begin{tabular}{|>{\raggedright}p{119pt}|>{\raggedright}p{78pt}|}
\hline
\subsection*{T\textbf{ask Heal }} & \subsection*{D\textbf{C}}\tabularnewline
\hline
First aid & 15\tabularnewline
\hline
Long-term care  & 15\tabularnewline
\hline
Treat wound from caltrop, \textit{spike growth, }or \textit{spike stones } & 15\tabularnewline
\hline
Treat poison  & Poison's save DC\tabularnewline
\hline
Treat disease  & Disease's save DC\tabularnewline
\hline
\end{tabular}

\vspace{12pt}
\textit{First Aid: }You usually use first aid to save a dying character. If a character 
has negative hit points and is losing hit points (at the rate of 1 per round, 1 
per hour, or 1 per day), you can make him or her stable. A stable character regains 
no hit points but stops losing them.

\textit{Long-Term Care: }Providing long-term care means treating a wounded person 
for a day or more. If your Heal check is successful, the patient recovers hit points 
or ability score points (lost to ability damage) at twice the normal rate: 2 hit 
points per level for a full 8 hours of rest in a day, or 4 hit points per level 
for each full day of complete rest; 2 ability score points for a full 8 hours of 
rest in a day, or 4 ability score points for each full day of complete rest.

You can tend as many as six patients at a time. You need a few items and supplies 
(bandages, salves, and so on) that are easy to come by in settled lands. Giving 
long-term care counts as light activity for the healer. You cannot give long-term 
care to yourself.

\textit{Treat Wound from Caltrop, Spike Growth, or Spike Stones: }A creature wounded 
by stepping on a caltrop moves at one-half normal speed. A successful Heal check 
removes this movement penalty.

A creature wounded by a \textit{spike growth }or \textit{spike stones }spell must 
succeed on a Reflex save or take injuries that reduce his speed by one-third. Another 
character can remove this penalty by taking 10 minutes to dress the victim's injuries 
and succeeding on a Heal check against the spell's save DC.

\textit{Treat Poison: }To treat poison means to tend a single character who has 
been poisoned and who is going to take more damage from the poison (or suffer some 
other effect). Every time the poisoned character makes a saving throw against the 
poison, you make a Heal check. The poisoned character uses your check result or 
his or her saving throw, whichever is higher.

\textit{Treat Disease: }To treat a disease means to tend a single diseased character. 
Every time he or she makes a saving throw against disease effects, you make a Heal 
check. The diseased character uses your check result or his or her saving throw, 
whichever is higher.

\textbf{Action:} Providing first aid, treating a wound, or treating poison is a 
standard action. Treating a disease or tending a creature wounded by a \textit{spike 
growth }or \textit{spike stones }spell takes 10 minutes of work. Providing long-term 
care requires 8 hours of light activity.

\textbf{Try Again:} Varies. Generally speaking, you can't try a Heal check again 
without proof of the original check's failure. You can always retry a check to 
provide first aid, assuming the target of the previous attempt is still alive.

\textbf{Special:} A character with the Self-Sufficient feat gets a +2 bonus on 
Heal checks.

A healer's kit gives you a +2 circumstance bonus on Heal checks.

\vspace{12pt}
HIDE (DEX; ARMOR CHECK PENALTY)

\textbf{Check:} Your Hide check is opposed by the Spot check of anyone who might 
see you. You can move up to one-half your normal speed and hide at no penalty. 
When moving at a speed greater than one-half but less than your normal speed, you 
take a -5 penalty. It's practically impossible (-20 penalty) to hide while attacking, 
running or charging.

A creature larger or smaller than Medium takes a size bonus or penalty on Hide 
checks depending on its size category: Fine +16, Diminutive +12, Tiny +8, Small 
+4, Large -4, Huge -8, Gargantuan -12, Colossal -16.

You need cover or concealment in order to attempt a Hide check. Total cover or 
total concealment usually (but not always; see Special, below) obviates the need 
for a Hide check, since nothing can see you anyway.

If people are observing you, even casually, you can't hide. You can run around 
a corner or behind cover so that you're out of sight and then hide, but the others 
then know at least where you went.

If your observers are momentarily distracted (such as by a Bluff check; see below), 
though, you can attempt to hide. While the others turn their attention from you, 
you can attempt a Hide check if you can get to a hiding place of some kind. (As 
a general guideline, the hiding place has to be within 1 foot per rank you have 
in Hide.) This check, however, is made at a -10 penalty because you have to move 
fast.

\textit{Sniping: }If you've already successfully hidden at least 10 feet from your 
target, you can make one ranged attack, then immediately hide again. You take a 
-20 penalty on your Hide check to conceal yourself after the shot.

\textit{Creating a Diversion to Hide: }You can use Bluff to help you hide. A successful 
Bluff check can give you the momentary diversion you need to attempt a Hide check 
while people are aware of you.

\textbf{Action:} Usually none. Normally, you make a Hide check as part of movement, 
so it doesn't take a separate action. However, hiding immediately after a ranged 
attack (see Sniping, above) is a move action.

\textbf{Special:} If you are invisible, you gain a +40 bonus on Hide checks if 
you are immobile, or a +20 bonus on Hide checks if you're moving.

If you have the Stealthy feat, you get a +2 bonus on Hide checks.

A 13th-level ranger can attempt a Hide check in any sort of natural terrain, even 
if it doesn't grant cover or concealment. A 17thlevel ranger can do this even while 
being observed.

\vspace{12pt}
INTIMIDATE (CHA)

\textbf{Check:} You can change another's behavior with a successful check. Your 
Intimidate check is opposed by the target's modified level check (1d20 + character 
level or Hit Dice + target's Wisdom bonus [if any] + target's modifiers on saves 
against fear). If you beat your target's check result, you may treat the target 
as friendly, but only for the purpose of actions taken while it remains intimidated. 
(That is, the target retains its normal attitude, but will chat, advise, offer 
limited help, or advocate on your behalf while intimidated. See the Diplomacy skill, 
above, for additional details.) The effect lasts as long as the target remains 
in your presence, and for 1d6\ensuremath{\times}10 minutes afterward. After this 
time, the target's default attitude toward you shifts to unfriendly (or, if normally 
unfriendly, to hostile).

If you fail the check by 5 or more, the target provides you with incorrect or useless 
information, or otherwise frustrates your efforts.

\textit{Demoralize Opponent: }You can also use Intimidate to weaken an opponent's 
resolve in combat. To do so, make an Intimidate check opposed by the target's modified 
level check (see above). If you win, the target becomes shaken for 1 round. A shaken 
character takes a -2 penalty on attack rolls, ability checks, and saving throws. 
You can intimidate only an opponent that you threaten in melee combat and that 
can see you.

\textbf{Action:} Varies. Changing another's behavior requires 1 minute of interaction. 
Intimidating an opponent in combat is a standard action.

\textbf{Try Again:} Optional, but not recommended because retries usually do not 
work. Even if the initial check succeeds, the other character can be intimidated 
only so far, and a retry doesn't help. If the initial check fails, the other character 
has probably become more firmly resolved to resist the intimidator, and a retry 
is futile.

\textbf{Special:} You gain a +4 bonus on your Intimidate check for every size category 
that you are larger than your target. Conversely, you take a -4 penalty on your 
Intimidate check for every size category that you are smaller than your target.

A character immune to fear can't be intimidated, nor can nonintelligent creatures.

If you have the Persuasive feat, you get a +2 bonus on Intimidate checks.

\textbf{Synergy:} If you have 5 or more ranks in Bluff, you get a +2 bonus on Intimidate 
checks.

\vspace{12pt}
JUMP (STR; ARMOR CHECK PENALTY)

\textbf{Check:} The DC and the distance you can cover vary according to the type 
of jump you are attempting (see below).

Your Jump check is modified by your speed. If your speed is 30 feet then no modifier 
based on speed applies to the check. If your speed is less than 30 feet, you take 
a -6 penalty for every 10 feet of speed less than 30 feet. If your speed is greater 
than 30 feet, you gain a +4 bonus for every 10 feet beyond 30 feet.

All Jump DCs given here assume that you get a running start, which requires that 
you move at least 20 feet in a straight line before attempting the jump. If you 
do not get a running start, the DC for the jump is doubled.

Distance moved by jumping is counted against your normal maximum movement in a 
round.

If you have ranks in Jump and you succeed on a Jump check, you land on your feet 
(when appropriate). If you attempt a Jump check untrained, you land prone unless 
you beat the DC by 5 or more.

\textit{Long Jump: }A long jump is a horizontal jump, made across a gap like a 
chasm or stream. At the midpoint of the jump, you attain a vertical height equal 
to one-quarter of the horizontal distance. The DC for the jump is equal to the 
distance jumped (in feet).

If your check succeeds, you land on your feet at the far end. If you fail the check 
by less than 5, you don't clear the distance, but you can make a DC 15 Reflex save 
to grab the far edge of the gap. You end your movement grasping the far edge. If 
that leaves you dangling over a chasm or gap, getting up requires a move action 
and a DC 15 Climb check.

\vspace{12pt}
\begin{tabular}{|>{\raggedright}p{87pt}|>{\raggedright}p{65pt}|}
\hline
L\textbf{ong Jump Distance} & J\textbf{ump DC}\textsuperscript{\textbf{1}}\tabularnewline
\hline
5 feet & 5\tabularnewline
\hline
10 feet & 10\tabularnewline
\hline
15 feet & 15\tabularnewline
\hline
20 feet & 20\tabularnewline
\hline
25 feet & 25\tabularnewline
\hline
30 feet & 30\tabularnewline
\hline
\multicolumn{2}{|p{152pt}|}{1 Requires a 20-foot running start. Without a running 
start, double the DC.}\tabularnewline
\hline
\end{tabular}

\vspace{12pt}
\textit{High Jump: }A high jump is a vertical leap made to reach a ledge high above 
or to grasp something overhead. The DC is equal to 4 times the distance to be cleared.

If you jumped up to grab something, a successful check indicates that you reached 
the desired height. If you wish to pull yourself up, you can do so with a move 
action and a DC 15 Climb check. If you fail the Jump check, you do not reach the 
height, and you land on your feet in the same spot from which you jumped. As with 
a long jump, the DC is doubled if you do not get a running start of at least 20 
feet.

\vspace{12pt}
\begin{tabular}{|>{\raggedright}p{101pt}|>{\raggedright}p{60pt}|}
\hline
H\textbf{igh Jump Distance}\textsuperscript{\textbf{1}}\textbf{ } & J\textbf{ump 
DC}\textsuperscript{\textbf{2}}\tabularnewline
\hline
1 foot  & 4\tabularnewline
\hline
2 feet  & 8\tabularnewline
\hline
3 feet  & 12\tabularnewline
\hline
4 feet  & 16\tabularnewline
\hline
5 feet  & 20\tabularnewline
\hline
6 feet  & 24\tabularnewline
\hline
7 feet  & 28\tabularnewline
\hline
8 feet  & 32\tabularnewline
\hline
\multicolumn{2}{|p{161pt}|}{1 Not including vertical reach; see below.}\tabularnewline
\hline
\multicolumn{2}{|p{161pt}|}{2 Requires a 20-foot running start. Without a running 
start, double the DC.}\tabularnewline
\hline
\end{tabular}

\vspace{12pt}
Obviously, the difficulty of reaching a given height varies according to the size 
of the character or creature. The maximum vertical reach (height the creature can 
reach without jumping) for an average creature of a given size is shown on the 
table below. (As a Medium creature, a typical human can reach 8 feet without jumping.)

Quadrupedal creatures don't have the same vertical reach as a bipedal creature; 
treat them as being one size category smaller.

\vspace{12pt}
\begin{tabular}{|>{\raggedright}p{57pt}|>{\raggedright}p{62pt}|}
\hline
\subsection*{C\textbf{reature Size }} & \subsection*{V\textbf{ertical Reach}}\tabularnewline
\hline
Colossal  & 128 ft.\tabularnewline
\hline
Gargantuan  & 64 ft.\tabularnewline
\hline
Huge  & 32 ft.\tabularnewline
\hline
Large  & 16 ft.\tabularnewline
\hline
Medium  & 8 ft.\tabularnewline
\hline
Small  & 4 ft.\tabularnewline
\hline
Tiny  & 2 ft.\tabularnewline
\hline
Diminutive  & 1 ft.\tabularnewline
\hline
Fine  & 1/2 ft.\tabularnewline
\hline
\end{tabular}

\vspace{12pt}
\textit{Hop Up: }You can jump up onto an object as tall as your waist, such as 
a table or small boulder, with a DC 10 Jump check. Doing so counts as 10 feet of 
movement, so if your speed is 30 feet, you could move 20 feet, then hop up onto 
a counter. You do not need to get a running start to hop up, so the DC is not doubled 
if you do not get a running start.

\textit{Jumping Down: }If you intentionally jump from a height, you take less damage 
than you would if you just fell. The DC to jump down from a height is 15. You do 
not have to get a running start to jump down, so the DC is not doubled if you do 
not get a running start.

If you succeed on the check, you take falling damage as if you had dropped 10 fewer 
feet than you actually did.

\textbf{Action:} None. A Jump check is included in your movement, so it is part 
of a move action. If you run out of movement mid-jump, your next action (either 
on this turn or, if necessary, on your next turn) must be a move action to complete 
the jump.

\textbf{Special:} Effects that increase your movement also increase your jumping 
distance, since your check is modified by your speed.

If you have the Run feat, you get a +4 bonus on Jump checks for any jumps made 
after a running start.

A halfling has a +2 racial bonus on Jump checks because halflings are agile and 
athletic.

If you have the Acrobatic feat, you get a +2 bonus on Jump checks.

\textbf{Synergy:} If you have 5 or more ranks in Tumble, you get a +2 bonus on 
Jump checks.

If you have 5 or more ranks in Jump, you get a +2 bonus on Tumble checks.

\vspace{12pt}
KNOWLEDGE (INT; TRAINED ONLY)

Like the Craft and Profession skills, Knowledge actually encompasses a number of 
unrelated skills. Knowledge represents a study of some body of lore, possibly an 
academic or even scientific discipline.

Below are listed typical fields of study.• 

\parindent=3pt
Arcana (ancient mysteries, magic traditions, arcane symbols, cryptic phrases, constructs, 
dragons, magical beasts)• 

Architecture and engineering (buildings, aqueducts, bridges, fortifications)• 

\parindent=7pt
Dungeoneering (aberrations, caverns, oozes, spelunking)• 

\parindent=3pt
Geography (lands, terrain, climate, people)• 

History (royalty, wars, colonies, migrations, founding of cities)• 

\parindent=7pt
Local (legends, personalities, inhabitants, laws, customs, traditions, humanoids)• 

\parindent=3pt
Nature (animals, fey, giants, monstrous humanoids, plants, seasons and cycles, 
weather, vermin)• 

Nobility and royalty (lineages, heraldry, family trees, mottoes, personalities)• 

\parindent=7pt
Religion (gods and goddesses, mythic history, ecclesiastic tradition, holy symbols, 
undead)• 

\parindent=3pt
The planes (the Inner Planes, the Outer Planes, the Astral Plane, the Ethereal 
Plane, outsiders, elementals, magic related to the planes)

\parindent=0pt
\textbf{Check:} Answering a question within your field of study has a DC of 10 
(for really easy questions), 15 (for basic questions), or 20 to 30 (for really 
tough questions).

In many cases, you can use this skill to identify monsters and their special powers 
or vulnerabilities. In general, the DC of such a check equals 10 + the monster's 
HD. A successful check allows you to remember a bit of useful information about 
that monster.

For every 5 points by which your check result exceeds the DC, you recall another 
piece of useful information.

\textbf{Action:} Usually none. In most cases, making a Knowledge check doesn't 
take an action---you simply know the answer or you don't.

\textbf{Try Again:} No. The check represents what you know, and thinking about 
a topic a second time doesn't let you know something that you never learned in 
the first place.

\textbf{Synergy:} If you have 5 or more ranks in Knowledge (arcana), you get a 
+2 bonus on Spellcraft checks.

If you have 5 or more ranks in Knowledge (architecture and engineering), you get 
a +2 bonus on Search checks made to find secret doors or hidden compartments.

If you have 5 or more ranks in Knowledge (geography), you get a +2 bonus on Survival 
checks made to keep from getting lost or to avoid natural hazards.

If you have 5 or more ranks in Knowledge (history), you get a +2 bonus on bardic 
knowledge checks.

If you have 5 or more ranks in Knowledge (local), you get a +2 bonus on Gather 
Information checks.

If you have 5 or more ranks in Knowledge (nature), you get a +2 bonus on Survival 
checks made in aboveground natural environments (aquatic, desert, forest, hill, 
marsh, mountains, or plains).

If you have 5 or more ranks in Knowledge (nobility and royalty), you get a +2 bonus 
on Diplomacy checks.

If you have 5 or more ranks in Knowledge (religion), you get a +2 bonus on turning 
checks against undead.

If you have 5 or more ranks in Knowledge (the planes), you get a +2 bonus on Survival 
checks made while on other planes.

If you have 5 or more ranks in Knowledge (dungeoneering), you get a +2 bonus on 
Survival checks made while underground.

If you have 5 or more ranks in Survival, you get a +2 bonus on Knowledge (nature) 
checks.

\textbf{Untrained:} An untrained Knowledge check is simply an Intelligence check. 
Without actual training, you know only common knowledge (DC 10 or lower).

\vspace{12pt}
LISTEN (WIS)

\textbf{Check:} Your Listen check is either made against a DC that reflects how 
quiet the noise is that you might hear, or it is opposed by your target's Move 
Silently check.

\vspace{12pt}
\begin{tabular}{|>{\raggedright}p{35pt}|>{\raggedright}p{291pt}|}
\hline
L\textbf{isten DC } & S\textbf{ound}\tabularnewline
\hline
-10  & A battle\tabularnewline
\hline
0  & People talking\textsuperscript{\textbf{1}}\tabularnewline
\hline
5  & A person in medium armor walking at a slow pace (10 ft./round) trying not 
to make any noise.\tabularnewline
\hline
10  & An unarmored person walking at a slow pace (15 ft./round) trying not to make 
any noise\tabularnewline
\hline
15  & A 1st-level rogue using Move Silently to sneak past the listener\tabularnewline
\hline
15  & People whispering\textsuperscript{\textbf{1}}\tabularnewline
\hline
19  & A cat stalking\tabularnewline
\hline
30  & An owl gliding in for a kill\tabularnewline
\hline
\multicolumn{2}{|p{326pt}|}{1 If you beat the DC by 10 or more, you can make out 
what's being said, assuming that you understand the language.}\tabularnewline
\hline
\end{tabular}

\vspace{12pt}
\begin{tabular}{|>{\raggedright}p{56pt}|>{\raggedright}p{92pt}|}
\hline
L\textbf{isten DC Modifier} & C\textbf{ondition}\tabularnewline
\hline
+5 & Through a door\tabularnewline
\hline
+15 & Through a stone wall\tabularnewline
\hline
+1 & Per 10 feet of distance\tabularnewline
\hline
+5 & Listener distracted\tabularnewline
\hline
\end{tabular}

\vspace{12pt}
In the case of people trying to be quiet, the DCs given on the table could be replaced 
by Move Silently checks, in which case the indicated DC would be their average 
check result. 

\textbf{Action:} Varies. Every time you have a chance to hear something in a reactive 
manner (such as when someone makes a noise or you move into a new area), you can 
make a Listen check without using an action. Trying to hear something you failed 
to hear previously is a move action.

\textbf{Try Again:} Yes. You can try to hear something that you failed to hear 
previously with no penalty.

\textbf{Special:} When several characters are listening to the same thing, a single 
1d20 roll can be used for all the individuals' Listen checks.

A fascinated creature takes a -4 penalty on Listen checks made as reactions.

If you have the Alertness feat, you get a +2 bonus on Listen checks.

A ranger gains a bonus on Listen checks when using this skill against a favored 
enemy.

An elf, gnome, or halfling has a +2 racial bonus on Listen checks. 

A half-elf has a +1 racial bonus on Listen checks..

A sleeping character may make Listen checks at a -10 penalty. A successful check 
awakens the sleeper.

\vspace{12pt}
MOVE SILENTLY (DEX; ARMOR CHECK PENALTY)

\textbf{Check:} Your Move Silently check is opposed by the Listen check of anyone 
who might hear you. You can move up to one-half your normal speed at no penalty. 
When moving at a speed greater than one-half but less than your full speed, you 
take a -5 penalty. It's practically impossible (-20 penalty) to move silently while 
running or charging.

Noisy surfaces, such as bogs or undergrowth, are tough to move silently across. 
When you try to sneak across such a surface, you take a penalty on your Move Silently 
check as indicated below.

\vspace{12pt}
\begin{tabular}{|>{\raggedright}p{254pt}|>{\raggedright}p{70pt}|}
\hline
\subsection*{S\textbf{urface }} & \subsection*{C\textbf{heck Modifier}}\tabularnewline
\hline
Noisy (scree, shallow or deep bog, undergrowth, dense rubble)- & 2\tabularnewline
\hline
Very noisy (dense undergrowth, deep snow) - & 5\tabularnewline
\hline
\end{tabular}

\vspace{12pt}
\textbf{Action:}None. A Move Silently check is included in your movement or other 
activity, so it is part of another action.

\textbf{Special:} The master of a cat familiar gains a +3 bonus on Move Silently 
checks.

A halfling has a +2 racial bonus on Move Silently checks.

If you have the Stealthy feat, you get a +2 bonus on Move Silently checks.

\vspace{12pt}
OPEN LOCK (DEX; TRAINED ONLY)

Attempting an Open Lock check without a set of thieves' tools imposes a -2 circumstance 
penalty on the check, even if a simple tool is employed. If you use masterwork 
thieves' tools, you gain a +2 circumstance bonus on the check.

\textbf{Check:} The DC for opening a lock varies from 20 to 40, depending on the 
quality of the lock, as given on the table below.

\vspace{12pt}
\begin{tabular}{|>{\raggedright}p{67pt}|>{\raggedright}p{13pt}|>{\raggedright}p{55pt}|>{\raggedright}p{13pt}|}
\hline
\subsection*{L\textbf{ock }} & \subsection*{D\textbf{C }} & \subsection*{L\textbf{ock 
}} & \subsection*{D\textbf{C}}\tabularnewline
\hline
Very simple lock  & 20  & Good lock  & 30\tabularnewline
\hline
Average lock &  25  & Amazing lock  & 40\tabularnewline
\hline
\end{tabular}

\vspace{12pt}
\textbf{Action:} Opening a lock is a full-round action.

\textbf{Special:} If you have the Nimble Fingers feat, you get a +2 bonus on Open 
Lock checks.

\textbf{Untrained:} You cannot pick locks untrained, but you might successfully 
force them open.

\vspace{12pt}
PERFORM (CHA)

Like Craft, Knowledge, and Profession, Perform is actually a number of separate 
skills.

You could have several Perform skills, each with its own ranks, each purchased 
as a separate skill.

Each of the nine categories of the Perform skill includes a variety of methods, 
instruments, or techniques, a small list of which is provided for each category 
below.• 

Act (comedy, drama, mime)• 

Comedy (buffoonery, limericks, joke-telling)• 

Dance (ballet, waltz, jig)• 

Keyboard instruments (harpsichord, piano, pipe organ)• 

Oratory (epic, ode, storytelling)• 

Percussion instruments (bells, chimes, drums, gong)• 

String instruments (fiddle, harp, lute, mandolin)• 

Wind instruments (flute, pan pipes, recorder, shawm, trumpet)• 

Sing (ballad, chant, melody)

\textbf{Check:} You can impress audiences with your talent and skill.

\begin{tabular}{|>{\raggedright}p{27pt}|>{\raggedright}p{298pt}|}
\hline
P\textbf{erform DC} & P\textbf{erformance}\tabularnewline
\hline
10 & Routine performance. Trying to earn money by playing in public is essentially 
begging. You can earn 1d10 cp/day.\tabularnewline
\hline
15 & Enjoyable performance. In a prosperous city, you can earn 1d10 sp/day.\tabularnewline
\hline
20 & Great performance. In a prosperous city, you can earn 3d10 sp/day. In time, 
you may be invited to join a professional troupe and may develop a regional reputation.\tabularnewline
\hline
25 & Memorable performance. In a prosperous city, you can earn 1d6 gp/day. In time, 
you may come to the attention of noble patrons and develop a national reputation.\tabularnewline
\hline
30 & Extraordinary performance. In a prosperous city, you can earn 3d6 gp/day. 
In time, you may draw attention from distant potential patrons, or even from extraplanar 
beings.\tabularnewline
\hline
\end{tabular}

\vspace{12pt}
A masterwork musical instrument gives you a +2 circumstance bonus on Perform checks 
that involve its use.

\textbf{Action:} Varies. Trying to earn money by playing in public requires anywhere 
from an evening's work to a full day's performance. The bard's special Perform-based 
abilities are described in that class's description.

\textbf{Try Again:} Yes. Retries are allowed, but they don't negate previous failures, 
and an audience that has been unimpressed in the past is likely to be prejudiced 
against future performances. (Increase the DC by 2 for each previous failure.)

\textbf{Special:} A bard must have at least 3 ranks in a Perform skill to inspire 
courage in his allies, or to use his countersong or his \textit{fascinate }ability. 
A bard needs 6 ranks in a Perform skill to inspire competence, 9 ranks to use his 
\textit{suggestion }ability, 12 ranks to inspire greatness, 15 ranks to use his 
\textit{song of freedom }ability, 18 ranks to inspire heroics, and 21 ranks to 
use his \textit{mass suggestion }ability. See Bardic Music in the bard class description.

In addition to using the Perform skill, you can entertain people with sleight of 
hand, tumbling, tightrope walking, and spells (especially illusions).

\vspace{12pt}
PROFESSION (WIS; TRAINED ONLY)

Like Craft, Knowledge, and Perform, Profession is actually a number of separate 
skills. You could have several Profession skills, each with its own ranks, each 
purchased as a separate skill. While a Craft skill represents ability in creating 
or making an item, a Profession skill represents an aptitude in a vocation requiring 
a broader range of less specific knowledge. 

\textbf{Check:} You can practice your trade and make a decent living, earning about 
half your Profession check result in gold pieces per week of dedicated work. You 
know how to use the tools of your trade, how to perform the profession's daily 
tasks, how to supervise helpers, and how to handle common problems.

\textbf{Action:} Not applicable. A single check generally represents a week of 
work.

\textbf{Try Again:} Varies. An attempt to use a Profession skill to earn an income 
cannot be retried. You are stuck with whatever weekly wage your check result brought 
you. Another check may be made after a week to determine a new income for the next 
period of time. An attempt to accomplish some specific task can usually be retried.

\textbf{Untrained:} Untrained laborers and assistants (that is, characters without 
any ranks in Profession) earn an average of 1 silver piece per day.

\vspace{12pt}
RIDE (DEX)

If you attempt to ride a creature that is ill suited as a mount, you take a -5 
penalty on your Ride checks.

\textbf{Check:} Typical riding actions don't require checks. You can saddle, mount, 
ride, and dismount from a mount without a problem.

The following tasks do require checks.

\vspace{12pt}
\begin{tabular}{|>{\raggedright}p{79pt}|>{\raggedright}p{35pt}|>{\raggedright}p{94pt}|>{\raggedright}p{35pt}|}
\hline
\subsection*{T\textbf{ask }} & \subsection*{R\textbf{ide DC }} & \subsection*{T\textbf{ask 
}} & \subsection*{R\textbf{ide DC}}\tabularnewline
\hline
Guide with knees  & 5 & Leap  & 15\tabularnewline
\hline
Stay in saddle  & 5 & Spur mount  & 15\tabularnewline
\hline
Fight with warhorse  & 10 &  Control mount in battle & 20\tabularnewline
\hline
Cover  & 15 & Fast mount or dismount & 20\textsuperscript{\textbf{1}}\tabularnewline
\hline
Soft fall & 15 &  & \tabularnewline
\hline
\multicolumn{4}{|p{245pt}|}{1 Armor check penalty applies.}\tabularnewline
\hline
\end{tabular}

\vspace{12pt}
\textit{Guide with Knees: }You can react instantly to guide your mount with your 
knees so that you can use both hands in combat. Make your Ride check at the start 
of your turn. If you fail, you can use only one hand this round because you need 
to use the other to control your mount.

\textit{Stay in Saddle: }You can react instantly to try to avoid falling when your 
mount rears or bolts unexpectedly or when you take damage. This usage does not 
take an action.

\textit{Fight with Warhorse: }If you direct your war-trained mount to attack in 
battle, you can still make your own attack or attacks normally. This usage is a 
free action.

\textit{Cover: }You can react instantly to drop down and hang alongside your mount, 
using it as cover. You can't attack or cast spells while using your mount as cover. 
If you fail your Ride check, you don't get the cover benefit. This usage does not 
take an action.

\textit{Soft Fall: }You can react instantly to try to take no damage when you fall 
off a mount---when it is killed or when it falls, for example. If you fail your 
Ride check, you take 1d6 points of falling damage. This usage does not take an 
action.

\textit{Leap: }You can get your mount to leap obstacles as part of its movement. 
Use your Ride modifier or the mount's Jump modifier, whichever is lower, to see 
how far the creature can jump. If you fail your Ride check, you fall off the mount 
when it leaps and take the appropriate falling damage (at least 1d6 points). This 
usage does not take an action, but is part of the mount's movement.

\textit{Spur Mount: }You can spur your mount to greater speed with a move action. 
A successful Ride check increases the mount's speed by 10 feet for 1 round but 
deals 1 point of damage to the creature. You can use this ability every round, 
but each consecutive round of additional speed deals twice as much damage to the 
mount as the previous round (2 points, 4 points, 8 points, and so on).

\textit{Control Mount in Battle: }As a move action, you can attempt to control 
a light horse, pony, heavy horse, or other mount not trained for combat riding 
while in battle. If you fail the Ride check, you can do nothing else in that round. 
You do not need to roll for warhorses or warponies.

\textit{Fast Mount or Dismount: }You can attempt to mount or dismount from a mount 
of up to one size category larger than yourself as a free action, provided that 
you still have a move action available that round. If you fail the Ride check, 
mounting or dismounting is a move action. You can't use fast mount or dismount 
on a mount more than one size category larger than yourself.

\textbf{Action:} Varies. Mounting or dismounting normally is a move action. Other 
checks are a move action, a free action, or no action at all, as noted above.

\textbf{Special:} If you are riding bareback, you take a -5 penalty on Ride checks.

If your mount has a military saddle you get a +2 circumstance bonus on Ride checks 
related to staying in the saddle.

The Ride skill is a prerequisite for the feats Mounted Archery, Mounted Combat, 
Ride-By Attack, Spirited Charge,

Trample.

If you have the Animal Affinity feat, you get a +2 bonus on Ride checks.

\textbf{Synergy:} If you have 5 or more ranks in Handle Animal, you get a +2 bonus 
on Ride checks.

\vspace{12pt}
SEARCH (INT)

\textbf{Check:} You generally must be within 10 feet of the object or surface to 
be searched. The table below gives DCs for typical tasks involving the Search skill.

\vspace{12pt}
\begin{tabular}{|>{\raggedright}p{177pt}|>{\raggedright}p{148pt}|}
\hline
\subsection*{T\textbf{ask}} & \subsection*{S\textbf{earch DC}}\tabularnewline
\hline
Ransack a chest full of junk to find a certain item & 10\tabularnewline
\hline
Notice a typical secret door or a simple trap  & 20\tabularnewline
\hline
Find a difficult nonmagical trap (rogue only)\textsuperscript{\textbf{1}}\textbf{ 
} & 21 or higher\tabularnewline
\hline
Find a magic trap (rogue only)\textsuperscript{\textbf{1 }} & 25 + level of spell 
used to create trap\tabularnewline
\hline
Notice a well-hidden secret door  & 30\tabularnewline
\hline
Find a footprint  & Varies\textsuperscript{\textbf{2}}\tabularnewline
\hline
\multicolumn{2}{|p{326pt}|}{1 Dwarves (even if they are not rogues) can use Search 
to find traps built into or out of stone.}\tabularnewline
\hline
\multicolumn{2}{|p{326pt}|}{2 A successful Search check can find a footprint or 
similar sign of a creature's passage, but it won't let you find or follow a trail. 
See the Track feat for the appropriate DC.}\tabularnewline
\hline
\end{tabular}

\vspace{12pt}
\textbf{Action:} It takes a full-round action to search a 5-foot-by-5-foot area 
or a volume of goods 5 feet on a side.

\textbf{Special:} An elf has a +2 racial bonus on Search checks, and a half-elf 
has a +1 racial bonus. An elf (but not a half-elf) who simply passes within 5 feet 
of a secret or concealed door can make a Search check to find that door.

If you have the Investigator feat, you get a +2 bonus on Search checks.

The spells \textit{explosive runes, fire trap, glyph of warding, symbol, }and \textit{teleportation 
circle }create magic traps that a rogue can find by making a successful Search 
check and then can attempt to disarm by using Disable Device. Identifying the location 
of a \textit{snare }spell has a DC of 23. \textit{Spike growth }and \textit{spike 
stones }create magic traps that can be found using Search, but against which Disable 
Device checks do not succeed. See the individual spell descriptions for details.

Active abjuration spells within 10 feet of each other for 24 hours or more create 
barely visible energy fluctuations. These fluctuations give you a +4 bonus on Search 
checks to locate such abjuration spells.

\textbf{Synergy:} If you have 5 or more ranks in Search, you get a +2 bonus on 
Survival checks to find or follow tracks.

If you have 5 or more ranks in Knowledge (architecture and engineering), you get 
a +2 bonus on Search checks to find secret doors or hidden compartments.

\textbf{Restriction:} While anyone can use Search to find a trap whose DC is 20 
or lower, only a rogue can use Search to locate traps with higher DCs. (\textit{Exception: 
}The spell \textit{find traps }temporarily enables a cleric to use the Search skill 
as if he were a rogue.)

A dwarf, even one who is not a rogue, can use the Search skill to find a difficult 
trap (one with a DC higher than 20) if the trap is built into or out of stone. 
He gains a +2 racial bonus on the Search check from his stonecunning ability.

\vspace{12pt}
SENSE MOTIVE (WIS)

\textbf{Check:} A successful check lets you avoid being bluffed (see the Bluff 
skill). You can also use this skill to determine when ``something is up'' (that 
is, something odd is going on) or to assess someone's trustworthiness. 

\vspace{12pt}
\begin{tabular}{|>{\raggedright}p{92pt}|>{\raggedright}p{72pt}|}
\hline
\subsection*{T\textbf{ask }} & \subsection*{S\textbf{ense Motive DC}}\tabularnewline
\hline
Hunch  & 20\tabularnewline
\hline
Sense enchantment  & 25 or 15\tabularnewline
\hline
Discern secret message  & Varies\tabularnewline
\hline
\end{tabular}

\vspace{12pt}
\textit{Hunch: }This use of the skill involves making a gut assessment of the social 
situation. You can get the feeling from another's behavior that something is wrong, 
such as when you're talking to an impostor. Alternatively, you can get the feeling 
that someone is trustworthy.

\textit{Sense Enchantment: }You can tell that someone's behavior is being influenced 
by an enchantment effect (by definition, a mind-affecting effect),\textit{ }even 
if that person isn't aware of it. The usual DC is 25, but if the target is dominated 
(see \textit{dominate person}), the DC is only 15 because of the limited range 
of the target's activities.

\textit{Discern Secret Message: }You may use Sense Motive to detect that a hidden 
message is being transmitted via the Bluff skill. In this case, your Sense Motive 
check is opposed by the Bluff check of the character transmitting the message. 
For each piece of information relating to the message that you are missing, you 
take a -2 penalty on your Sense Motive check. If you succeed by 4 or less, you 
know that something hidden is being communicated, but you can't learn anything 
specific about its content. If you beat the DC by 5 or more, you intercept and 
understand the message. If you fail by 4 or less, you don't detect any hidden communication. 
If you fail by 5 or more, you infer some false information.

\textbf{Action:} Trying to gain information with Sense Motive generally takes at 
least 1 minute, and you could spend a whole evening trying to get a sense of the 
people around you.

\textbf{Try Again:} No, though you may make a Sense Motive check for each Bluff 
check made against you.

\textbf{Special:} A ranger gains a bonus on Sense Motive checks when using this 
skill against a favored enemy.

If you have the Negotiator feat, you get a +2 bonus on Sense Motive checks.

\textbf{Synergy:} If you have 5 or more ranks in Sense Motive, you get a +2 bonus 
on Diplomacy checks.

\vspace{12pt}
SLEIGHT OF HAND (DEX; TRAINED ONLY; ARMOR CHECK PENALTY)

\textbf{Check:} A DC 10 Sleight of Hand check lets you palm a coin-sized, unattended 
object. Performing a minor feat of legerdemain, such as making a coin disappear, 
also has a DC of 10 unless an observer is determined to note where the item went.

When you use this skill under close observation, your skill check is opposed by 
the observer's Spot check. The observer's success doesn't prevent you from performing 
the action, just from doing it unnoticed.

You can hide a small object (including a light weapon or an easily concealed ranged 
weapon, such as a dart, sling, or hand crossbow) on your body. Your Sleight of 
Hand check is opposed by the Spot check of anyone observing you or the Search check 
of anyone frisking you. In the latter case, the searcher gains a +4 bonus on the 
Search check, since it's generally easier to find such an object than to hide it. 
A dagger is easier to hide than most light weapons, and grants you a +2 bonus on 
your Sleight of Hand check to conceal it. An extraordinarily small object, such 
as a coin, shuriken, or ring, grants you a +4 bonus on your Sleight of Hand check 
to conceal it, and heavy or baggy clothing (such as a cloak) grants you a +2 bonus 
on the check.

Drawing a hidden weapon is a standard action and doesn't provoke an attack of opportunity.

If you try to take something from another creature, you must make a DC 20 Sleight 
of Hand check to obtain it. The opponent makes a Spot check to detect the attempt, 
opposed by the same Sleight of Hand check result you achieved when you tried to 
grab the item. An opponent who succeeds on this check notices the attempt, regardless 
of whether you got the item.

You can also use Sleight of Hand to entertain an audience as though you were using 
the Perform skill. In such a case, your ``act'' encompasses elements of legerdemain, 
juggling, and the like.

\vspace{12pt}
\begin{tabular}{|>{\raggedright}p{82pt}|>{\raggedright}p{190pt}|}
\hline
\subsection*{S\textbf{leight of Hand DC }} & \subsection*{T\textbf{ask}}\tabularnewline
\hline
10 & Palm a coin-sized object, make a coin disappear\tabularnewline
\hline
20 & Lift a small object from a person\tabularnewline
\hline
\end{tabular}

\vspace{12pt}
\textbf{Action:} Any Sleight of Hand check normally is a standard action. However, 
you may perform a Sleight of Hand check as a free action by taking a -20 penalty 
on the check.

\textbf{Try Again:} Yes, but after an initial failure, a second Sleight of Hand 
attempt against the same target (or while you are being watched by the same observer 
who noticed your previous attempt) increases the DC for the task by 10.

\textbf{Special:} If you have the Deft Hands feat, you get a +2 bonus on Sleight 
of Hand checks.

\textbf{Synergy:} If you have 5 or more ranks in Bluff, you get a +2 bonus on Sleight 
of Hand checks.

\textbf{Untrained:} An untrained Sleight of Hand check is simply a Dexterity check. 
Without actual training, you can't succeed on any Sleight of Hand check with a 
DC higher than 10, except for hiding an object on your body.

\vspace{12pt}
SPEAK LANGUAGE (NONE; TRAINED ONLY)

\begin{tabular}{|>{\raggedright}p{58pt}|>{\raggedright}p{163pt}|>{\raggedright}p{38pt}|}
\hline
\multicolumn{3}{|p{260pt}|}{C\textbf{ommon Languages and Their Alphabets}}\tabularnewline
\hline
L\textbf{anguage } & T\textbf{ypical Speakers } & A\textbf{lphabet}\tabularnewline
\hline
Abyssal  & Demons, chaotic evil outsiders  & Infernal\tabularnewline
\hline
Aquan  & Water-based creatures  & Elven\tabularnewline
\hline
Auran  & Air-based creatures  & Draconic\tabularnewline
\hline
Celestial  & Good outsiders  & Celestial\tabularnewline
\hline
Common  & Humans, halflings, half-elves, half-orcs  & Common\tabularnewline
\hline
Draconic  & Kobolds, troglodytes, lizardfolk, dragons & Draconic\tabularnewline
\hline
Druidic  & Druids (only)  & Druidic\tabularnewline
\hline
Dwarven  & Dwarves  & Dwarven\tabularnewline
\hline
Elven  & Elves  & Elven\tabularnewline
\hline
Giant  & Ogres, giants  & Dwarven\tabularnewline
\hline
Gnome  & Gnomes  & Dwarven\tabularnewline
\hline
Goblin  & Goblins, hobgoblins, bugbears  & Dwarven\tabularnewline
\hline
Gnoll  & Gnolls  & Common\tabularnewline
\hline
Halfling  & Halflings  & Common\tabularnewline
\hline
Ignan  & Fire-based creatures  & Draconic\tabularnewline
\hline
Infernal  & Devils, lawful evil outsiders  & Infernal\tabularnewline
\hline
Orc  & Orcs  & Dwarven\tabularnewline
\hline
Sylvan  & Dryads, brownies, leprechauns  & Elven\tabularnewline
\hline
Terran  & Xorns and other earth-based creatures & Dwarven\tabularnewline
\hline
Undercommon  & Drow & Elven\tabularnewline
\hline
\end{tabular}

\vspace{12pt}
\textbf{Action:} Not applicable.

\textbf{Try Again:} Not applicable. There are no Speak Language checks to fail.

The Speak Language skill doesn't work like other skills. Languages work as follows.• 

\parindent=3pt
You start at 1st level knowing one or two languages (based on your race), plus 
an additional number of languages equal to your starting Intelligence bonus.• 

You can purchase Speak Language just like any other skill, but instead of buying 
a rank in it, you choose a new language that you can speak.• 

\parindent=7pt
You don't make Speak Language checks. You either know a language or you don't.• 

\parindent=3pt
A literate character (anyone but a barbarian who has not spent skill points to 
become literate) can read and write any language she speaks. Each language has 
an alphabet, though sometimes several spoken languages share a single alphabet.

\vspace{12pt}
\parindent=0pt
SPELLCRAFT (INT; TRAINED ONLY)

Use this skill to identify spells as they are cast or spells already in place.

\vspace{12pt}
\begin{tabular}{|>{\raggedright}p{46pt}|>{\raggedright}p{280pt}|}
\hline
\subsection*{S\textbf{pellcraft DC }} & \subsection*{T\textbf{ask}}\tabularnewline
\hline
13  & When using \textit{read magic, }identify a \textit{glyph of warding. }No 
action required.\tabularnewline
\hline
15 + spell level  & Identify a spell being cast. (You must see or hear the spell's 
verbal or somatic components.) No action required. No retry. \tabularnewline
\hline
15 + spell level  & Learn a spell from a spellbook or scroll (wizard only). No 
retry for that spell until you gain at least 1 rank in Spellcraft (even if you 
find another source to try to learn the spell from). Requires 8 hours.\tabularnewline
\hline
15 + spell level  & Prepare a spell from a borrowed spellbook (wizard only). One 
try per day. No extra time required. \tabularnewline
\hline
15 + spell level  & When casting \textit{detect magic, }determine the school of 
magic involved in the aura of a single item or creature you can see. (If the aura 
is not a spell effect, the DC is 15 + one-half caster level.) No action required.\tabularnewline
\hline
19  & When using \textit{read magic, }identify a \textit{symbol. }No action required.\tabularnewline
\hline
20 + spell level  & Identify a spell that's already in place and in effect. You 
must be able to see or detect the effects of the spell. No action required. No 
retry.\tabularnewline
\hline
20 + spell level  & Identify materials created or shaped by magic, such as noting 
that an iron wall is the result of a \textit{wall of iron }spell. No action required. 
No retry.\tabularnewline
\hline
20 + spell level  & Decipher a written spell (such as a scroll) without using \textit{read 
magic. }One try per day. Requires a full-round action.\tabularnewline
\hline
25 + spell level  & After rolling a saving throw against a spell targeted on you, 
determine what that spell was. No action required. No retry.\tabularnewline
\hline
25  & Identify a potion. Requires 1 minute. No retry.\tabularnewline
\hline
20  & Draw a diagram to allow \textit{dimensional anchor }to be cast on a \textit{magic 
circle }spell. Requires 10 minutes. No retry. This check is made secretly so you 
do not know the result.\tabularnewline
\hline
30 or higher  & Understand a strange or unique magical effect, such as the effects 
of a magic stream. Time required varies. No retry.\tabularnewline
\hline
\end{tabular}

\vspace{12pt}
\textbf{Check:} You can identify spells and magic effects. The DCs for Spellcraft 
checks relating to various tasks are summarized on the table above.

\textbf{Action:} Varies, as noted above.

\textbf{Try Again:} See above.

\textbf{Special:} If you are a specialist wizard, you get a +2 bonus on Spellcraft 
checks when dealing with a spell or effect from your specialty school. You take 
a -5 penalty when dealing with a spell or effect from a prohibited school (and 
some tasks, such as learning a prohibited spell, are just impossible).

If you have the Magical Aptitude feat, you get a +2 bonus on Spellcraft checks.

\textbf{Synergy:} If you have 5 or more ranks in Knowledge (arcana), you get a 
+2 bonus on Spellcraft checks.

If you have 5 or more ranks in Use Magic Device, you get a +2 bonus on Spellcraft 
checks to decipher spells on scrolls.

If you have 5 or more ranks in Spellcraft, you get a +2 bonus on Use Magic Device 
checks related to scrolls.

Additionally, certain spells allow you to gain information about magic, provided 
that you make a successful Spellcraft check as detailed in the spell description.

\vspace{12pt}
SPOT (WIS)

\textbf{Check:} The Spot skill is used primarily to detect characters or creatures 
who are hiding. Typically, your Spot check is opposed by the Hide check of the 
creature trying not to be seen. Sometimes a creature isn't intentionally hiding 
but is still difficult to see, so a successful Spot check is necessary to notice 
it.

A Spot check result higher than 20 generally lets you become aware of an invisible 
creature near you, though you can't actually see it.

Spot is also used to detect someone in disguise (see the Disguise skill), and to 
read lips when you can't hear or understand what someone is saying.

Spot checks may be called for to determine the distance at which an encounter begins. 
A penalty applies on such checks, depending on the distance between the two individuals 
or groups, and an additional penalty may apply if the character making the Spot 
check is distracted (not concentrating on being observant).

\vspace{12pt}
\begin{tabular}{|>{\raggedright}p{88pt}|>{\raggedright}p{31pt}|}
\hline
\subsection*{C\textbf{ondition }} & \subsection*{P\textbf{enalty}}\tabularnewline
\hline
Per 10 feet of distance - & 1\tabularnewline
\hline
Spotter distracted - & 5\tabularnewline
\hline
\end{tabular}

\vspace{12pt}
\textit{Read Lips: }To understand what someone is saying by reading lips, you must 
be within 30 feet of the speaker, be able to see him or her speak, and understand 
the speaker's language. (This use of the skill is language-dependent.) The base 
DC is 15, but it increases for complex speech or an inarticulate speaker. You must 
maintain a line of sight to the lips being read.

If your Spot check succeeds, you can understand the general content of a minute's 
worth of speaking, but you usually still miss certain details. If the check fails 
by 4 or less, you can't read the speaker's lips. If the check fails by 5 or more, 
you draw some incorrect conclusion about the speech. The check is rolled secretly 
in this case, so that you don't know whether you succeeded or missed by 5.

\textbf{Action: }Varies. Every time you have a chance to spot something in a reactive 
manner you can make a Spot check without using an action. Trying to spot something 
you failed to see previously is a move action. To read lips, you must concentrate 
for a full minute before making a Spot check, and you can't perform any other action 
(other than moving at up to half speed) during this minute.

\textbf{Try Again:} Yes. You can try to spot something that you failed to see previously 
at no penalty. You can attempt to read lips once per minute.

\textbf{Special:} A fascinated creature takes a -4 penalty on Spot checks made 
as reactions.

If you have the Alertness feat, you get a +2 bonus on Spot checks.

A ranger gains a bonus on Spot checks when using this skill against a favored enemy.

An elf has a +2 racial bonus on Spot checks.

A half-elf has a +1 racial bonus on Spot checks.

The master of a hawk familiar gains a +3 bonus on Spot checks in daylight or other 
lighted areas.

The master of an owl familiar gains a +3 bonus on Spot checks in shadowy or other 
darkened areas.

\vspace{12pt}
SURVIVAL (WIS)

\textbf{Check:} You can keep yourself and others safe and fed in the wild. The 
table below gives the DCs for various tasks that require Survival checks.

Survival does not allow you to follow difficult tracks unless you are a ranger 
or have the Track feat (see the Restriction section below).

\vspace{12pt}
\begin{tabular}{|>{\raggedright}p{29pt}|>{\raggedright}p{297pt}|}
\hline
S\textbf{urvival DC } & \subsection*{T\textbf{ask}}\tabularnewline
\hline
10  & Get along in the wild. Move up to one-half your overland speed while hunting 
and foraging (no food or water supplies needed). You can provide food and water 
for one other person for every 2 points by which your check result exceeds 10.\tabularnewline
\hline
15 & Gain a +2 bonus on all Fortitude saves against severe weather while moving 
up to one-half your overland speed, or gain a +4 bonus if you remain stationary. 
You may grant the same bonus to one other character for every 1 point by which 
your Survival check result exceeds 15.\tabularnewline
\hline
15  & Keep from getting lost or avoid natural hazards, such as quicksand.\tabularnewline
\hline
15  & Predict the weather up to 24 hours in advance. For every 5 points by which 
your Survival check result exceeds 15, you can predict the weather for one additional 
day in advance.\tabularnewline
\hline
Varies  & Follow tracks (see the Track feat).\tabularnewline
\hline
\end{tabular}

\vspace{12pt}
\textbf{Action:} Varies. A single Survival check may represent activity over the 
course of hours or a full day. A Survival check made to find tracks is at least 
a full-round action, and it may take even longer.

\textbf{Try Again:} Varies. For getting along in the wild or for gaining the Fortitude 
save bonus noted in the table above, you make a Survival check once every 24 hours. 
The result of that check applies until the next check is made. To avoid getting 
lost or avoid natural hazards, you make a Survival check whenever the situation 
calls for one. Retries to avoid getting lost in a specific situation or to avoid 
a specific natural hazard are not allowed. For finding tracks, you can retry a 
failed check after 1 hour (outdoors) or 10 minutes(indoors) of searching.

\textbf{Restriction:} While anyone can use Survival to find tracks (regardless 
of the DC), or to follow tracks when the DC for the task is 10 or lower, only a 
ranger (or a character with the Track feat) can use Survival to follow tracks when 
the task has a higher DC.

\textbf{Special:} If you have 5 or more ranks in Survival, you can automatically 
determine where true north lies in relation to yourself.

A ranger gains a bonus on Survival checks when using this skill to find or follow 
the tracks of a favored enemy.

If you have the Self-Sufficient feat, you get a +2 bonus on Survival checks.

\textbf{Synergy:} If you have 5 or more ranks in Survival, you get a +2 bonus on 
Knowledge (nature) checks.

If you have 5 or more ranks in Knowledge (dungeoneering), you get a +2 bonus on 
Survival checks made while underground.

If you have 5 or more ranks in Knowledge (nature), you get a +2 bonus on Survival 
checks in aboveground natural environments (aquatic, desert, forest, hill, marsh, 
mountains, and plains).

If you have 5 or more ranks in Knowledge (geography), you get a +2 bonus on Survival 
checks made to keep from getting lost or to avoid natural hazards.

If you have 5 or more ranks in Knowledge (the planes), you get a +2 bonus on Survival 
checks made while on other planes.

If you have 5 or more ranks in Search, you get a +2 bonus on Survival checks to 
find or follow tracks.

\vspace{12pt}
SWIM (STR; ARMOR CHECK PENALTY)

\textbf{Check:} Make a Swim check once per round while you are in the water. Success 
means you may swim at up to one-half your speed (as a full-round action) or at 
one-quarter your speed (as a move action). If you fail by 4 or less, you make no 
progress through the water. If you fail by 5 or more, you go underwater.

If you are underwater, either because you failed a Swim check or because you are 
swimming underwater intentionally, you must hold your breath. You can hold your 
breath for a number of rounds equal to your Constitution score, but only if you 
do nothing other than take move actions or free actions. If you take a standard 
action or a full-round action (such as making an attack), the remainder of the 
duration for which you can hold your breath is reduced by 1 round. (Effectively, 
a character in combat can hold his or her breath only half as long as normal.) 
After that period of time, you must make a DC 10 Constitution check every round 
to continue holding your breath. Each round, the DC for that check increases by 
1. If you fail the Constitution check, you begin to drown.

The DC for the Swim check depends on the water, as given on the table below.

\vspace{12pt}
\begin{tabular}{|>{\raggedright}p{101pt}|>{\raggedright}p{60pt}|}
\hline
\subsection*{W\textbf{ater}} & \subsection*{S\textbf{wim DC}}\tabularnewline
\hline
Calm water  & 10\tabularnewline
\hline
Rough water  & 15\tabularnewline
\hline
Stormy water  & 20\textsuperscript{\textbf{1}}\tabularnewline
\hline
\multicolumn{2}{|p{161pt}|}{1 You can't take 10 on a Swim check in stormy water, 
even if you aren't otherwise being threatened or distracted.}\tabularnewline
\hline
\end{tabular}

\vspace{24pt}
Each hour that you swim, you must make a DC 20 Swim check or take 1d6 points of 
nonlethal damage from fatigue.

\textbf{Action:} A successful Swim check allows you to swim one-quarter of your 
speed as a move action or one-half your speed as a full-round action.

\textbf{Special:} Swim checks are subject to double the normal armor check penalty 
and encumbrance penalty.

If you have the Athletic feat, you get a +2 bonus on Swim checks.

If you have the Endurance feat, you get a +4 bonus on Swim checks made to avoid 
taking nonlethal damage from fatigue.

A creature with a swim speed can move through water at its indicated speed without 
making Swim checks. It gains a +8 racial bonus on any Swim check to perform a special 
action or avoid a hazard. The creature always can choose to take 10 on a Swim check, 
even if distracted or endangered when swimming. Such a creature can use the run 
action while swimming, provided that it swims in a straight line.

\vspace{12pt}
TUMBLE (DEX; TRAINED ONLY; ARMOR CHECK PENALTY)

You can't use this skill if your speed has been reduced by armor, excess equipment, 
or loot.

\textbf{Check:} You can land softly when you fall or tumble past opponents. You 
can also tumble to entertain an audience (as though using the Perform skill). The 
DCs for various tasks involving the Tumble skill are given on the table below.

\vspace{12pt}
\begin{tabular}{|>{\raggedright}p{23pt}|>{\raggedright}p{303pt}|}
\hline
\subsection*{T\textbf{umble DC }} & \subsection*{T\textbf{ask}}\tabularnewline
\hline
15  & Treat a fall as if it were 10 feet shorter than it really is when determining 
damage.\tabularnewline
\hline
15  & Tumble at one-half speed as part of normal movement, provoking no attacks 
of opportunity while doing so. Failure means you provoke attacks of opportunity 
normally. Check separately for each opponent you move past, in the order in which 
you pass them (player's choice of order in case of a tie).\linebreak{}
Each additional enemy after the first adds +2 to the Tumble DC.\tabularnewline
\hline
25  & Tumble at one-half speed through an area occupied by an enemy (over, under, 
or around the opponent) as part of normal movement, provoking no attacks of opportunity 
while doing so. Failure means you stop before entering the enemy-occupied area 
and provoke an attack of opportunity from that enemy.\linebreak{}
Check separately for each opponent. Each additional enemy after the first adds 
+2 to the Tumble DC.\tabularnewline
\hline
\end{tabular}

\vspace{12pt}
Obstructed or otherwise treacherous surfaces, such as natural cavern floors or 
undergrowth, are tough to tumble through. The DC for any Tumble check made to tumble 
into such a square is modified as indicated below.

\vspace{12pt}
\begin{tabular}{|>{\raggedright}p{275pt}|>{\raggedright}p{51pt}|}
\hline
S\textbf{urface Is . . . } & D\textbf{C Modifier}\tabularnewline
\hline
Lightly obstructed (scree, light rubble, shallow bog\textsuperscript{\textbf{1}}, 
undergrowth)  & +2\tabularnewline
\hline
Severely obstructed (natural cavern floor, dense rubble, dense undergrowth)  & +5\tabularnewline
\hline
Lightly slippery (wet floor)  & +2\tabularnewline
\hline
Severely slippery (ice sheet)  & +5\tabularnewline
\hline
Sloped or angled  & +2\tabularnewline
\hline
\multicolumn{2}{|p{326pt}|}{1 Tumbling is impossible in a deep bog.}\tabularnewline
\hline
\end{tabular}

\vspace{12pt}
\textit{Accelerated Tumbling: }You try to tumble past or through enemies more quickly 
than normal. By accepting a -10 penalty on your Tumble checks, you can move at 
your full speed instead of one-half your speed.

\textbf{Action:} Not applicable. Tumbling is part of movement, so a Tumble check 
is part of a move action.

\textbf{Try Again:} Usually no. An audience, once it has judged a tumbler as an 
uninteresting performer, is not receptive to repeat performances.

You can try to reduce damage from a fall as an instant reaction only once per fall.

\textbf{Special:} If you have 5 or more ranks in Tumble, you gain a +3 dodge bonus 
to AC when fighting defensively instead of the usual +2 dodge bonus to AC.

If you have 5 or more ranks in Tumble, you gain a +6 dodge bonus to AC when executing 
the total defense standard action instead of the usual +4 dodge bonus to AC.

If you have the Acrobatic feat, you get a +2 bonus on Tumble checks.

\textbf{Synergy:} If you have 5 or more ranks in Tumble, you get a +2 bonus on 
Balance and Jump checks.

If you have 5 or more ranks in Jump, you get a +2 bonus on Tumble checks.

\vspace{12pt}
USE MAGIC DEVICE (CHA; TRAINED ONLY)

Use this skill to activate magic

\textbf{Check:} You can use this skill to read a spell or to activate a magic item. 
Use Magic Device lets you use a magic item as if you had the spell ability or class 
features of another class, as if you were a different race, or as if you were of 
a different alignment.

You make a Use Magic Device check each time you activate a device such as a wand. 
If you are using the check to emulate an alignment or some other quality in an 
ongoing manner, you need to make the relevant Use Magic Device check once per hour.

You must consciously choose which requirement to emulate. That is, you must know 
what you are trying to emulate when you make a Use Magic Device check for that 
purpose. The DCs for various tasks involving Use Magic Device checks are summarized 
on the table below.

\vspace{12pt}
\begin{tabular}{|>{\raggedright}p{95pt}|>{\raggedright}p{91pt}|}
\hline
\subsection*{T\textbf{ask }} & \subsection*{U\textbf{se Magic Device DC}}\tabularnewline
\hline
Activate blindly  & 25\tabularnewline
\hline
Decipher a written spell  & 25 + spell level\tabularnewline
\hline
Use a scroll  & 20 + caster level\tabularnewline
\hline
Use a wand  & 20\tabularnewline
\hline
Emulate a class feature  & 20\tabularnewline
\hline
Emulate an ability score  & See text\tabularnewline
\hline
Emulate a race  & 25\tabularnewline
\hline
Emulate an alignment  & 30\tabularnewline
\hline
\end{tabular}

\vspace{12pt}
\textit{Activate Blindly: }Some magic items are activated by special words, thoughts, 
or actions. You can activate such an item as if you were using the activation word, 
thought, or action, even when you're not and even if you don't know it. You do 
have to perform some equivalent activity in order to make the check. That is, you 
must speak, wave the item around, or otherwise attempt to get it to activate. You 
get a special +2 bonus on your Use Magic Device check if you've activated the item 
in question at least once before. If you fail by 9 or less, you can't activate 
the device. If you fail by 10 or more, you suffer a mishap. A mishap means that 
magical energy gets released but it doesn't do what you wanted it to do. The default 
mishaps are that the item affects the wrong target or that uncontrolled magical 
energy is released, dealing 2d6 points of damage to you. This mishap is in addition 
to the chance for a mishap that you normally run when you cast a spell from a scroll 
that you could not otherwise cast yourself.

\textit{Decipher a Written Spell: }This usage works just like deciphering a written 
spell with the Spellcraft skill, except that the DC is 5 points higher. Deciphering 
a written spell requires 1 minute of concentration.

\textit{Emulate an Ability Score: }To cast a spell from a scroll, you need a high 
score in the appropriate ability (Intelligence for wizard spells, Wisdom for divine 
spells, or Charisma for sorcerer or bard spells). Your effective ability score 
(appropriate to the class you're emulating when you try to cast the spell from 
the scroll) is your Use Magic Device check result minus 15. If you already have 
a high enough score in the appropriate ability, you don't need to make this check.

\textit{Emulate an Alignment: }Some magic items have positive or negative effects 
based on the user's alignment. Use Magic Device lets you use these items as if 
you were of an alignment of your choice. You can emulate only one alignment at 
a time.

\textit{Emulate a Class Feature: }Sometimes you need to use a class feature to 
activate a magic item. In this case, your effective level in the emulated class 
equals your Use Magic Device check result minus 20.  This skill does not let you 
actually use the class feature of another class. It just lets you activate items 
as if you had that class feature. If the class whose feature you are emulating 
has an alignment requirement, you must meet it, either honestly or by emulating 
an appropriate alignment with a separate Use Magic Device check (see above).

\textit{Emulate a Race: }Some magic items work only for members of certain races, 
or work better for members of those races. You can use such an item as if you were 
a race of your choice. You can emulate only one race at a time.

\textit{Use a Scroll: }If you are casting a spell from a scroll, you have to decipher 
it first. Normally, to cast a spell from a scroll, you must have the scroll's spell 
on your class spell list. Use Magic Device allows you to use a scroll as if you 
had a particular spell on your class spell list. The DC is equal to 20 + the caster 
level of the spell you are trying to cast from the scroll. In addition, casting 
a spell from a scroll requires a minimum score (10 + spell level) in the appropriate 
ability. If you don't have a sufficient score in that ability, you must emulate 
the ability score with a separate Use Magic Device check (see above).

This use of the skill also applies to other spell completion magic items.

\textit{Use a Wand: }Normally, to use a wand, you must have the wand's spell on 
your class spell list. This use of the skill allows you to use a wand as if you 
had a particular spell on your class spell list. This use of the skill also applies 
to other spell trigger magic items, such as staffs.

\textbf{Action:} None. The Use Magic Device check is made as part of the action 
(if any) required to activate the magic item.

\textbf{Try Again:} Yes, but if you ever roll a natural 1 while attempting to activate 
an item and you fail, then you can't try to activate that item again for 24 hours.

\textbf{Special:} You cannot take 10 with this skill.

You can't aid another on Use Magic Device checks. Only the user of the item may 
attempt such a check.

If you have the Magical Aptitude feat, you get a +2 bonus on Use Magic Device checks.

\textbf{Synergy:} If you have 5 or more ranks in Spellcraft, you get a +2 bonus 
on Use Magic Device checks related to scrolls.

If you have 5 or more ranks in Decipher Script, you get a +2 bonus on Use Magic 
Device checks related to scrolls.

If you have 5 or more ranks in Use Magic Device, you get a +2 bonus to Spellcraft 
checks made to decipher spells on scrolls.

\vspace{12pt}
USE ROPE (DEX)

\textbf{Check:} Most tasks with a rope are relatively simple. The DCs for various 
tasks utilizing this skill are summarized on the table below.

\vspace{12pt}
\begin{tabular}{|>{\raggedright}p{54pt}|>{\raggedright}p{272pt}|}
\hline
U\textbf{se Rope DC} & T\textbf{ask}\tabularnewline
\hline
10 & Tie a firm knot\tabularnewline
\hline
10\textsuperscript{\textbf{1}} & Secure a grappling hook\tabularnewline
\hline
15 & Tie a special knot, such as one that slips, slides slowly, or loosens with 
a tug\tabularnewline
\hline
15 & Tie a rope around yourself one-handed\tabularnewline
\hline
15 & Splice two ropes together\tabularnewline
\hline
Varies & Bind a character\tabularnewline
\hline
\multicolumn{2}{|p{326pt}|}{1 Add 2 to the DC for every 10 feet the hook is thrown; 
see below.}\tabularnewline
\hline
\end{tabular}

\vspace{12pt}
\textit{Secure a Grappling Hook: }Securing a grappling hook requires a Use Rope 
check (DC 10, +2 for every 10 feet of distance the grappling hook is thrown, to 
a maximum DC of 20 at 50 feet). Failure by 4 or less indicates that the hook fails 
to catch and falls, allowing you to try again. Failure by 5 or more indicates that 
the grappling hook initially holds, but comes loose after 1d4 rounds of supporting 
weight. This check is made secretly, so that you don't know whether the rope will 
hold your weight.

\textit{Bind a Character: }When you bind another character with a rope, any Escape 
Artist check that the bound character makes is opposed by your Use Rope check.

You get a +10 bonus on this check because it is easier to bind someone than to 
escape from bonds. You don't even make your Use Rope check until someone tries 
to escape.

\textbf{Action:} Varies. Throwing a grappling hook is a standard action that provokes 
an attack of opportunity. Tying a knot, tying a special knot, or tying a rope around 
yourself one-handed is a full-round action that provokes an attack of opportunity. 
Splicing two ropes together takes 5 minutes. Binding a character takes 1 minute.

\textbf{Special:} A silk rope gives you a +2 circumstance bonus on Use Rope checks. 
If you cast an \textit{animate rope }spell on a rope, you get a +2 circumstance 
bonus on any Use Rope checks you make when using that rope.

These bonuses stack.

If you have the Deft Hands feat, you get a +2 bonus on Use Rope checks.

\textbf{Synergy:} If you have 5 or more ranks in Use Rope, you get a +2 bonus on 
Climb checks made to climb a rope, a knotted rope, or a rope-and-wall combination.

If you have 5 or more ranks in Use Rope, you get a +2 bonus on Escape Artist checks 
when escaping from rope bonds.

If you have 5 or more ranks in Escape Artist, you get a +2 bonus on checks made 
to bind someone.

\newpage

\end{document}
