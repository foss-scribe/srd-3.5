%&pdfLaTeX
% !TEX encoding = UTF-8 Unicode
\documentclass{article}
\usepackage{ifxetex}
\ifxetex
\usepackage{fontspec}
\setmainfont[Mapping=tex-text]{STIXGeneral}
\else
\usepackage[T1]{fontenc}
\usepackage[utf8]{inputenc}
\fi
\usepackage{textcomp}

\usepackage{array}
\usepackage{amssymb}
\usepackage{fancyhdr}
\renewcommand{\headrulewidth}{0pt}
\renewcommand{\footrulewidth}{0pt}

\begin{document}

This material is Open Game Content, and is licensed for public use under the terms 
of the Open Game License v1.0a.

{\LARGE{}FEATS}

\vspace{12pt}
{\LARGE{}PREREQUISITES}

Some feats have prerequisites. Your character must have the indicated ability score, 
class feature, feat, skill, base attack bonus, or other quality designated in order 
to select or use that feat. A character can gain a feat at the same level at which 
he or she gains the prerequisite.

A character can't use a feat if he or she has lost a prerequisite.

\vspace{12pt}
{\LARGE{}TYPES OF FEATS}

Some feats are general, meaning that no special rules govern them as a group. Others 
are item creation feats, which allow spellcasters to create magic items of all 
sorts. A metamagic feat lets a spellcaster prepare and cast a spell with greater 
effect, albeit as if the spell were a higher spell level than it actually is.

\vspace{12pt}
FIGHTER BONUS FEATS

Any feat designated as a fighter feat can be selected as a fighter's bonus feat. 
This designation does not restrict characters of other classes from selecting these 
feats, assuming that they meet any prerequisites.

\vspace{12pt}
ITEM CREATION FEATS

An item creation feat lets a spellcaster create a magic item of a certain type. 
Regardless of the type of items they involve, the various item creation feats all 
have certain features in common.

\textbf{XP Cost:} Experience that the spellcaster would normally keep is expended 
when making a magic item. The XP cost equals 1/25 of the cost of the item in gold 
pieces. A character cannot spend so much XP on an item that he or she loses a level. 
However, upon gaining enough XP to attain a new level, he or she can immediately 
expend XP on creating an item rather than keeping the XP to advance a level.

\textbf{Raw Materials Cost:} The cost of creating a magic item equals one-half 
the sale cost of the item.

Using an item creation feat also requires access to a laboratory or magical workshop, 
special tools, and so on. A character generally has access to what he or she needs 
unless unusual circumstances apply.

\textbf{Time:} The time to create a magic item depends on the feat and the cost 
of the item. The minimum time is one day.

\textbf{Item Cost:} Brew Potion, Craft Wand, and Scribe Scroll create items that 
directly reproduce spell effects, and the power of these items depends on their 
caster level---that is, a spell from such an item has the power it would have if 
cast by a spellcaster of that level. The price of these items (and thus the XP 
cost and the cost of the raw materials) also depends on the caster level. The caster 
level must be high enough that the spellcaster creating the item can cast the spell 
at that level. To find the final price in each case, multiply the caster level 
by the spell level, then multiply the result by a constant, as shown below:

\textit{Scrolls: }Base price = spell level x caster level x $ $25 gp.

\textit{Potions: }Base price = spell level x $ $caster level x $ $50 gp.

\textit{Wands: }Base price = spell level x $ $caster level x $ $750 gp.

A 0-level spell is considered to have a spell level of 1/2 for the purpose of this 
calculation.

\textbf{Extra Costs: }Any potion, scroll, or wand that stores a spell with a costly 
material component or an XP cost also carries a commensurate cost. For potions 
and scrolls, the creator must expend the material component or pay the XP cost 
when creating the item.

For a wand, the creator must expend fifty copies of the material component or pay 
fifty times the XP cost.

Some magic items similarly incur extra costs in material components or XP, as noted 
in their descriptions.

\vspace{12pt}
METAMAGIC FEATS

As a spellcaster's knowledge of magic grows, she can learn to cast spells in ways 
slightly different from the ways in which the spells were originally designed or 
learned. Preparing and casting a spell in such a way is harder than normal but, 
thanks to metamagic feats, at least it is possible.  Spells modified by a metamagic 
feat use a spell slot higher than normal. This does not change the level of the 
spell, so the DC for saving throws against it does not go up.

\textbf{Wizards and Divine Spellcasters:} Wizards and divine spellcasters must 
prepare their spells in advance. During preparation, the character chooses which 
spells to prepare with metamagic feats (and thus which ones take up higher-level 
spell slots than normal).

\textbf{Sorcerers and Bards:} Sorcerers and bards choose spells as they cast them. 
They can choose when they cast their spells whether to apply their metamagic feats 
to improve them. As with other spellcasters, the improved spell uses up a higher-level 
spell slot. But because the sorcerer or bard has not prepared the spell in a metamagic 
form in advance, he must apply the metamagic feat on the spot. Therefore, such 
a character must also take more time to cast a metamagic spell (one enhanced by 
a metamagic feat) than he does to cast a regular spell. If the spell's normal casting 
time is 1 action, casting a metamagic version is a full-round action for a sorcerer 
or bard. (This isn't the same as a 1-round casting time.)

For a spell with a longer casting time, it takes an extra full-round action to 
cast the spell.

\textbf{Spontaneous Casting and Metamagic Feats: }A cleric spontaneously casting 
a \textit{cure }or \textit{inflict }spell can cast a metamagic version of it instead. 
Extra time is also required in this case. Casting a 1-action metamagic spell spontaneously 
is a full-round action, and a spell with a longer casting time takes an extra full-round 
action to cast.

\textbf{Effects of Metamagic Feats on a Spell:} In all ways, a metamagic spell 
operates at its original spell level, even though it is prepared and cast as a 
higher-level spell. Saving throw modifications are not changed unless stated otherwise 
in the feat description.

The modifications made by these feats only apply to spells cast directly by the 
feat user. A spellcaster can't use a metamagic feat to alter a spell being cast 
from a wand, scroll, or other device.

Metamagic feats that eliminate components of a spell don't eliminate the attack 
of opportunity provoked by casting a spell while threatened. However, casting a 
spell modified by Quicken Spell does not provoke an attack of opportunity.

Metamagic feats cannot be used with all spells. See the specific feat descriptions 
for the spells that a particular feat can't modify.

\vspace{12pt}
\textbf{Multiple Metamagic Feats on a Spell:} A spellcaster can apply multiple 
metamagic feats to a single spell. Changes to its level are cumulative. You can't 
apply the same metamagic feat more than once to a single spell.

\vspace{12pt}
\textbf{Magic Items and Metamagic Spells:} With the right item creation feat, you 
can store a metamagic version of a spell in a scroll, potion, or wand. Level limits 
for potions and wands apply to the spell's higher spell level (after the application 
of the metamagic feat). A character doesn't need the metamagic feat to activate 
an item storing a metamagic version of a spell.

\textbf{Counterspelling Metamagic Spells:} Whether or not a spell has been enhanced 
by a metamagic feat does not affect its vulnerability to counterspelling or its 
ability to counterspell another spell.

\vspace{12pt}
{\LARGE{}FEAT DESCRIPTIONS}

Here is the format for feat descriptions.

\vspace{12pt}
FEAT NAME [TYPE OF FEAT]

\textbf{Prerequisite:} A minimum ability score, another feat or feats, a minimum 
base attack bonus, a minimum number of ranks in one or more skills, or a class 
level that a character must have in order to acquire this feat. This entry is absent 
if a feat has no prerequisite. A feat may have more than one prerequisite.

\textbf{Benefit:} What the feat enables the character (``you'' in the feat description) 
to do. If a character has the same feat more than once, its benefits do not stack 
unless indicated otherwise in the description.

In general, having a feat twice is the same as having it once.

\textbf{Normal:} What a character who does not have this feat is limited to or 
restricted from doing. If not having the feat causes no particular drawback, this 
entry is absent.

\textbf{Special: }Additional facts about the feat that may be helpful when you 
decide whether to acquire the feat.

\vspace{24pt}
ACROBATIC [GENERAL]

\textbf{Benefit:} You get a +2 bonus on all Jump checks and Tumble checks.

\vspace{12pt}
AGILE [GENERAL]

\textbf{Benefit:} You get a +2 bonus on all Balance checks and Escape Artist checks.

\vspace{12pt}
ALERTNESS [GENERAL]

\textbf{Benefit:} You get a +2 bonus on all Listen checks and Spot checks.

\textbf{Special: }The master of a familiar gains the benefit of the Alertness feat 
whenever the familiar is within arm's reach.

\vspace{12pt}
ANIMAL AFFINITY [GENERAL]

\textbf{Benefit:} You get a +2 bonus on all Handle Animal checks and Ride checks.

\vspace{12pt}
ARMOR PROFICIENCY (HEAVY) [GENERAL]

\textbf{Prerequisites:} Armor Proficiency (light), Armor Proficiency (medium).

\textbf{Benefit:} See Armor Proficiency (light).

\textbf{Normal:} See Armor Proficiency (light).

\textbf{Special:} Fighters, paladins, and clerics automatically have Armor Proficiency 
(heavy) as a bonus feat. They need not select it.

\vspace{12pt}
ARMOR PROFICIENCY (LIGHT) [GENERAL]

\textbf{Benefit:} When you wear a type of armor with which you are proficient, 
the armor check penalty for that armor applies only to Balance, Climb, Escape Artist, 
Hide, Jump, Move Silently, Sleight of Hand, and Tumble checks.

\textbf{Normal:} A character who is wearing armor with which she is not proficient 
applies its armor check penalty to attack rolls and to all skill checks that involve 
moving, including Ride.

\textbf{Special:} All characters except wizards, sorcerers, and monks automatically 
have Armor Proficiency (light) as a bonus feat. They need not select it.

\vspace{12pt}
ARMOR PROFICIENCY (MEDIUM) [GENERAL]

\textbf{Prerequisite:} Armor Proficiency (light).

\textbf{Benefit:} See Armor Proficiency (light).

\textbf{Normal:} See Armor Proficiency (light).

\textbf{Special:} Fighters, barbarians, paladins, clerics, druids, and bards automatically 
have Armor Proficiency (medium) as a bonus feat. They need not select it.

\vspace{12pt}
ATHLETIC [GENERAL]

\textbf{Benefit:} You get a +2 bonus on all Climb checks and Swim checks.

\vspace{12pt}
AUGMENT SUMMONING [GENERAL]

\textbf{Prerequisite:} Spell Focus (conjuration).

\textbf{Benefit:} Each creature you conjure with any \textit{summon }spell gains 
a +4 enhancement bonus to Strength and Constitution for the duration of the spell 
that summoned it.

\vspace{12pt}
BLIND-FIGHT [GENERAL]

\textbf{Benefit:} In melee, every time you miss because of concealment, you can 
reroll your miss chance percentile roll one time to see if you actually hit.

An invisible attacker gets no advantages related to hitting you in melee. That 
is, you don't lose your Dexterity bonus to Armor Class, and the attacker doesn't 
get the usual +2 bonus for being invisible. The invisible attacker's bonuses do 
still apply for ranged attacks, however.

You take only half the usual penalty to speed for being unable to see. Darkness 
and poor visibility in general reduces your speed to three-quarters normal, instead 
of one-half.

\textbf{Normal:} Regular attack roll modifiers for invisible attackers trying to 
hit you apply, and you lose your Dexterity bonus to AC. The speed reduction for 
darkness and poor visibility also applies.

\textbf{Special:} The Blind-Fight feat is of no use against a character who is 
the subject of a \textit{blink }spell.

A fighter may select Blind-Fight as one of his fighter bonus feats.

\vspace{12pt}
BREW POTION [ITEM CREATION]

\textbf{Prerequisite:} Caster level 3rd.

\textbf{Benefit:} You can create a potion of any 3rd-level or lower spell that 
you know and that targets one or more creatures. Brewing a potion takes one day. 
When you create a potion, you set the caster level, which must be sufficient to 
cast the spell in question and no higher than your own level. The base price of 
a potion is its spell level x $ $its caster level x $ $50 gp. To brew a potion, 
you must spend 1/25 of this base price in XP and use up raw materials costing one 
half this base price.

When you create a potion, you make any choices that you would normally make when 
casting the spell. Whoever drinks the potion is the target of the spell.

Any potion that stores a spell with a costly material component or an XP cost also 
carries a commensurate cost. In addition to the costs derived from the base price, 
you must expend the material component or pay the XP when creating the potion.

\vspace{12pt}
CLEAVE [GENERAL]

\textbf{Prerequisites:} Str 13, Power Attack.

\textbf{Benefit:} If you deal a creature enough damage to make it drop (typically 
by dropping it to below 0 hit points or killing it), you get an immediate, extra 
melee attack against another creature within reach. You cannot take a 5-foot step 
before making this extra attack. The extra attack is with the same weapon and at 
the same bonus as the attack that dropped the previous creature. You can use this 
ability once per round.

\textbf{Special:} A fighter may select Cleave as one of his fighter bonus feats.

\vspace{12pt}
COMBAT CASTING [GENERAL]

\textbf{Benefit:} You get a +4 bonus on Concentration checks made to cast a spell 
or use a spell-like ability while on the defensive or while you are grappling or 
pinned.

\vspace{12pt}
COMBAT EXPERTISE [GENERAL]

\textbf{Prerequisite:} Int 13.

\textbf{Benefit:} When you use the attack action or the full attack action in melee, 
you can take a penalty of as much as -5 on your attack roll and add the same number 
(+5 or less) as a dodge bonus to your Armor Class. This number may not exceed your 
base attack bonus. The changes to attack rolls and Armor Class last until your 
next action.

\textbf{Normal:} A character without the Combat Expertise feat can fight defensively 
while using the attack or full attack action to take a -4 penalty on attack rolls 
and gain a +2 dodge bonus to Armor Class.

\textbf{Special:} A fighter may select Combat Expertise as one of his fighter bonus 
feats.

\vspace{12pt}
COMBAT REFLEXES [GENERAL]

\textbf{Benefit:} You may make a number of additional attacks of opportunity equal 
to your Dexterity bonus.

With this feat, you may also make attacks of opportunity while flat-footed.

\textbf{Normal:} A character without this feat can make only one attack of opportunity 
per round and can't make attacks of opportunity while flat-footed.

\textbf{Special:} The Combat Reflexes feat does not allow a rogue to use her opportunist 
ability more than once per round.

A fighter may select Combat Reflexes as one of his fighter bonus feats.

A monk may select Combat Reflexes as a bonus feat at 2nd level.

\vspace{12pt}
CRAFT MAGIC ARMS AND ARMOR [ITEM CREATION]

\textbf{Prerequisite:} Caster level 5th.

\textbf{Benefit:} You can create any magic weapon, armor, or shield whose prerequisites 
you meet. Enhancing a weapon, suit of armor, or shield takes one day for each 1,000 
gp in the price of its magical features. To enhance a weapon, suit of armor, or 
shield, you must spend 1/25 of its features' total price in XP and use up raw materials 
costing one-half of this total price.

The weapon, armor, or shield to be enhanced must be a masterwork item that you 
provide. Its cost is not included in the above cost.

You can also mend a broken magic weapon, suit of armor, or shield if it is one 
that you could make. Doing so costs half the XP, half the raw materials, and half 
the time it would take to craft that item in the first place.

\vspace{12pt}
CRAFT ROD [ITEM CREATION]

\textbf{Prerequisite:} Caster level 9th.

\textbf{Benefit:} You can create any rod whose prerequisites you meet. Crafting 
a rod takes one day for each 1,000 gp in its base price. To craft a rod, you must 
spend 1/25 of its base price in XP and use up raw materials costing one-half of 
its base price.

Some rods incur extra costs in material components or XP, as noted in their descriptions. 
These costs are in addition to those derived from the rod's base price.

\vspace{12pt}
CRAFT STAFF [ITEM CREATION]

\textbf{Prerequisite:} Caster level 12th.

\textbf{Benefit:} You can create any staff whose prerequisites you meet.

Crafting a staff takes one day for each 1,000 gp in its base price. To craft a 
staff, you must spend 1/25 of its base price in XP and use up raw materials costing 
one-half of its base price. A newly created staff has 50 charges.

Some staffs incur extra costs in material components or XP, as noted in their descriptions. 
These costs are in addition to those derived from the staff 's base price.

\vspace{12pt}
CRAFT WAND [ITEM CREATION]

\textbf{Prerequisite:} Caster level 5th.

\textbf{Benefit:} You can create a wand of any 4th-level or lower spell that you 
know. Crafting a wand takes one day for each 1,000 gp in its base price. The base 
price of a wand is its caster level x $ $the spell level x 750 gp. To craft a wand, 
you must spend 1/25 of this base price in XP and use up raw materials costing one-half 
of this base price. A newly created wand has 50 charges.

Any wand that stores a spell with a costly material component or an XP cost also 
carries a commensurate cost. In addition to the cost derived from the base price, 
you must expend fifty copies of the material component or pay fifty times the XP 
cost.

\vspace{12pt}
CRAFT WONDROUS ITEM [ITEM CREATION]

\textbf{Prerequisite:} Caster level 3rd.

\textbf{Benefit:} You can create any wondrous item whose prerequisites you meet. 
Enchanting a wondrous item takes one day for each 1,000 gp in its price. To enchant 
a wondrous item, you must spend 1/25 of the item's price in XP and use up raw materials 
costing half of this price.

You can also mend a broken wondrous item if it is one that you could make. Doing 
so costs half the XP, half the raw materials, and half the time it would take to 
craft that item in the first place.

Some wondrous items incur extra costs in material components or XP, as noted in 
their descriptions. These costs are in addition to those derived from the item's 
base price. You must pay such a cost to create an item or to mend a broken one.

\vspace{12pt}
DECEITFUL [GENERAL]

\textbf{Benefit:} You get a +2 bonus on all Disguise checks and Forgery checks.

\vspace{12pt}
DEFLECT ARROWS [GENERAL]

\textbf{Prerequisites:} Dex 13, Improved Unarmed Strike.

\textbf{Benefit:} You must have at least one hand free (holding nothing) to use 
this feat. Once per round when you would normally be hit with a ranged weapon, 
you may deflect it so that you take no damage from it. You must be aware of the 
attack and not flatfooted.

Attempting to deflect a ranged weapon doesn't count as an action. Unusually massive 
ranged weapons and ranged attacks generated by spell effects can't be deflected.

\textbf{Special:} A monk may select Deflect Arrows as a bonus feat at 2nd level, 
even if she does not meet the prerequisites.

A fighter may select Deflect Arrows as one of his fighter bonus feats.

\vspace{12pt}
DEFT HANDS [GENERAL]

\textbf{Benefit:} You get a +2 bonus on all Sleight of Hand checks and Use Rope 
checks.

\vspace{12pt}
DIEHARD [GENERAL]

\textbf{Prerequisite:} Endurance.

\textbf{Benefit:} When reduced to between -1 and -9 hit points, you automatically 
become stable. You don't have to roll d\% to see if you lose 1 hit point each round.

When reduced to negative hit points, you may choose to act as if you were disabled, 
rather than dying. You must make this decision as soon as you are reduced to negative 
hit points (even if it isn't your turn). If you do not choose to act as if you 
were disabled, you immediately fall unconscious.

When using this feat, you can take either a single move or standard action each 
turn, but not both, and you cannot take a full round action. You can take a move 
action without further injuring yourself, but if you perform any standard action 
(or any other action deemed as strenuous, including some free actions, such as 
casting a quickened spell) you take 1 point of damage after completing the act. 
If you reach -10 hit points, you immediately die.

\textbf{Normal:} A character without this feat who is reduced to between -1 and 
-9 hit points is unconscious and dying.

\vspace{12pt}
DILIGENT [GENERAL]

\textbf{Benefit:} You get a +2 bonus on all Appraise checks and Decipher Script 
checks.

\vspace{12pt}
DODGE [GENERAL]

\textbf{Prerequisite:} Dex 13.

\textbf{Benefit:} During your action, you designate an opponent and receive a +1 
dodge bonus to Armor Class against attacks from that opponent. You can select a 
new opponent on any action.

A condition that makes you lose your Dexterity bonus to Armor Class (if any) also 
makes you lose dodge bonuses. Also, dodge bonuses stack with each other, unlike 
most other types of bonuses.

\textbf{Special:} A fighter may select Dodge as one of his fighter bonus feats.

\vspace{12pt}
EMPOWER SPELL [METAMAGIC]

\textbf{Benefit:} All variable, numeric effects of an empowered spell are increased 
by one-half.

Saving throws and opposed rolls are not affected, nor are spells without random 
variables. An empowered spell uses up a spell slot two levels higher than the spell's 
actual level.

\vspace{12pt}
ENDURANCE [GENERAL]

\textbf{Benefit:} You gain a +4 bonus on the following checks and saves: Swim checks 
made to resist nonlethal damage, Constitution checks made to continue running, 
Constitution checks made to avoid nonlethal damage from a forced march, Constitution 
checks made to hold your breath, Constitution checks made to avoid nonlethal damage 
from starvation or thirst, Fortitude saves made to avoid nonlethal damage from 
hot or cold environments, and Fortitude saves made to resist damage from suffocation. 
Also, you may sleep in light or medium armor without becoming fatigued.

\textbf{Normal:} A character without this feat who sleeps in medium or heavier 
armor is automatically fatigued the next day.

\textbf{Special:} A ranger automatically gains Endurance as a bonus feat at 3rd 
level. He need not select it.

\vspace{12pt}
ENLARGE SPELL [METAMAGIC]

\textbf{Benefit:} You can alter a spell with a range of close, medium, or long 
to increase its range by 100\%. An enlarged spell with a range of close now has 
a range of 50 ft. + 5 ft./level, while medium-range spells have a range of 200 
ft. + 20 ft./level and long-range spells have a range of 800 ft. + 80 ft./level. 
An enlarged spell uses up a spell slot one level higher than the spell's actual 
level.

Spells whose ranges are not defined by distance, as well as spells whose ranges 
are not close, medium, or long, do not have increased ranges.

\vspace{12pt}
ESCHEW MATERIALS [GENERAL]

\textbf{Benefit:} You can cast any spell that has a material component costing 
1 gp or less without needing that component. (The casting of the spell still provokes 
attacks of opportunity as normal.) If the spell requires a material component that 
costs more than 1 gp, you must have the material component at hand to cast the 
spell, just as normal.

\vspace{12pt}
EXOTIC WEAPON PROFICIENCY [GENERAL]

Choose a type of exotic weapon. You understand how to use that type of exotic weapon 
in combat.

\textbf{Prerequisite:} Base attack bonus +1 (plus Str 13 for bastard sword or dwarven 
waraxe).

\textbf{Benefit:} You make attack rolls with the weapon normally.

\textbf{Normal:} A character who uses a weapon with which he or she is not proficient 
takes a -4 penalty on attack rolls.

\textbf{Special:} You can gain Exotic Weapon Proficiency multiple times. Each time 
you take the feat, it applies to a new type of exotic weapon. Proficiency with 
the bastard sword or the dwarven waraxe has an additional prerequisite of Str 13.

A fighter may select Exotic Weapon Proficiency as one of his fighter bonus feats.

\vspace{12pt}
EXTEND SPELL [METAMAGIC]

\textbf{Benefit:} An extended spell lasts twice as long as normal. A spell with 
a duration of concentration, instantaneous, or permanent is not affected by this 
feat. An extended spell uses up a spell slot one level higher than the spell's 
actual level.

\vspace{12pt}
EXTRA TURNING [GENERAL]

\textbf{Prerequisite:} Ability to turn or rebuke creatures.

\textbf{Benefit:} Each time you take this feat, you can use your ability to turn 
or rebuke creatures four more times per day than normal.

If you have the ability to turn or rebuke more than one kind of creature each of 
your turning or rebuking abilities gains four additional uses per day.

\textbf{Normal:} Without this feat, a character can typically turn or rebuke undead 
(or other creatures) a number of times per day equal to 3 + his or her Charisma 
modifier.

\textbf{Special:} You can gain Extra Turning multiple times. Its effects stack. 
Each time you take the feat, you can use each of your turning or rebuking abilities 
four additional times per day.

\vspace{12pt}
FAR SHOT [GENERAL]

\textbf{Prerequisite:} Point Blank Shot.

\textbf{Benefit:} When you use a projectile weapon, such as a bow, its range increment 
increases by one-half (multiply by 1-1/2). When you use a thrown weapon, its range 
increment is doubled.

\textbf{Special:} A fighter may select Far Shot as one of his fighter bonus feats.

\vspace{12pt}
FORGE RING [ITEM CREATION]

\textbf{Prerequisite:} Caster level 12th.

\textbf{Benefit:} You can create any ring whose prerequisites you meet. Crafting 
a ring takes one day for each 1,000 gp in its base price. To craft a ring, you 
must spend 1/25 of its base price in XP and use up raw materials costing one-half 
of its base price.

You can also mend a broken ring if it is one that you could make. Doing so costs 
half the XP, half the raw materials, and half the time it would take to forge that 
ring in the first place.

Some magic rings incur extra costs in material components or XP, as noted in their 
descriptions. You must pay such a cost to forge such a ring or to mend a broken 
one.

\vspace{12pt}
GREAT CLEAVE [GENERAL]

\textbf{Prerequisites:} Str 13, Cleave, Power Attack, base attack bonus +4.

\textbf{Benefit:} This feat works like Cleave, except that there is no limit to 
the number of times you can use it per round.

\textbf{Special:} A fighter may select Great Cleave as one of his fighter bonus 
feats.

\vspace{12pt}
GREAT FORTITUDE [GENERAL]

\textbf{Benefit:} You get a +2 bonus on all Fortitude saving throws.

\vspace{12pt}
GREATER SPELL FOCUS [GENERAL]

Choose a school of magic to which you already have applied the Spell Focus feat.

\textbf{Benefit:} Add +1 to the Difficulty Class for all saving throws against 
spells from the school of magic you select. This bonus stacks with the bonus from 
Spell Focus.

\textbf{Special:} You can gain this feat multiple times. Its effects do not stack. 
Each time you take the feat, it applies to a new school of magic to which you already 
have applied the Spell Focus feat.

\vspace{12pt}
GREATER SPELL PENETRATION [GENERAL]

\textbf{Prerequisite:} Spell Penetration.

\textbf{Benefit:} You get a +2 bonus on caster level checks (1d20 + caster level) 
made to overcome a creature's spell resistance. This bonus stacks with the one 
from Spell Penetration.

\vspace{12pt}
GREATER TWO-WEAPON FIGHTING [GENERAL]

\textbf{Prerequisites:} Dex 19, Improved Two-Weapon Fighting, Two-Weapon Fighting, 
base attack bonus +11.

\textbf{Benefit:} You get a third attack with your off-hand weapon, albeit at a 
-10 penalty.

\textbf{Special:} A fighter may select Greater Two-Weapon Fighting as one of his 
fighter bonus feats.

An 11th-level ranger who has chosen the two-weapon combat style is treated as having 
Greater Two-Weapon Fighting, even if he does not have the prerequisites for it, 
but only when he is wearing light or no armor.

\vspace{12pt}
GREATER WEAPON FOCUS [GENERAL]

Choose one type of weapon for which you have already selected Weapon Focus. You 
can also choose unarmed strike or grapple as your weapon for purposes of this feat. 

\parindent=3pt
\textbf{Prerequisites:} Proficiency with selected weapon, Weapon Focus with selected 
weapon, fighter level 8th.

\parindent=0pt
\textbf{Benefit:} You gain a +1 bonus on all attack rolls you make using the selected 
weapon. This bonus stacks with other bonuses on attack rolls, including the one 
from Weapon Focus (see below).

\textbf{Special:} You can gain Greater Weapon Focus multiple times. Its effects 
do not stack. Each time you take the feat, it applies to a new type of weapon.

A fighter must have Greater Weapon Focus with a given weapon to gain the Greater 
Weapon Specialization feat for that weapon. 

A fighter may select Greater Weapon Focus as one of his fighter bonus feats.

\vspace{12pt}
GREATER WEAPON SPECIALIZATION [GENERAL]

Choose one type of weapon for which you have already selected Weapon Specialization. 
You can also choose unarmed strike or grapple as your weapon for purposes of this 
feat. 

\parindent=3pt
\textbf{Prerequisites:} Proficiency with selected weapon, Greater Weapon Focus 
with selected weapon, Weapon Focus with selected weapon, Weapon Specialization 
with selected weapon, fighter level 12th.

\parindent=0pt
\textbf{Benefit:} You gain a +2 bonus on all damage rolls you make using the selected 
weapon. This bonus stacks with other bonuses on damage rolls, including the one 
from Weapon Specialization (see below).

\textbf{Special:} You can gain Greater Weapon Specialization multiple times. Its 
effects do not stack. Each time you take the feat, it applies to a new type of 
weapon.

A fighter may select Greater Weapon Specialization as one of his fighter bonus 
feats.

\vspace{12pt}
HEIGHTEN SPELL [METAMAGIC]

\textbf{Benefit:} A heightened spell has a higher spell level than normal (up to 
a maximum of 9th level). Unlike other metamagic feats, Heighten Spell actually 
increases the effective level of the spell that it modifies. All effects dependent 
on spell level (such as saving throw DCs and ability to penetrate a \textit{lesser 
globe of invulnerability}) are calculated according to the heightened level. The 
heightened spell is as difficult to prepare and cast as a spell of its effective 
level. 

\vspace{12pt}
IMPROVED BULL RUSH [GENERAL]

\textbf{Prerequisites:} Str 13, Power Attack.

\textbf{Benefit:} When you perform a bull rush you do not provoke an attack of 
opportunity from the defender. You also gain a +4 bonus on the opposed Strength 
check you make to push back the defender.

\textbf{Special:} A fighter may select Improved Bull Rush as one of his fighter 
bonus feats.

\vspace{12pt}
IMPROVED COUNTERSPELL [GENERAL]

\textbf{Benefit: }When counterspelling, you may use a spell of the same school 
that is one or more spell levels higher than the target spell.

\textbf{Normal:} Without this feat, you may counter a spell only with the same 
spell or with a spell specifically designated as countering the target spell.

\vspace{12pt}
IMPROVED CRITICAL [GENERAL]

Choose one type of weapon.

\textbf{Prerequisite:} Proficient with weapon, base attack bonus +8.

\textbf{Benefit:} When using the weapon you selected, your threat range is doubled.

\textbf{Special:} You can gain Improved Critical multiple times. The effects do 
not stack. Each time you take the feat, it applies to a new type of weapon.

This effect doesn't stack with any other effect that expands the threat range of 
a weapon.

A fighter may select Improved Critical as one of his fighter bonus feats.

\vspace{12pt}
IMPROVED DISARM [GENERAL]

\textbf{Prerequisites:} Int 13, Combat Expertise.

\textbf{Benefit:} You do not provoke an attack of opportunity when you attempt 
to disarm an opponent, nor does the opponent have a chance to disarm you. You also 
gain a +4 bonus on the opposed attack roll you make to disarm your opponent.

\textbf{Normal:} See the normal disarm rules.

\textbf{Special:} A fighter may select Improved Disarm as one of his fighter bonus 
feats.

A monk may select Improved Disarm as a bonus feat at 6th level, even if she does 
not meet the prerequisites.

\vspace{12pt}
IMPROVED FAMILIAR [GENERAL]

This feat allows spellcasters to acquire a new familiar from a nonstandard list, 
but only when they could normally acquire a new familiar.

\textbf{Prerequisites:} Ability to acquire a new familiar, compatible alignment, 
sufficiently high level (see below).

\textbf{Benefit:} When choosing a familiar, the creatures listed below are also 
available to the spellcaster. The spellcaster may choose a familiar with an alignment 
up to one step away on each of the alignment axes (lawful through chaotic, good 
through evil).

\vspace{12pt}
\begin{tabular}{|>{\raggedright}p{63pt}|>{\raggedright}p{57pt}|>{\raggedright}p{72pt}|}
\hline
F\textbf{amiliar } & A\textbf{lignment } & A\textbf{rcane Spellcaster Level}\tabularnewline
\hline
Shocker lizard  & Neutral  & 5th\tabularnewline
\hline
Stirge  & Neutral  & 5th\tabularnewline
\hline
Formian worker  & Lawful neutral  & 7th\tabularnewline
\hline
Imp  & Lawful evil  & 7th\tabularnewline
\hline
Pseudodragon  & Neutral good  & 7th\tabularnewline
\hline
Quasit  & Chaotic evil  & 7th\tabularnewline
\hline
\end{tabular}

\vspace{12pt}
Improved familiars otherwise use the rules for regular familiars\textit{, }with 
two exceptions: If the creature's type is something other than animal, its type 
does not change; and improved familiars do not gain the ability to speak with other 
creatures of their kind (although many of them already have the ability to communicate).

\vspace{12pt}
The list in the table above presents only a few possible improved familiars. Almost 
any creature of the same general size and power as those on the list makes a suitable 
familiar. Nor is the master's alignment the only possible categorization. For instance, 
improved familiars could be assigned by the master's creature type or subtype, 
as shown below.

\vspace{12pt}
\begin{tabular}{|>{\raggedright}p{159pt}|>{\raggedright}p{63pt}|>{\raggedright}p{79pt}|}
\hline
F\textbf{amiliar} & T\textbf{ype/Subtype} & A\textbf{rcane Spellcaster Level}\tabularnewline
\hline
Celestial hawk\textsuperscript{\textbf{1}} & Good & 3rd\tabularnewline
\hline
Fiendish Tiny viper snake\textsuperscript{\textbf{2}} & Evil & 3rd\tabularnewline
\hline
Air elemental, Small & Air & 5th\tabularnewline
\hline
Earth elemental, Small & Earth & 5th\tabularnewline
\hline
Fire elemental, Small & Fire & 5th\tabularnewline
\hline
Shocker lizard & Electricity & 5th\tabularnewline
\hline
Water elemental, Small & Water & 5th\tabularnewline
\hline
Homunculus\textsuperscript{\textbf{3}} & Undead & 7th\tabularnewline
\hline
Ice mephit & Cold & 7th\tabularnewline
\hline
\multicolumn{3}{|p{302pt}|}{1 Or other celestial animal from the standard familiar 
list.}\tabularnewline
\hline
\multicolumn{3}{|p{302pt}|}{2 Or other fiendish animal from the standard familiar 
list.}\tabularnewline
\hline
\multicolumn{3}{|p{302pt}|}{3 The master must first create the homunculus, substituting 
ichor or another part of the master's body for blood if necessary.}\tabularnewline
\hline
\end{tabular}

\vspace{12pt}
IMPROVED FEINT [GENERAL]

\textbf{Prerequisites:} Int 13, Combat Expertise.

\textbf{Benefit:} You can make a Bluff check to feint in combat as a move action.

\textbf{Normal:} Feinting in combat is a standard action.

A fighter may select Improved Feint as one of his fighter bonus feats.

\vspace{12pt}
IMPROVED GRAPPLE [GENERAL]

\textbf{Prerequisites:} Dex 13, Improved Unarmed Strike.

\textbf{Benefit:} You do not provoke an attack of opportunity when you make a touch 
attack to start a grapple. You also gain a +4 bonus on all grapple checks, regardless 
of whether you started the grapple.

\textbf{Normal:} Without this feat, you provoke an attack of opportunity when you 
make a touch attack to start a grapple.

\textbf{Special:} A fighter may select Improved Grapple as one of his fighter bonus 
feats.

A monk may select Improved Grapple as a bonus feat at 1st level, even if she does 
not meet the prerequisites.

\vspace{12pt}
IMPROVED INITIATIVE [GENERAL]

\textbf{Benefit:} You get a +4 bonus on initiative checks.

\textbf{Special:} A fighter may select Improved Initiative as one of his fighter 
bonus feats.

\vspace{12pt}
IMPROVED OVERRUN [GENERAL]

\textbf{Prerequisites:} Str 13, Power Attack.

\textbf{Benefit:} When you attempt to overrun an opponent, the target may not choose 
to avoid you. You also gain a +4 bonus on your Strength check to knock down your 
opponent.

\textbf{Normal:} Without this feat, the target of an overrun can choose to avoid 
you or to block you.

\textbf{Special:} A fighter may select Improved Overrun as one of his fighter bonus 
feats.

\vspace{12pt}
IMPROVED PRECISE SHOT [GENERAL]

\textbf{Prerequisites:} Dex 19, Point Blank Shot, Precise Shot, base attack bonus 
+11.

\textbf{Benefit:} Your ranged attacks ignore the AC bonus granted to targets by 
anything less than total cover, and the miss chance granted to targets by anything 
less than total concealment. Total cover and total concealment provide their normal 
benefits against your ranged attacks.

In addition, when you shoot or throw ranged weapons at a grappling opponent, you 
automatically strike at the opponent you have chosen.

\textbf{Normal:} See the normal rules on the effects of cover and concealment. 
Without this feat, a character who shoots or throws a ranged weapon at a target 
involved in a grapple must roll randomly to see which grappling combatant the attack 
strikes.

\textbf{Special:} A fighter may select Improved Precise Shot as one of his fighter 
bonus feats.

An 11th-level ranger who has chosen the archery combat style is treated as having 
Improved Precise Shot, even if he does not have the prerequisites for it, but only 
when he is wearing light or no armor.

\vspace{12pt}
IMPROVED SHIELD BASH [GENERAL]

\textbf{Prerequisite:} Shield Proficiency.

\textbf{Benefit:} When you perform a shield bash, you may still apply the shield's 
shield bonus to your AC.

\textbf{Normal:} Without this feat, a character who performs a shield bash loses 
the shield's shield bonus to AC until his or her next turn.

\textbf{Special:} A fighter may select Improved Shield Bash as one of his fighter 
bonus feats.

\vspace{12pt}
IMPROVED SUNDER [GENERAL]

\textbf{Prerequisites:} Str 13, Power Attack.

\textbf{Benefit:} When you strike at an object held or carried by an opponent (such 
as a weapon or shield), you do not provoke an attack of opportunity.

You also gain a +4 bonus on any attack roll made to attack an object held or carried 
by another character.

\textbf{Normal:} Without this feat, you provoke an attack of opportunity when you 
strike at an object held or carried by another character.

\textbf{Special:} A fighter may select Improved Sunder as one of his fighter bonus 
feats.

\vspace{12pt}
IMPROVED TRIP [GENERAL]

\textbf{Prerequisites:} Int 13, Combat Expertise.

\textbf{Benefit:} You do not provoke an attack of opportunity when you attempt 
to trip an opponent while you are unarmed. You also gain a +4 bonus on your Strength 
check to trip your opponent.

If you trip an opponent in melee combat, you immediately get a melee attack against 
that opponent as if you hadn't used your attack for the trip attempt. 

\textbf{Normal:} Without this feat, you provoke an attack of opportunity when you 
attempt to trip an opponent while you are unarmed.

\textbf{Special:} At 6th level, a monk may select Improved Trip as a bonus feat, 
even if she does not have the prerequisites.

A fighter may select Improved Trip as one of his fighter bonus feats.

\vspace{12pt}
IMPROVED TURNING [GENERAL]

\textbf{Prerequisite:} Ability to turn or rebuke creatures.

\textbf{Benefit:} You turn or rebuke creatures as if you were one level higher 
than you are in the class that grants you the ability.

\vspace{12pt}
IMPROVED TWO-WEAPON FIGHTING [GENERAL]

\textbf{Prerequisites:} Dex 17, Two-Weapon Fighting, base attack bonus +6.

\textbf{Benefit:} In addition to the standard single extra attack you get with 
an off-hand weapon, you get a second attack with it, albeit at a -5 penalty.

\textbf{Normal:} Without this feat, you can only get a single extra attack with 
an off-hand weapon.

\textbf{Special:} A fighter may select Improved Two-Weapon Fighting as one of his 
fighter bonus feats.

A 6th-level ranger who has chosen the two-weapon combat style is treated as having 
Improved Two-Weapon Fighting, even if he does not have the prerequisites for it, 
but only when he is wearing light or no armor.

\vspace{12pt}
IMPROVED UNARMED STRIKE [GENERAL]

\textbf{Benefit:} You are considered to be armed even when unarmed ---that is, 
you do not provoke attacks or opportunity from armed opponents when you attack 
them while unarmed. However, you still get an attack of opportunity against any 
opponent who makes an unarmed attack on you.

In addition, your unarmed strikes can deal lethal or nonlethal damage, at your 
option.

\textbf{Normal: }Without this feat, you are considered unarmed when attacking with 
an unarmed strike, and you can deal only nonlethal damage with such an attack.

\textbf{Special:} A monk automatically gains Improved Unarmed Strike as a bonus 
feat at 1st level. She need not select it. 

A fighter may select Improved Unarmed Strike as one of his fighter bonus feats.

\vspace{12pt}
INVESTIGATOR [GENERAL]

\textbf{Benefit:} You get a +2 bonus on all Gather Information checks and Search 
checks.

\vspace{12pt}
IRON WILL [GENERAL]

\textbf{Benefit:} You get a +2 bonus on all Will saving throws.

\vspace{12pt}
LEADERSHIP [GENERAL]

\textbf{Prerequisite:} Character level 6th.

\textbf{Benefits:} Having this feat enables the character to attract loyal companions 
and devoted followers, subordinates who assist her. See the table below for what 
sort of cohort and how many followers the character can recruit.

\textbf{Leadership Modifiers: }Several factors can affect a character's Leadership 
score, causing it to vary from the base score (character level + Cha modifier). 
A character's reputation (from the point of view of the cohort or follower he is 
trying to attract) raises or lowers his Leadership score:

\vspace{12pt}
\begin{tabular}{|>{\raggedright}p{93pt}|>{\raggedright}p{36pt}|}
\hline
L\textbf{eader's Reputation } & \subsubsection*{M\textbf{odifier}}\tabularnewline
\hline
Great renown  & +2\tabularnewline
\hline
Fairness and generosity  & +1\tabularnewline
\hline
Special power  & +1\tabularnewline
\hline
Failure - & 1\tabularnewline
\hline
Aloofness - & 1\tabularnewline
\hline
Cruelty - & 2\tabularnewline
\hline
\end{tabular}

\vspace{12pt}
Other modifiers may apply when the character tries to attract a cohort:

\begin{tabular}{|>{\raggedright}p{206pt}|>{\raggedright}p{36pt}|}
\hline
T\textbf{he Leader . . . } & M\textbf{odifier}\tabularnewline
\hline
Has a familiar, special mount, or animal companion - & 2\tabularnewline
\hline
Recruits a cohort of a different alignment- & 1\tabularnewline
\hline
Caused the death of a cohort - & 2*\tabularnewline
\hline
\multicolumn{2}{|p{242pt}|}{* Cumulative per cohort killed.}\tabularnewline
\hline
\end{tabular}

\vspace{12pt}
Followers have different priorities from cohorts. When the character tries to attract 
a new follower, use any of the following modifiers that apply.

\begin{tabular}{|>{\raggedright}p{238pt}|>{\raggedright}p{36pt}|}
\hline
T\textbf{he Leader . . . } & M\textbf{odifier}\tabularnewline
\hline
Has a stronghold, base of operations, guildhouse, or the like & +2\tabularnewline
\hline
Moves around a lot - & 1\tabularnewline
\hline
Caused the death of other followers - & 1.\tabularnewline
\hline
\end{tabular}

\vspace{12pt}
\begin{tabular}{|>{\raggedright}p{56pt}|>{\raggedright}p{33pt}|>{\raggedright}p{19pt}|>{\raggedright}p{19pt}|>{\raggedright}p{19pt}|>{\raggedright}p{19pt}|>{\raggedright}p{19pt}|>{\raggedright}p{19pt}|}
\hline
\subsection*{L\textbf{eadership Score}} & \subsection*{C\textbf{ohort Level}}------ & \multicolumn{6}{p{117pt}|}{\subsection*{ 
\textbf{Number of Followers by Level ------}}}\tabularnewline
\hline
\subsection*{} & \subsection*{} & 1\textbf{st} & 2\textbf{nd} & 3\textbf{rd} & 4\textbf{th} & 5\textbf{th} & 6\textbf{th}\tabularnewline
\hline
1 or lower--- & --- & --- & --- & --- & --- & --- & \tabularnewline
\hline
2 & 1st--- & --- & --- & --- & --- & --- & \tabularnewline
\hline
3 & 2nd--- & --- & --- & --- & --- & --- & \tabularnewline
\hline
4 & 3rd--- & --- & --- & --- & --- & --- & \tabularnewline
\hline
5 & 3rd--- & --- & --- & --- & --- & --- & \tabularnewline
\hline
6 & 4th--- & --- & --- & --- & --- & --- & \tabularnewline
\hline
7 & 5th--- & --- & --- & --- & --- & --- & \tabularnewline
\hline
8 & 5th--- & --- & --- & --- & --- & --- & \tabularnewline
\hline
9 & 6th--- & --- & --- & --- & --- & --- & \tabularnewline
\hline
10 & 7th & 5--- & --- & --- & --- & --- & \tabularnewline
\hline
11 & 7th & 6--- & --- & --- & --- & --- & \tabularnewline
\hline
12 & 8th & 8--- & --- & --- & --- & --- & \tabularnewline
\hline
13 & 9th & 10 & 1--- & --- & --- & --- & \tabularnewline
\hline
14 & 10th & 15 & 1--- & --- & --- & --- & \tabularnewline
\hline
15 & 10th & 20 & 2 & 1--- & --- & --- & \tabularnewline
\hline
16 & 11th & 25 & 2 & 1--- & --- & --- & \tabularnewline
\hline
17 & 12th & 30 & 3 & 1 & 1--- & --- & \tabularnewline
\hline
18 & 12th & 35 & 3 & 1 & 1--- & --- & \tabularnewline
\hline
19 & 13th & 40 & 4 & 2 & 1 & 1--- & \tabularnewline
\hline
20 & 14th & 50 & 5 & 3 & 2 & 1--- & \tabularnewline
\hline
21 & 15th & 60 & 6 & 3 & 2 & 1 & 1\tabularnewline
\hline
22 & 15th & 75 & 7 & 4 & 2 & 2 & 1\tabularnewline
\hline
23 & 16th & 90 & 9 & 5 & 3 & 2 & 1\tabularnewline
\hline
24 & 17th & 110 & 11 & 6 & 3 & 2 & 1\tabularnewline
\hline
25 or higher & 17th & 135 & 13 & 7 & 4 & 2 & 2\tabularnewline
\hline
\end{tabular}

\vspace{12pt}
\textit{Leadership Score: }A character's base Leadership score equals his level 
plus any Charisma modifier. In order to take into account negative Charisma modifiers, 
this table allows for very low Leadership scores, but the character must still 
be 6th level or higher in order to gain the Leadership feat. Outside factors can 
affect a character's Leadership score, as detailed above.

\textit{Cohort Level: }The character can attract a cohort of up to this level. 
Regardless of a character's Leadership score, he can only recruit a cohort who 
is two or more levels lower than himself. The cohort should be equipped with gear 
appropriate for its level. A character can try to attract a cohort of a particular 
race, class, and alignment. The cohort's alignment may not be opposed to the leader's 
alignment on either the law-vs-chaos or good-vs-evil axis, and the leader takes 
a Leadership penalty if he recruits a cohort of an alignment different from his 
own.

Cohorts earn XP as follows:

The cohort does not count as a party member when determining the party's XP.

Divide the cohort's level by the level of the PC with whom he or she is associated 
(the character with the Leadership feat who attracted the cohort).

Multiply this result by the total XP awarded to the PC and add that number of experience 
points to the cohort's total.

If a cohort gains enough XP to bring it to a level one lower than the associated 
PC's character level, the cohort does not gain the new level---its new XP total 
is 1 less than the amount needed attain the next level. 

\vspace{12pt}
\textit{Number of Followers by Level: }The character can lead up to the indicated 
number of characters of each level. Followers are similar to cohorts, except they're 
generally low-level NPCs. Because they're generally five or more levels behind 
the character they follow, they're rarely effective in combat.

Followers don't earn experience and thus don't gain levels. However, when a character 
with Leadership attains a new level, the player consults the table above to determine 
if she has acquired more followers, some of which may be higher level than the 
existing followers. (You don't consult the table to see if your cohort gains levels, 
however, because cohorts earn experience on their own.)

\vspace{12pt}
LIGHTNING REFLEXES [GENERAL]

\textbf{Benefit:} You get a +2 bonus on all Reflex saving throws.

\vspace{12pt}
MAGICAL APTITUDE [GENERAL]

\textbf{Benefit:} You get a +2 bonus on all Spellcraft checks and Use Magic Device 
checks.

\vspace{12pt}
MANYSHOT [GENERAL]

\textbf{Prerequisites:} Dex 17, Point Blank Shot, Rapid Shot, base attack bonus 
+6

\textbf{Benefit:} As a standard action, you may fire two arrows at a single opponent 
within 30 feet. Both arrows use the same attack roll (with a -4 penalty) to determine 
success and deal damage normally (but see Special).

For every five points of base attack bonus you have above +6, you may add one additional 
arrow to this attack, to a maximum of four arrows at a base attack bonus of +16. 
However, each arrow after the second adds a cumulative -2 penalty on the attack 
roll (for a total penalty of -6 for three arrows and -8 for four).

Damage reduction and other resistances apply separately against each arrow fired.

\textbf{Special:} Regardless of the number of arrows you fire, you apply precision-based 
damage only once. If you score a critical hit, only the first arrow fired deals 
critical damage; all others deal regular damage.

A fighter may select Manyshot as one of his fighter bonus feats.

A 6th-level ranger who has chosen the archery combat style is treated as having 
Manyshot even if he does not have the prerequisites for it, but only when he is 
wearing light or no armor.

\vspace{12pt}
MARTIAL WEAPON PROFICIENCY [GENERAL]

Choose a type of martial weapon. You understand how to use that type of martial 
weapon in combat.

\textbf{Benefit:} You make attack rolls with the selected weapon normally.

\textbf{Normal:} When using a weapon with which you are not proficient, you take 
a -4 penalty on attack rolls.

\textbf{Special:} Barbarians, fighters, paladins, and rangers are proficient with 
all martial weapons. They need not select this feat.

You can gain Martial Weapon Proficiency multiple times. Each time you take the 
feat, it applies to a new type of weapon. 

A cleric who chooses the War domain automatically gains the Martial Weapon Proficiency 
feat related to his deity's favored weapon as a bonus feat, if the weapon is a 
martial one. He need not select it.

\vspace{12pt}
MAXIMIZE SPELL [METAMAGIC]

\textbf{Benefit:} All variable, numeric effects of a spell modified by this feat 
are maximized. Saving throws and opposed rolls are not affected, nor are spells 
without random variables. A maximized spell uses up a spell slot three levels higher 
than the spell's actual level.

An empowered, maximized spell gains the separate benefits of each feat: the maximum 
result plus one-half the normally rolled result.

\vspace{12pt}
MOBILITY [GENERAL]

\textbf{Prerequisites:} Dex 13, Dodge.

\textbf{Benefit:} You get a +4 dodge bonus to Armor Class against attacks of opportunity 
caused when you move out of or within a threatened area. A condition that makes 
you lose your Dexterity bonus to Armor Class (if any) also makes you lose dodge 
bonuses.

Dodge bonuses stack with each other, unlike most types of bonuses.

\textbf{Special:} A fighter may select Mobility as one of his fighter bonus feats.

\vspace{12pt}
MOUNTED ARCHERY [GENERAL]

\textbf{Prerequisites:} Ride 1 rank, Mounted Combat.

\textbf{Benefit:} The penalty you take when using a ranged weapon while mounted 
is halved: -2 instead of -4 if your mount is taking a double move, and -4 instead 
of -8 if your mount is running.

\textbf{Special:} A fighter may select Mounted Archery as one of his fighter bonus 
feats.

\vspace{12pt}
MOUNTED COMBAT [GENERAL]

\textbf{Prerequisite:} Ride 1 rank.

\textbf{Benefit:} Once per round when your mount is hit in combat, you may attempt 
a Ride check (as a reaction) to negate the hit. The hit is negated if your Ride 
check result is greater than the opponent's attack roll. (Essentially, the Ride 
check result becomes the mount's Armor Class if it's higher than the mount's regular 
AC.)

\textbf{Special:} A fighter may select Mounted Combat as one of his fighter bonus 
feats.

\vspace{12pt}
NATURAL SPELL [GENERAL]

\textbf{Prerequisites: }Wis 13, wild shape ability.

\textbf{Benefit:} You can complete the verbal and somatic components of spells 
while in a wild shape. You substitute various noises and gestures for the normal 
verbal and somatic components of a spell.

You can also use any material components or focuses you possess, even if such items 
are melded within your current form. This feat does not permit the use of magic 
items while you are in a form that could not ordinarily use them, and you do not 
gain the ability to speak while in a wild shape.

\vspace{12pt}
NEGOTIATOR [GENERAL]

\textbf{Benefit:} You get a +2 bonus on all Diplomacy checks and Sense Motive checks.

\vspace{12pt}
NIMBLE FINGERS [GENERAL]

\textbf{Benefit:} You get a +2 bonus on all Disable Device checks and Open Lock 
checks.

\vspace{12pt}
PERSUASIVE [GENERAL]

\textbf{Benefit:} You get a +2 bonus on all Bluff checks and Intimidate checks.

\vspace{12pt}
POINT BLANK SHOT [GENERAL]

\textbf{Benefit:} You get a +1 bonus on attack and damage rolls with ranged weapons 
at ranges of up to 30 feet.

\textbf{Special:} A fighter may select Point Blank Shot as one of his fighter bonus 
feats.

\vspace{12pt}
POWER ATTACK [GENERAL]

\textbf{Prerequisite:} Str 13.

\textbf{Benefit:} On your action, before making attack rolls for a round, you may 
choose to subtract a number from all melee attack rolls and add the same number 
to all melee damage rolls. This number may not exceed your base attack bonus. The 
penalty on attacks and bonus on damage apply until your next turn.

\textbf{Special:} If you attack with a two-handed weapon, or with a one-handed 
weapon wielded in two hands, instead add twice the number subtracted from your 
attack rolls. You can't add the bonus from Power Attack to the damage dealt with 
a light weapon (except with unarmed strikes or natural weapon attacks), even though 
the penalty on attack rolls still applies. (Normally, you treat a double weapon 
as a one-handed weapon and a light weapon. If you choose to use a double weapon 
like a two-handed weapon, attacking with only one end of it in a round, you treat 
it as a two-handed weapon.)

A fighter may select Power Attack as one of his fighter bonus feats.

\vspace{12pt}
PRECISE SHOT [GENERAL]

\textbf{Prerequisite:} Point Blank Shot.

\textbf{Benefit:} You can shoot or throw ranged weapons at an opponent engaged 
in melee without taking the standard -4 penalty on your attack roll.

\textbf{Special:} A fighter may select Precise Shot as one of his fighter bonus 
feats.

\vspace{12pt}
QUICK DRAW [GENERAL]

\textbf{Prerequisite:} Base attack bonus +1.

\textbf{Benefit:} You can draw a weapon as a free action instead of as a move action. 
You can draw a hidden weapon (see the Sleight of Hand skill) as a move action.

A character who has selected this feat may throw weapons at his full normal rate 
of attacks (much like a character with a bow).

\textbf{Normal: }Without this feat, you may draw a weapon as a move action, or 
(if your base attack bonus is +1 or higher) as a free action as part of movement. 
Without this feat, you can draw a hidden weapon as a standard action.

\textbf{Special:} A fighter may select Quick Draw as one of his fighter bonus feats.

\vspace{12pt}
QUICKEN SPELL [METAMAGIC]

\textbf{Benefit:} Casting a quickened spell is a free action. You can perform another 
action, even casting another spell, in the same round as you cast a quickened spell. 
You may cast only one quickened spell per round. A spell whose casting time is 
more than 1 full round action cannot be quickened. A quickened spell uses up a 
spell slot four levels higher than the spell's actual level. Casting a quickened 
spell doesn't provoke an attack of opportunity.

\textbf{Special:} This feat can't be applied to any spell cast spontaneously (including 
sorcerer spells, bard spells, and cleric or druid spells cast spontaneously), since 
applying a metamagic feat to a spontaneously cast spell automatically increases 
the casting time to a full-round action.

\vspace{12pt}
RAPID RELOAD [GENERAL]

Choose a type of crossbow (hand, light, or heavy).

\textbf{Prerequisite:} Weapon Proficiency (crossbow type chosen).

\textbf{Benefit:} The time required for you to reload your chosen type of crossbow 
is reduced to a free action (for a hand or light crossbow) or a move action (for 
a heavy crossbow). Reloading a crossbow still provokes an attack of opportunity.

If you have selected this feat for hand crossbow or light crossbow, you may fire 
that weapon as many times in a full attack action as you could attack if you were 
using a bow.

\textbf{Normal:} A character without this feat needs a move action to reload a 
hand or light crossbow, or a full-round action to reload a heavy crossbow. 

\textbf{Special:} You can gain Rapid Reload multiple times. Each time you take 
the feat, it applies to a new type of crossbow.

A fighter may select Rapid Reload as one of his fighter bonus feats.

\vspace{12pt}
RAPID SHOT [GENERAL]

\textbf{Prerequisites:} Dex 13, Point Blank Shot.

\textbf{Benefit:} You can get one extra attack per round with a ranged weapon. 
The attack is at your highest base attack bonus, but each attack you make in that 
round (the extra one and the normal ones) takes a -2 penalty. You must use the 
full attack action to use this feat.

\textbf{Special:} A fighter may select Rapid Shot as one of his fighter bonus feats.

A 2nd-level ranger who has chosen the archery combat style is treated as having 
Rapid Shot, even if he does not have the prerequisites for it, but only when he 
is wearing light or no armor.

\vspace{12pt}
RIDE-BY ATTACK [GENERAL]

\textbf{Prerequisites:} Ride 1 rank, Mounted Combat.

\textbf{Benefit:} When you are mounted and use the charge action, you may move 
and attack as if with a standard charge and then move again (continuing the straight 
line of the charge). Your total movement for the round can't exceed double your 
mounted speed. You and your mount do not provoke an attack of opportunity from 
the opponent that you attack.

\textbf{Special:} A fighter may select Ride-By Attack as one of his fighter bonus 
feats.

\vspace{12pt}
RUN [GENERAL]

\textbf{Benefit:} When running, you move five times your normal speed (if wearing 
medium, light, or no armor and carrying no more than a medium load) or four times 
your speed (if wearing heavy armor or carrying a heavy load). If you make a jump 
after a running start (see the Jump skill description), you gain a +4 bonus on 
your Jump check. While running, you retain your Dexterity bonus to AC.

\textbf{Normal:} You move four times your speed while running (if wearing medium, 
light, or no armor and carrying no more than a medium load) or three times your 
speed (if wearing heavy armor or carrying a heavy load), and you lose your Dexterity 
bonus to AC.

\vspace{12pt}
SCRIBE SCROLL [ITEM CREATION]

\textbf{Prerequisite:} Caster level 1st.

\textbf{Benefit:} You can create a scroll of any spell that you know. Scribing 
a scroll takes one day for each 1,000 gp in its base price. The base price of a 
scroll is its spell level x $ $its caster level x $ $25 gp. To scribe a scroll, 
you must spend 1/25 of this base price in XP and use up raw materials costing one-half 
of this base price.

Any scroll that stores a spell with a costly material component or an XP cost also 
carries a commensurate cost. In addition to the costs derived from the base price, 
you must expend the material component or pay the XP when scribing the scroll.

\vspace{12pt}
SELF-SUFFICIENT [GENERAL]

\textbf{Benefit:} You get a +2 bonus on all Heal checks and Survival checks.

\vspace{12pt}
SHIELD PROFICIENCY [GENERAL]

\textbf{Benefit:} You can use a shield and take only the standard penalties.

\textbf{Normal:} When you are using a shield with which you are not proficient, 
you take the shield's armor check penalty on attack rolls and on all skill checks 
that involve moving, including Ride checks.

\textbf{Special:} Barbarians, bards, clerics, druids, fighters, paladins, and rangers 
automatically have Shield Proficiency as a bonus feat. They need not select it.

\vspace{12pt}
SHOT ON THE RUN [GENERAL]

\textbf{Prerequisites:} Dex 13, Dodge, Mobility, Point Blank Shot, base attack 
bonus +4.

\textbf{Benefit:} When using the attack action with a ranged weapon, you can move 
both before and after the attack, provided that your total distance moved is not 
greater than your speed.

\textbf{Special:} A fighter may select Shot on the Run as one of his fighter bonus 
feats.

\vspace{12pt}
SILENT SPELL [METAMAGIC]

\textbf{Benefit:} A silent spell can be cast with no verbal components. Spells 
without verbal components are not affected. A silent spell uses up a spell slot 
one level higher than the spell's actual level.

\textbf{Special:} Bard spells cannot be enhanced by this metamagic feat.

\vspace{12pt}
SIMPLE WEAPON PROFICIENCY [GENERAL]

\textbf{Benefit:} You make attack rolls with simple weapons normally.

\textbf{Normal:} When using a weapon with which you are not proficient, you take 
a -4 penalty on attack rolls.

\textbf{Special:} All characters except for druids, monks, and wizards are automatically 
proficient with all simple weapons. They need not select this feat.

\vspace{12pt}
SKILL FOCUS [GENERAL]

Choose a skill.

\textbf{Benefit:} You get a +3 bonus on all checks involving that skill.

\textbf{Special:} You can gain this feat multiple times. Its effects do not stack. 
Each time you take the feat, it applies to a new skill.

\vspace{12pt}
SNATCH ARROWS [GENERAL]

\textbf{Prerequisites:} Dex 15, Deflect Arrows, Improved Unarmed Strike.

\textbf{Benefit:} When using the Deflect Arrows feat you may catch the weapon instead 
of just deflecting it. Thrown weapons can immediately be thrown back at the original 
attacker (even though it isn't your turn) or kept for later use.

You must have at least one hand free (holding nothing) to use this feat.

\textbf{Special:} A fighter may select Snatch Arrows as one of his fighter bonus 
feats.

\vspace{12pt}
SPELL FOCUS [GENERAL]

Choose a school of magic.

\textbf{Benefit:} Add +1 to the Difficulty Class for all saving throws against 
spells from the school of magic you select.

\textbf{Special:} You can gain this feat multiple times. Its effects do not stack. 
Each time you take the feat, it applies to a new school of magic.

\vspace{12pt}
SPELL MASTERY [SPECIAL]

\textbf{Prerequisite: }Wizard level 1st.

\textbf{Benefit:} Each time you take this feat, choose a number of spells equal 
to your Intelligence modifier that you already know. From that point on, you can 
prepare these spells without referring to a spellbook.

\textbf{Normal: }Without this feat, you must use a spellbook to prepare all your 
spells, except \textit{read magic.}

\vspace{12pt}
SPELL PENETRATION [GENERAL]

\textbf{Benefit:} You get a +2 bonus on caster level checks (1d20 + caster level) 
made to overcome a creature's spell resistance.

\vspace{12pt}
SPIRITED CHARGE [GENERAL]

\textbf{Prerequisites:} Ride 1 rank, Mounted Combat, Ride-By Attack.

\textbf{Benefit:} When mounted and using the charge action, you deal double damage 
with a melee weapon (or triple damage with a lance).

\textbf{Special:} A fighter may select Spirited Charge as one of his fighter bonus 
feats.

\vspace{12pt}
SPRING ATTACK [GENERAL]

\textbf{Prerequisites:} Dex 13, Dodge, Mobility, base attack bonus +4.

\textbf{Benefit:} When using the attack action with a melee weapon, you can move 
both before and after the attack, provided that your total distance moved is not 
greater than your speed. Moving in this way does not provoke an attack of opportunity 
from the defender you attack, though it might provoke attacks of opportunity from 
other creatures, if appropriate. You can't use this feat if you are wearing heavy 
armor.

You must move at least 5 feet both before and after you make your attack in order 
to utilize the benefits of Spring Attack.

\textbf{Special:} A fighter may select Spring Attack as one of his fighter bonus 
feats.

\vspace{12pt}
STEALTHY [GENERAL]

\textbf{Benefit:} You get a +2 bonus on all Hide checks and Move Silently checks.

\vspace{12pt}
STILL SPELL [METAMAGIC]

\textbf{Benefit:} A stilled spell can be cast with no somatic components.

Spells without somatic components are not affected. A stilled spell uses up a spell 
slot one level higher than the spell's actual level.

\vspace{12pt}
STUNNING FIST [GENERAL]

\textbf{Prerequisites:} Dex 13, Wis 13, Improved Unarmed Strike, base attack bonus 
+8.

\textbf{Benefit:} You must declare that you are using this feat before you make 
your attack roll (thus, a failed attack roll ruins the attempt). Stunning Fist 
forces a foe damaged by your unarmed attack to make a Fortitude saving throw (DC 
10 + 1/2 your character level + your Wis modifier), in addition to dealing damage 
normally. A defender who fails this saving throw is stunned for 1 round (until 
just before your next action). A stunned character can't act, loses any Dexterity 
bonus to AC, and takes a -2 penalty to AC. You may attempt a stunning attack once 
per day for every four levels you have attained (but see Special), and no more 
than once per round. Constructs, oozes, plants, undead, incorporeal creatures, 
and creatures immune to critical hits cannot be stunned.

\textbf{Special:} A monk may select Stunning Fist as a bonus feat at 1st level, 
even if she does not meet the prerequisites. A monk who selects this feat may attempt 
a stunning attack a number of times per day equal to her monk level, plus one more 
time per day for every four levels she has in classes other than monk.

A fighter may select Stunning Fist as one of his fighter bonus feats.

\vspace{12pt}
TOUGHNESS [GENERAL]

\textbf{Benefit:} You gain +3 hit points.

\textbf{Special:} A character may gain this feat multiple times. Its effects stack.

\vspace{12pt}
TOWER SHIELD PROFICIENCY [GENERAL]

\textbf{Prerequisite:} Shield Proficiency.

\textbf{Benefit:} You can use a tower shield and suffer only the standard penalties.

\textbf{Normal:} A character who is using a shield with which he or she is not 
proficient takes the shield's armor check penalty on attack rolls and on all skill 
checks that involve moving, including Ride.

\textbf{Special:} Fighters automatically have Tower Shield Proficiency as a bonus 
feat. They need not select it.

\vspace{12pt}
TRACK [GENERAL]

\textbf{Benefit:} To find tracks or to follow them for 1 mile requires a successful 
Survival check. You must make another Survival check every time the tracks become 
difficult to follow.

You move at half your normal speed (or at your normal speed with a -5 penalty on 
the check, or at up to twice your normal speed with a -20 penalty on the check). 
The DC depends on the surface and the prevailing conditions, as given on the table 
below:

\vspace{12pt}
\begin{tabular}{|>{\raggedright}p{67pt}|>{\raggedright}p{51pt}|>{\raggedright}p{70pt}|>{\raggedright}p{13pt}|}
\hline
\section*{S\textbf{urface}} & \section*{S\textbf{urvival DC }} & \section*{S\textbf{urface 
Survival }} & \section*{D\textbf{C}}\tabularnewline
\hline
Very soft ground  & 5  & Firm ground  & 15\tabularnewline
\hline
Soft ground  & 10  & Hard ground  & 20\tabularnewline
\hline
\end{tabular}

\vspace{12pt}
\textit{Very Soft Ground: }Any surface (fresh snow, thick dust, wet mud) that holds 
deep, clear impressions of footprints.

\textit{Soft Ground: }Any surface soft enough to yield to pressure, but firmer 
than wet mud or fresh snow, in which a creature leaves frequent but shallow footprints.

\textit{Firm Ground: }Most normal outdoor surfaces (such as lawns, fields, woods, 
and the like) or exceptionally soft or dirty indoor surfaces (thick rugs and very 
dirty or dusty floors). The creature might leave some traces (broken branches or 
tufts of hair), but it leaves only occasional or partial footprints.

\textit{Hard Ground: }Any surface that doesn't hold footprints at all, such as 
bare rock or an indoor floor. Most streambeds fall into this category, since any 
footprints left behind are obscured or washed away. The creature leaves only traces 
(scuff marks or displaced pebbles). 

Several modifiers may apply to the Survival check, as given on the table below.

\vspace{12pt}
\begin{tabular}{|>{\raggedright}p{216pt}|>{\raggedright}p{99pt}|}
\hline
\section*{C\textbf{ondition }} & \section*{S\textbf{urvival DC Modifier}}\tabularnewline
\hline
Every three creatures in the group being tracked - & 1\tabularnewline
\hline
Size of creature or creatures being tracked:\textsuperscript{\textbf{1}} & \tabularnewline
\hline
Fine  & +8\tabularnewline
\hline
Diminutive  & +4\tabularnewline
\hline
Tiny  & +2\tabularnewline
\hline
Small  & +1\tabularnewline
\hline
Medium  & +0\tabularnewline
\hline
Large - & 1\tabularnewline
\hline
Huge - & 2\tabularnewline
\hline
Gargantuan - & 4\tabularnewline
\hline
Colossal - & 8\tabularnewline
\hline
Every 24 hours since the trail was made  & +1\tabularnewline
\hline
Every hour of rain since the trail was made  & +1\tabularnewline
\hline
Fresh snow cover since the trail was made  & +10\tabularnewline
\hline
Poor visibility:\textsuperscript{\textbf{2}} & \tabularnewline
\hline
Overcast or moonless night  & +6\tabularnewline
\hline
Moonlight  & +3\tabularnewline
\hline
Fog or precipitation  & +3\tabularnewline
\hline
Tracked party hides trail (and moves at half speed)  & +5\tabularnewline
\hline
\multicolumn{2}{|p{315pt}|}{1 For a group of mixed sizes, apply only the modifier 
for the largest size category.}\tabularnewline
\hline
\multicolumn{2}{|p{315pt}|}{2 Apply only the largest modifier from this category.}\tabularnewline
\hline
\end{tabular}

\vspace{12pt}
If you fail a Survival check, you can retry after 1 hour (outdoors) or 10 minutes 
(indoors) of searching.

\textbf{Normal:} Without this feat, you can use the Survival skill to find tracks, 
but you can follow them only if the DC for the task is 10 or lower. Alternatively, 
you can use the Search skill to find a footprint or similar sign of a creature's 
passage using the DCs given above, but you can't use Search to follow tracks, even 
if someone else has already found them.

\textbf{Special:} A ranger automatically has Track as a bonus feat. He need not 
select it.

This feat does not allow you to find or follow the tracks made by a subject of 
a \textit{pass without trace }spell.

\vspace{12pt}
TRAMPLE [GENERAL]

\textbf{Prerequisites:} Ride 1 rank, Mounted Combat.

\textbf{Benefit:} When you attempt to overrun an opponent while mounted, your target 
may not choose to avoid you. Your mount may make one hoof attack against any target 
you knock down, gaining the standard +4 bonus on attack rolls against prone targets.

\textbf{Special:} A fighter may select Trample as one of his fighter bonus feats.

\vspace{12pt}
TWO-WEAPON DEFENSE [GENERAL]

\textbf{Prerequisites:} Dex 15, Two-Weapon Fighting.

\textbf{Benefit: }When wielding a double weapon or two weapons (not including natural 
weapons or unarmed strikes), you gain a +1 shield bonus to your AC.

When you are fighting defensively or using the total defense action, this shield 
bonus increases to +2.

\textbf{Special:} A fighter may select Two-Weapon Defense as one of his fighter 
bonus feats.

\vspace{12pt}
TWO-WEAPON FIGHTING [GENERAL]

You can fight with a weapon in each hand. You can make one extra attack each round 
with the second weapon.

\textbf{Prerequisite:} Dex 15.

\textbf{Benefit:} Your penalties on attack rolls for fighting with two weapons 
are reduced. The penalty for your primary hand lessens by 2 and the one for your 
off hand lessens by 6.

\textbf{Normal:} If you wield a second weapon in your off hand, you can get one 
extra attack per round with that weapon. When fighting in this way you suffer a 
-6 penalty with your regular attack or attacks with your primary hand and a -10 
penalty to the attack with your off hand. If your off-hand weapon is light the 
penalties are reduced by 2 each. (An unarmed strike is always considered light.)

\textbf{Special:} A 2nd-level ranger who has chosen the two-weapon combat style 
is treated as having Two-Weapon Fighting, even if he does not have the prerequisite 
for it, but only when he is wearing light or no armor.

A fighter may select Two-Weapon Fighting as one of his fighter bonus feats.

\vspace{12pt}
WEAPON FINESSE [GENERAL]

\textbf{Prerequisite:} Base attack bonus +1.

\textbf{Benefit:} With a light weapon, rapier, whip, or spiked chain made for a 
creature of your size category, you may use your Dexterity modifier instead of 
your Strength modifier on attack rolls. If you carry a shield, its armor check 
penalty applies to your attack rolls.

\textbf{Special:} A fighter may select Weapon Finesse as one of his fighter bonus 
feats.

Natural weapons are always considered light weapons.

\vspace{12pt}
WEAPON FOCUS [GENERAL]

Choose one type of weapon. You can also choose unarmed strike or grapple (or ray, 
if you are a spellcaster) as your weapon for purposes of this feat. 

\parindent=3pt
\textbf{Prerequisites:} Proficiency with selected weapon, base attack bonus +1.

\parindent=0pt
\textbf{Benefit:} You gain a +1 bonus on all attack rolls you make using the selected 
weapon.

\textbf{Special:} You can gain this feat multiple times. Its effects do not stack. 
Each time you take the feat, it applies to a new type of weapon.

A fighter may select Weapon Focus as one of his fighter bonus feats. He must have 
Weapon Focus with a weapon to gain the Weapon Specialization feat for that weapon.

\vspace{12pt}
WEAPON SPECIALIZATION [GENERAL]

Choose one type of weapon for which you have already selected the Weapon Focus 
feat. You can also choose unarmed strike or grapple as your weapon for purposes 
of this feat. You deal extra damage when using this weapon. 

\parindent=3pt
\textbf{Prerequisites:} Proficiency with selected weapon, Weapon Focus with selected 
weapon, fighter level 4th.

\parindent=0pt
\textbf{Benefit:} You gain a +2 bonus on all damage rolls you make using the selected 
weapon.

\textbf{Special:} You can gain this feat multiple times. Its effects do not stack. 
Each time you take the feat, it applies to a new type of weapon.

A fighter may select Weapon Specialization as one of his fighter bonus feats.

\vspace{12pt}
WHIRLWIND ATTACK [GENERAL]

\textbf{Prerequisites:} Dex 13, Int 13, Combat Expertise, Dodge, Mobility, Spring 
Attack, base attack bonus +4.

\textbf{Benefit:} When you use the full attack action, you can give up your regular 
attacks and instead make one melee attack at your full base attack bonus against 
each opponent within reach.

When you use the Whirlwind Attack feat, you also forfeit any bonus or extra attacks 
granted by other feats, spells, or abilities.

\textbf{Special:} A fighter may select Whirlwind Attack as one of his fighter bonus 
feats.

\vspace{12pt}
WIDEN SPELL [METAMAGIC]

\textbf{Benefit:} You can alter a burst, emanation, line, or spread shaped spell 
to increase its area. Any numeric measurements of the spell's area increase by 
100\%.A widened spell uses up a spell slot three levels higher than the spell's 
actual level.

Spells that do not have an area of one of these four sorts are not affected by 
this feat.

\newpage

\end{document}
