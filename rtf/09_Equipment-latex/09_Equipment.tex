%&pdfLaTeX
% !TEX encoding = UTF-8 Unicode
\documentclass{article}
\usepackage{ifxetex}
\ifxetex
\usepackage{fontspec}
\setmainfont[Mapping=tex-text]{STIXGeneral}
\else
\usepackage[T1]{fontenc}
\usepackage[utf8]{inputenc}
\fi
\usepackage{textcomp}

\usepackage{array}
\usepackage{amssymb}
\usepackage{fancyhdr}
\renewcommand{\headrulewidth}{0pt}
\renewcommand{\footrulewidth}{0pt}

\begin{document}

This material is Open Game Content, and is licensed for public use under the terms 
of the Open Game License v1.0a.

{\LARGE{}EQUIPMENT}

Assume a character owns at least one outfit of normal clothes. Pick any one of 
the following clothing outfits: artisan's outfit, entertainer's outfit, explorer's 
outfit, monk's outfit, peasant's outfit, scholar's outfit, or traveler's outfit.

\vspace{12pt}
{\LARGE{}WEALTH AND MONEY}

COINS

The most common coin is the gold piece (gp). A gold piece is worth 10 silver pieces. 
Each silver piece is worth 10 copper pieces (cp). In addition to copper, silver, 
and gold coins, there are also platinum pieces (pp), which are each worth 10 gp.

The standard coin weighs about a third of an ounce (fifty to the pound).

\vspace{12pt}
\begin{tabular}{|>{\raggedright}p{85pt}|>{\raggedright}p{30pt}|>{\raggedright}p{24pt}|>{\raggedright}p{31pt}|>{\raggedright}p{40pt}|}
\hline
T\textbf{able: Coins} &  &  &  & \tabularnewline
\hline
 --------------- & \multicolumn{4}{p{127pt}|}{ \textbf{Exchange Value ------------}}\tabularnewline
\hline
  & C\textbf{P} & S\textbf{P} & G\textbf{P} & P\textbf{P}\tabularnewline
\hline
Copper piece (cp) = & 1 & 1/10 & 1/100 & 1/1,000\tabularnewline
\hline
Silver piece (sp) = & 10 & 1 & 1/10 & 1/100\tabularnewline
\hline
Gold piece (gp) = & 100 & 10 & 1 & 1/10\tabularnewline
\hline
Platinum piece (pp) = & 1,000 & 100 & 10 & 1\tabularnewline
\hline
\end{tabular}

\vspace{12pt}
WEALTH OTHER THAN COINS

Merchants commonly exchange trade goods without using currency. As a means of comparison, 
some trade goods are detailed below.

\vspace{12pt}
\begin{tabular}{|>{\raggedright}p{26pt}|>{\raggedright}p{177pt}|}
\hline
\multicolumn{2}{|p{204pt}|}{\section*{T\textbf{able: Trade Goods}}}\tabularnewline
\hline
C\textbf{ost} & I\textbf{tem}\tabularnewline
\hline
1 cp & One pound of wheat\tabularnewline
\hline
2 cp & One pound of flour, or one chicken\tabularnewline
\hline
1 sp & One pound of iron\tabularnewline
\hline
5 sp & One pound of tobacco or copper\tabularnewline
\hline
1 gp & One pound of cinnamon, or one goat\tabularnewline
\hline
2 gp & One pound of ginger or pepper, or one sheep\tabularnewline
\hline
3 gp & One pig\tabularnewline
\hline
4 gp & One square yard of linen\tabularnewline
\hline
5 gp & One pound of salt or silver\tabularnewline
\hline
10 gp & One square yard of silk, or one cow\tabularnewline
\hline
15 gp & One pound of saffron or cloves, or one ox\tabularnewline
\hline
50 gp & One pound of gold\tabularnewline
\hline
500 gp & One pound of platinum\tabularnewline
\hline
\end{tabular}

\vspace{12pt}
SELLING LOOT

In general, a character can sell something for half its listed price.

Trade goods are the exception to the half-price rule. A trade good, in this sense, 
is a valuable good that can be easily exchanged almost as if it were cash itself.

\vspace{12pt}
{\LARGE{}WEAPONS}

WEAPON CATEGORIES

Weapons are grouped into several interlocking sets of categories.

These categories pertain to what training is needed to become proficient in a weapon's 
use (simple, martial, or exotic), the weapon's usefulness either in close combat 
(melee) or at a distance (ranged, which includes both thrown and projectile weapons), 
its relative encumbrance (light, one-handed, or two-handed), and its size (Small, 
Medium, or Large).

\textbf{Simple, Martial, and Exotic Weapons:} Anybody but a druid, monk, rogue, 
or wizard is proficient with all simple weapons. Barbarians, fighters, paladins, 
and rangers are proficient with all simple and all martial weapons. Characters 
of other classes are proficient with an assortment of mainly simple weapons and 
possibly also some martial or even exotic weapons. A character who uses a weapon 
with which he or she is not proficient takes a -4 penalty on attack rolls.

\textbf{Melee and Ranged Weapons:} Melee weapons are used for making melee attacks, 
though some of them can be thrown as well. Ranged weapons are thrown weapons or 
projectile weapons that are not effective in melee.

\textit{Reach Weapons: }Glaives, guisarmes, lances, longspears, ranseurs, spiked 
chains, and whips are reach weapons. A reach weapon is a melee weapon that allows 
its wielder to strike at targets that aren't adjacent to him or her. Most reach 
double the wielder's natural reach, meaning that a typical Small or Medium wielder 
of such a weapon can attack a creature 10 feet away, but not a creature in an adjacent 
square. A typical Large character wielding a reach weapon of the appropriate size 
can attack a creature 15 or 20 feet away, but not adjacent creatures or creatures 
up to 10 feet away.

\textit{Double Weapons: }Dire flails, dwarven urgroshes, gnome hooked hammers, 
orc double axes, quarterstaffs, and two-bladed swords are double weapons. A character 
can fight with both ends of a double weapon as if fighting with two weapons, but 
he or she incurs all the normal attack penalties associated with two-weapon combat, 
just as though the character were wielding a one-handed weapon and a light weapon.

The character can also choose to use a double weapon two handed, attacking with 
only one end of it. A creature wielding a double weapon in one hand can't use it 
as a double weapon---only one end of the weapon can be used in any given round.

\textit{Thrown Weapons: }Daggers, clubs, shortspears, spears, darts, javelins, 
throwing axes, light hammers, tridents, shuriken, and nets are thrown weapons. 
The wielder applies his or her Strength modifier to damage dealt by thrown weapons 
(except for splash weapons). It is possible to throw a weapon that isn't designed 
to be thrown (that is, a melee weapon that doesn't have a numeric entry in the 
Range Increment column on Table: Weapons), but a character who does so takes a 
-4 penalty on the attack roll. Throwing a light or one-handed weapon is a standard 
action, while throwing a two-handed weapon is a full-round action. Regardless of 
the type of weapon, such an attack scores a threat only on a natural roll of 20 
and deals double damage on a critical hit. Such a weapon has a range increment 
of 10 feet.

\textit{Projectile Weapons: }Light crossbows, slings, heavy crossbows, shortbows, 
composite shortbows, longbows, composite longbows, hand crossbows, and repeating 
crossbows are projectile weapons. Most projectile weapons require two hands to 
use (see specific weapon descriptions). A character gets no Strength bonus on damage 
rolls with a projectile weapon unless it's a specially built composite shortbow, 
specially built composite longbow, or sling. If the character has a penalty for 
low Strength, apply it to damage rolls when he or she uses a bow or a sling.

\textit{Ammunition: }Projectile weapons use ammunition: arrows (for bows), bolts 
(for crossbows), or sling bullets (for slings). When using a bow, a character can 
draw ammunition as a free action; crossbows and slings require an action for reloading. 
Generally speaking, ammunition that hits its target is destroyed or rendered useless, 
while normal ammunition that misses has a 50\% chance of being destroyed or lost.

Although they are thrown weapons, shuriken are treated as ammunition for the purposes 
of drawing them, crafting masterwork or otherwise special versions of them (see 
Masterwork Weapons), and what happens to them after they are thrown.

\vspace{12pt}
\textbf{Light, One-Handed, and Two-Handed Melee Weapons:} This designation is a 
measure of how much effort it takes to wield a weapon in combat. It indicates whether 
a melee weapon, when wielded by a character of the weapon's size category, is considered 
a light weapon, a one-handed weapon, or a two-handed weapon.

\textit{Light: }A light weapon is easier to use in one's off hand than a one-handed 
weapon is, and it can be used while grappling. A light weapon is used in one hand. 
Add the wielder's Strength bonus (if any) to damage rolls for melee attacks with 
a light weapon if it's used in the primary hand, or one-half the wielder's Strength 
bonus if it's used in the off hand. Using two hands to wield a light weapon gives 
no advantage on damage; the Strength bonus applies as though the weapon were held 
in the wielder's primary hand only.

An unarmed strike is always considered a light weapon.

\textit{One-Handed: }A one-handed weapon can be used in either the primary hand 
or the off hand. Add the wielder's Strength bonus to damage rolls for melee attacks 
with a one-handed weapon if it's used in the primary hand, or 1/2 his or her Strength 
bonus if it's used in the off hand. If a one-handed weapon is wielded with two 
hands during melee combat, add 1-1/2 times the character's Strength bonus to damage 
rolls.

\textit{Two-Handed: }Two hands are required to use a two-handed melee weapon effectively. 
Apply 1-1/2 times the character's Strength bonus to damage rolls for melee attacks 
with such a weapon. 

\vspace{12pt}
\textbf{Weapon Size:} Every weapon has a size category. This designation indicates 
the size of the creature for which the weapon was designed.

A weapon's size category isn't the same as its size as an object. Instead, a weapon's 
size category is keyed to the size of the intended wielder. In general, a light 
weapon is an object two size categories smaller than the wielder, a one-handed 
weapon is an object one size category smaller than the wielder, and a two-handed 
weapon is an object of the same size category as the wielder.

\textit{Inappropriately Sized Weapons: }A creature can't make optimum use of a 
weapon that isn't properly sized for it. A cumulative -2 penalty applies on attack 
rolls for each size category of difference between the size of its intended wielder 
and the size of its actual wielder. If the creature isn't proficient with the weapon 
a -4 nonproficiency penalty also applies.

The measure of how much effort it takes to use a weapon (whether the weapon is 
designated as a light, one-handed, or two-handed weapon for a particular wielder) 
is altered by one step for each size category of difference between the wielder's 
size and the size of the creature for which the weapon was designed. If a weapon's 
designation would be changed to something other than light, one-handed, or two-handed 
by this alteration, the creature can't wield the weapon at all.

\vspace{12pt}
\textbf{Improvised Weapons:} Sometimes objects not crafted to be weapons nonetheless 
see use in combat. Because such objects are not designed for this use, any creature 
that uses one in combat is considered to be nonproficient with it and takes a -4 
penalty on attack rolls made with that object. To determine the size category and 
appropriate damage for an improvised weapon, compare its relative size and damage 
potential to the weapon list to find a reasonable match. An improvised weapon scores 
a threat on a natural roll of 20 and deals double damage on a critical hit. An 
improvised thrown weapon has a range increment of 10 feet.

\vspace{12pt}
WEAPON QUALITIES

Here is the format for weapon entries (given as column headings on Table: Weapons, 
below).

\textbf{Cost:} This value is the weapon's cost in gold pieces (gp) or silver pieces 
(sp). The cost includes miscellaneous gear that goes with the weapon.

This cost is the same for a Small or Medium version of the weapon. A Large version 
costs twice the listed price.

\textbf{Damage:} The Damage columns give the damage dealt by the weapon on a successful 
hit. The column labeled ``Dmg (S)'' is for Small weapons. The column labeled ``Dmg 
(M)'' is for Medium weapons. If two damage ranges are given then the weapon is 
a double weapon. Use the second damage figure given for the double weapon's extra 
attack. Table: Tiny and Large Weapon Damage gives weapon damage values for weapons 
of those sizes.

\vspace{12pt}
\begin{tabular}{|>{\raggedright}p{65pt}|>{\raggedright}p{60pt}|>{\raggedright}p{60pt}|}
\hline
\multicolumn{3}{|p{185pt}|}{T\textbf{able: Tiny and Large Weapon Damage}}\tabularnewline
\hline
M\textbf{edium Weapon Damage} & T\textbf{iny Weapon Damage} & L\textbf{arge Weapon 
Damage}\tabularnewline
\hline
1d2--- &  & 1d3\tabularnewline
\hline
1d3 & 1 & 1d4\tabularnewline
\hline
1d4 & 1d2 & 1d6\tabularnewline
\hline
1d6 & 1d3 & 1d8\tabularnewline
\hline
1d8 & 1d4 & 2d6\tabularnewline
\hline
1d10 & 1d6 & 2d8\tabularnewline
\hline
1d12 & 1d8 & 3d6\tabularnewline
\hline
2d4 & 1d4 & 2d6\tabularnewline
\hline
2d6 & 1d8 & 3d6\tabularnewline
\hline
2d8 & 1d10 & 3d8\tabularnewline
\hline
2d10 & 2d6 & 4d8\tabularnewline
\hline
\end{tabular}

\vspace{12pt}
\textbf{Critical:} The entry in this column notes how the weapon is used with the 
rules for critical hits. When your character scores a critical hit, roll the damage 
two, three, or four times, as indicated by its critical multiplier (using all applicable 
modifiers on each roll), and add all the results together.

\textit{Exception: }Extra damage over and above a weapon's normal damage is not 
multiplied when you score a critical hit.

x\textit{2: }The weapon deals double damage on a critical hit.

x\textit{3: }The weapon deals triple damage on a critical hit.

x\textit{3/}x\textit{4: }One head of this double weapon deals triple damage on 
a critical hit. The other head deals quadruple damage on a critical hit.

x\textit{4: }The weapon deals quadruple damage on a critical hit.

\textit{19-20/}x\textit{2: }The weapon scores a threat on a natural roll of 19 
or 20 (instead of just 20) and deals double damage on a critical hit. (The weapon 
has a threat range of 19-20.)

\textit{18-20/}x\textit{2: }The weapon scores a threat on a natural roll of 18, 
19, or 20 (instead of just 20) and deals double damage on a critical hit. (The 
weapon has a threat range of 18-20.)

\textbf{Range Increment:} Any attack at less than this distance is not penalized 
for range. However, each full range increment imposes a cumulative -2 penalty on 
the attack roll. A thrown weapon has a maximum range of five range increments. 
A projectile weapon can shoot out to ten range increments.

\textbf{Weight:} This column gives the weight of a Medium version of the weapon. 
Halve this number for Small weapons and double it for Large weapons.

\textbf{Type:} Weapons are classified according to the type of damage they deal: 
bludgeoning, piercing, or slashing. Some monsters may be resistant or immune to 
attacks from certain types of weapons.

Some weapons deal damage of multiple types. If a weapon is of two types, the damage 
it deals is not half one type and half another; all of it is both types. Therefore, 
a creature would have to be immune to both types of damage to ignore any of the 
damage from such a weapon.

In other cases, a weapon can deal either of two types of damage. In a situation 
when the damage type is significant, the wielder can choose which type of damage 
to deal with such a weapon.

\textbf{Special:} Some weapons have special features. See the weapon descriptions 
for details.

\vspace{12pt}
WEAPON DESCRIPTIONS

\begin{tabular}{|>{\raggedright}p{64pt}|>{\raggedright}p{17pt}|>{\raggedright}p{21pt}|>{\raggedright}p{24pt}|>{\raggedright}p{-2pt}|>{\raggedright}p{20pt}|>{\raggedright}p{32pt}|>{\raggedright}p{22pt}|>{\raggedright}p{54pt}|}
\hline
\multicolumn{9}{|p{255pt}|}{\section*{T\textbf{able: Weapons}}}\tabularnewline
\hline
S\textbf{imple Weapons} & C\textbf{ost} & D\textbf{mg (S)} & D\textbf{mg (M)} & \multicolumn{2}{p{18pt}|}{C\textbf{ritical}} & R\textbf{ange 
Increment} & W\textbf{eight}\textsuperscript{\textbf{1}} & T\textbf{ype}\textsuperscript{\textbf{2}}\tabularnewline
\hline
\multicolumn{9}{|p{255pt}|}{U\textit{narmed Attacks}}\tabularnewline
\hline
Gauntlet & 2 gp & 1d2 & 1d3 & \multicolumn{2}{p{18pt}|}{x2}--- &  & 1 lb. & Bludgeoning\tabularnewline
\hline
Unarmed strike--- &  & 1d2\textsuperscript{\textbf{3}} & 1d3\textsuperscript{\textbf{3}} & \multicolumn{2}{p{18pt}|}{x2}--- & --- &  & Bludgeoning\tabularnewline
\hline
\multicolumn{9}{|p{255pt}|}{L\textit{ight Melee Weapons}}\tabularnewline
\hline
Dagger & 2 gp & 1d3 & 1d4 & \multicolumn{2}{p{18pt}|}{19-20/x2} & 10 ft. & 1 lb. & Piercing 
or slashing\tabularnewline
\hline
Dagger, punching & 2 gp & 1d3 & 1d4 & \multicolumn{2}{p{18pt}|}{x3}--- &  & 1 lb. & Piercing\tabularnewline
\hline
Gauntlet, spiked & 5 gp & 1d3 & 1d4 & \multicolumn{2}{p{18pt}|}{x2}--- &  & 1 lb. & Piercing\tabularnewline
\hline
Mace, light & 5 gp & 1d4 & 1d6 & \multicolumn{2}{p{18pt}|}{x2}--- &  & 4 lb. & Bludgeoning\tabularnewline
\hline
Sickle & 6 gp & 1d4 & 1d6 & \multicolumn{2}{p{18pt}|}{x2}--- &  & 2 lb. & Slashing\tabularnewline
\hline
\multicolumn{9}{|p{255pt}|}{O\textit{ne-Handed Melee Weapons}}\tabularnewline
\hline
Club--- &  & 1d4 & 1d6 & \multicolumn{2}{p{18pt}|}{x2} & 10 ft. & 3 lb. & Bludgeoning\tabularnewline
\hline
Mace, heavy & 12 gp & 1d6 & 1d8 & \multicolumn{2}{p{18pt}|}{x2}--- &  & 8 lb. & Bludgeoning\tabularnewline
\hline
Morningstar & 8 gp & 1d6 & 1d8 & \multicolumn{2}{p{18pt}|}{x2}--- &  & 6 lb. & Bludgeoning 
and piercing\tabularnewline
\hline
Shortspear & 1 gp & 1d4 & 1d6 & \multicolumn{2}{p{18pt}|}{x2} & 20 ft. & 3 lb. & Piercing\tabularnewline
\hline
\multicolumn{9}{|p{255pt}|}{T\textit{wo-Handed Melee Weapons}}\tabularnewline
\hline
Longspear\textsuperscript{\textbf{4}} & 5 gp & 1d6 & 1d8 & \multicolumn{2}{p{18pt}|}{x3}--- &  & 9 
lb. & Piercing\tabularnewline
\hline
Quarterstaff\textsuperscript{\textbf{5}}--- &  & 1d4/1d4 & 1d6/1d6 & \multicolumn{2}{p{18pt}|}{x2}--- &  & 4 
lb. & Bludgeoning\tabularnewline
\hline
Spear & 2 gp & 1d6 & 1d8 & \multicolumn{2}{p{18pt}|}{x3} & 20 ft. & 6 lb. & Piercing\tabularnewline
\hline
\multicolumn{9}{|p{255pt}|}{R\textit{anged Weapons}}\tabularnewline
\hline
Crossbow, heavy & 50 gp & 1d8 & 1d10 & \multicolumn{2}{p{18pt}|}{19-20/x2} & 120 
ft. & 8 lb. & Piercing\tabularnewline
\hline
Bolts, crossbow (10) & 1 gp--- & --- & --- & \multicolumn{2}{p{18pt}|}{}--- &  & 1 
lb.--- & \tabularnewline
\hline
Crossbow, light & 35 gp & 1d6 & 1d8 & \multicolumn{2}{p{18pt}|}{19-20/x2} & 80 
ft. & 4 lb. & Piercing\tabularnewline
\hline
Bolts, crossbow (10) & 1 gp--- & --- & --- & \multicolumn{2}{p{18pt}|}{}--- &  & 1 
lb.--- & \tabularnewline
\hline
Dart & 5 sp & 1d3 & 1d4 & \multicolumn{2}{p{18pt}|}{x2} & 20 ft. & 1/2 lb. & Piercing\tabularnewline
\hline
Javelin & 1 gp & 1d4 & 1d6 & \multicolumn{2}{p{18pt}|}{x2} & 30 ft. & 2 lb. & Piercing\tabularnewline
\hline
Sling--- &  & 1d3 & 1d4 & \multicolumn{2}{p{18pt}|}{x2} & 50 ft. & 0 lb. & Bludgeoning\tabularnewline
\hline
Bullets, sling (10) & 1 sp--- & --- & --- & \multicolumn{2}{p{18pt}|}{}--- &  & 5 
lb.--- & \tabularnewline
\hline
M\textbf{artial Weapons} & C\textbf{ost} & D\textbf{mg (S)} & D\textbf{mg (M)} & \multicolumn{2}{p{18pt}|}{C\textbf{ritical}} & R\textbf{ange 
Increment} & W\textbf{eight1} & T\textbf{ype}\textsuperscript{\textbf{2}}\tabularnewline
\hline
\multicolumn{9}{|p{255pt}|}{L\textit{ight Melee Weapons}}\tabularnewline
\hline
Axe, throwing & 8 gp & 1d4 & 1d6 & \multicolumn{2}{p{18pt}|}{x2} & 10 ft. & 2 lb. & Slashing\tabularnewline
\hline
Hammer, light & 1 gp & 1d3 & 1d4 & \multicolumn{2}{p{18pt}|}{x2} & 20 ft. & 2 lb. & Bludgeoning\tabularnewline
\hline
Handaxe & 6 gp & 1d4 & 1d6 & \multicolumn{2}{p{18pt}|}{x3}--- &  & 3 lb. & Slashing\tabularnewline
\hline
Kukri & 8 gp & 1d3 & 1d4 & \multicolumn{2}{p{18pt}|}{18-20/x2}--- &  & 2 lb. & Slashing\tabularnewline
\hline
Pick, light & 4 gp & 1d3 & 1d4 & \multicolumn{2}{p{18pt}|}{x4}--- &  & 3 lb. & Piercing\tabularnewline
\hline
Sap & 1 gp & 1d4\textsuperscript{\textbf{3}} & 1d6\textsuperscript{\textbf{3}} & \multicolumn{2}{p{18pt}|}{x2}--- &  & 2 
lb. & Bludgeoning\tabularnewline
\hline
Shield, light & special & 1d2 & 1d3 & \multicolumn{2}{p{18pt}|}{x2}--- &  & special & Bludgeoning\tabularnewline
\hline
Spiked armor & special & 1d4 & 1d6 & \multicolumn{2}{p{18pt}|}{x2}--- &  & special & Piercing\tabularnewline
\hline
Spiked shield, light & special & 1d3 & 1d4 & \multicolumn{2}{p{18pt}|}{x2}--- &  & special & Piercing\tabularnewline
\hline
Sword, short & 10 gp & 1d4 & 1d6 & \multicolumn{2}{p{18pt}|}{19-20/x2}--- &  & 2 
lb. & Piercing\tabularnewline
\hline
\multicolumn{9}{|p{255pt}|}{O\textit{ne-Handed Melee Weapons}}\tabularnewline
\hline
Battleaxe & 10 gp & 1d6 & 1d8 & \multicolumn{2}{p{18pt}|}{x3}--- &  & 6 lb. & Slashing\tabularnewline
\hline
Flail & 8 gp & 1d6 & 1d8 & \multicolumn{2}{p{18pt}|}{x2}--- &  & 5 lb. & Bludgeoning\tabularnewline
\hline
Longsword & 15 gp & 1d6 & 1d8 & \multicolumn{2}{p{18pt}|}{19-20/x2}--- &  & 4 lb. & Slashing\tabularnewline
\hline
Pick, heavy & 8 gp & 1d4 & 1d6 & \multicolumn{2}{p{18pt}|}{x4}--- &  & 6 lb. & Piercing\tabularnewline
\hline
Rapier & 20 gp & 1d4 & 1d6 & \multicolumn{2}{p{18pt}|}{18-20/x2}--- &  & 2 lb. & Piercing\tabularnewline
\hline
Scimitar & 15 gp & 1d4 & 1d6 & \multicolumn{2}{p{18pt}|}{18-20/x2}--- &  & 4 lb. & Slashing\tabularnewline
\hline
Shield, heavy & special & 1d3 & 1d4 & \multicolumn{2}{p{18pt}|}{x2}--- &  & special & Bludgeoning\tabularnewline
\hline
Spiked shield, heavy & special & 1d4 & 1d6 & \multicolumn{2}{p{18pt}|}{x2}--- &  & special & Piercing\tabularnewline
\hline
Trident & 15 gp & 1d6 & 1d8 & \multicolumn{2}{p{18pt}|}{x2} & 10 ft. & 4 lb. & Piercing\tabularnewline
\hline
Warhammer & 12 gp & 1d6 & 1d8 & \multicolumn{2}{p{18pt}|}{x3}--- &  & 5 lb. & Bludgeoning\tabularnewline
\hline
\multicolumn{9}{|p{255pt}|}{T\textit{wo-Handed Melee Weapons}}\tabularnewline
\hline
Falchion & 75 gp & 1d6 & 2d4 & \multicolumn{2}{p{18pt}|}{18-20/x2}--- &  & 8 lb. & Slashing\tabularnewline
\hline
Glaive\textsuperscript{\textbf{4}} & 8 gp & 1d8 & 1d10 & \multicolumn{2}{p{18pt}|}{x3}--- &  & 10 
lb. & Slashing\tabularnewline
\hline
Greataxe & 20 gp & 1d10 & 1d12 & \multicolumn{2}{p{18pt}|}{x3}--- &  & 12 lb. & Slashing\tabularnewline
\hline
Greatclub & 5 gp & 1d8 & 1d10 & \multicolumn{2}{p{18pt}|}{x2}--- &  & 8 lb. & Bludgeoning\tabularnewline
\hline
Flail, heavy & 15 gp & 1d8 & 1d10 & \multicolumn{2}{p{18pt}|}{19-20/x2}--- &  & 10 
lb. & Bludgeoning\tabularnewline
\hline
Greatsword & 50 gp & 1d10 & 2d6 & \multicolumn{2}{p{18pt}|}{19-20/x2}--- &  & 8 
lb. & Slashing\tabularnewline
\hline
Guisarme\textsuperscript{\textbf{4}} & 9 gp & 1d6 & 2d4 & \multicolumn{2}{p{18pt}|}{x3}--- &  & 12 
lb. & Slashing\tabularnewline
\hline
Halberd & 10 gp & 1d8 & 1d10 & \multicolumn{2}{p{18pt}|}{x3}--- &  & 12 lb. & Piercing 
or slashing\tabularnewline
\hline
Lance\textsuperscript{\textbf{4}} & 10 gp & 1d6 & 1d8 & \multicolumn{2}{p{18pt}|}{x3}--- &  & 10 
lb. & Piercing\tabularnewline
\hline
Ranseur\textsuperscript{\textbf{4}} & 10 gp & 1d6 & 2d4 & \multicolumn{2}{p{18pt}|}{x3}--- &  & 12 
lb. & Piercing\tabularnewline
\hline
Scythe & 18 gp & 1d6 & 2d4 & \multicolumn{2}{p{18pt}|}{x4}--- &  & 10 lb. & Piercing 
or slashing\tabularnewline
\hline
\multicolumn{9}{|p{255pt}|}{R\textit{anged Weapons}}\tabularnewline
\hline
Longbow & 75 gp & 1d6 & \multicolumn{2}{p{21pt}|}{1d8} & x3 & 100 ft. & 3 lb. & Piercing\tabularnewline
\hline
Arrows (20) & 1 gp--- & --- & \multicolumn{2}{p{21pt}|}{}--- & --- &  & 3 lb.--- & \tabularnewline
\hline
Longbow, composite & 100 gp & 1d6 & \multicolumn{2}{p{21pt}|}{1d8} & x3 & 110 ft. & 3 
lb. & Piercing\tabularnewline
\hline
Arrows (20) & 1 gp--- & --- & \multicolumn{2}{p{21pt}|}{}--- & --- &  & 3 lb.--- & \tabularnewline
\hline
Shortbow & 30 gp & 1d4 & \multicolumn{2}{p{21pt}|}{1d6} & x3 & 60 ft. & 2 lb. & Piercing\tabularnewline
\hline
Arrows (20) & 1 gp--- & --- & \multicolumn{2}{p{21pt}|}{}--- & --- &  & 3 lb.--- & \tabularnewline
\hline
Shortbow, composite & 75 gp & 1d4 & \multicolumn{2}{p{21pt}|}{1d6} & x3 & 70 ft. & 2 
lb. & Piercing\tabularnewline
\hline
Arrows (20) & 1 gp--- & --- & \multicolumn{2}{p{21pt}|}{}--- & --- &  & 3 lb.--- & \tabularnewline
\hline
E\textbf{xotic Weapons} & C\textbf{ost} & D\textbf{mg (S)} & \multicolumn{2}{p{21pt}|}{D\textbf{mg 
(M)}} & C\textbf{ritical} & R\textbf{ange Increment} & W\textbf{eight}\textsuperscript{\textbf{1}} & T\textbf{ype}\textsuperscript{\textbf{2}}\tabularnewline
\hline
\multicolumn{9}{|p{255pt}|}{L\textit{ight Melee Weapons}}\tabularnewline
\hline
Kama & 2 gp & 1d4 & 1d6 & \multicolumn{2}{p{18pt}|}{x2}--- &  & 2 lb. & Slashing\tabularnewline
\hline
Nunchaku & 2 gp & 1d4 & 1d6 & \multicolumn{2}{p{18pt}|}{x2}--- &  & 2 lb. & Bludgeoning\tabularnewline
\hline
Sai & 1 gp & 1d3 & 1d4 & \multicolumn{2}{p{18pt}|}{x2} & 10 ft. & 1 lb. & Bludgeoning\tabularnewline
\hline
Siangham & 3 gp & 1d4 & 1d6 & \multicolumn{2}{p{18pt}|}{x2}--- &  & 1 lb. & Piercing\tabularnewline
\hline
\multicolumn{9}{|p{255pt}|}{O\textit{ne-Handed Melee Weapons}}\tabularnewline
\hline
Sword, bastard & 35 gp & 1d8 & 1d10 & \multicolumn{2}{p{18pt}|}{19-20/x2}--- &  & 6 
lb. & Slashing\tabularnewline
\hline
Waraxe, dwarven & 30 gp & 1d8 & 1d10 & \multicolumn{2}{p{18pt}|}{x3}--- &  & 8 
lb. & Slashing\tabularnewline
\hline
Whip\textsuperscript{\textbf{4}} & 1 gp & 1d2\textsuperscript{\textbf{3}} & 1d3\textsuperscript{\textbf{3}} & \multicolumn{2}{p{18pt}|}{x2} &  & 2 
lb. & Slashing\tabularnewline
\hline
\multicolumn{9}{|p{255pt}|}{\subsection*{T\textit{wo-Handed Melee Weapons}}}\tabularnewline
\hline
Axe, orc double\textsuperscript{\textbf{5}} & 60 gp & 1d6/1d6 & 1d8/1d8 & \multicolumn{2}{p{18pt}|}{x3}--- &  & 15 
lb. & Slashing\tabularnewline
\hline
Chain, spiked\textsuperscript{\textbf{4}} & 25 gp & 1d6 & 2d4 & \multicolumn{2}{p{18pt}|}{x2}--- &  & 10 
lb. & Piercing\tabularnewline
\hline
Flail, dire\textsuperscript{\textbf{5}} & 90 gp & 1d6/1d6 & 1d8/1d8 & \multicolumn{2}{p{18pt}|}{x2}--- &  & 10 
lb. & Bludgeoning\tabularnewline
\hline
Hammer, \linebreak{}
gnome hooked\textsuperscript{\textbf{5}} & 20 gp & 1d6/1d4 & 1d8/1d6 & \multicolumn{2}{p{18pt}|}{x3/x4}--- &  & 6 
lb. & Bludgeoning and piercing\tabularnewline
\hline
Sword, two-bladed\textsuperscript{\textbf{5}} & 100 gp & 1d6/1d6 & 1d8/1d8 & \multicolumn{2}{p{18pt}|}{19-20/x2}--- &  & 10 
lb. & Slashing\tabularnewline
\hline
Urgrosh, dwarven\textsuperscript{\textbf{5}} & 50 gp & 1d6/1d4 & 1d8/1d6 & \multicolumn{2}{p{18pt}|}{x3}--- &  & 12 
lb. & Slashing or piercing\tabularnewline
\hline
\multicolumn{9}{|p{255pt}|}{R\textit{anged Weapons}}\tabularnewline
\hline
Bolas & 5 gp & 1d3\textsuperscript{\textbf{3}} & 1d4\textsuperscript{\textbf{3}} & \multicolumn{2}{p{18pt}|}{x2} & 10 
ft. & 2 lb. & Bludgeoning\tabularnewline
\hline
Crossbow, hand & 100 gp & 1d3 & 1d4 & \multicolumn{2}{p{18pt}|}{19-20/x2} & 30 
ft. & 2 lb. & Piercing\tabularnewline
\hline
Bolts (10) & 1 gp--- & --- & --- & \multicolumn{2}{p{18pt}|}{}--- &  & 1 lb.--- & \tabularnewline
\hline
Crossbow, \linebreak{}
repeating heavy & 400 gp & 1d8 & 1d10 & \multicolumn{2}{p{18pt}|}{19-20/x2} & 120 
ft. & 12 lb. & Piercing\tabularnewline
\hline
Bolts (5) & 1 gp--- & --- & --- & \multicolumn{2}{p{18pt}|}{} & 1 lb.--- &  & \tabularnewline
\hline
Crossbow, \linebreak{}
repeating light & 250 gp & 1d6 & 1d8 & \multicolumn{2}{p{18pt}|}{19-20/x2} & 80 
ft. & 6 lb. & Piercing\tabularnewline
\hline
Bolts (5) & 1 gp--- & --- & --- & \multicolumn{2}{p{18pt}|}{} & 1 lb.--- &  & \tabularnewline
\hline
Net & 20 gp--- & --- &  & \multicolumn{2}{p{18pt}|}{10 ft.} & 6 lb.--- &  & \tabularnewline
\hline
Shuriken (5) & 1 gp & 1 & 1d2 & \multicolumn{2}{p{18pt}|}{x2} & 10 ft. & 1/2 lb. & Piercing\tabularnewline
\hline
\multicolumn{9}{|p{255pt}|}{1 Weight figures are for Medium weapons. A Small weapon 
weighs half as much, and a Large weapon weighs twice as much.}\tabularnewline
\hline
\multicolumn{9}{|p{255pt}|}{2 When two types are given, the weapon is both types 
if the entry specifies ``and,'' or either type (player's choice at time of attack) 
if the entry specifies ``or.''}\tabularnewline
\hline
\multicolumn{9}{|p{255pt}|}{3 The weapon deals nonlethal damage rather than lethal 
damage.}\tabularnewline
\hline
\multicolumn{9}{|p{255pt}|}{4 Reach weapon.}\tabularnewline
\hline
\multicolumn{9}{|p{255pt}|}{5 Double weapon.}\tabularnewline
\hline
\end{tabular}

\vspace{12pt}
Weapons found on Table: Weapons that have special options for the wielder (``you'') 
are described below. Splash weapons are described under Special Substances and 
Items.

\textbf{Arrows:} An arrow used as a melee weapon is treated as a light improvised 
weapon (-4 penalty on attack rolls) and deals damage as a dagger of its size (critical 
multiplier x2). Arrows come in a leather quiver that holds 20 arrows. An arrow 
that hits its target is destroyed; one that misses has a 50\% chance of being destroyed 
or lost.

\textbf{Axe, Orc Double:} An orc double axe is a double weapon. You can fight with 
it as if fighting with two weapons, but if you do, you incur all the normal attack 
penalties associated with fighting with two weapons, just as if you were using 
a one-handed weapon and a light weapon.

A creature wielding an orc double axe in one hand can't use it as a double weapon---only 
one end of the weapon can be used in any given round.

\textbf{Bolas:} You can use this weapon to make a ranged trip attack against an 
opponent. You can't be tripped during your own trip attempt when using a set of 
bolas.

\textbf{Bolts:} A crossbow bolt used as a melee weapon is treated as a light improvised 
weapon (-4 penalty on attack rolls) and deals damage as a dagger of its size (crit 
x2). Bolts come in a wooden case that holds 10 bolts (or 5, for a repeating crossbow). 
A bolt that hits its target is destroyed; one that misses has a 50\% chance of 
being destroyed or lost.

\textbf{Bullets, Sling:} Bullets come in a leather pouch that holds 10 bullets. 
A bullet that hits its target is destroyed; one that misses has a 50\% chance of 
being destroyed or lost.

\textbf{Chain, Spiked:} A spiked chain has reach, so you can strike opponents 10 
feet away with it. In addition, unlike most other weapons with reach, it can be 
used against an adjacent foe.

You can make trip attacks with the chain. If you are tripped during your own trip 
attempt, you can drop the chain to avoid being tripped.

When using a spiked chain, you get a +2 bonus on opposed attack rolls made to disarm 
an opponent (including the roll to avoid being disarmed if such an attempt fails).

You can use the Weapon Finesse feat to apply your Dexterity modifier instead of 
your Strength modifier to attack rolls with a spiked chain sized for you, even 
though it isn't a light weapon for you.

\textbf{Crossbow, Hand:} You can draw a hand crossbow back by hand. Loading a hand 
crossbow is a move action that provokes attacks of opportunity.

You can shoot, but not load, a hand crossbow with one hand at no penalty. You can 
shoot a hand crossbow with each hand, but you take a penalty on attack rolls as 
if attacking with two light weapons.

\textbf{Crossbow, Heavy:} You draw a heavy crossbow back by turning a small winch. 
Loading a heavy crossbow is a full-round action that provokes attacks of opportunity.

Normally, operating a heavy crossbow requires two hands. However, you can shoot, 
but not load, a heavy crossbow with one hand at a -4 penalty on attack rolls. You 
can shoot a heavy crossbow with each hand, but you take a penalty on attack rolls 
as if attacking with two one-handed weapons. This penalty is cumulative with the 
penalty for one-handed firing.

\textbf{Crossbow, Light:} You draw a light crossbow back by pulling a lever. Loading 
a light crossbow is a move action that provokes attacks of opportunity.

Normally, operating a light crossbow requires two hands. However, you can shoot, 
but not load, a light crossbow with one hand at a -2 penalty on attack rolls. You 
can shoot a light crossbow with each hand, but you take a penalty on attack rolls 
as if attacking with two light weapons. This penalty is cumulative with the penalty 
for one-handed firing.

\textbf{Crossbow, Repeating:} The repeating crossbow (whether heavy or light) holds 
5 crossbow bolts. As long as it holds bolts, you can reload it by pulling the reloading 
lever (a free action). Loading a new case of 5 bolts is a full-round action that 
provokes attacks of opportunity.

You can fire a repeating crossbow with one hand or fire a repeating crossbow in 
each hand in the same manner as you would a normal crossbow of the same size. However, 
you must fire the weapon with two hands in order to use the reloading lever, and 
you must use two hands to load a new case of bolts.

\textbf{Dagger:} You get a +2 bonus on Sleight of Hand checks made to conceal a 
dagger on your body (see the Sleight of Hand skill).

\textbf{Flail, Dire: }A dire flail is a double weapon. You can fight with it as 
if fighting with two weapons, but if you do, you incur all the normal attack penalties 
associated with fighting with two weapons, just as if you were using a one-handed 
weapon and a light weapon. A creature wielding a dire flail in one hand can't use 
it as a double weapon--- only one end of the weapon can be used in any given round.

When using a dire flail, you get a +2 bonus on opposed attack rolls made to disarm 
an enemy (including the opposed attack roll to avoid being disarmed if such an 
attempt fails).

You can also use this weapon to make trip attacks. If you are tripped during your 
own trip attempt, you can drop the dire flail to avoid being tripped.

\textbf{Flail or Heavy Flail:} With a flail, you get a +2 bonus on opposed attack 
rolls made to disarm an enemy (including the roll to avoid being disarmed if such 
an attempt fails).

You can also use this weapon to make trip attacks. If you are tripped during your 
own trip attempt, you can drop the flail to avoid being tripped.

\textbf{Gauntlet:} This metal glove lets you deal lethal damage rather than nonlethal 
damage with unarmed strikes. A strike with a gauntlet is otherwise considered an 
unarmed attack. The cost and weight given are for a single gauntlet. Medium and 
heavy armors (except breastplate) come with gauntlets.

\textbf{Gauntlet, Spiked:} Your opponent cannot use a disarm action to disarm you 
of spiked gauntlets. The cost and weight given are for a single gauntlet. An attack 
with a spiked gauntlet is considered an armed attack.

\textbf{Glaive:} A glaive has reach. You can strike opponents 10 feet away with 
it, but you can't use it against an adjacent foe.

\textbf{Guisarme: }A guisarme has reach. You can strike opponents 10 feet away 
with it, but you can't use it against an adjacent foe.

You can also use it to make trip attacks. If you are tripped during your own trip 
attempt, you can drop the guisarme to avoid being tripped.

\textbf{Halberd:} If you use a ready action to set a halberd against a charge, 
you deal double damage on a successful hit against a charging character.

You can use a halberd to make trip attacks. If you are tripped during your own 
trip attempt, you can drop the halberd to avoid being tripped.

\textbf{Hammer, Gnome Hooked:} A gnome hooked hammer is a double weapon. You can 
fight with it as if fighting with two weapons, but if you do, you incur all the 
normal attack penalties associated with fighting with two weapons, just as if you 
were using a one-handed weapon and a light weapon. The hammer's blunt head is a 
bludgeoning weapon that deals 1d6 points of damage (crit x3). Its hook is a piercing 
weapon that deals 1d4 points of damage (crit x4). You can use either head as the 
primary weapon. The other head is the offhand weapon. A creature wielding a gnome 
hooked hammer in one hand can't use it as a double weapon---only one end of the 
weapon can be used in any given round.

You can use a gnome hooked hammer to make trip attacks. If you are tripped during 
your own trip attempt, you can drop the gnome hooked hammer to avoid being tripped.

Gnomes treat gnome hooked hammers as martial weapons.

\textbf{Javelin:} Since it is not designed for melee, you are treated as nonproficient 
with it and take a -4 penalty on attack rolls if you use a javelin as a melee weapon.

\textbf{Kama:} The kama is a special monk weapon. This designation gives a monk 
wielding a kama special options.

You can use a kama to make trip attacks. If you are tripped during your own trip 
attempt, you can drop the kama to avoid being tripped.

\textbf{Lance:} A lance deals double damage when used from the back of a charging 
mount. It has reach, so you can strike opponents 10 feet away with it, but you 
can't use it against an adjacent foe.

While mounted, you can wield a lance with one hand.

\textbf{Longbow:} You need at least two hands to use a bow, regardless of its size. 
A longbow is too unwieldy to use while you are mounted. If you have a penalty for 
low Strength, apply it to damage rolls when you use a longbow. If you have a bonus 
for high Strength, you can apply it to damage rolls when you use a composite longbow 
(see below) but not a regular longbow.

\textbf{Longbow, Composite:} You need at least two hands to use a bow, regardless 
of its size. You can use a composite longbow while mounted. All composite bows 
are made with a particular strength rating (that is, each requires a minimum Strength 
modifier to use with proficiency). If your Strength bonus is less than the strength 
rating of the composite bow, you can't effectively use it, so you take a -2 penalty 
on attacks with it. The default composite longbow requires a Strength modifier 
of +0 or higher to use with proficiency. A composite longbow can be made with a 
high strength rating to take advantage of an above-average Strength score; this 
feature allows you to add your Strength bonus to damage, up to the maximum bonus 
indicated for the bow. Each point of Strength bonus granted by the bow adds 100 
gp to its cost.

For purposes of weapon proficiency and similar feats, a composite longbow is treated 
as if it were a longbow.

\textbf{Longspear: }A longspear has reach. You can strike opponents 10 feet away 
with it, but you can't use it against an adjacent foe. If you use a ready action 
to set a longspear against a charge, you deal double damage on a successful hit 
against a charging character.

\textbf{Net:} A net is used to entangle enemies. When you throw a net, you make 
a ranged touch attack against your target. A net's maximum range is 10 feet. If 
you hit, the target is entangled. An entangled creature takes a -2 penalty on attack 
rolls and a -4 penalty on Dexterity, can move at only half speed, and cannot charge 
or run. If you control the trailing rope by succeeding on an opposed Strength check 
while holding it, the entangled creature can move only within the limits that the 
rope allows. If the entangled creature attempts to cast a spell, it must make a 
DC 15 Concentration check or be unable to cast the spell.

An entangled creature can escape with a DC 20 Escape Artist check (a full-round 
action). The net has 5 hit points and can be burst with a DC 25 Strength check 
(also a full-round action).

A net is useful only against creatures within one size category of you.

A net must be folded to be thrown effectively. The first time you throw your net 
in a fight, you make a normal ranged touch attack roll. After the net is unfolded, 
you take a -4 penalty on attack rolls with it. It takes 2 rounds for a proficient 
user to fold a net and twice that long for a nonproficient one to do so.

\textbf{Nunchaku:} The nunchaku is a special monk weapon. This designation gives 
a monk wielding a nunchaku special options. With a nunchaku, you get a +2 bonus 
on opposed attack rolls made to disarm an enemy (including the roll to avoid being 
disarmed if such an attempt fails).

\textbf{Quarterstaff:} A quarterstaff is a double weapon. You can fight with it 
as if fighting with two weapons, but if you do, you incur all the normal attack 
penalties associated with fighting with two weapons, just as if you were using 
a one-handed weapon and a light weapon. A creature wielding a quarterstaff in one 
hand can't use it as a double weapon---only one end of the weapon can be used in 
any given round.

The quarterstaff is a special monk weapon. This designation gives a monk wielding 
a quarterstaff special options.

\textbf{Ranseur:} A ranseur has reach. You can strike opponents 10 feet away with 
it, but you can't use it against an adjacent foe.

With a ranseur, you get a +2 bonus on opposed attack rolls made to disarm an opponent 
(including the roll to avoid being disarmed if such an attempt fails).

\textbf{Rapier: }You can use the Weapon Finesse feat to apply your Dexterity modifier 
instead of your Strength modifier to attack rolls with a rapier sized for you, 
even though it isn't a light weapon for you. You can't wield a rapier in two hands 
in order to apply 1-1/2 times your Strength bonus to damage.

\textbf{Sai:} With a sai, you get a +4 bonus on opposed attack rolls made to disarm 
an enemy (including the roll to avoid being disarmed if such an attempt fails).

The sai is a special monk weapon. This designation gives a monk wielding a sai 
special options.

\textbf{Scythe:} A scythe can be used to make trip attacks. If you are tripped 
during your own trip attempt, you can drop the scythe to avoid being tripped.

\textbf{Shield, Heavy or Light:} You can bash with a shield instead of using it 
for defense. See Armor for details.

\textbf{Shortbow: }You need at least two hands to use a bow, regardless of its 
size. You can use a shortbow while mounted. If you have a penalty for low Strength, 
apply it to damage rolls when you use a shortbow. If you have a bonus for high 
Strength, you can apply it to damage rolls when you use a composite shortbow (see 
below) but not a regular shortbow.

\textbf{Shortbow, Composite: }You need at least two hands to use a bow, regardless 
of its size. You can use a composite shortbow while mounted. All composite bows 
are made with a particular strength rating (that is, each requires a minimum Strength 
modifier to use with proficiency). If your Strength bonus is lower than the strength 
rating of the composite bow, you can't effectively use it, so you take a -2 penalty 
on attacks with it. The default composite shortbow requires a Strength modifier 
of +0 or higher to use with proficiency. A composite shortbow can be made with 
a high strength rating to take advantage of an above-average Strength score; this 
feature allows you to add your Strength bonus to damage, up to the maximum bonus 
indicated for the bow. Each point of Strength bonus granted by the bow adds 75 
gp to its cost. 

For purposes of weapon proficiency and similar feats, a composite shortbow is treated 
as if it were a shortbow.

\textbf{Shortspear:} A shortspear is small enough to wield one-handed. It may also 
be thrown.

\textbf{Shuriken:} A shuriken is a special monk weapon. This designation gives 
a monk wielding shuriken special options. A shuriken can't be used as a melee weapon.

Although they are thrown weapons, shuriken are treated as ammunition for the purposes 
of drawing them, crafting masterwork or otherwise special versions of them and 
what happens to them after they are thrown.

\textbf{Siangham: }The siangham is a special monk weapon. This designation gives 
a monk wielding a siangham special options.

\textbf{Sickle: }A sickle can be used to make trip attacks. If you are tripped 
during your own trip attempt, you can drop the sickle to avoid being tripped.

\textbf{Sling: }Your Strength modifier applies to damage rolls when you use a sling, 
just as it does for thrown weapons. You can fire, but not load, a sling with one 
hand. Loading a sling is a move action that requires two hands and provokes attacks 
of opportunity.

You can hurl ordinary stones with a sling, but stones are not as dense or as round 
as bullets. Thus, such an attack deals damage as if the weapon were designed for 
a creature one size category smaller than you and you take a -1 penalty on attack 
rolls.

\textbf{Spear:} A spear can be thrown. If you use a ready action to set a spear 
against a charge, you deal double damage on a successful hit against a charging 
character.

\textbf{Spiked Armor:} You can outfit your armor with spikes, which can deal damage 
in a grapple or as a separate attack. See Armor for details.

\textbf{Spiked Shield, Heavy or Light:} You can bash with a spiked shield instead 
of using it for defense. See Armor for details.

\textbf{Strike, Unarmed: }A Medium character deals 1d3 points of nonlethal damage 
with an unarmed strike. A Small character deals 1d2 points of nonlethal damage. 
A monk or any character with the Improved Unarmed Strike feat can deal lethal or 
nonlethal damage with unarmed strikes, at her option. The damage from an unarmed 
strike is considered weapon damage for the purposes of effects that give you a 
bonus on weapon damage rolls.

An unarmed strike is always considered a light weapon. Therefore, you can use the 
Weapon Finesse feat to apply your Dexterity modifier instead of your Strength modifier 
to attack rolls with an unarmed strike.

\textbf{Sword, Bastard:} A bastard sword is too large to use in one hand without 
special training; thus, it is an exotic weapon. A character can use a bastard sword 
two-handed as a martial weapon.

\textbf{Sword, Two-Bladed:} A two-bladed sword is a double weapon. You can fight 
with it as if fighting with two weapons, but if you do, you incur all the normal 
attack penalties associated with fighting with two weapons, just as if you were 
using a one-handed weapon and a light weapon. A creature wielding a two-bladed 
sword in one hand can't use it as a double weapon---only one end of the weapon 
can be used in any given round.

\textbf{Trident:} This weapon can be thrown. If you use a ready action to set a 
trident against a charge, you deal double damage on a successful hit against a 
charging character.

\textbf{Urgrosh, Dwarven:} A dwarven urgrosh is a double weapon. You can fight 
with it as if fighting with two weapons, but if you do, you incur all the normal 
attack penalties associated with fighting with two weapons, just as if you were 
using a one-handed weapon and a light weapon. The urgrosh's axe head is a slashing 
weapon that deals 1d8 points of damage. Its spear head is a piercing weapon that 
deals 1d6 points of damage. You can use either head as the primary weapon. The 
other is the off-hand weapon. A creature wielding a dwarven urgrosh in one hand 
can't use it as a double weapon---only one end of the weapon can be used in any 
given round.

If you use a ready action to set an urgrosh against a charge, you deal double damage 
if you score a hit against a charging character. If you use an urgrosh against 
a charging character, the spear head is the part of the weapon that deals damage.

Dwarves treat dwarven urgroshes as martial weapons.

\textbf{Waraxe, Dwarven: }A dwarven waraxe is too large to use in one hand without 
special training; thus, it is an exotic weapon. A Medium character can use a dwarven 
waraxe two-handed as a martial weapon, or a Large creature can use it one-handed 
in the same way. A dwarf treats a dwarven waraxe as a martial weapon even when 
using it in one hand.

\textbf{Whip:} A whip deals nonlethal damage. It deals no damage to any creature 
with an armor bonus of +1 or higher or a natural armor bonus of +3 or higher. The 
whip is treated as a melee weapon with 15-foot reach, though you don't threaten 
the area into which you can make an attack. In addition, unlike most other weapons 
with reach, you can use it against foes anywhere within your reach (including adjacent 
foes).

Using a whip provokes an attack of opportunity, just as if you had used a ranged 
weapon.

You can make trip attacks with a whip. If you are tripped during your own trip 
attempt, you can drop the whip to avoid being tripped.

When using a whip, you get a +2 bonus on opposed attack rolls made to disarm an 
opponent (including the roll to keep from being disarmed if the attack fails).

You can use the Weapon Finesse feat to apply your Dexterity modifier instead of 
your Strength modifier to attack rolls with a whip sized for you, even though it 
isn't a light weapon for you.

\vspace{12pt}
MASTERWORK WEAPONS

A masterwork weapon is a finely crafted version of a normal weapon. Wielding it 
provides a +1 enhancement bonus on attack rolls.

You can't add the masterwork quality to a weapon after it is created; it must be 
crafted as a masterwork weapon (see the Craft skill). The masterwork quality adds 
300 gp to the cost of a normal weapon (or 6 gp to the cost of a single unit of 
ammunition). Adding the masterwork quality to a double weapon costs twice the normal 
increase (+600 gp).

Masterwork ammunition is damaged (effectively destroyed) when used. The enhancement 
bonus of masterwork ammunition does not stack with any enhancement bonus of the 
projectile weapon firing it.

All magic weapons are automatically considered to be of masterwork quality. The 
enhancement bonus granted by the masterwork quality doesn't stack with the enhancement 
bonus provided by the weapon's magic.

Even though some types of armor and shields can be used as weapons, you can't create 
a masterwork version of such an item that confers an enhancement bonus on attack 
rolls. Instead, masterwork armor and shields have lessened armor check penalties.

\vspace{12pt}
{\LARGE{}ARMOR}

\vspace{12pt}
ARMOR QUALITIES

To wear heavier armor effectively, a character can select the Armor Proficiency 
feats, but most classes are automatically proficient with the armors that work 
best for them.

Armor and shields can take damage from some types of attacks. 

Here is the format for armor entries (given as column headings on Table: Armor 
and Shields, below).

\textbf{Cost:} The cost of the armor for Small or Medium humanoid creatures. See 
Armor for Unusual Creatures, below, for armor prices for other creatures.

\vspace{12pt}
\textbf{Armor/Shield Bonus:} Each armor grants an armor bonus to AC, while shields 
grant a shield bonus to AC. The armor bonus from a suit of armor doesn't stack 
with other effects or items that grant an armor bonus. Similarly, the shield bonus 
from a shield doesn't stack with other effects that grant a shield bonus.

\textbf{Maximum Dex Bonus:} This number is the maximum Dexterity bonus to AC that 
this type of armor allows. Heavier armors limit mobility, reducing the wearer's 
ability to dodge blows. This restriction doesn't affect any other Dexterity-related 
abilities.

Even if a character's Dexterity bonus to AC drops to 0 because of armor, this situation 
does not count as losing a Dexterity bonus to AC. 

Your character's encumbrance (the amount of gear he or she carries) may also restrict 
the maximum Dexterity bonus that can be applied to his or her Armor Class.

\textit{Shields: }Shields do not affect a character's maximum Dexterity bonus.

\textbf{Armor Check Penalty:} Any armor heavier than leather hurts a character's 
ability to use some skills. An armor check penalty number is the penalty that applies 
to Balance, Climb, Escape Artist, Hide, Jump, Move Silently, Sleight of Hand, and 
Tumble checks by a character wearing a certain kind of armor. Double the normal 
armor check penalty is applied to Swim checks. A character's encumbrance (the amount 
of gear carried, including armor) may also apply an armor check penalty.

\textit{Shields: }If a character is wearing armor and using a shield, both armor 
check penalties apply.

\textit{Nonproficient with Armor Worn: }A character who wears armor and/or uses 
a shield with which he or she is not proficient takes the armor's (and/or shield's) 
armor check penalty on attack rolls and on all Strength-based and Dexterity-based 
ability and skill checks. The penalty for nonproficiency with armor stacks with 
the penalty for nonproficiency with shields.

\textit{Sleeping in Armor: }A character who sleeps in medium or heavy armor is 
automatically fatigued the next day. He or she takes a -2 penalty on Strength and 
Dexterity and can't charge or run. Sleeping in light armor does not cause fatigue.

\textbf{Arcane Spell Failure:} Armor interferes with the gestures that a spellcaster 
must make to cast an arcane spell that has a somatic component. Arcane spellcasters 
face the possibility of arcane spell failure if they're wearing armor. Bards can 
wear light armor without incurring any arcane spell failure chance for their bard 
spells.

\textit{Casting an Arcane Spell in Armor: }A character who casts an arcane spell 
while wearing armor must usually make an arcane spell failure roll. The number 
in the Arcane Spell Failure Chance column on Table: Armor and Shields is the chance 
that the spell fails and is ruined. If the spell lacks a somatic component, however, 
it can be cast with no chance of arcane spell failure.

\textit{Shields: }If a character is wearing armor and using a shield, add the two 
numbers together to get a single arcane spell failure chance.

\textbf{Speed:} Medium or heavy armor slows the wearer down. The number on Table: 
Armor and Shields is the character's speed while wearing the armor. Humans, elves, 
half-elves, and half-orcs have an unencumbered speed of 30 feet.

They use the first column. Dwarves, gnomes, and halflings have an unencumbered 
speed of 20 feet. They use the second column. Remember, however, that a dwarf 's 
land speed remains 20 feet even in medium or heavy armor or when carrying a medium 
or heavy load.

\textit{Shields: }Shields do not affect a character's speed.

\textbf{Weight:} This column gives the weight of the armor sized for a Medium wearer. 
Armor fitted for Small characters weighs half as much, and armor for Large characters 
weighs twice as much.

\vspace{12pt}
\begin{tabular}{|>{\raggedright}p{58pt}|>{\raggedright}p{19pt}|>{\raggedright}p{36pt}|>{\raggedright}p{28pt}|>{\raggedright}p{35pt}|>{\raggedright}p{38pt}|>{\raggedright}p{17pt}|>{\raggedright}p{16pt}|>{\raggedright}p{20pt}|}
\hline
\multicolumn{9}{|p{272pt}|}{T\textbf{able: Armor and Shields}}\tabularnewline
\hline
  &  &  &  &  & ---- & \multicolumn{2}{p{33pt}|}{ \textbf{Speed ----}} & \tabularnewline
\hline
A\textbf{rmor} & C\textbf{ost} & A\textbf{rmor/Shield }\linebreak{}
\textbf{Bonus} & M\textbf{aximum }\linebreak{}
\textbf{Dex Bonus} & A\textbf{rmor }\linebreak{}
\textbf{Check Penalty} & A\textbf{rcane Spell }\linebreak{}
\textbf{Failure Chance} & (\textbf{30 ft.)} & (\textbf{20 ft.)} & W\textbf{eight}\textsuperscript{\textbf{1}}\tabularnewline
\hline
Light armor &  &  &  &  &  &  &  & \tabularnewline
\hline
Padded & 5 gp & +1 & +8 & 0 & 5\% & 30 ft. & 20 ft. & 10 lb.\tabularnewline
\hline
Leather & 10 gp & +2 & +6 & 0 & 10\% & 30 ft. & 20 ft. & 15 lb.\tabularnewline
\hline
Studded leather & 25 gp & +3 & +5- & 1 & 15\% & 30 ft. & 20 ft. & 20 lb.\tabularnewline
\hline
Chain shirt & 100 gp & +4 & +4- & 2 & 20\% & 30 ft. & 20 ft. & 25 lb.\tabularnewline
\hline
Medium armor &  &  &  &  &  &  &  & \tabularnewline
\hline
Hide & 15 gp & +3 & +4- & 3 & 20\% & 20 ft. & 15 ft. & 25 lb.\tabularnewline
\hline
Scale mail & 50 gp & +4 & +3- & 4 & 25\% & 20 ft. & 15 ft. & 30 lb.\tabularnewline
\hline
Chainmail & 150 gp & +5 & +2- & 5 & 30\% & 20 ft. & 15 ft. & 40 lb.\tabularnewline
\hline
Breastplate & 200 gp & +5 & +3- & 4 & 25\% & 20 ft. & 15 ft. & 30 lb.\tabularnewline
\hline
Heavy armor &  &  &  &  &  &  &  & \tabularnewline
\hline
Splint mail & 200 gp & +6 & +0- & 7 & 40\% & 20 ft.\textsuperscript{\textbf{2}} & 15 
ft.\textsuperscript{\textbf{2}} & 45 lb.\tabularnewline
\hline
Banded mail & 250 gp & +6 & +1- & 6 & 35\% & 20 ft.\textsuperscript{\textbf{2}} & 15 
ft.\textsuperscript{\textbf{2}} & 35 lb.\tabularnewline
\hline
Half-plate & 600 gp & +7 & +0- & 7 & 40\% & 20 ft.\textsuperscript{\textbf{2}} & 15 
ft.\textsuperscript{\textbf{2}} & 50 lb.\tabularnewline
\hline
Full plate & 1,500 gp & +8 & +1- & 6 & 35\% & 20 ft.\textsuperscript{\textbf{2}} & 15 
ft.\textsuperscript{\textbf{2}} & 50 lb.\tabularnewline
\hline
Shields &  &  &  &  &  &  &  & \tabularnewline
\hline
Buckler & 15 gp & +1--- & - & 1 & 5\%--- & --- &  & 5 lb.\tabularnewline
\hline
Shield, light wooden & 3 gp & +1--- & - & 1 & 5\%--- & --- &  & 5 lb.\tabularnewline
\hline
Shield, light steel & 9 gp & +1--- & - & 1 & 5\%--- & --- &  & 6 lb.\tabularnewline
\hline
Shield, heavy wooden & 7 gp & +2--- & - & 2 & 15\%--- & --- &  & 10 lb.\tabularnewline
\hline
Shield, heavy steel & 20 gp & +2--- & - & 2 & 15\%--- & --- &  & 15 lb.\tabularnewline
\hline
Shield, tower & 30 gp & +4\textsuperscript{\textbf{3}} & +2- & 10 & 50\%--- & --- &  & 45 
lb.\tabularnewline
\hline
Extras &  &  &  &  &  &  &  & \tabularnewline
\hline
Armor spikes & +50 gp--- & --- & --- & --- & --- & --- &  & +10 lb.\tabularnewline
\hline
Gauntlet, locked & 8 gp--- & --- &  & Special & 4--- & --- &  & +5 lb.\tabularnewline
\hline
Shield spikes & +10 gp--- & --- & --- & --- & --- & --- &  & +5 lb.\tabularnewline
\hline
\multicolumn{9}{|p{272pt}|}{1 Weight figures are for armor sized to fit Medium 
characters. Armor fitted for Small characters weighs half as much, and armor fitted 
for Large characters weighs twice as much.}\tabularnewline
\hline
\multicolumn{9}{|p{272pt}|}{2 When running in heavy armor, you move only triple 
your speed, not quadruple.}\tabularnewline
\hline
\multicolumn{9}{|p{272pt}|}{3 A tower shield can instead grant you cover. See the 
description.}\tabularnewline
\hline
\multicolumn{9}{|p{272pt}|}{4 Hand not free to cast spells.}\tabularnewline
\hline
\end{tabular}

\vspace{12pt}
ARMOR DESCRIPTIONS

Any special benefits or accessories to the types of armor found on Table: Armor 
and Shields are described below.

\textbf{Armor Spikes:} You can have spikes added to your armor, which allow you 
to deal extra piercing damage (see Table: Weapons) on a successful grapple attack. 
The spikes count as a martial weapon. If you are not proficient with them, you 
take a -4 penalty on grapple checks when you try to use them. You can also make 
a regular melee attack (or off-hand attack) with the spikes, and they count as 
a light weapon in this case. (You can't also make an attack with armor spikes if 
you have already made an attack with another off-hand weapon, and vice versa.)

An enhancement bonus to a suit of armor does not improve the spikes' effectiveness, 
but the spikes can be made into magic weapons in their own right.

\textbf{Banded Mail: }The suit includes gauntlets.

\textbf{Breastplate: }It comes with a helmet and greaves. 

\textbf{Buckler:} This small metal shield is worn strapped to your forearm. You 
can use a bow or crossbow without penalty while carrying it. You can also use your 
shield arm to wield a weapon (whether you are using an off-hand weapon or using 
your off hand to help wield a two-handed weapon), but you take a -1 penalty on 
attack rolls while doing so. This penalty stacks with those that may apply for 
fighting with your off hand and for fighting with two weapons. In any case, if 
you use a weapon in your off hand, you don't get the buckler's AC bonus for the 
rest of the round.

You can't bash someone with a buckler.

\textbf{Chain Shirt: }A chain shirt comes with a steel cap.

\textbf{Chainmail: }The suit includes gauntlets.

\textbf{Full Plate:} The suit includes gauntlets, heavy leather boots, a visored 
helmet, and a thick layer of padding that is worn underneath the armor. Each suit 
of full plate must be individually fitted to its owner by a master armorsmith, 
although a captured suit can be resized to fit a new owner at a cost of 200 to 
800 (2d4x100) gold pieces.

\textbf{Gauntlet, Locked:} This armored gauntlet has small chains and braces that 
allow the wearer to attach a weapon to the gauntlet so that it cannot be dropped 
easily. It provides a +10 bonus on any roll made to keep from being disarmed in 
combat. Removing a weapon from a locked gauntlet or attaching a weapon to a locked 
gauntlet is a full-round action that provokes attacks of opportunity.

The price given is for a single locked gauntlet. The weight given applies only 
if you're wearing a breastplate, light armor, or no armor. Otherwise, the locked 
gauntlet replaces a gauntlet you already have as part of the armor.

While the gauntlet is locked, you can't use the hand wearing it for casting spells 
or employing skills. (You can still cast spells with somatic components, provided 
that your other hand is free.)

Like a normal gauntlet, a locked gauntlet lets you deal lethal damage rather than 
nonlethal damage with an unarmed strike.

\textbf{Half-Plate:} The suit includes gauntlets.

\textbf{Scale Mail:} The suit includes gauntlets.

\textbf{Shield, Heavy, Wooden or Steel:} You strap a shield to your forearm and 
grip it with your hand. A heavy shield is so heavy that you can't use your shield 
hand for anything else.

\textit{Wooden or Steel: }Wooden and steel shields offer the same basic protection, 
though they respond differently to special attacks.

\textit{Shield Bash Attacks: }You can bash an opponent with a heavy shield, using 
it as an off-hand weapon. See Table: Weapons for the damage dealt by a shield bash. 
Used this way, a heavy shield is a martial bludgeoning weapon. For the purpose 
of penalties on attack rolls, treat a heavy shield as a one-handed weapon. If you 
use your shield as a weapon, you lose its AC bonus until your next action (usually 
until the next round). An enhancement bonus on a shield does not improve the effectiveness 
of a shield bash made with it, but the shield can be made into a magic weapon in 
its own right.

\textbf{Shield, Light, Wooden or Steel:} You strap a shield to your forearm and 
grip it with your hand. A light shield's weight lets you carry other items in that 
hand, although you cannot use weapons with it.

\textit{Wooden or Steel: }Wooden and steel shields offer the same basic protection, 
though they respond differently to special attacks.

\textit{Shield Bash Attacks: }You can bash an opponent with a light shield, using 
it as an off-hand weapon. See Table: Weapons for the damage dealt by a shield bash. 
Used this way, a light shield is a martial bludgeoning weapon. For the purpose 
of penalties on attack rolls, treat a light shield as a light weapon. If you use 
your shield as a weapon, you lose its AC bonus until your next action (usually 
until the next round). An enhancement bonus on a shield does not improve the effectiveness 
of a shield bash made with it, but the shield can be made into a magic weapon in 
its own right.

\textbf{Shield, Tower:} This massive wooden shield is nearly as tall as you are. 
In most situations, it provides the indicated shield bonus to your AC. However, 
you can instead use it as total cover, though you must give up your attacks to 
do so. The shield does not, however, provide cover against targeted spells; a spellcaster 
can cast a spell on you by targeting the shield you are holding. You cannot bash 
with a tower shield, nor can you use your shield hand for anything else.

When employing a tower shield in combat, you take a -2 penalty on attack rolls 
because of the shield's encumbrance.

\textbf{Shield Spikes:} When added to your shield, these spikes turn it into a 
martial piercing weapon that increases the damage dealt by a shield bash as if 
the shield were designed for a creature one size category larger than you. You 
can't put spikes on a buckler or a tower shield. Otherwise, attacking with a spiked 
shield is like making a shield bash attack (see above).

An enhancement bonus on a spiked shield does not improve the effectiveness of a 
shield bash made with it, but a spiked shield can be made into a magic weapon in 
its own right.

\textbf{Splint Mail:} The suit includes gauntlets.

\vspace{12pt}
MASTERWORK ARMOR

Just as with weapons, you can purchase or craft masterwork versions of armor or 
shields. Such a well-made item functions like the normal version, except that its 
armor check penalty is lessened by 1. 

A masterwork suit of armor or shield costs an extra 150 gp over and above the normal 
cost for that type of armor or shield.

The masterwork quality of a suit of armor or shield never provides a bonus on attack 
or damage rolls, even if the armor or shield is used as a weapon.

All magic armors and shields are automatically considered to be of masterwork quality.

You can't add the masterwork quality to armor or a shield after it is created; 
it must be crafted as a masterwork item.

\vspace{12pt}
ARMOR FOR UNUSUAL CREATURES

Armor and shields for unusually big creatures, unusually little creatures, and 
nonhumanoid creatures have different costs and weights from those given on Table: 
Armor and Shields. Refer to the appropriate line on the table below and apply the 
multipliers to cost and weight for the armor type in question.

\begin{tabular}{|>{\raggedright}p{63pt}|>{\raggedright}p{18pt}|>{\raggedright}p{29pt}|>{\raggedright}p{18pt}|>{\raggedright}p{29pt}|}
\hline
  & \multicolumn{2}{p{48pt}|}{H\textbf{umanoid}} & \multicolumn{2}{p{48pt}|}{N\textbf{onhumanoid}}\tabularnewline
\hline
S\textbf{ize} & C\textbf{ost} & W\textbf{eight} & C\textbf{ost} & W\textbf{eight}\tabularnewline
\hline
Tiny or smaller\textsuperscript{\textbf{1}} & x1/2 & x1/10 & x1 & x1/10\tabularnewline
\hline
Small & x1 & x1/2 & x2 & x1/2\tabularnewline
\hline
Medium & x1 & x1 & x2 & x1\tabularnewline
\hline
Large & x2 & x2 & x4 & x2\tabularnewline
\hline
Huge & x4 & x5 & x8 & x5\tabularnewline
\hline
Gargantuan & x8 & x8 & x16 & x8\tabularnewline
\hline
Colossal & x16 & x12 & x32 & x12\tabularnewline
\hline
\multicolumn{5}{|p{160pt}|}{1 Divide armor bonus by 2.}\tabularnewline
\hline
\end{tabular}

\vspace{12pt}
GETTING INTO AND OUT OF ARMOR

The time required to don armor depends on its type; see Table: Donning Armor.

\textbf{Don:} This column tells how long it takes a character to put the armor 
on. (One minute is 10 rounds.) Readying (strapping on) a shield is only a move 
action.

\textbf{Don Hastily:} This column tells how long it takes to put the armor on in 
a hurry. The armor check penalty and armor bonus for hastily donned armor are each 
1 point worse than normal. 

\textbf{Remove:} This column tells how long it takes to get the armor off. Loosing 
a shield (removing it from the arm and dropping it) is only a move action.

\begin{tabular}{|>{\raggedright}p{178pt}|>{\raggedright}p{43pt}|>{\raggedright}p{36pt}|>{\raggedright}p{44pt}|}
\hline
\multicolumn{4}{|p{302pt}|}{T\textbf{able: Donning Armor}}\tabularnewline
\hline
A\textbf{rmor Type} & D\textbf{on} & D\textbf{on Hastily} & R\textbf{emove}\tabularnewline
\hline
Shield (any) & 1 move action & n/a & 1 move action\tabularnewline
\hline
Padded, leather, hide, studded leather, or chain shirt & 1 minute & 5 rounds & 1 
minute\textsuperscript{\textbf{1}}\tabularnewline
\hline
Breastplate, scale mail, chainmail, banded mail, or splint mail & 4 minutes\textsuperscript{\textbf{1}} & 1 
minute & 1 minute\textsuperscript{\textbf{1}}\tabularnewline
\hline
Half-plate or full plate & 4 minutes\textsuperscript{\textbf{2}} & 4 minutes\textsuperscript{\textbf{1}} & 1d4+1 
minutes\textsuperscript{\textbf{1}}\tabularnewline
\hline
\multicolumn{4}{|p{302pt}|}{1 If the character has some help, cut this time in 
half. A single character doing nothing else can help one or two adjacent characters. 
Two characters can't help each other don armor at the same time.}\tabularnewline
\hline
\multicolumn{4}{|p{302pt}|}{2 The wearer must have help to don this armor. Without 
help, it can be donned only hastily.}\tabularnewline
\hline
\end{tabular}

\vspace{12pt}
{\LARGE{}GOODS AND SERVICES}

\begin{tabular}{|>{\raggedright}p{182pt}|>{\raggedright}p{44pt}|>{\raggedright}p{40pt}|}
\hline
T\textbf{able: Goods and Services} &  & \tabularnewline
\hline
A\textbf{dventuring Gear} &  & \tabularnewline
\hline
\section*{I\textbf{tem}} & C\textbf{ost} & W\textbf{eight}\tabularnewline
\hline
Backpack (empty) & 2 gp & 2 lb.\textsuperscript{\textbf{1}}\tabularnewline
\hline
Barrel (empty) & 2 gp & 30 lb.\tabularnewline
\hline
Basket (empty) & 4 sp & 1 lb.\tabularnewline
\hline
Bedroll & 1 sp & 5 lb.\textsuperscript{\textbf{1}}\tabularnewline
\hline
Bell & 1 gp--- & \tabularnewline
\hline
Blanket, winter & 5 sp & 3 lb.\textsuperscript{\textbf{1}}\tabularnewline
\hline
Block and tackle & 5 gp & 5 lb.\tabularnewline
\hline
Bottle, wine, glass & 2 gp--- & \tabularnewline
\hline
Bucket (empty) & 5 sp & 2 lb.\tabularnewline
\hline
Caltrops & 1 gp & 2 lb.\tabularnewline
\hline
Candle & 1 cp--- & \tabularnewline
\hline
Canvas (sq. yd.) & 1 sp & 1 lb.\tabularnewline
\hline
Case, map or scroll & 1 gp & 1/2 lb.\tabularnewline
\hline
Chain (10 ft.) & 30 gp & 2 lb.\tabularnewline
\hline
Chalk, 1 piece & 1 cp--- & \tabularnewline
\hline
Chest (empty) & 2 gp & 25 lb.\tabularnewline
\hline
Crowbar & 2 gp & 5 lb.\tabularnewline
\hline
Firewood (per day) & 1 cp & 20 lb.\tabularnewline
\hline
Fishhook & 1 sp--- & \tabularnewline
\hline
Fishing net, 25 sq. ft. & 4 gp & 5 lb.\tabularnewline
\hline
Flask (empty) & 3 cp & 1-1/2 lb.\tabularnewline
\hline
Flint and steel & 1 gp--- & \tabularnewline
\hline
Grappling hook & 1 gp & 4 lb.\tabularnewline
\hline
Hammer & 5 sp & 2 lb.\tabularnewline
\hline
Ink (1 oz. vial) & 8 gp--- & \tabularnewline
\hline
Inkpen & 1 sp--- & \tabularnewline
\hline
Jug, clay & 3 cp & 9 lb.\tabularnewline
\hline
Ladder, 10-foot & 5 cp & 20 lb.\tabularnewline
\hline
Lamp, common & 1 sp & 1 lb.\tabularnewline
\hline
Lantern, bullseye & 12 gp & 3 lb.\tabularnewline
\hline
Lantern, hooded & 7 gp & 2 lb.\tabularnewline
\hline
Lock &   & 1 lb.\tabularnewline
\hline
Very simple & 20 gp & 1 lb.\tabularnewline
\hline
Average & 40 gp & 1 lb.\tabularnewline
\hline
Good & 80 gp & 1 lb.\tabularnewline
\hline
Amazing & 150 gp & 1 lb.\tabularnewline
\hline
Manacles & 15 gp & 2 lb.\tabularnewline
\hline
Manacles, masterwork & 50 gp & 2 lb.\tabularnewline
\hline
Mirror, small steel & 10 gp & 1/2 lb.\tabularnewline
\hline
Mug/Tankard, clay & 2 cp & 1 lb.\tabularnewline
\hline
Oil (1-pint flask) & 1 sp & 1 lb.\tabularnewline
\hline
Paper (sheet) & 4 sp--- & \tabularnewline
\hline
Parchment (sheet) & 2 sp--- & \tabularnewline
\hline
Pick, miner's & 3 gp & 10 lb.\tabularnewline
\hline
Pitcher, clay & 2 cp & 5 lb.\tabularnewline
\hline
Piton & 1 sp & 1/2 lb.\tabularnewline
\hline
Pole, 10-foot & 2 sp & 8 lb.\tabularnewline
\hline
Pot, iron & 5 sp & 10 lb.\tabularnewline
\hline
Pouch, belt (empty) & 1 gp & 1/2 lb.\textsuperscript{\textbf{1}}\tabularnewline
\hline
Ram, portable & 10 gp & 20 lb.\tabularnewline
\hline
Rations, trail (per day) & 5 sp & 1 lb.\textsuperscript{\textbf{1}}\tabularnewline
\hline
Rope, hempen (50 ft.) & 1 gp & 10 lb.\tabularnewline
\hline
Rope, silk (50 ft.) & 10 gp & 5 lb.\tabularnewline
\hline
Sack (empty) & 1 sp & 1/2 lb.\textsuperscript{\textbf{1}}\tabularnewline
\hline
Sealing wax & 1 gp & 1 lb.\tabularnewline
\hline
Sewing needle & 5 sp--- & \tabularnewline
\hline
Signal whistle & 8 sp--- & \tabularnewline
\hline
Signet ring & 5 gp--- & \tabularnewline
\hline
Sledge & 1 gp & 10 lb.\tabularnewline
\hline
Soap (per lb.) & 5 sp & 1 lb.\tabularnewline
\hline
Spade or shovel & 2 gp & 8 lb.\tabularnewline
\hline
Spyglass & 1,000 gp & 1 lb.\tabularnewline
\hline
Tent & 10 gp & 20 lb.\textsuperscript{\textbf{1}}\tabularnewline
\hline
Torch & 1 cp & 1 lb.\tabularnewline
\hline
Vial, ink or potion & 1 gp & 1/10 lb.\tabularnewline
\hline
Waterskin & 1 gp & 4 lb.\textsuperscript{\textbf{1}}\tabularnewline
\hline
Whetstone & 2 cp & 1 lb.\tabularnewline
\hline
S\textbf{pecial Substances and Items} &  & \tabularnewline
\hline
I\textbf{tem} & C\textbf{ost} & W\textbf{eight}\tabularnewline
\hline
Acid (flask) & 10 gp & 1 lb.\tabularnewline
\hline
Alchemist's fire (flask) & 20 gp & 1 lb.\tabularnewline
\hline
Antitoxin (vial) & 50 gp--- & \tabularnewline
\hline
Everburning torch & 110 gp & 1 lb.\tabularnewline
\hline
Holy water (flask) & 25 gp & 1 lb.\tabularnewline
\hline
Smokestick & 20 gp & 1/2 lb.\tabularnewline
\hline
Sunrod & 2 gp & 1 lb.\tabularnewline
\hline
Tanglefoot bag & 50 gp & 4 lb.\tabularnewline
\hline
Thunderstone & 30 gp & 1 lb.\tabularnewline
\hline
Tindertwig & 1 gp--- & \tabularnewline
\hline
T\textbf{ools and Skill Kits} &  & \tabularnewline
\hline
I\textbf{tem} & C\textbf{ost} & W\textbf{eight}\tabularnewline
\hline
Alchemist's lab & 500 gp & 40 lb.\tabularnewline
\hline
Artisan's tools & 5 gp & 5 lb.\tabularnewline
\hline
Artisan's tools, masterwork & 55 gp & 5 lb.\tabularnewline
\hline
Climber's kit & 80 gp & 5 lb.\textsuperscript{\textbf{1}}\tabularnewline
\hline
Disguise kit & 50 gp & 8 lb.\textsuperscript{\textbf{1}}\tabularnewline
\hline
Healer's kit & 50 gp & 1 lb.\tabularnewline
\hline
Holly and mistletoe--- & --- & \tabularnewline
\hline
Holy symbol, wooden & 1 gp--- & \tabularnewline
\hline
Holy symbol, silver & 25 gp & 1 lb.\tabularnewline
\hline
Hourglass & 25 gp & 1 lb.\tabularnewline
\hline
Magnifying glass & 100 gp--- & \tabularnewline
\hline
Musical instrument, common & 5 gp & 3 lb.\textsuperscript{\textbf{1}}\tabularnewline
\hline
Musical instrument, masterwork & 100 gp & 3 lb.\textsuperscript{\textbf{1}}\tabularnewline
\hline
Scale, merchant's & 2 gp & 1 lb.\tabularnewline
\hline
Spell component pouch & 5 gp & 2 lb.\tabularnewline
\hline
Spellbook, wizard's (blank) & 15 gp & 3 lb.\tabularnewline
\hline
Thieves' tools & 30 gp & 1 lb.\tabularnewline
\hline
Thieves' tools, masterwork & 100 gp & 2 lb.\tabularnewline
\hline
Tool, masterwork & 50 gp & 1 lb.\tabularnewline
\hline
Water clock & 1,000 gp & 200 lb.\tabularnewline
\hline
C\textbf{lothing} &  & \tabularnewline
\hline
I\textbf{tem} & C\textbf{ost} & W\textbf{eight}\tabularnewline
\hline
Artisan's outfit & 1 gp & 4 lb.\textsuperscript{\textbf{1}}\tabularnewline
\hline
Cleric's vestments & 5 gp & 6 lb.\textsuperscript{\textbf{1}}\tabularnewline
\hline
Cold weather outfit & 8 gp & 7 lb.\textsuperscript{\textbf{1}}\tabularnewline
\hline
Courtier's outfit & 30 gp & 6 lb.\textsuperscript{\textbf{1}}\tabularnewline
\hline
Entertainer's outfit & 3 gp & 4 lb.\textsuperscript{\textbf{1}}\tabularnewline
\hline
Explorer's outfit & 10 gp & 8 lb.\textsuperscript{\textbf{1}}\tabularnewline
\hline
Monk's outfit & 5 gp & 2 lb.\textsuperscript{\textbf{1}}\tabularnewline
\hline
Noble's outfit & 75 gp & 10 lb.\textsuperscript{\textbf{1}}\tabularnewline
\hline
Peasant's outfit & 1 sp & 2 lb.\textsuperscript{\textbf{1}}\tabularnewline
\hline
Royal outfit & 200 gp & 15 lb.\textsuperscript{\textbf{1}}\tabularnewline
\hline
Scholar's outfit & 5 gp & 6 lb.\textsuperscript{\textbf{1}}\tabularnewline
\hline
Traveler's outfit & 1 gp & 5 lb.\textsuperscript{\textbf{1}}\tabularnewline
\hline
F\textbf{ood, Drink, and Lodging} &  & \tabularnewline
\hline
I\textbf{tem} & C\textbf{ost} & W\textbf{eight}\tabularnewline
\hline
Ale &  & \tabularnewline
\hline
Gallon & 2 sp & 8 lb.\tabularnewline
\hline
Mug & 4 cp & 1 lb.\tabularnewline
\hline
Banquet (per person) & 10 gp--- & \tabularnewline
\hline
Bread, per loaf & 2 cp & 1/2 lb.\tabularnewline
\hline
Cheese, hunk of & 1 sp & 1/2 lb.\tabularnewline
\hline
Inn stay (per day) &  & \tabularnewline
\hline
Good & 2 gp--- & \tabularnewline
\hline
Common & 5 sp--- & \tabularnewline
\hline
Poor & 2 sp--- & \tabularnewline
\hline
Meals (per day) &  & \tabularnewline
\hline
Good & 5 sp--- & \tabularnewline
\hline
Common & 3 sp--- & \tabularnewline
\hline
Poor & 1 sp--- & \tabularnewline
\hline
Meat, chunk of & 3 sp & 1/2 lb.\tabularnewline
\hline
Wine &  & \tabularnewline
\hline
Common (pitcher) & 2 sp & 6 lb.\tabularnewline
\hline
Fine (bottle) & 10 gp & 1-1/2 lb.\tabularnewline
\hline
M\textbf{ounts and Related Gear} &  & \tabularnewline
\hline
I\textbf{tem} & C\textbf{ost} & W\textbf{eight}\tabularnewline
\hline
Barding &  & \tabularnewline
\hline
Medium creature & x2 & x1\tabularnewline
\hline
Large creature & x4 & x2\tabularnewline
\hline
Bit and bridle & 2 gp & 1 lb.\tabularnewline
\hline
Dog, guard & 25 gp--- & \tabularnewline
\hline
Dog, riding & 150 gp--- & \tabularnewline
\hline
Donkey or mule & 8 gp--- & \tabularnewline
\hline
Feed (per day) & 5 cp & 10 lb.\tabularnewline
\hline
Horse &  & \tabularnewline
\hline
Horse, heavy & 200 gp--- & \tabularnewline
\hline
Horse, light & 75 gp--- & \tabularnewline
\hline
Pony & 30 gp--- & \tabularnewline
\hline
Warhorse, heavy & 400 gp--- & \tabularnewline
\hline
Warhorse, light & 150 gp--- & \tabularnewline
\hline
Warpony & 100 gp--- & \tabularnewline
\hline
Saddle &  & \tabularnewline
\hline
Military & 20 gp & 30 lb.\tabularnewline
\hline
Pack & 5 gp & 15 lb.\tabularnewline
\hline
Riding & 10 gp & 25 lb.\tabularnewline
\hline
Saddle, Exotic &  & \tabularnewline
\hline
Military & 60 gp & 40 lb.\tabularnewline
\hline
Pack & 15 gp & 20 lb.\tabularnewline
\hline
Riding & 30 gp & 30 lb.\tabularnewline
\hline
Saddlebags & 4 gp & 8 lb.\tabularnewline
\hline
Stabling (per day) & 5 sp--- & \tabularnewline
\hline
T\textbf{ransport} &  & \tabularnewline
\hline
I\textbf{tem} & C\textbf{ost} & W\textbf{eight}\tabularnewline
\hline
Carriage & 100 gp & 600 lb.\tabularnewline
\hline
Cart & 15 gp & 200 lb.\tabularnewline
\hline
Galley & 30,000 gp--- & \tabularnewline
\hline
Keelboat & 3,000 gp--- & \tabularnewline
\hline
Longship & 10,000 gp--- & \tabularnewline
\hline
Rowboat & 50 gp & 100 lb.\tabularnewline
\hline
Oar & 2 gp & 10 lb.\tabularnewline
\hline
Sailing ship & 10,000 gp--- & \tabularnewline
\hline
Sled & 20 gp & 300 lb.\tabularnewline
\hline
Wagon & 35 gp & 400 lb.\tabularnewline
\hline
Warship & 25,000 gp--- & \tabularnewline
\hline
S\textbf{pellcasting and Services} &  & \tabularnewline
\hline
S\textbf{ervice} & \multicolumn{2}{p{84pt}|}{C\textbf{ost}}\tabularnewline
\hline
Coach cab & \multicolumn{2}{p{84pt}|}{3 cp per mile}\tabularnewline
\hline
Hireling, trained & \multicolumn{2}{p{84pt}|}{3 sp per day}\tabularnewline
\hline
Hireling, untrained & \multicolumn{2}{p{84pt}|}{1 sp per day}\tabularnewline
\hline
Messenger & \multicolumn{2}{p{84pt}|}{2 cp per mile}\tabularnewline
\hline
Road or gate toll & \multicolumn{2}{p{84pt}|}{1 cp}\tabularnewline
\hline
Ship's passage & \multicolumn{2}{p{84pt}|}{1 sp per mile}\tabularnewline
\hline
Spell, 0-level & \multicolumn{2}{p{84pt}|}{Caster level x$ $5 gp\textsuperscript{\textbf{2}}}\tabularnewline
\hline
Spell, 1st-level & \multicolumn{2}{p{84pt}|}{Caster level x$ $10 gp\textsuperscript{\textbf{2}}}\tabularnewline
\hline
Spell, 2nd-level & \multicolumn{2}{p{84pt}|}{Caster level x$ $20 gp\textsuperscript{\textbf{2}}}\tabularnewline
\hline
Spell, 3rd-level & \multicolumn{2}{p{84pt}|}{Caster level x$ $30 gp\textsuperscript{\textbf{2}}}\tabularnewline
\hline
Spell, 4th-level & \multicolumn{2}{p{84pt}|}{Caster level x$ $40 gp\textsuperscript{\textbf{2}}}\tabularnewline
\hline
Spell, 5th-level & \multicolumn{2}{p{84pt}|}{Caster level x$ $50 gp\textsuperscript{\textbf{2}}}\tabularnewline
\hline
Spell, 6th-level & \multicolumn{2}{p{84pt}|}{Caster level x$ $60 gp\textsuperscript{\textbf{2}}}\tabularnewline
\hline
Spell, 7th-level & \multicolumn{2}{p{84pt}|}{Caster level x$ $70 gp\textsuperscript{\textbf{2}}}\tabularnewline
\hline
Spell, 8th-level & \multicolumn{2}{p{84pt}|}{Caster level x$ $80 gp\textsuperscript{\textbf{2}}}\tabularnewline
\hline
Spell, 9th-level & \multicolumn{2}{p{84pt}|}{Caster level x$ $90 gp\textsuperscript{\textbf{2}}}\tabularnewline
\hline
\multicolumn{3}{|p{266pt}|}{--- No weight, or no weight worth noting.}\tabularnewline
\hline
\multicolumn{3}{|p{266pt}|}{1 These items weigh one-quarter this amount when made 
for Small characters. Containers for Small characters also carry one-quarter the 
normal amount.}\tabularnewline
\hline
\multicolumn{3}{|p{266pt}|}{2 See spell description for additional costs. If the 
additional costs put the spell's total cost above 3,000 gp, that spell is not generally 
available.}\tabularnewline
\hline
\end{tabular}

\vspace{12pt}
ADVENTURING GEAR

few of the pieces of adventuring gear found on Table: Goods and Services are described 
below, along with any special benefits they confer on the user (``you'').

\textbf{Caltrops:} A caltrop is a four-pronged iron spike crafted so that one prong 
faces up no matter how the caltrop comes to rest. You scatter caltrops on the ground 
in the hope that your enemies step on them or are at least forced to slow down 
to avoid them. One 2- pound bag of caltrops covers an area 5 feet square.

Each time a creature moves into an area covered by caltrops (or spends a round 
fighting while standing in such an area), it might step on one. The caltrops make 
an attack roll (base attack bonus +0) against the creature. For this attack, the 
creature's shield, armor, and deflection bonuses do not count. If the creature 
is wearing shoes or other footwear, it gets a +2 armor bonus to AC. If the caltrops 
succeed on the attack, the creature has stepped on one. The caltrop deals 1 point 
of damage, and the creature's speed is reduced by one-half because its foot is 
wounded. This movement penalty lasts for 24 hours, or until the creature is successfully 
treated with a DC 15 Heal check, or until it receives at least 1 point of magical 
curing. A charging or running creature must immediately stop if it steps on a caltrop. 
Any creature moving at half speed or slower can pick its way through a bed of caltrops 
with no trouble.

Caltrops may not be effective against unusual opponents.

\textbf{Candle:} A candle dimly illuminates a 5-foot radius and burns for 1 hour.

\textbf{Chain:} Chain has hardness 10 and 5 hit points. It can be burst with a 
DC 26 Strength check.

\textbf{Crowbar:} A crowbar it grants a +2 circumstance bonus on Strength checks 
made for such purposes. If used in combat, treat a crowbar as a one-handed improvised 
weapon that deals bludgeoning damage equal to that of a club of its size.

\textbf{Flint and Steel:} Lighting a torch with flint and steel is a full-round 
action, and lighting any other fire with them takes at least that long.

\textbf{Grappling Hook:} Throwing a grappling hook successfully requires a Use 
Rope check (DC 10, +2 per 10 feet of distance thrown).

\textbf{Hammer:} If a hammer is used in combat, treat it as a one-handed improvised 
weapon that deals bludgeoning damage equal to that of a spiked gauntlet of its 
size.

\textbf{Ink:} This is black ink. You can buy ink in other colors, but it costs 
twice as much.

\textbf{Jug, Clay:} This basic ceramic jug is fitted with a stopper and holds 1 
gallon of liquid.

\textbf{Lamp, Common:} A lamp clearly illuminates a 15-foot radius, provides shadowy 
illumination out to a 30-foot radius, and burns for 6 hours on a pint of oil. You 
can carry a lamp in one hand. 

\textbf{Lantern, Bullseye:} A bullseye lantern provides clear illumination in a 
60-foot cone and shadowy illumination in a 120-foot cone. It burns for 6 hours 
on a pint of oil. You can carry a bullseye lantern in one hand.

\textbf{Lantern, Hooded:} A hooded lantern clearly illuminates a 30-foot radius 
and provides shadowy illumination in a 60-foot radius. It burns for 6 hours on 
a pint of oil. You can carry a hooded lantern in one hand.

\textbf{Lock:} The DC to open a lock with the Open Lock skill depends on the lock's 
quality: simple (DC 20), average (DC 25), good (DC 30), or superior (DC 40).

\textbf{Manacles and Manacles, Masterwork:} Manacles can bind a Medium creature. 
A manacled creature can use the Escape Artist skill to slip free (DC 30, or DC 
35 for masterwork manacles). Breaking the manacles requires a Strength check (DC 
26, or DC 28 for masterwork manacles). Manacles have hardness 10 and 10 hit points.

Most manacles have locks; add the cost of the lock you want to the cost of the 
manacles.

For the same cost, you can buy manacles for a Small creature.

For a Large creature, manacles cost ten times the indicated amount, and for a Huge 
creature, one hundred times this amount. Gargantuan, Colossal, Tiny, Diminutive, 
and Fine creatures can be held only by specially made manacles.

\textbf{Oil:} A pint of oil burns for 6 hours in a lantern. You can use a flask 
of oil as a splash weapon. Use the rules for alchemist's fire, except that it takes 
a full round action to prepare a flask with a fuse. Once it is thrown, there is 
a 50\% chance of the flask igniting successfully.

You can pour a pint of oil on the ground to cover an area 5 feet square, provided 
that the surface is smooth. If lit, the oil burns for 2 rounds and deals 1d3 points 
of fire damage to each creature in the area.

\textbf{Ram, Portable:} This iron-shod wooden beam gives you a +2 circumstance 
bonus on Strength checks made to break open a door and it allows a second person 
to help you without having to roll, increasing your bonus by 2.

\textbf{Rope, Hempen:} This rope has 2 hit points and can be burst with a DC 23 
Strength check.

\textbf{Rope, Silk:} This rope has 4 hit points and can be burst with a DC 24 Strength 
check. It is so supple that it provides a +2 circumstance bonus on Use Rope checks.

\textbf{Spyglass:} Objects viewed through a spyglass are magnified to twice their 
size.

\textbf{Torch:} A torch burns for 1 hour, clearly illuminating a 20-foot radius 
and providing shadowy illumination out to a 40- foot radius. If a torch is used 
in combat, treat it as a one-handed improvised weapon that deals bludgeoning damage 
equal to that of a gauntlet of its size, plus 1 point of fire damage.

\textbf{Vial:} A vial holds 1 ounce of liquid. The stoppered container usually 
is no more than 1 inch wide and 3 inches high.

\vspace{12pt}
SPECIAL SUBSTANCES AND ITEMS

Any of these substances except for the everburning torch and holy water can be 
made by a character with the Craft (alchemy) skill.

\textbf{Acid: }You can throw a flask of acid as a splash weapon. Treat this attack 
as a ranged touch attack with a range increment of 10 feet. A direct hit deals 
1d6 points of acid damage. Every creature within 5 feet of the point where the 
acid hits takes 1 point of acid damage from the splash.

\textbf{Alchemist's Fire:} You can throw a flask of alchemist's fire as a splash 
weapon. Treat this attack as a ranged touch attack with a range increment of 10 
feet.

A direct hit deals 1d6 points of fire damage. Every creature within 5 feet of the 
point where the flask hits takes 1 point of fire damage from the splash. On the 
round following a direct hit, the target takes an additional 1d6 points of damage. 
If desired, the target can use a full-round action to attempt to extinguish the 
flames before taking this additional damage. Extinguishing the flames requires 
a DC 15 Reflex save. Rolling on the ground provides the target a +2 bonus on the 
save. Leaping into a lake or magically extinguishing the flames automatically smothers 
the fire.

\textbf{Antitoxin:} If you drink antitoxin, you get a +5 alchemical bonus on Fortitude 
saving throws against poison for 1 hour.

\textbf{Everburning Torch:} This otherwise normal torch has a \textit{continual 
flame }spell cast upon it. An everburning torch clearly illuminates a 20-foot radius 
and provides shadowy illumination out to a 40-foot radius.

\textbf{Holy Water:} Holy water damages undead creatures and evil outsiders almost 
as if it were acid. A flask of holy water can be thrown as a splash weapon.

Treat this attack as a ranged touch attack with a range increment of 10 feet. A 
flask breaks if thrown against the body of a corporeal creature, but to use it 
against an incorporeal creature, you must open the flask and pour the holy water 
out onto the target. Thus, you can douse an incorporeal creature with holy water 
only if you are adjacent to it. Doing so is a ranged touch attack that does not 
provoke attacks of opportunity.

A direct hit by a flask of holy water deals 2d4 points of damage to an undead creature 
or an evil outsider. Each such creature within 5 feet of the point where the flask 
hits takes 1 point of damage from the splash.

Temples to good deities sell holy water at cost (making no profit).

\textbf{Smokestick:} This alchemically treated wooden stick instantly creates thick, 
opaque smoke when ignited. The smoke fills a 10- foot cube (treat the effect as 
a \textit{fog cloud }spell, except that a moderate or stronger wind dissipates 
the smoke in 1 round). The stick is consumed after 1 round, and the smoke dissipates 
naturally.

\textbf{Sunrod:} This 1-foot-long, gold-tipped, iron rod glows brightly when struck. 
It clearly illuminates a 30-foot radius and provides shadowy illumination in a 
60-foot radius. It glows for 6 hours, after which the gold tip is burned out and 
worthless.

\textbf{Tanglefoot Bag:} When you throw a tanglefoot bag at a creature (as a ranged 
touch attack with a range increment of 10 feet), the bag comes apart and the goo 
bursts out, entangling the target and then becoming tough and resilient upon exposure 
to air. An entangled creature takes a -2 penalty on attack rolls and a -4 penalty 
to Dexterity and must make a DC 15 Reflex save or be glued to the floor, unable 
to move. Even on a successful save, it can move only at half speed. Huge or larger 
creatures are unaffected by a tanglefoot bag. A flying creature is not stuck to 
the floor, but it must make a DC 15 Reflex save or be unable to fly (assuming it 
uses its wings to fly) and fall to the ground. A tanglefoot bag does not function 
underwater.

A creature that is glued to the floor (or unable to fly) can break free by making 
a DC 17 Strength check or by dealing 15 points of damage to the goo with a slashing 
weapon. A creature trying to scrape goo off itself, or another creature assisting, 
does not need to make an attack roll; hitting the goo is automatic, after which 
the creature that hit makes a damage roll to see how much of the goo was scraped 
off. Once free, the creature can move (including flying) at half speed. A character 
capable of spellcasting who is bound by the goo must make a DC 15 Concentration 
check to cast a spell. The goo becomes brittle and fragile after 2d4 rounds, cracking 
apart and losing its effectiveness. An application of \textit{universal solvent} 
to a stuck creature dissolves the alchemical goo immediately.

\textbf{Thunderstone:} You can throw this stone as a ranged attack with a range 
increment of 20 feet. When it strikes a hard surface (or is struck hard), it creates 
a deafening bang that is treated as a sonic attack. Each creature within a 10-foot-radius 
spread must make a DC 15 Fortitude save or be deafened for 1 hour. A deafened creature, 
in addition to the obvious effects, takes a -4 penalty on initiative and has a 
20\% chance to miscast and lose any spell with a verbal component that it tries 
to cast.

Since you don't need to hit a specific target, you can simply aim at a particular 
5-foot square. Treat the target square as AC 5.

\textbf{Tindertwig:} The alchemical substance on the end of this small, wooden 
stick ignites when struck against a rough surface. Creating a flame with a tindertwig 
is much faster than creating a flame with flint and steel (or a magnifying glass) 
and tinder. Lighting a torch with a tindertwig is a standard action (rather than 
a full-round action), and lighting any other fire with one is at least a standard 
action.

\vspace{12pt}
TOOLS AND SKILL KITS

\textbf{Alchemist's Lab:} An alchemist's lab always has the perfect tool for making 
alchemical items, so it provides a +2 circumstance bonus on Craft (alchemy) checks. 
It has no bearing on the costs related to the Craft (alchemy) skill. Without this 
lab, a character with the Craft (alchemy) skill is assumed to have enough tools 
to use the skill but not enough to get the +2 bonus that the lab provides.

\textbf{Artisan's Tools:} These special tools include the items needed to pursue 
any craft. Without them, you have to use improvised tools (-2 penalty on Craft 
checks), if you can do the job at all.

\textbf{Artisan's Tools, Masterwork:} These tools serve the same purpose as artisan's 
tools (above), but masterwork artisan's tools are the perfect tools for the job, 
so you get a +2 circumstance bonus on Craft checks made with them.

\textbf{Climber's Kit:} This is the perfect tool for climbing and gives you a +2 
circumstance bonus on Climb checks.

\textbf{Disguise Kit:} The kit is the perfect tool for disguise and provides a 
+2 circumstance bonus on Disguise checks. A disguise kit is exhausted after ten 
uses.

\textbf{Healer's Kit: }It is the perfect tool for healing and provides a +2 circumstance 
bonus on Heal checks. A healer's kit is exhausted after ten uses.

\textbf{Holy Symbol, Silver or Wooden: }A holy symbol focuses positive energy. 
A cleric or paladin uses it as the focus for his spells and as a tool for turning 
undead. Each religion has its own holy symbol.

\textit{Unholy Symbols: }An unholy symbol is like a holy symbol except that it 
focuses negative energy and is used by evil clerics (or by neutral clerics who 
want to cast evil spells or command undead).

\textbf{Magnifying Glass:} This simple lens allows a closer look at small objects. 
It is also useful as a substitute for flint and steel when starting fires. Lighting 
a fire with a magnifying glass requires light as bright as sunlight to focus, tinder 
to ignite, and at least a full-round action. A magnifying glass grants a +2 circumstance 
bonus on Appraise checks

involving any item that is small or highly detailed.

\textbf{Musical Instrument, Common or Masterwork:} A masterwork instrument grants 
a +2 circumstance bonus on Perform checks involving its use.

\textbf{Scale, Merchant's: }A scale grants a +2 circumstance bonus on Appraise 
checks involving items that are valued by weight, including anything made of precious 
metals.

\textbf{Spell Component Pouch: }A spellcaster with a spell component pouch is assumed 
to have all the material components and focuses needed for spellcasting, except 
for those components that have a specific cost, divine focuses, and focuses that 
wouldn't fit in a pouch.

\textbf{Spellbook, Wizard's (Blank):} A spellbook has 100 pages of parchment, and 
each spell takes up one page per spell level (one page each for 0-level spells).

\textbf{Thieves' Tools: }This kit contains the tools you need to use the Disable 
Device and Open Lock skills. Without these tools, you must improvise tools, and 
you take a -2 circumstance penalty on Disable Device and Open Locks checks.

\textbf{Thieves' Tools, Masterwork:} This kit contains extra tools and tools of 
better make, which grant a +2 circumstance bonus on Disable Device and Open Lock 
checks.

\textbf{Tool, Masterwork:} This well-made item is the perfect tool for the job. 
It grants a +2 circumstance bonus on a related skill check (if any). Bonuses provided 
by multiple masterwork items used toward the same skill check do not stack.

\textbf{Water Clock: }This large, bulky contrivance gives the time accurate to 
within half an hour per day since it was last set. It requires a source of water, 
and it must be kept still because it marks time by the regulated flow of droplets 
of water.

\vspace{12pt}
CLOTHING

\textbf{Artisan's Outfit: }This outfit includes a shirt with buttons, a skirt or 
pants with a drawstring, shoes, and perhaps a cap or hat. It may also include a 
belt or a leather or cloth apron for carrying tools.

\textbf{Cleric's Vestments: }These ecclesiastical clothes are for performing priestly 
functions, not for adventuring.

\textbf{Cold Weather Outfit:} A cold weather outfit includes a wool coat, linen 
shirt, wool cap, heavy cloak, thick pants or skirt, and

boots. This outfit grants a +5 circumstance bonus on Fortitude saving throws against 
exposure to cold weather.

\textbf{Courtier's Outfit:} This outfit includes fancy, tailored clothes in whatever 
fashion happens to be the current style in the courts of the nobles. Anyone trying 
to influence nobles or courtiers while wearing street dress will have a hard time 
of it (-2 penalty on Charisma-based skill checks to influence such individuals). 
If you wear this outfit without jewelry (costing an additional 50 gp), you look 
like an out-of-place commoner.

\textbf{Entertainer's Outfit:} This set of flashy, perhaps even gaudy, clothes 
is for entertaining. While the outfit looks whimsical, its practical design lets 
you tumble, dance, walk a tightrope, or just run (if the audience turns ugly).

\textbf{Explorer's Outfit:} This is a full set of clothes for someone who never 
knows what to expect. It includes sturdy boots, leather breeches or a skirt, a 
belt, a shirt (perhaps with a vest or jacket), gloves, and a cloak. Rather than 
a leather skirt, a leather overtunic may be worn over a cloth skirt. The clothes 
have plenty of pockets (especially the cloak). The outfit also includes any extra 
items you might need, such as a scarf or a wide-brimmed hat.

\textbf{Monk's Outfit:} This simple outfit includes sandals, loose breeches, and 
a loose shirt, and is all bound together with sashes. The outfit is designed to 
give you maximum mobility, and it's made of high-quality fabric. You can hide small 
weapons in pockets hidden in the folds, and the sashes are strong enough to serve 
as short ropes.

\textbf{Noble's Outfit:} This set of clothes is designed specifically to be expensive 
and to show it. Precious metals and gems are worked into the clothing. To fit into 
the noble crowd, every would-be noble also needs a signet ring (see Adventuring 
Gear, above) and jewelry (worth at least 100 gp).

\textbf{Peasant's Outfit:} This set of clothes consists of a loose shirt and baggy 
breeches, or a loose shirt and skirt or overdress. Cloth wrappings are used for 
shoes.

\textbf{Royal Outfit:} This is just the clothing, not the royal scepter, crown, 
ring, and other accoutrements. Royal clothes are ostentatious, with gems, gold, 
silk, and fur in abundance.

\textbf{Scholar's Outfit: }Perfect for a scholar, this outfit includes a robe, 
a belt, a cap, soft shoes, and possibly a cloak.

\textbf{Traveler's Outfit:} This set of clothes consists of boots, a wool skirt 
or breeches, a sturdy belt, a shirt (perhaps with a vest or jacket), and an ample 
cloak with a hood.

\vspace{12pt}
FOOD, DRINK, AND LODGING

\textbf{Inn:} Poor accommodations at an inn amount to a place on the floor near 
the hearth. Common accommodations consist of a place on a raised, heated floor, 
the use of a blanket and a pillow. Good accommodations consist of a small, private 
room with one bed, some amenities, and a covered chamber pot in the corner.

\textbf{Meals:} Poor meals might be composed of bread, baked turnips, onions, and 
water. Common meals might consist of bread, chicken stew, carrots, and watered-down 
ale or wine. Good meals might be composed of bread and pastries, beef, peas, and 
ale or wine.

\vspace{12pt}
MOUNTS AND RELATED GEAR

\textbf{Barding, Medium Creature and Large Creature: }Barding is a type of armor 
that covers the head, neck, chest, body, and possibly legs of a horse or other 
mount. Barding made of medium or heavy armor provides better protection than light 
barding, but at the expense of speed. Barding can be made of any of the armor types 
found on Table: Armor and Shields.

Armor for a horse (a Large nonhumanoid creature) costs four times as much as armor 
for a human (a Medium humanoid creature) and also weighs twice as much as the armor 
found on Table: Armor and Shields (see Armor for Unusual Creatures). If the barding 
is for a pony or other Medium mount, the cost is only double, and the weight is 
the same as for Medium armor worn by a humanoid. Medium or heavy barding slows 
a mount that wears it, as shown on the table below.

\vspace{12pt}
\begin{tabular}{|>{\raggedright}p{37pt}|>{\raggedright}p{42pt}|>{\raggedright}p{42pt}|>{\raggedright}p{42pt}|}
\hline
 ------------ & \multicolumn{3}{p{126pt}|}{ \textbf{Base Speed -----------}}\tabularnewline
\hline
B\textbf{arding} & (\textbf{40 ft.)} & (\textbf{50 ft.)} & (\textbf{60 ft.)}\tabularnewline
\hline
Medium & 30 ft. & 35 ft. & 40 ft.\tabularnewline
\hline
Heavy & 30 ft.\textsuperscript{\textbf{1}} & 35 ft.\textsuperscript{\textbf{1}} & 40 
ft.\textsuperscript{\textbf{1}}\tabularnewline
\hline
\multicolumn{4}{|p{164pt}|}{1 A mount wearing heavy armor moves at only triple 
its normal speed when running instead of quadruple.}\tabularnewline
\hline
\end{tabular}

\vspace{12pt}
Flying mounts can't fly in medium or heavy barding.

Removing and fitting barding takes five times as long as the figures given on Table: 
Donning Armor. A barded animal cannot be used to carry any load other than the 
rider and normal saddlebags.

\textbf{Dog, Riding:} This Medium dog is specially trained to carry a Small humanoid 
rider. It is brave in combat like a warhorse. You take no damage when you fall 
from a riding dog.

\textbf{Donkey or Mule:} Donkeys and mules are stolid in the face of danger, hardy, 
surefooted, and capable of carrying heavy loads over vast distances. Unlike a horse, 
a donkey or a mule is willing (though not eager) to enter dungeons and other strange 
or threatening places.

\textbf{Feed:} Horses, donkeys, mules, and ponies can graze to sustain themselves, 
but providing feed for them is much better. If you have a riding dog, you have 
to feed it at least some meat.

\textbf{Horse:} A horse (other than a pony) is suitable as a mount for a human, 
dwarf, elf, half-elf, or half-orc. A pony is smaller than a horse and is a suitable 
mount for a gnome or halfling.

Warhorses and warponies can be ridden easily into combat. Light horses, ponies, 
and heavy horses are hard to control in combat.

\textbf{Saddle, Exotic:} An exotic saddle is like a normal saddle of the same sort 
except that it is designed for an unusual mount. Exotic saddles come in military, 
pack, and riding styles.

\textbf{Saddle, Military:} A military saddle braces the rider, providing a +2 circumstance 
bonus on Ride checks related to staying in the saddle. If you're knocked unconscious 
while in a military saddle, you have a 75\% chance to stay in the saddle (compared 
to 50\% for a riding saddle).

\textbf{Saddle, Pack:} A pack saddle holds gear and supplies, but not a rider. 
It holds as much gear as the mount can carry.

\textbf{Saddle, Riding:} The standard riding saddle supports a rider.

\vspace{12pt}
TRANSPORT

\textbf{Carriage}: This four-wheeled vehicle can transport as many as four people 
within an enclosed cab, plus two drivers. In general, two horses (or other beasts 
of burden) draw it. A carriage comes with the harness needed to pull it.

\textbf{Cart:} This two-wheeled vehicle can be drawn by a single horse (or other 
beast of burden). It comes with a harness.

\textbf{Galley:} This three-masted ship has seventy oars on either side and requires 
a total crew of 200. A galley is 130 feet long and 20 feet wide, and it can carry 
150 tons of cargo or 250 soldiers. For 8,000 gp more, it can be fitted with a ram 
and castles with firing platforms fore, aft, and amidships. This ship cannot make 
sea voyages and sticks to the coast. It moves about 4 miles per hour when being 
rowed or under sail.

\textbf{Keelboat:} This 50- to 75-foot-long ship is 15 to 20 feet wide and has 
a few oars to supplement its single mast with a square sail. It has a crew of eight 
to fifteen and can carry 40 to 50 tons of cargo or 100 soldiers. It can make sea 
voyages, as well as sail down rivers (thanks to its flat bottom). It moves about 
1 mile per hour.

\textbf{Longship:} This 75-foot-long ship with forty oars requires a total crew 
of 50. It has a single mast and a square sail, and it can carry 50 tons of cargo 
or 120 soldiers. A longship can make sea voyages. It moves about 3 miles per hour 
when being rowed or under sail.

\textbf{Rowboat: }This 8- to 12-foot-long boat holds two or three Medium passengers. 
It moves about 1-1/2 miles per hour.

\textbf{Sailing Ship:} This larger, seaworthy ship is 75 to 90 feet long and 20 
feet wide and has a crew of 20. It can carry 150 tons of cargo. It has square sails 
on its two masts and can make sea voyages. It moves about 2 miles per hour.

\textbf{Sled: }This is a wagon on runners for moving through snow and over ice. 
In general, two horses (or other beasts of burden) draw it. A sled comes with the 
harness needed to pull it.

\textbf{Wagon:} This is a four-wheeled, open vehicle for transporting heavy loads. 
In general, two horses (or other beasts of burden) draw it. A wagon comes with 
the harness needed to pull it.

\textbf{Warship: }This 100-foot-long ship has a single mast, although oars can 
also propel it. It has a crew of 60 to 80 rowers. This ship can carry 160 soldiers, 
but not for long distances, since there isn't room for supplies to support that 
many people. The warship cannot make sea voyages and sticks to the coast. It is 
not used for cargo. It moves about 2-1/2 miles per hour when being rowed or under 
sail.

\vspace{12pt}
SPELLCASTING AND SERVICES

Sometimes the best solution for a problem is to hire someone else to take care 
of it.

\textbf{Coach Cab:} The price given is for a ride in a coach that transports people 
(and light cargo) between towns. For a ride in a cab that transports passengers 
within a city, 1 copper piece usually takes you anywhere you need to go.

\textbf{Hireling, Trained:} The amount given is the typical daily wage for mercenary 
warriors, masons, craftsmen, scribes, teamsters, and other trained hirelings. This 
value represents a minimum wage; many such hirelings require significantly higher 
pay.

\textbf{Hireling, Untrained:} The amount shown is the typical daily wage for laborers, 
porters, cooks, maids, and other menial workers.

\textbf{Messenger:} This entry includes horse-riding messengers and runners. Those 
willing to carry a message to a place they were going anyway may ask for only half 
the indicated amount.

\textbf{Road or Gate Toll:} A toll is sometimes charged to cross a well-trodden, 
well-kept, and well-guarded road to pay for patrols on it and for its upkeep. Occasionally, 
a large walled city charges a toll to enter or exit (or sometimes just to enter).

\textbf{Ship's Passage:} Most ships do not specialize in passengers, but many have 
the capability to take a few along when transporting cargo. Double the given cost 
for creatures larger than Medium or creatures that are otherwise difficult to bring 
aboard a ship.

\textbf{Spell: }The indicated amount is how much it costs to get a spellcaster 
to cast a spell for you. This cost assumes that you can go to the spellcaster and 
have the spell cast at his or her convenience (generally at least 24 hours later, 
so that the spellcaster has time to prepare the spell in question). If you want 
to bring the spellcaster somewhere to cast a spell you need to negotiate with him 
or her, and the default answer is no.

The cost given is for a spell with no cost for a material component or focus component 
and no XP cost. If the spell includes a material component, add the cost of that 
component to the cost of the spell.

If the spell has a focus component (other than a divine focus), add 1/10 the cost 
of that focus to the cost of the spell. If the spell has an XP cost, add 5 gp per 
XP lost. 

Furthermore, if a spell has dangerous consequences, the spellcaster will certainly 
require proof that you can and will pay for dealing with any such consequences 
(that is, assuming that the spellcaster even agrees to cast such a spell, which 
isn't certain). In the case of spells that transport the caster and characters 
over a distance, you will likely have to pay for two castings of the spell, even 
if you aren't returning with the caster.

In addition, not every town or village has a spellcaster of sufficient level to 
cast any spell. In general, you must travel to a small town (or larger settlement) 
to be reasonably assured of finding a spellcaster capable of casting 1st-level 
spells, a large town for 2nd-level spells, a small city for 3rd- or 4th-level spells, 
a large city for 5th- or 6th-level spells, and a metropolis for 7th- or 8th-level 
spells. Even a metropolis isn't guaranteed to have a local spellcaster able to 
cast 9th-level spells.

\newpage

\end{document}
