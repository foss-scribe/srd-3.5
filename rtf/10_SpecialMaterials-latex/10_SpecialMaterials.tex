%&pdfLaTeX
% !TEX encoding = UTF-8 Unicode
\documentclass{article}
\usepackage{ifxetex}
\ifxetex
\usepackage{fontspec}
\setmainfont[Mapping=tex-text]{STIXGeneral}
\else
\usepackage[T1]{fontenc}
\usepackage[utf8]{inputenc}
\fi
\usepackage{textcomp}

\usepackage{array}
\usepackage{amssymb}
\usepackage{fancyhdr}
\renewcommand{\headrulewidth}{0pt}
\renewcommand{\footrulewidth}{0pt}

\begin{document}

This material is Open Game Content, and is licensed for public use under the terms 
of the Open Game License v1.0a.

\subsubsection*{{\LARGE{}SPECIAL MATERIALS}}

In addition to magic items created with spells, some substances have innate special 
properties.

If you make a suit of armor or weapon out of more than one special material, you 
get the benefit of only the most prevalent material. However, you can build a double 
weapon with each head made of a different special material. 

\vspace{12pt}
SPECIAL WEAPONS MATERIALS

Each of the special materials described below has a definite game effect. Some 
creatures have damage reduction based on their creature type or core concept. Some 
are resistant to all but a special type of damage, such as that dealt by evil-aligned 
weapons or bludgeoning weapons. Others are vulnerable to weapons of a particular 
material. Characters may choose to carry several different types of weapons, depending 
upon the campaign and types of creatures they most commonly encounter. 

\vspace{12pt}
\textbf{Adamantine:} This ultrahard metal adds to the quality of a weapon or suit 
of armor. Weapons fashioned from adamantine have a natural ability to bypass hardness 
when sundering weapons or attacking objects, ignoring hardness less than 20. Armor 
made from adamantine grants its wearer damage reduction of 1/- if it's light armor, 
2/- if it's medium armor, and 3/- if it's heavy armor. Adamantine is so costly 
that weapons and armor made from it are always of masterwork quality; the masterwork 
cost is included in the prices given below. Thus, adamantine weapons and ammunition 
have a +1 enhancement bonus on attack rolls, and the armor check penalty of adamantine 
armor is lessened by 1 compared to ordinary armor of its type. Items without metal 
parts cannot be made from adamantine. An arrow could be made of adamantine, but 
a quarterstaff could not.

Only weapons, armor, and shields normally made of metal can be fashioned from adamantine. 
Weapons, armor and shields normally made of steel that are made of adamantine have 
one-third more hit points than normal. Adamantine has 40 hit points per inch of 
thickness and hardness 20.

\begin{tabular}{|>{\raggedright}p{114pt}|>{\raggedright}p{86pt}|}
\hline
T\textbf{ype of Adamantine Item} & \section*{I\textbf{tem Cost Modifier}}\tabularnewline
\hline
Ammunition & +60 gp\tabularnewline
\hline
Light armor & +5,000 gp\tabularnewline
\hline
Medium armor & +10,000 gp\tabularnewline
\hline
Heavy armor & +15,000 gp\tabularnewline
\hline
Weapon & +3,000 gp\tabularnewline
\hline
\end{tabular}

\vspace{12pt}
\textbf{Darkwood:} This rare magic wood is as hard as normal wood but very light. 
Any wooden or mostly wooden item (such as a bow, an arrow, or a spear) made from 
darkwood is considered a masterwork item and weighs only half as much as a normal 
wooden item of that type. Items not normally made of wood or only partially of 
wood (such as a battleaxe or a mace) either cannot be made from darkwood or do 
not gain any special benefit from being made of darkwood. The armor check penalty 
of a darkwood shield is lessened by 2 compared to an ordinary shield of its type. 
To determine the price of a darkwood item, use the original weight but add 10 gp 
per pound to the price of a masterwork version of that item.

Darkwood has 10 hit points per inch of thickness and hardness 5.

\vspace{12pt}
\textbf{Dragonhide:} Armorsmiths can work with the hides of dragons to produce 
armor or shields of masterwork quality. One dragon produces enough hide for a single 
suit of masterwork hide armor for a creature one size category smaller than the 
dragon. By selecting only choice scales and bits of hide, an armorsmith can produce 
one suit of masterwork banded mail for a creature two sizes smaller, one suit of 
masterwork half-plate for a creature three sizes smaller, or one masterwork breastplate 
or suit of full plate for a creature four sizes smaller. In each case, enough hide 
is available to produce a small or large masterwork shield in addition to the armor, 
provided that the dragon is Large or larger.

Because dragonhide armor isn't made of metal, druids can wear it without penalty.

Dragonhide armor costs double what masterwork armor of that type ordinarily costs, 
but it takes no longer to make than ordinary armor of that type.

Dragonhide has 10 hit points per inch of thickness and hardness 10.

\vspace{12pt}
\textbf{Iron, Cold:} This iron, mined deep underground, known for its effectiveness 
against fey creatures, is forged at a lower temperature to preserve its delicate 
properties. Weapons made of cold iron cost twice as much to make as their normal 
counterparts. Also, any magical enhancements cost an additional 2,000 gp. 

Items without metal parts cannot be made from cold iron. An arrow could be made 
of cold iron, but a quarterstaff could not.

A double weapon that has only half of it made of cold iron increases its cost by 
50\%.

Cold iron has 30 hit points per inch of thickness and hardness 10.

\vspace{12pt}
\textbf{Mithral:} Mithral is a very rare silvery, glistening metal that is lighter 
than iron but just as hard. When worked like steel, it becomes a wonderful material 
from which to create armor and is occasionally used for other items as well. Most 
mithral armors are one category lighter than normal for purposes of movement and 
other limitations. Heavy armors are treated as medium, and medium armors are treated 
as light, but light armors are still treated as light. Spell failure chances for 
armors and shields made from mithral are decreased by 10\%, maximum Dexterity bonus 
is increased by 2, and armor check penalties are lessened by 3 (to a minimum of 
0).

An item made from mithral weighs half as much as the same item made from other 
metals. In the case of weapons, this lighter weight does not change a weapon's 
size category or the ease with which it can be wielded (whether it is light, one-handed, 
or two-handed). Items not primarily of metal are not meaningfully affected by being 
partially made of mithral. (A longsword can be a mithral weapon, while a scythe 
cannot be.)

Weapons or armors fashioned from mithral are always masterwork items as well; the 
masterwork cost is included in the prices given below.

Mithral has 30 hit points per inch of thickness and hardness 15.

\begin{tabular}{|>{\raggedright}p{94pt}|>{\raggedright}p{86pt}|}
\hline
\subsection*{T\textbf{ype of Mithral Item}} & \subsection*{I\textbf{tem Cost Modifier}}\tabularnewline
\hline
Light armor & +1,000 gp\tabularnewline
\hline
Medium armor & +4,000 gp\tabularnewline
\hline
Heavy armor & +9,000 gp\tabularnewline
\hline
Shield & +1,000 gp\tabularnewline
\hline
Other items & +500 gp/lb.\tabularnewline
\hline
\end{tabular}

\vspace{12pt}
\textbf{Silver, Alchemical:} A complex process involving metallurgy and alchemy 
can bond silver to a weapon made of steel so that it bypasses the damage reduction 
of creatures such as lycanthropes.

On a successful attack with a silvered weapon, the wielder takes a -1 penalty on 
the damage roll (with the usual minimum of 1 point of damage). The alchemical silvering 
process can't be applied to nonmetal items, and it doesn't work on rare metals 
such as adamantine, cold iron, and mithral.

Alchemical silver has 10 hit points per inch of thickness and hardness 8.

\begin{tabular}{|>{\raggedright}p{222pt}|>{\raggedright}p{81pt}|}
\hline
T\textbf{ype of Alchemical Silver Item} & \section*{I\textbf{tem Cost Modifier}}\tabularnewline
\hline
Ammunition & +2 gp\tabularnewline
\hline
Light weapon & +20 gp\tabularnewline
\hline
One-handed weapon, or one head of a double weapon & +90 gp\tabularnewline
\hline
Two-handed weapon, or both heads of a double weapon & +180 gp\tabularnewline
\hline
\end{tabular}

\newpage

\end{document}
