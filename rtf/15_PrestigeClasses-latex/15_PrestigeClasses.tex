%&pdfLaTeX
% !TEX encoding = UTF-8 Unicode
\documentclass{article}
\usepackage{ifxetex}
\ifxetex
\usepackage{fontspec}
\setmainfont[Mapping=tex-text]{STIXGeneral}
\else
\usepackage[T1]{fontenc}
\usepackage[utf8]{inputenc}
\fi
\usepackage{textcomp}

\usepackage{array}
\usepackage{amssymb}
\usepackage{fancyhdr}
\renewcommand{\headrulewidth}{0pt}
\renewcommand{\footrulewidth}{0pt}

\begin{document}

This material is Open Game Content, and is licensed for public use under the terms 
of the Open Game License v1.0a.

{\LARGE{}PRESTIGE CLASSES}

\vspace{12pt}
Prestige classes offer a new form of multiclassing. Unlike the basic classes\textit{, 
}characters must meet Requirements before they can take their first level of a 
prestige class. The rules for level advancement apply to this system, meaning the 
first step of advancement is always choosing a class. If a character does not meet 
the Requirements for a prestige class before that first step, that character cannot 
take the first level of that prestige class.  Taking a prestige class does not 
incur the experience point penalties normally associated with multiclassing.

\vspace{12pt}
Definitions of Terms

Here are definitions of some terms used in this section.

\textbf{Base Class:} One of the standard eleven classes.

\textbf{Caster Level:} Generally equal to the number of class levels (see below) 
in a spellcasting class. Some prestige classes add caster levels to an existing 
class.

\textbf{Character Level:} The total level of the character, which is the sum of 
all class levels held by that character.

\textbf{Class Level:} The level of a character in a particular class. For a character 
with levels in only one class, class level and character level are the same.

\vspace{12pt}
ARCANE ARCHER

\textbf{Hit Die:} d8.

\section*{\textbf{Requirements}}

To qualify to become an arcane archer, a character must fulfill all the following 
criteria.

\textbf{Race:} Elf or half-elf.

\textbf{Base Attack Bonus:} +6.

\textbf{Feats:} Point Blank Shot, Precise Shot, Weapon Focus (longbow or shortbow).

\textbf{Spells:} Ability to cast 1st-level arcane spells.

\section*{\textbf{Class Skills}}

The arcane archer's class skills (and the key ability for each skill) are Craft 
(Int), Hide (Dex). Listen (Wis), Move Silently (Dex), Ride (Dex), Spot (Wis), Survival 
(Wis), and Use Rope (Dex).

\textbf{Skill Points at Each Level:} 4 + Int modifier.

\vspace{12pt}
\begin{tabular}{|>{\raggedright}p{29pt}|>{\raggedright}p{35pt}|>{\raggedright}p{23pt}|>{\raggedright}p{23pt}|>{\raggedright}p{23pt}|>{\raggedright}p{76pt}|}
\hline
\multicolumn{6}{|p{211pt}|}{\subsection*{T\textbf{able: The Arcane Archer}}}\tabularnewline
\hline
L\textbf{evel } & B\textbf{ase}\linebreak{}
\textbf{Attack}\linebreak{}
\textbf{Bonus} & F\textbf{ort}\linebreak{}
\textbf{Save} & R\textbf{ef}\linebreak{}
\textbf{Save} & W\textbf{ill}\linebreak{}
\textbf{Save} & S\textbf{pecial}\tabularnewline
\hline
1st & +1 & +2 & +2 & +0 & Enhance arrow +1\tabularnewline
\hline
2nd & +2 & +3 & +3 & +0 & I\textit{mbue arrow}\tabularnewline
\hline
3rd & +3 & +3 & +3 & +1 & Enhance arrow +2\tabularnewline
\hline
4th & +4 & +4 & +4 & +1 & S\textit{eeker arrow}\tabularnewline
\hline
5th & +5 & +4 & +4 & +1 & Enhance arrow +3\tabularnewline
\hline
6th & +6 & +5 & +5 & +2 & P\textit{hase arrow}\tabularnewline
\hline
7th & +7 & +5 & +5 & +2 & Enhance arrow +4\tabularnewline
\hline
8th & +8 & +6 & +6 & +2 & H\textit{ail of arrows}\tabularnewline
\hline
9th & +9 & +6 & +6 & +3 & Enhance arrow +5\tabularnewline
\hline
10th & +10 & +7 & +7 & +3 & A\textit{rrow of death}\tabularnewline
\hline
\end{tabular}

\vspace{12pt}
\subsubsection*{\textbf{Class Features}}

All of the following are Class Features of the arcane archer prestige class.

\textbf{Weapon and Armor Proficiency:} An arcane archer is proficient with all 
simple and martial weapons, light armor, medium armor, and shields.

\textbf{Enhance Arrow (Su):} At 1st level, every nonmagical arrow an arcane archer 
nocks and lets fly becomes magical, gaining a +1 enhancement bonus. Unlike magic 
weapons created by normal means, the archer need not spend experience points or 
gold pieces to accomplish this task. However, an archer's magic arrows only function 
for her. For every two levels the character advances past 1st level in the prestige 
class, the magic arrows she creates gain +1 greater potency (+1 at 1st level, +2 
at 3rd level, +3 at 5th level, +4 at 7th level, and +5 at 9th level).

\textit{\textbf{Imbue Arrow }}\textbf{(Sp): }At 2nd level, an arcane archer gains 
the ability to place an area spell upon an arrow. When the arrow is fired, the 
spell's area is centered on where the arrow lands, even if the spell could normally 
be centered only on the caster. This ability allows the archer to use the bow's 
range rather than the spell's range. It takes a standard action to cast the spell 
and fire the arrow. The arrow must be fired in the round the spell is cast, or 
the spell is wasted.

\textit{\textbf{Seeker Arrow }}\textbf{(Sp):} At 4th level, an arcane archer can 
launch an arrow once per day at a target known to her within range, and the arrow 
travels to the target, even around corners. Only an unavoidable obstacle or the 
limit of the arrow's range prevents the arrow's flight. This ability negates cover 
and concealment modifiers, but otherwise the attack is rolled normally. Using this 
ability is a standard action (and shooting the arrow is part of the action).

\textit{\textbf{Phase Arrow }}\textbf{(Sp):} At 6th level, an arcane archer can 
launch an arrow once per day at a target known to her within range, and the arrow 
travels to the target in a straight path, passing through any nonmagical barrier 
or wall in its way. (Any magical barrier stops the arrow.) This ability negates 
cover, concealment, and even armor modifiers, but otherwise the attack is rolled 
normally.

Using this ability is a standard action (and shooting the arrow is part of the 
action).

\textit{\textbf{Hail of Arrows }}\textbf{(Sp):} In lieu of her regular attacks, 
once per day an arcane archer of 8th level or higher can fire an arrow at each 
and every target within range, to a maximum of one target for every arcane archer 
level she has earned. Each attack uses the archer's primary attack bonus, and each 
enemy may only be targeted by a single arrow.

\textit{\textbf{Arrow of Death }}\textbf{(Sp):} At 10th level, an arcane archer 
can create an \textit{arrow of death }that forces the target, if damaged by the 
arrow's attack, to make a DC 20 Fortitude save or be slain immediately. It takes 
one day to make an \textit{arrow of death}, and the arrow only functions for the 
arcane archer who created it. The \textit{arrow of death }lasts no longer than 
one year, and the archer can only have one such arrow in existence at a time.

\vspace{12pt}
ARCANE TRICKSTER

\textbf{Hit Die:} d4.

\textbf{Requirements}

To qualify to become an arcane trickster, a character must fulfill all of the following 
criteria.

\textbf{Alignment:} Any nonlawful.

\textbf{Skills:} Decipher Script 7 ranks, Disable Device 7 ranks, Escape Artist 
7 ranks, Knowledge (arcana) 4 ranks.

\textbf{Spells:} Ability to cast \textit{mage hand }and at least one arcane spell 
of 3rd level or higher.

\textbf{Special:} Sneak attack +2d6.

\subsubsection*{\textbf{Class Skills}}

The arcane trickster's class skills (and the key ability for each skill) are Appraise 
(Int), Balance (Dex), Bluff (Cha), Climb (Str), Concentration (Con), Craft (Int), 
Decipher Script (Int), Diplomacy (Cha), Disable Device (Int), Disguise (Cha), Escape 
Artist (Dex), Gather Information (Cha), Hide (Dex), Jump (Str), Knowledge (all 
skills taken individually) (Int), Listen (Wis), Move Silently (Dex), Open Lock 
(Dex), Profession (Wis), Sense Motive (Wis), Search (Int), Sleight of Hand (Dex), 
Speak Language (Int), Spellcraft (Int), Spot (Wis), Swim (Str), Tumble (Dex), and 
Use Rope (Dex).

\textbf{Skill Points at Each Level:} 4 + Int modifier.

\vspace{12pt}
\begin{tabular}{|>{\raggedright}p{21pt}|>{\raggedright}p{24pt}|>{\raggedright}p{17pt}|>{\raggedright}p{17pt}|>{\raggedright}p{17pt}|>{\raggedright}p{91pt}|>{\raggedright}p{76pt}|}
\hline
\multicolumn{7}{|p{266pt}|}{\subsection*{T\textbf{able: The Arcane Trickster}}}\tabularnewline
\hline
L\textbf{evel} & B\textbf{ase}\linebreak{}
\textbf{Attack}\linebreak{}
\textbf{Bonus} & F\textbf{ort}\linebreak{}
\textbf{Save} & R\textbf{ef}\linebreak{}
\textbf{Save} & W\textbf{ill}\linebreak{}
\textbf{Save} & S\textbf{pecial} & S\textbf{pells per Day}\tabularnewline
\hline
1st & +0 & +0 & +2 & +2 & Ranged legerdemain 1/day & +1 level of existing class\tabularnewline
\hline
2nd & +1 & +0 & +3 & +3 & Sneak attack +1d6 & +1 level of existing class\tabularnewline
\hline
3rd & +1 & +1 & +3 & +3 & Impromptu sneak attack 1/day & +1 level of existing class\tabularnewline
\hline
4th & +2 & +1 & +4 & +4 & Sneak attack +2d6 & +1 level of existing class\tabularnewline
\hline
5th & +2 & +1 & +4 & +4 & Ranged legerdemain 2/day & +1 level of existing class\tabularnewline
\hline
6th & +3 & +2 & +5 & +5 & Sneak attack +3d6 & +1 level of existing class\tabularnewline
\hline
7th & +3 & +2 & +5 & +5 & Impromptu sneak attack 2/day & +1 level of existing class\tabularnewline
\hline
8th & +4 & +2 & +6 & +6 & Sneak attack +4d6 & +1 level of existing class\tabularnewline
\hline
9th & +4 & +3 & +6 & +6 & Ranged legerdemain 3/day & +1 level of existing class\tabularnewline
\hline
10th & +5 & +3 & +7 & +7 & Sneak attack +5d6 & +1 level of existing class\tabularnewline
\hline
\end{tabular}

\vspace{12pt}
\textbf{Class Features}

All of the following are Class Features of the arcane trickster prestige class.

\textbf{Weapon and Armor Proficiency:} Arcane tricksters gain no proficiency with 
any weapon or armor.

\textbf{Spells per Day:} When a new arcane trickster level is gained, the character 
gains new spells per day as if he had also gained a level in a spellcasting class 
he belonged to before adding the prestige class. He does not, however, gain any 
other benefit a character of that class would have gained, except for an increased 
effective level of spellcasting. If a character had more than one spellcasting 
class before becoming an arcane trickster, he must decide to which class he adds 
the new level for purposes of determining spells per day.

\textbf{Ranged Legerdemain:} An arcane trickster can perform one of the following 
class skills at a range of 30 feet: Disable Device, Open Lock, or Sleight of Hand. 
Working at a distance increases the normal skill check DC by 5, and an arcane trickster 
cannot take 10 on this check. Any object to be manipulated must weigh 5 pounds 
or less.

An arcane trickster can use ranged legerdemain once per day initially, twice per 
day upon attaining 5th level, and three times per day at 9th level or higher. He 
can make only one ranged legerdemain skill check each day, and only if he has at 
least 1 rank in the skill being used.

\textbf{Sneak Attack:} This is exactly like the rogue ability of the same name. 
The extra damage dealt increases by +1d6 every other level (2nd, 4th, 6th, 8th, 
and 10th). If an arcane trickster gets a sneak attack bonus from another source 
the bonuses on damage stack.

\textbf{Impromptu Sneak Attack:} Beginning at 3rd level, once per day an arcane 
trickster can declare one melee or ranged attack he makes to be a sneak attack 
(the target can be no more than 30 feet distant if the impromptu sneak attack is 
a ranged attack). The target of an impromptu sneak attack loses any Dexterity bonus 
to AC, but only against that attack. The power can be used against any target, 
but creatures that are not subject to critical hits take no extra damage (though 
they still lose any Dexterity bonus to AC against the attack).

At 7th level, an arcane trickster can use this ability twice per day.

\vspace{12pt}
ARCHMAGE

\textbf{Hit Die:} d4.

\textbf{Requirements}

To qualify to become an archmage, a character must fulfill all the following criteria. 

\parindent=3pt
\textbf{Skills:} Knowledge (arcana) 15 ranks, Spellcraft 15 ranks.

\parindent=0pt
\textbf{Feats:} Skill Focus (Spellcraft), Spell Focus in two schools of magic.

\textbf{Spells:} Ability to cast 7th-level arcane spells, knowledge of 5th-level 
or higher spells from at least five schools.

\vspace{12pt}
\subsubsection*{\textbf{Class Skills}}

The archmage's class skills (and the key ability for each skill) are Concentration 
(Con), Craft (alchemy) (Int), Knowledge (all skills taken individually) (Int), 
Profession (Wis), Search (Int), and Spellcraft (Int). 

\parindent=3pt
\textbf{Skill Points at Each Level:} 2 + Int modifier.

\vspace{12pt}
\parindent=0pt
\begin{tabular}{|>{\raggedright}p{20pt}|>{\raggedright}p{23pt}|>{\raggedright}p{16pt}|>{\raggedright}p{16pt}|>{\raggedright}p{17pt}|>{\raggedright}p{37pt}|>{\raggedright}p{129pt}|>{\raggedright}p{-7pt}|}
\hline
\multicolumn{8}{|p{254pt}|}{T\textbf{able: The Archmage}}\tabularnewline
\hline
\subsection*{L\textbf{evel}} & \subsection*{B\textbf{ase}}\linebreak{}
\subsection*{\textbf{Attack}}\linebreak{}
\subsection*{\textbf{Bonus}} & \subsection*{F\textbf{ort}}\linebreak{}
\subsection*{\textbf{Save}} & \subsection*{R\textbf{ef}}\linebreak{}
\subsection*{\textbf{Save}} & \subsection*{W\textbf{ill}}\linebreak{}
\subsection*{\textbf{Save}} & \subsection*{S\textbf{pecial}} & \multicolumn{2}{p{122pt}|}{\subsection*{S\textbf{pells 
per Day}}}\tabularnewline
\hline
1st & +0 & +0 & +0 & +2 & High arcana & +1 level of existing arcane spellcasting 
class\tabularnewline
\hline
2nd & +1 & +0 & +0 & +3 & High arcana & \multicolumn{2}{p{122pt}|}{+1 level of 
existing arcane spellcasting class}\tabularnewline
\hline
3rd & +1 & +1 & +1 & +3 & High arcana & \multicolumn{2}{p{122pt}|}{+1\textbf{ }level 
of existing arcane spellcasting class}\tabularnewline
\hline
4th & +2 & +1 & +1 & +4 & High arcana & \multicolumn{2}{p{122pt}|}{+1 level of 
existing arcane spellcasting class}\tabularnewline
\hline
5th & +2 & +1 & +1 & +4 & High arcana & \multicolumn{2}{p{122pt}|}{+1 level of 
existing arcane spellcasting class}\tabularnewline
\hline
\end{tabular}

\vspace{12pt}
\textbf{Class Features}

All the following are \textbf{Class Features} of the archmage prestige class.

\textbf{Weapon and Armor Proficiency:} Archmages gain no proficiency with any weapon 
or armor.

\textbf{Spells per Day/Spells Known:} When a new archmage level is gained, the 
character gains new spells per day (and spells known, if applicable) as if he had 
also gained a level in whatever arcane spellcasting class in which he could cast 
7th-level spells before he added the prestige class level. He does not, however, 
gain any other benefit a character of that class would have gained. If a character 
had more than one arcane spellcasting class in which he could cast 7th-level spells 
before he became an archmage, he must decide to which class he adds each level 
of archmage for the purpose of determining spells per day.

\textbf{High Arcana:} An archmage gains the opportunity to select a special ability 
from among those described below by permanently eliminating one existing spell 
slot (she cannot eliminate a spell slot of higher level than the highest-level 
spell she can cast). Each special ability has a minimum required spell slot level, 
as specified in its description.

An archmage may choose to eliminate a spell slot of a higher level than that required 
to gain a type of high arcana.

\textit{Arcane Fire (Su): }The archmage gains the ability to change arcane spell 
energy into arcane fire, manifesting it as a bolt of raw magical energy. The bolt 
is a ranged touch attack with long range (400 feet + 40 feet/level of archmage) 
that deals 1d6 points of damage per class level of the archmage plus 1d6 points 
of damage per level of the spell used to create the effect. This ability costs 
one 9th-level spell slot.

\textit{Arcane Reach (Su): }The archmage can use spells with a range of touch on 
a target up to 30 feet away. The archmage must make a ranged touch attack. Arcane 
reach can be selected a second time as a special ability, in which case the range 
increases to 60 feet. This ability costs one 7th-level spell slot.

\textit{Mastery of Counterspelling: }When the archmage counterspells a spell, it 
is turned back upon the caster as if it were fully affected by a \textit{spell 
turning }spell. If the spell cannot be affected by \textit{spell turning}, then 
it is merely counterspelled. This ability costs one 7th-level spell slot.

\textit{Mastery of Elements: }The archmage can alter an arcane spell when cast 
so that it utilizes a different element from the one it normally uses. This ability 
can only alter a spell with the acid, cold, fire, electricity, or sonic descriptor. 
The spell's casting time is unaffected. The caster decides whether to alter the 
spell's energy type and chooses the new energy type when he begins casting. This 
ability costs one 8th-level spell slot.

\textit{Mastery of Shaping: }The archmage can alter area and effect spells that 
use one of the following shapes: burst, cone, cylinder, emanation, or spread. The 
alteration consists of creating spaces within the spell's area or effect that are 
not subject to the spell. The minimum dimension for these spaces is a 5-foot cube. 
Furthermore, any shapeable spells have a minimum dimension of 5 feet instead of 
10 feet. This ability costs one 6th-level spell slot.

\textit{Spell Power: }This ability increases the archmage's effective caster level 
by +1 (for purposes of determining level-dependent spell variables such as damage 
dice or range, and caster level checks only). This ability costs one 5th-level 
spell slot.

\textit{Spell-Like Ability: }An archmage who selects this type of high arcana can 
use one of her arcane spell slots (other than a slot expended to learn this or 
any other type of high arcana) to permanently prepare one of her arcane spells 
as a spell-like ability that can be used twice per day. The archmage does not use 
any components when casting the spell, although a spell that costs XP to cast still 
does so and a spell with a costly material component instead costs her 10 times 
that amount in XP. This ability costs one 5th-level spell slot. 

The spell-like ability normally uses a spell slot of the spell's level, although 
the archmage can choose to make a spell modified by a metamagic feat into a spell-like 
ability at the appropriate spell level. 

The archmage may use an available higher-level spell slot in order to use the spell-like 
ability more often. Using a slot three levels higher than the chosen spell allows 
her to use the spell-like ability four times per day, and a slot six levels higher 
lets her use it six times per day. 

If spell-like ability is selected more than one time as a high arcana choice, this 
ability can apply to the same spell chosen the first time (increasing the number 
of times per day it can be used) or to a different spell.

\vspace{12pt}
ASSASSIN

\textbf{Hit Die:} d6.

\textbf{Requirements}

To qualify to become an assassin, a character must fulfill all the following criteria.

\textbf{Alignment:} Any evil.

\textbf{Skills:} Disguise 4 ranks, Hide 8 ranks, Move Silently 8 ranks.

\textbf{Special:} The character must kill someone for no other reason than to join 
the assassins.

\vspace{12pt}
\subsection*{\textbf{Class Skills}}

The assassin's class skills (and the key ability for each skill) are Balance (Dex), 
Bluff (Cha), Climb (Str), Craft (Int), Decipher Script (Int), Diplomacy (Cha), 
Disable Device (Int), Disguise (Cha), Escape Artist (Dex), Forgery (Int), Gather 
Information (Cha), Hide (Dex), Intimidate (Cha), Jump (Str), Listen (Wis), Move 
Silently (Dex), Open Lock (Dex), Search (Int), Sense Motive (Wis), Sleight of Hand 
(Dex), Spot (Wis), Swim (Str), Tumble (Dex), Use Magic Device (Cha), and Use Rope 
(Dex).  

\parindent=3pt
\textbf{Skill Points at Each Level:} 4 + Int modifier.

\vspace{12pt}
\parindent=0pt
\begin{tabular}{|>{\raggedright}p{17pt}|>{\raggedright}p{19pt}|>{\raggedright}p{13pt}|>{\raggedright}p{13pt}|>{\raggedright}p{13pt}|>{\raggedright}p{110pt}|>{\raggedright}p{10pt}|>{\raggedright}p{10pt}|>{\raggedright}p{10pt}|>{\raggedright}p{10pt}|}
\hline
\multicolumn{10}{|p{230pt}|}{T\textbf{able: The Assassin}}\tabularnewline
\hline
  &   &   &   &   &  ------ & \multicolumn{4}{p{42pt}|}{ \textbf{Spells per Day 
------}}\tabularnewline
\hline
L\textbf{evel} & B\textbf{ase}\linebreak{}
\textbf{Attack}\linebreak{}
\textbf{Bonus} & F\textbf{ort}\linebreak{}
\textbf{Save} & R\textbf{ef}\linebreak{}
\textbf{Save} & W\textbf{ill}\linebreak{}
\textbf{Save} & S\textbf{pecial} & 1\textbf{st} & 2\textbf{nd} & 3\textbf{rd} & 4\textbf{th}\tabularnewline
\hline
1st & +0 & +0 & +2 & +0 & Sneak attack +1d6, death attack, poison use, spells & 0--- & --- & --- & \tabularnewline
\hline
2nd & +1 & +0 & +3 & +0 & +1 save against poison, uncanny dodge & 1--- & --- & --- & \tabularnewline
\hline
3rd & +2 & +1 & +3 & +1 & Sneak attack +2d6 & 2 & 0--- & --- & \tabularnewline
\hline
4th & +3 & +1 & +4 & +1 & +2 save against poison & 3 & 1--- & --- & \tabularnewline
\hline
5th & +3 & +1 & +4 & +1 & Improved uncanny dodge, sneak attack +3d6 & 3 & 2 & 0--- & \tabularnewline
\hline
6th & +4 & +2 & +5 & +2 & +3 save against poison & 3 & 3 & 1--- & \tabularnewline
\hline
7th & +5 & +2 & +5 & +2 & Sneak attack +4d6 & 3 & 3 & 2 & 0\tabularnewline
\hline
8th & +6 & +2 & +6 & +2 & +4 save against poison, hide in plain sight & 3 & 3 & 3 & 1\tabularnewline
\hline
9th & +6 & +3 & +6 & +3 & Sneak attack +5d6 & 3 & 3 & 3 & 2\tabularnewline
\hline
10th & +7 & +3 & +7 & +3 & +5 save against poison & 3 & 3 & 3 & 3\tabularnewline
\hline
\end{tabular}

\vspace{12pt}
\textbf{Class Features}

All of the following are Class Features of the assassin prestige class.

\textbf{Weapon and Armor Proficiency:} Assassins are proficient with the crossbow 
(hand, light, or heavy), dagger (any type), dart, rapier, sap, shortbow (normal 
and composite), and short sword. Assassins are proficient with light armor but 
not with shields.

\textbf{Sneak Attack:} This is exactly like the rogue ability of the same name. 
The extra damage dealt increases by +1d6 every other level (2nd, 4th, 6th, 8th, 
and 10th). If an assassin gets a sneak attack bonus from another source the bonuses 
on damage stack.

\textbf{Death Attack:} If an assassin studies his victim for 3 rounds and then 
makes a sneak attack with a melee weapon that successfully deals damage, the sneak 
attack has the additional effect of possibly either paralyzing or killing the target 
(assassin's choice). While studying the victim, the assassin can undertake other 
actions so long as his attention stays focused on the target and the target does 
not detect the assassin or recognize the assassin as an enemy. If the victim of 
such an attack fails a Fortitude save (DC 10 + the assassin's class level + the 
assassin's Int modifier) against the kill effect, she dies. If the saving throw 
fails against the paralysis effect, the victim is rendered helpless and unable 
to act for 1d6 rounds plus 1 round per level of the assassin. If the victim's saving 
throw succeeds, the attack is just a normal sneak attack. Once the assassin has 
completed the 3 rounds of study, he must make the death attack within the next 
3 rounds.

If a death attack is attempted and fails (the victim makes her save) or if the 
assassin does not launch the attack within 3 rounds of completing the study, 3 
new rounds of study are required before he can attempt another death attack.

\textbf{Poison Use:} Assassins are trained in the use of poison and never risk 
accidentally poisoning themselves when applying poison to a blade.

\textbf{Spells:} Beginning at 1st level, an assassin gains the ability to cast 
a number of arcane spells. To cast a spell, an assassin must have an Intelligence 
score of at least 10 + the spell's level, so an assassin with an Intelligence of 
10 or lower cannot cast these spells. Assassin bonus spells are based on Intelligence, 
and saving throws against these spells have a DC of 10 + spell level + the assassin's 
Intelligence bonus. When the assassin gets 0 spells per day of a given spell level 
he gains only the bonus spells he would be entitled to based on his Intelligence 
score for that spell level.

The assassin's spell list appears below. An assassin casts spells just as a bard 
does.

Upon reaching 6th level, at every even-numbered level after that (8th and 10th), 
an assassin can choose to learn a new spell in place of one he already knows. The 
new spell's level must be the same as that of the spell being exchanged, and it 
must be at least two levels lower than the highest-level assassin spell the assassin 
can cast. An assassin may swap only a single spell at any given level, and must 
choose whether or not to swap the spell at the same time that he gains new spells 
known for that level.

\vspace{12pt}
\begin{tabular}{|>{\raggedright}p{26pt}|>{\raggedright}p{-10pt}|>{\raggedright}p{28pt}|>{\raggedright}p{28pt}|>{\raggedright}p{28pt}|>{\raggedright}p{28pt}|}
\hline
\multicolumn{6}{|p{131pt}|}{\subsection*{T\textbf{able: Assassin Spells Known}}}\tabularnewline
\hline
 ------ & \multicolumn{5}{p{104pt}|}{ \textbf{Spells Known ------}}\tabularnewline
\hline
\multicolumn{2}{|p{16pt}|}{L\textbf{evel}} & 1\textbf{st} & 2\textbf{nd} & 3\textbf{rd} & 4\textbf{th}\tabularnewline
\hline
\multicolumn{2}{|p{16pt}|}{1st} & 2\textsuperscript{1}--- & --- & --- & \tabularnewline
\hline
\multicolumn{2}{|p{16pt}|}{2nd} & 3--- & --- & --- & \tabularnewline
\hline
\multicolumn{2}{|p{16pt}|}{3rd} & 3 & 2\textsuperscript{1}--- & --- & \tabularnewline
\hline
\multicolumn{2}{|p{16pt}|}{4th} & 4 & 3--- & --- & \tabularnewline
\hline
\multicolumn{2}{|p{16pt}|}{5th} & 4 & 3 & 2\textsuperscript{1}--- & \tabularnewline
\hline
\multicolumn{2}{|p{16pt}|}{6th} & 4 & 4 & 3--- & \tabularnewline
\hline
\multicolumn{2}{|p{16pt}|}{7th} & 4 & 4 & 3 & 2\textsuperscript{1}\tabularnewline
\hline
\multicolumn{2}{|p{16pt}|}{8th} & 4 & 4 & 4 & 3\tabularnewline
\hline
\multicolumn{2}{|p{16pt}|}{9th} & 4 & 4 & 4 & 3\tabularnewline
\hline
\multicolumn{2}{|p{16pt}|}{10th} & 4 & 4 & 4 & 4\tabularnewline
\hline
\multicolumn{6}{|p{131pt}|}{1 Provided the assassin has sufficient Intelligence 
to have a bonus spell of this level.}\tabularnewline
\hline
\end{tabular}

\vspace{12pt}
\textbf{Save Bonus against Poison:} The assassin gains a natural saving throw bonus 
to all poisons gained at 2nd level that increases by +1 for every two additional 
levels the assassin gains.

\textbf{Uncanny Dodge (Ex):} Starting at 2nd level, an assassin retains his Dexterity 
bonus to AC (if any) regardless of being caught flat-footed or struck by an invisible 
attacker. (He still loses any Dexterity bonus to AC if immobilized.)

If a character gains uncanny dodge from a second class the character automatically 
gains improved uncanny dodge (see below).

\textbf{Improved Uncanny Dodge (Ex):} At 5th level, an assassin can no longer be 
flanked, since he can react to opponents on opposite sides of him as easily as 
he can react to a single attacker. This defense denies rogues the ability to use 
flank attacks to sneak attack the assassin. The exception to this defense is that 
a rogue at least four levels higher than the assassin can flank him (and thus sneak 
attack him).

If a character gains uncanny dodge (see above) from a second class the character 
automatically gains improved uncanny dodge, and the levels from those classes stack 
to determine the minimum rogue level required to flank the character.

\textbf{Hide in Plain Sight (Su):} At 8th level, an assassin can use the Hide skill 
even while being observed. As long as he is within 10 feet of some sort of shadow, 
an assassin can hide himself from view in the open without having anything to actually 
hide behind.

He cannot, however, hide in his own shadow.

\subsection*{\textbf{Assassin Spell List}}

Assassins choose their spells from the following list:

1st Level: \textit{disguise self, detect poison, feather fall, ghost sound, jump, 
obscuring mist, sleep, true strike.}

2nd Level: \textit{alter self, cat's grace, darkness, fox's cunning, illusory script, 
invisibility, pass without trace, spider climb, undetectable alignment.}

3rd Level: \textit{deep slumber, deeper darkness, false life, magic circle against 
good, misdirection, nondetection.}

4th Level: \textit{clairaudience/clairvoyance, dimension door, freedom of movement, 
glibness, greater invisibility, locate creature, modify memory, poison.}

\vspace{12pt}
BLACKGUARD

\textbf{Hit Die:} d10.

\textbf{Requirements}

To qualify to become a blackguard, a character must fulfill all the following criteria.

\textbf{Alignment:} Any evil.

\textbf{Base Attack Bonus:} +6. 

\parindent=3pt
\textbf{Skills:} Hide 5 ranks, Knowledge (religion) 2 ranks. 

\parindent=0pt
\textbf{Feats:} Cleave, Improved Sunder, Power Attack.

\textbf{Special:} The character must have made peaceful contact with an evil outsider 
who was summoned by him or someone else.

\subsection*{\textbf{Class Skills}}

The blackguard's class skills (and the key ability for each skill) are Concentration 
(Con), Craft (Int), Diplomacy (Cha), Handle Animal (Cha), Heal (Wis), Hide (Dex), 
Intimidate (Cha), Knowledge (religion) (Int), Profession (Wis), and Ride (Dex). 

\textbf{Skill Points at Each Level:} 2 + Int modifier.

\vspace{12pt}
\begin{tabular}{|>{\raggedright}p{19pt}|>{\raggedright}p{21pt}|>{\raggedright}p{15pt}|>{\raggedright}p{15pt}|>{\raggedright}p{15pt}|>{\raggedright}p{98pt}|>{\raggedright}p{10pt}|>{\raggedright}p{12pt}|>{\raggedright}p{11pt}|>{\raggedright}p{11pt}|}
\hline
\multicolumn{10}{|p{230pt}|}{T\textbf{able: The Blackguard}}\tabularnewline
\hline
  &   &   &   &   &  --- & \multicolumn{4}{p{45pt}|}{ \textbf{Spells per Day ---}}\tabularnewline
\hline
L\textbf{evel} & B\textbf{ase}\linebreak{}
\textbf{Attack}\linebreak{}
\textbf{Bonus} & F\textbf{ort}\linebreak{}
\textbf{Save} & R\textbf{ef}\linebreak{}
\textbf{Save} & W\textbf{ill}\linebreak{}
\textbf{Save} & S\textbf{pecial} & 1\textbf{st} & 2\textbf{nd} & 3\textbf{rd} & 4\textbf{th}\tabularnewline
\hline
1st & +1 & +2 & 0 & 0 & Aura of evil, \textit{detect good}, poison use & 0--- & --- & --- & \tabularnewline
\hline
2nd & +2 & +3 & 0 & 0 & Dark blessing, smite good 1/day & 1--- & --- & --- & \tabularnewline
\hline
3rd & +3 & +3 & +1 & +1 & Command undead, aura of despair & 1 & 0--- & --- & \tabularnewline
\hline
4th & +4 & +4 & +1 & +1 & Sneak attack +1d6 & 1 & 1--- & --- & \tabularnewline
\hline
5th & +5 & +4 & +1 & +1 & Fiendish servant, smite good 2/day & 1 & 1 & 0--- & \tabularnewline
\hline
6th & +6 & +5 & +2 & +2 &  & 1 & 1 & 1--- & \tabularnewline
\hline
7th & +7 & +5 & +2 & +2 & Sneak attack +2d6 & 2 & 1 & 1 & 0\tabularnewline
\hline
8th & +8 & +6 & +2 & +2 &  & 2 & 1 & 1 & 1\tabularnewline
\hline
9th & +9 & +6 & +3 & +3 &  & 2 & 2 & 1 & 1\tabularnewline
\hline
10th & +10 & +7 & +3 & +3 & Sneak attack +3d6, smite good 3/day & 2 & 2 & 2 & 1\tabularnewline
\hline
\end{tabular}

\vspace{12pt}
\textbf{Class Features}

All of the following are Class Features of the blackguard prestige class.

\textbf{Weapon and Armor Proficiency:} Blackguards are proficient with all simple 
and martial weapons, with all types of armor, and with shields.

\textbf{Aura of Evil (Ex):} The power of a blackguard's aura of evil (see the \textit{detect 
evil }spell) is equal to his class level plus his cleric level, if any.

\textit{\textbf{Detect Good }}\textbf{(Sp):} At will, a blackguard can use \textit{detect 
good }as a spell-like ability, duplicating the effect of the \textit{detect good 
}spell.

\textbf{Poison Use:} Blackguards are skilled in the use of poison and never risk 
accidentally poisoning themselves when applying poison to a blade.

\textbf{Dark Blessing (Su):} A blackguard applies his Charisma modifier (if positive) 
as a bonus on all saving throws.

\textbf{Spells:} A blackguard has the ability to cast a small number of divine 
spells. To cast a blackguard spell, a blackguard must have a Wisdom score of at 
least 10 + the spell's level, so a blackguard with a Wisdom of 10 or lower cannot 
cast these spells.

Blackguard bonus spells are based on Wisdom, and saving throws against these spells 
have a DC of 10 + spell level + the blackguard's Wisdom modifier. When the blackguard 
gets 0 spells per day of a given spell level he gains only the bonus spells he 
would be entitled to based on his Wisdom score for that spell level. The blackguard's 
spell list appears below. A blackguard has access to any spell on the list and 
can freely choose which to prepare, just as a cleric. A blackguard prepares and 
casts spells just as a cleric does (though a blackguard cannot spontaneously cast 
\textit{cure }or \textit{inflict }spells).

\textbf{Smite Good (Su):} Once a day, a blackguard of 2nd level or higher may attempt 
to smite good with one normal melee attack.

He adds his Charisma modifier (if positive) to his attack roll and deals 1 extra 
point of damage per class level. If a blackguard accidentally smites a creature 
that is not good, the smite has no effect but it is still used up for that day.

At 5th level, and again at 10th level, a blackguard may smite good one additional 
time per day.

\textbf{Aura of Despair (Su):} Beginning at 3rd level, the blackguard radiates 
a malign aura that causes enemies within 10 feet of him to take a -2 penalty on 
all saving throws.

\textbf{Command Undead (Su):} When a blackguard reaches 3rd level, he gains the 
supernatural ability to command and rebuke undead. He commands undead as would 
a cleric of two levels lower.

\textbf{Sneak Attack:} This ability, gained at 4th level, is like the rogue ability 
of the same name. The extra damage increases by +1d6 every third level beyond 4th 
(7th and 10th). If a blackguard gets a sneak attack bonus from another source the 
bonuses on damage stack.

\subsection*{\textbf{Blackguard Spell List}}

Blackguards choose their spells from the following list:

1st Level: \textit{cause fear, corrupt weapon, cure light wounds, doom, inflict 
light wounds, magic weapon, summon monster I*.}

2nd Level: \textit{bull's strength, cure moderate wounds, darkness, death knell, 
eagle's splendor, inflict moderate wounds, shatter, summon monster II*.}

3rd Level: \textit{contagion, cure serious wounds, deeper darkness, inflict serious 
wounds, protection from elements, summon monster III*.}

4th Level: \textit{cure critical wounds, freedom of movement, inflict critical 
wounds, poison, summon monster IV*.}

* Evil creatures only.

\subsection*{\textbf{Corrupt Weapon}}

Blackguards have access to a special spell, \textit{corrupt weapon, }which is the 
opposing counterpart of the paladin spell \textit{bless weapon}. Instead of improving 
a weapon's effectiveness against evil foes \textit{corrupt weapon }makes a weapon 
more effective against good foes.

\vspace{12pt}
\subsection*{\textbf{Fallen Paladins}}

Blackguards who have levels in the paladin class (that is to say, are now ex-paladins) 
gain extra abilities the more levels of paladin they have.

A fallen paladin who becomes a blackguard gains all of the following abilities 
that apply, according to the number of paladin levels the character has.

1-2: Smite good 1/day. (This is in addition to the ability granted to all blackguards 
at 2nd level.

3-4: Lay on hands. Once per day, the blackguard can use this supernatural ability 
to cure himself or his fiendish servant of damage equal to his Charisma bonus x 
his level.

5-6: Sneak attack damage increased by +1d6. Smite good 2/day.

7-8: Fiendish summoning. Once per day, the blackguard can use a \textit{summon 
monster I }spell to call forth an evil creature. For this spell, the caster level 
is double the blackguard's class level.

9-10: Undead companion. In addition to the fiendish servant, the blackguard gains 
(at 5th level) a Medium-size skeleton or zombie as a companion. This companion 
cannot be turned or rebuked and gains all special bonuses as a fiendish servant 
when the blackguard gains levels. Smite good 3/day.

11 or more: A fallen paladin of this stature immediately gains a blackguard level 
for each level of paladin he trades in. 

The character level of the character does not change. With the loss of paladin 
levels, the character no longer gains as many extra abilities for being a fallen 
paladin. 

\vspace{12pt}
\subsection*{\textbf{The Blackguard's Fiendish Servant}}

Upon or after reaching 5th level, a blackguard can call a fiendish bat, cat, dire 
rat, horse, pony, raven, or toad to serve him. The blackguard's servant further 
gains HD and special abilities based on the blackguard's character level (see the 
table below).

A blackguard may have only one fiendish servant at a time.

Should the blackguard's servant die, he may call for another one after a year and 
a day. The new fiendish servant has all the accumulated abilities due a servant 
of the blackguard's current level.

\vspace{12pt}
\begin{tabular}{|>{\raggedright}p{45pt}|>{\raggedright}p{19pt}|>{\raggedright}p{33pt}|>{\raggedright}p{13pt}|>{\raggedright}p{10pt}|>{\raggedright}p{156pt}|}
\hline
C\textbf{haracter Level } & B\textbf{onus}\linebreak{}
\textbf{HD} & N\textbf{atural}\linebreak{}
\textbf{Armor Adj.} & S\textbf{tr}\linebreak{}
\textbf{Adj.} & I\textbf{nt} & S\textbf{pecial}\tabularnewline
\hline
12th or lower  & +2 & +1 & +1 & 6 & Empathic link, improved evasion, share saving 
throws, share spells\tabularnewline
\hline
13th-15th  & +4 & +3 & +2 & 7 & Speak with blackguard\tabularnewline
\hline
16th-18th  & +6 & +5 & +3 & 8 & Blood bond\tabularnewline
\hline
19th-20th  & +8 & +7 & +4 & 9 & Spell resistance\tabularnewline
\hline
\end{tabular}

\textit{Character Level: }The character level of the blackguard (his blackguard 
level plus his original class level).

\textit{Bonus HD: }Extra eight-sided (d8) Hit Dice, each of which gains a Constitution 
modifier, as normal. Extra Hit Dice improve the servant's base attack and base 
save bonuses, as normal.

\textit{Natural Armor Adj.: }This is an improvement to the servant's existing natural 
armor bonus.

\textit{Str Adj.: }Add this figure to the servant's Strength score.

\textit{Int: }The servant's Intelligence score. (A fiendish servant is smarter 
than normal animals of its kind.)

\vspace{12pt}
The abilities mentioned in the ``Special'' column of the accompanying table are 
described below.

\textbf{Empathic Link (Su): }The blackguard has an empathic link with his servant 
out to a distance of up to 1 mile. The blackguard cannot see through the servant's 
eyes, but they can communicate empathically. Because of the limited nature of the 
link, only general emotional content can be communicated.

Because of the empathic link between the servant and the blackguard, the blackguard 
has the same connection to a place or an item that the servant does.

\textbf{Improved Evasion (Ex):} If the servant is subjected to an attack that normally 
allows a Reflex saving throw for half damage, it takes no damage on a successful 
saving throw and only half damage on a failed saving throw. Improved evasion is 
an extraordinary ability.

\textbf{Share Saving Throws:} For each of its saving throws, the servant uses either 
its own base save bonus or the blackguard's, whichever is higher. The servant applies 
its own ability modifiers to saves, and it doesn't share any other bonuses on saves 
that the blackguard might have.

\textbf{Share Spells:} At the blackguard's option, he may have any spell (but not 
any spell-like ability) he casts on himself also affect his servant. The servant 
must be within 5 feet at the time of casting to receive the benefit. If the spell 
has a duration other than instantaneous, it stops affecting the servant if it moves 
farther than 5 feet away and will not affect the servant again even if the servant 
returns to the blackguard before the duration expires. Additionally, the blackguard 
may cast a spell with a target of ``You'' on his servant (as a touch range spell) 
instead of on himself. A blackguard and his servant can share spells even if the 
spells normally do not affect creatures of the servant's type (magical beast).

\textbf{Speak with Blackguard (Ex):} If the blackguard's character level is 13th 
or higher, the blackguard and servant can communicate verbally as if they were 
using a common language. Other creatures do not understand the communication without 
magical help.

\textbf{Blood Bond (Ex): }If the blackguard's character level is 16th or higher, 
the servant gains a +2 bonus on all attack rolls, checks, and saves if it witnesses 
the blackguard being threatened or harmed.

This bonus lasts as long as the threat is immediate and apparent.

\textbf{Spell Resistance (Ex):} If the blackguard's character level is 19th or 
higher, the servant gains spell resistance equal to the blackguard's level + 5. 
To affect the servant with a spell, another spellcaster must get a result on a 
caster level check (1d20 + caster level) that equals or exceeds the servant's spell 
resistance.

\vspace{12pt}
DRAGON DISCIPLE

\textbf{Hit Die:} d12.

\textbf{Requirements}

To qualify to become a dragon disciple, a character must fulfill all the following 
criteria.

\textbf{Race:} Any nondragon (cannot already be a half-dragon).

\textbf{Skills:} Knowledge (arcana) 8 ranks.

\textbf{Languages:} Draconic.

\textbf{Spellcasting:} Ability to cast arcane spells without preparation.

\textbf{Special:} The player chooses a dragon variety when taking the first level 
in this prestige class.

\section*{\textbf{Class Skills}}

The dragon disciple's class skills (and the key ability for each skill) are Concentration 
(Con), Craft (Int), Diplomacy (Cha), Escape Artist (Dex), Gather Information (Cha), 
Knowledge (all skills, taken individually) (Int) Listen (Wis), Profession (Wis), 
Search (Int), Speak Language (Int), Spellcraft (Int), and Spot (Wis).  

\parindent=3pt
\textbf{Skill Points at Each Level:} 2 + Int modifier.

\vspace{12pt}
\parindent=0pt
\begin{tabular}{|>{\raggedright}p{19pt}|>{\raggedright}p{22pt}|>{\raggedright}p{15pt}|>{\raggedright}p{15pt}|>{\raggedright}p{15pt}|>{\raggedright}p{138pt}|>{\raggedright}p{38pt}|}
\hline
\multicolumn{7}{|p{266pt}|}{T\textbf{able: The Dragon Disciple}}\tabularnewline
\hline
L\textbf{evel} & B\textbf{ase}\linebreak{}
\textbf{Attack}\linebreak{}
\textbf{Bonus} & F\textbf{ort}\linebreak{}
\textbf{Save} & R\textbf{ef}\linebreak{}
\textbf{Save} & W\textbf{ill}\linebreak{}
\textbf{Save} & S\textbf{pecial} & B\textbf{onus Spells}\tabularnewline
\hline
1st & +0 & +2 & +0 & +2 & Natural armor increase (+1) & 1\tabularnewline
\hline
2nd & +1 & +3 & +0 & +3 & Ability boost (Str +2), claws and bite & 1\tabularnewline
\hline
3rd & +2 & +3 & +1 & +3 & Breath weapon (2d8) & 0\tabularnewline
\hline
4th & +3 & +4 & +1 & +4 & Ability boost (Str +2), natural armor increase (+2) & 1\tabularnewline
\hline
5th & +3 & +4 & +1 & +4 & Blindsense 30 ft. & 1\tabularnewline
\hline
6th & +4 & +5 & +2 & +5 & Ability boost (Con +2) & 1\tabularnewline
\hline
7th & +5 & +5 & +2 & +5 & Breath weapon (4d8), natural armor increase (+3) & 0\tabularnewline
\hline
8th & +6 & +6 & +2 & +6 & Ability boost (Int +2) & 1\tabularnewline
\hline
9th & +6 & +6 & +3 & +6 & Wings & 1\tabularnewline
\hline
10th & +7 & +7 & +3 & +7 & Blindsense 60 ft., dragon apotheosis & 0\tabularnewline
\hline
\end{tabular}

\vspace{12pt}
\textbf{Class Features}

All of the following are Class Features of the dragon disciple prestige class.

\textbf{Weapon and Armor Proficiency:} Dragon disciples gain no proficiency with 
any weapon or armor.

\textbf{Bonus Spells:} Dragon disciples gain bonus spells as they gain levels in 
this prestige class, as if from having a high ability score, as given on Table: 
The Dragon Disciple. A bonus spell can be added to any level of spells the disciple 
already has the ability to cast.

If a character has more than one spellcasting class, he must decide to which class 
he adds each bonus spell as it is gained. Once a bonus spell has been applied, 
it cannot be shifted.

\textbf{Natural Armor Increase (Ex):} At 1st, 4th, and 7th level, a gains an increase 
to the character's existing natural armor (if any), as indicated on Table: The 
Dragon Disciple (the numbers represent the total increase gained to that point). 
As his skin thickens, a dragon disciple takes on more and more of his progenitor's 
physical aspect.

\textbf{Claws and Bite (Ex):} At 2nd level, a dragon disciple gains claw and bite 
attacks if he does not already have them. Use the values below or the disciple's 
base claw and bite damage values, whichever are greater.

\vspace{12pt}
\begin{tabular}{|>{\raggedright}p{39pt}|>{\raggedright}p{60pt}|>{\raggedright}p{63pt}|}
\hline
\subsection*{S\textbf{ize }} & \subsection*{B\textbf{ite Damage}} & \subsection*{C\textbf{law 
Damage}}\tabularnewline
\hline
Small  & 1d4 & 1d3\tabularnewline
\hline
Medium  & 1d6 & 1d4\tabularnewline
\hline
Large  & 1d8 & 1d6\tabularnewline
\hline
\end{tabular}

\vspace{12pt}
A dragon disciple is considered proficient with these attacks. When making a full 
attack, a dragon disciple uses his full base attack bonus with his bite attack 
but takes a -5 penalty on claw attacks. The Multiattack feat reduces this penalty 
to only -2.

\textbf{Ability Boost (Ex):} As a dragon disciple gains levels in this prestige 
class, his ability scores increase as noted on Table: The Dragon Disciple.

These increases stack and are gained as if through level advancement.

\textbf{Breath Weapon (Su):} At 3rd level, a dragon disciple gains a minor breath 
weapon. The type and shape depend on the dragon variety whose heritage he enjoys 
(see below). Regardless of the ancestor, the breath weapon deals 2d8 points of 
damage of the appropriate energy type.

At 7th level, the damage increases to 4d8, and when a disciple attains dragon apotheosis 
at 10th level it reaches its full power at 6d8. Regardless of its strength, the 
breath weapon can be used only once per day. Use all the rules for dragon breath 
weapons except as specified here.

The DC of the breath weapon is 10 + class level + Con modifier.

A line-shaped breath weapon is 5 feet high, 5 feet wide, and 60 feet long. A cone-shaped 
breath weapon is 30 feet long.

\vspace{12pt}
\begin{tabular}{|>{\raggedright}p{77pt}|>{\raggedright}p{119pt}|}
\hline
D\textbf{ragon Variety* } & B\textbf{reath Weapon}\tabularnewline
\hline
Black  & Line of acid\tabularnewline
\hline
Blue  & Line of lightning\tabularnewline
\hline
Green  & Cone of corrosive gas (acid)\tabularnewline
\hline
Red  & Cone of fire\tabularnewline
\hline
White  & Cone of cold\tabularnewline
\hline
Brass  & Line of fire\tabularnewline
\hline
Bronze  & Line of lightning\tabularnewline
\hline
Copper  & Line of acid\tabularnewline
\hline
Gold  & Cone of fire\tabularnewline
\hline
Silver  & Cone of cold\tabularnewline
\hline
\multicolumn{2}{|p{197pt}|}{* Other varieties of dragon disciple are possible, 
using other dragon varieties as ancestors.}\tabularnewline
\hline
\end{tabular}

\vspace{12pt}
\textbf{Blindsense (Ex):} At 5th level, the dragon disciple gains blindsense with 
a range of 30 feet. Using nonvisual senses the dragon disciple notices things it 
cannot see. He usually does not need to make Spot or Listen checks to notice and 
pinpoint the location of creatures within range of his blindsense ability, provided 
that he has line of effect to that creature.

Any opponent the dragon disciple cannot see still has total concealment against 
him, and the dragon disciple still has the normal miss chance when attacking foes 
that have concealment. Visibility still affects the movement of a creature with 
blindsense. A creature with blindsense is still denied its Dexterity bonus to Armor 
Class against attacks from creatures it cannot see. At 10th level, the range of 
this ability increases to 60 feet.

\textbf{Wings (Ex):} At 9th level, a dragon disciple grows a set of draconic wings. 
He may now fly at a speed equal to his normal land speed, with average maneuverability.

\textbf{Dragon Apotheosis:} At 10th level, a dragon disciple takes on the half-dragon 
template. His breath weapon reaches full strength (as noted above), and he gains 
+4 to Strength and +2 to Charisma. His natural armor bonus increases to +4, and 
he acquires low-light vision, 60-foot darkvision, immunity to \textit{sleep }and 
paralysis effects, and immunity to the energy type used by his breath weapon (see 
above).

\vspace{12pt}
DUELIST

\textbf{Hit Die:} d10.

\textbf{Requirements}

To qualify to become a duelist, a character must fulfill all the following criteria.

\textbf{Base Attack Bonus:} +6. 

\parindent=3pt
\textbf{Skills:} Perform 3 ranks, Tumble 5 ranks.

\parindent=0pt
\textbf{Feats:} Dodge, Mobility, Weapon Finesse.

\subsection*{\textbf{Class Skills}}

The duelist's class skills (and the key ability for each skill) are Balance (Dex), 
Bluff (Cha), Escape Artist (Dex), Jump (Str), Listen (Wis), Perform (Cha), Sense 
Motive (Wis), Spot (Wis), and Tumble (Dex). 

\parindent=3pt
\textbf{Skill Points at Each Level:} 4 + Int modifier.

\vspace{12pt}
\parindent=0pt
\begin{tabular}{|>{\raggedright}p{22pt}|>{\raggedright}p{33pt}|>{\raggedright}p{23pt}|>{\raggedright}p{23pt}|>{\raggedright}p{23pt}|>{\raggedright}p{90pt}|}
\hline
\multicolumn{6}{|p{215pt}|}{T\textbf{able: The Duelist}}\tabularnewline
\hline
L\textbf{evel} & B\textbf{ase}\linebreak{}
\textbf{Attack}\linebreak{}
\textbf{Bonus} & F\textbf{ort}\linebreak{}
\textbf{Save} & R\textbf{ef}\linebreak{}
\textbf{Save} & W\textbf{ill}\linebreak{}
\textbf{Save} & S\textbf{pecial}\tabularnewline
\hline
1st & +1 & +0 & +2 & +0 & Canny defense\tabularnewline
\hline
2nd & +2 & +0 & +3 & +0 & Improved reaction +2\tabularnewline
\hline
3rd & +3 & +1 & +3 & +1 & Enhanced mobility\tabularnewline
\hline
4th & +4 & +1 & +4 & +1 & Grace\tabularnewline
\hline
5th & +5 & +1 & +4 & +1 & Precise strike +1d6\tabularnewline
\hline
6th & +6 & +2 & +5 & +2 & Acrobatic charge\tabularnewline
\hline
7th & +7 & +2 & +5 & +2 & Elaborate parry\tabularnewline
\hline
8th & +8 & +2 & +6 & +2 & Improved reaction +4\tabularnewline
\hline
9th & +9 & +3 & +6 & +3 & Deflect Arrows\tabularnewline
\hline
10th & +10 & +3 & +7 & +3 & Precise strike +2d6\tabularnewline
\hline
\end{tabular}

\vspace{12pt}
\textbf{Class Features}

\textbf{Weapon and Armor Proficiency:} The duelist is proficient with all simple 
and martial weapons, but no type of armor or shield.

\textbf{Canny Defense (Ex):} When not wearing armor or using a shield, a duelist 
adds 1 point of Intelligence bonus (if any) per duelist class level to her Dexterity 
bonus to modify Armor Class while wielding a melee weapon. If a duelist is caught 
flat-footed or otherwise denied her Dexterity bonus, she also loses this bonus.

\textbf{Improved Reaction (Ex):} At 2nd level, a duelist gains a +2 bonus on initiative 
checks.

At 8th level, the bonus increases to +4. This bonus stacks with the benefit provided 
by the Improved Initiative feat.

\textbf{Enhanced Mobility (Ex):} When wearing no armor and not using a shield, 
a duelist gains an additional +4 bonus to AC against attacks of opportunity caused 
when she moves out of a threatened square.

\textbf{Grace (Ex):} At 4th level, a duelist gains an additional +2 competence 
bonus on all Reflex saving throws. This ability functions for a duelist only when 
she is wearing no armor and not using a shield.

\textbf{Precise Strike (Ex):} At 5th level, a duelist gains the ability to strike 
precisely with a light or one-handed piercing weapon, gaining an extra 1d6 damage 
added to her normal damage roll.

When making a precise strike, a duelist cannot attack with a weapon in her other 
hand or use a shield. A duelist's precise strike only works against living creatures 
with discernible anatomies. Any creature that is immune to critical hits is not 
vulnerable to a precise strike, and any item or ability that protects a creature 
from critical hits\textit{ }also protects a creature from a precise strike. At 
10th level, the extra damage on a precise strike increases to +2d6.

\textbf{Acrobatic Charge (Ex):} At 6th level, a duelist gains the ability to charge 
in situations where others cannot. She may charge over difficult terrain that normally 
slows movement. Depending on the circumstance, she may still need to make appropriate 
checks to successfully move over the terrain.

\textbf{Elaborate Parry (Ex):} At 7th level and higher, if a duelist chooses to 
fight defensively or use total defense in melee combat, she gains an additional 
+1 dodge bonus to AC for each level of duelist she has.

Deflect Arrows: At 9th level, a duelist gains the benefit of the Deflect Arrows 
feat when using a light or one-handed piercing weapon.

\vspace{12pt}
DWARVEN DEFENDER

\textbf{Hit Die:} d12.

\textbf{Requirements}

To qualify to become a defender, a character must fulfill all the following criteria.

\textbf{Race:} Dwarf.

\textbf{Alignment:} Any lawful.

\textbf{Base Attack Bonus:} +7.

\textbf{Feats:} Dodge, Endurance, Toughness.

\section*{\textbf{Class Skills}}

The defender's class skills (and the key ability for each skill) are Craft (Int), 
Listen (Wis), Sense Motive (Wis), and Spot (Wis). 

\parindent=3pt
\textbf{Skill Points at Each Level:} 2 + Int modifier.

\vspace{12pt}
\parindent=0pt
\begin{tabular}{|>{\raggedright}p{20pt}|>{\raggedright}p{23pt}|>{\raggedright}p{16pt}|>{\raggedright}p{16pt}|>{\raggedright}p{16pt}|>{\raggedright}p{33pt}|>{\raggedright}p{140pt}|}
\hline
\multicolumn{7}{|p{266pt}|}{T\textbf{able: The Dwarven Defender}}\tabularnewline
\hline
L\textbf{evel} & B\textbf{ase}\linebreak{}
\textbf{Attack}\linebreak{}
\textbf{Bonus} & F\textbf{ort}\linebreak{}
\textbf{Save} & R\textbf{ef}\linebreak{}
\textbf{Save} & W\textbf{ill}\linebreak{}
\textbf{Save} & A\textbf{C Bonus} & S\textbf{pecial}\tabularnewline
\hline
1st & +1 & +2 & +0 & +2 & +1 & Defensive stance 1/day\tabularnewline
\hline
2nd & +2 & +3 & +0 & +3 & +1 & Uncanny dodge\tabularnewline
\hline
3rd & +3 & +3 & +1 & +3 & +1 & Defensive stance 2/day\tabularnewline
\hline
4th & +4 & +4 & +1 & +4 & +2 & Trap sense +1\tabularnewline
\hline
5th & +5 & +4 & +1 & +4 & +2 & Defensive stance 3/day\tabularnewline
\hline
6th & +6 & +5 & +2 & +5 & +2 & Damage reduction 3/-, improved uncanny dodge\tabularnewline
\hline
7th & +7 & +5 & +2 & +5 & +3 & Defensive stance 4/day\tabularnewline
\hline
8th & +8 & +6 & +2 & +6 & +3 & Mobile defense, trap sense +2\tabularnewline
\hline
9th & +9 & +6 & +3 & +6 & +3 & Defensive stance 5/day\tabularnewline
\hline
10th & +10 & +7 & +3 & +7 & +4 & Damage reduction 6/-\tabularnewline
\hline
\end{tabular}

\vspace{12pt}
\textbf{Class Features}

All of the following are Class Features of the dwarven defender prestige class.

\textbf{AC Bonus (Ex):} The dwarven defender receives a dodge bonus to Armor Class 
that starts at +1 and improves as the defender gains levels, until it reaches +4 
at 10th level.

\textbf{Weapon and Armor Proficiency:} A dwarven defender is proficient with all 
simple and martial weapons, all types of armor, and shields.

\textbf{Defensive Stance: } When he adopts a defensive stance, a defender gains 
phenomenal strength and durability, but he cannot move from the spot he is defending. 
He gains +2 to Strength, +4 to Constitution, a +2 resistance bonus on all saves, 
and a +4 dodge bonus to AC. The increase in Constitution increases the defender's 
hit points by 2 points per level, but these hit points go away at the end of the 
defensive stance when the Constitution score drops back 4 points. These extra hit 
points are not lost first the way temporary hit points are. While in a defensive 
stance, a defender cannot use skills or abilities that would require him to shift 
his position. A defensive stance lasts for a number of rounds equal to 3 + the 
character's (newly improved) Constitution modifier. A defender may end his defensive 
stance voluntarily prior to this limit. At the end of the defensive stance, the 
defender is winded and takes a -2 penalty to Strength for the duration of that 
encounter. A defender can only use his defensive stance a certain number of times 
per day as determined by his level (see Table: The Dwarven Defender). Using the 
defensive stance takes no time itself, but a defender can only do so during his 
action.

\textbf{Uncanny Dodge (Ex):} Starting at 2nd level, a dwarven defender retains 
his Dexterity bonus to AC (if any) regardless of being caught flat-footed or struck 
by an invisible attacker. (He still loses any Dexterity bonus to AC if immobilized.)

If a character gains uncanny dodge from a second class, the character automatically 
gains improved uncanny dodge (see below).

\textbf{Trap Sense (Ex):} At 4th level, a dwarven defender gains a +1 bonus on 
Reflex saves made to avoid traps and a +1 dodge bonus to AC against attacks by 
traps. At 8th level, these bonuses rise to +2. These bonuses stack with trap sense 
bonuses gained from other classes.

\textbf{Damage Reduction (Ex):} At 6th level, a dwarven defender gains damage reduction. 
Subtract 3 points from the damage the dwarven defender takes each time he is dealt 
damage. At 10th level, this damage reduction rises to 6/-. Damage reduction can 
reduce damage to 0 but not below 0.

\textbf{Improved Uncanny Dodge (Ex):} At 6th level, a dwarven defender can no longer 
be flanked. This defense denies rogues the ability to use flank attacks to sneak 
attack the dwarven defender.

The exception to this defense is that a rogue at least four levels higher than 
the dwarven defender can flank him (and thus sneak attack him).

If a character gains uncanny dodge (see above) from a second class the character 
automatically gains improved uncanny dodge, and the levels from those classes stack 
to determine the minimum rogue level required to flank the character.

\textbf{Mobile Defense (Ex):} At 8th level, a dwarven defender can adjust his position 
while maintaining a defensive stance. While in a defensive stance, he can take 
one 5-foot step each round without losing the benefit of the stance.

\vspace{12pt}
ELDRITCH KNIGHT

\textbf{Hit Die:} d6.

\textbf{Requirements}

To qualify to become an eldritch knight, a character must fulfill all the following 
criteria.

\textbf{Weapon Proficiency:} Must be proficient with all martial weapons.

\textbf{Spells:} Able to cast 3rd-level arcane spells.

\section*{\textbf{Class Skills}}

The eldritch knight's class skills (and the key ability for each skill) are Concentration 
(Con), Craft (Int), Decipher Script (Int), Jump (Str), Knowledge (arcana) (Int), 
Knowledge (nobility and royalty) (Int), Ride (Dex), Sense Motive (Wis), Spellcraft 
(Int), and Swim (Str).  

\parindent=3pt
\textbf{Skill Points at Each Level:} 2 + Int modifier.

\vspace{12pt}
\parindent=0pt
\begin{tabular}{|>{\raggedright}p{24pt}|>{\raggedright}p{29pt}|>{\raggedright}p{20pt}|>{\raggedright}p{20pt}|>{\raggedright}p{20pt}|>{\raggedright}p{161pt}|}
\hline
\multicolumn{6}{|p{278pt}|}{T\textbf{able: The Eldritch Knight}}\tabularnewline
\hline
L\textbf{evel} & B\textbf{ase}\linebreak{}
\textbf{Attack}\linebreak{}
\textbf{Bonus} & F\textbf{ort}\linebreak{}
\textbf{Save} & R\textbf{ef}\linebreak{}
\textbf{Save} & W\textbf{ill}\linebreak{}
\textbf{Save} & S\textbf{pecial}\tabularnewline
\hline
1st & +1 & +2 & +0 & +0 & Bonus feat\tabularnewline
\hline
2nd & +2 & +3 & +0 & +0 & +1 level of existing arcane spellcasting class\tabularnewline
\hline
3rd & +3 & +3 & +1 & +1 & +1 level of existing arcane spellcasting class\tabularnewline
\hline
4th & +4 & +4 & +1 & +1 & +1 level of existing arcane spellcasting class\tabularnewline
\hline
5th & +5 & +4 & +1 & +1 & +1 level of existing arcane spellcasting class\tabularnewline
\hline
6th & +6 & +5 & +2 & +2 & +1 level of existing arcane spellcasting class\tabularnewline
\hline
7th & +7 & +5 & +2 & +2 & +1 level of existing arcane spellcasting class\tabularnewline
\hline
8th & +8 & +6 & +2 & +2 & +1 level of existing arcane spellcasting class\tabularnewline
\hline
9th & +9 & +6 & +3 & +3 & +1 level of existing arcane spellcasting class\tabularnewline
\hline
10th & +10 & +7 & +3 & +3 & +1 level of existing arcane spellcasting class\tabularnewline
\hline
\end{tabular}

\vspace{12pt}
\textbf{Class Features}

All of the following are features of the eldritch knight prestige class.

\textbf{Weapon and Armor Proficiency:} Eldritch knights gain no proficiency with 
any weapon or armor.

\textbf{Bonus Feat:} At 1st level, an eldritch knight may choose a bonus feat from 
the list of feats available to fighters. This is in addition to the feats that 
a character of any class normally gets from advancing levels. The character must 
still meet any prerequisites for these bonus feats, including levels of fighter 
for the Weapon Specialization, Greater Weapon Focus, and Greater Weapon Specialization 
feats.

\textbf{Spells per Day:} From 2nd level on, when a new eldritch knight level is 
gained, the character gains new spells per day as if she had also gained a level 
in whatever arcane spellcasting class she belonged to before she added the prestige 
class. She does not, however, gain any other benefit a character of that class 
would have gained. This essentially means that she adds the level of eldritch knight 
to the level of whatever other arcane spellcasting class the character has, then 
determines spells per day and caster level accordingly.

If a character had more than one arcane spellcasting class before she became an 
eldritch knight, she must decide to which class she adds each level of eldritch 
knight for the purpose of determining spells per day.

\vspace{12pt}
HIEROPHANT

\textbf{Hit Die:} d8.

\textbf{Requirements}

To qualify to become a hierophant, a character must fulfill all the following criteria.

\textbf{Skills:} Knowledge (religion) 15 ranks.

\textbf{Feats:} Any metamagic feat.

\textbf{Spells:} Able to cast 7th-level divine spells.

\section*{\textbf{Class Skills}}

The hierophant's class skills (and the key ability for each skill) are Concentration 
(Con), Craft (Int), Diplomacy (Cha), Heal (Wis), Knowledge (arcana) (Int), Knowledge 
(religion) (Int), Profession (Wis), and Spellcraft (Int).  

\parindent=3pt
\textbf{Skill Points at Each Level:} 2 + Int modifier.

\vspace{12pt}
\parindent=0pt
\begin{tabular}{|>{\raggedright}p{29pt}|>{\raggedright}p{33pt}|>{\raggedright}p{23pt}|>{\raggedright}p{23pt}|>{\raggedright}p{23pt}|>{\raggedright}p{60pt}|}
\hline
\multicolumn{6}{|p{192pt}|}{T\textbf{able: The Hierophant}}\tabularnewline
\hline
L\textbf{evel } & B\textbf{ase}\linebreak{}
\textbf{Attack}\linebreak{}
\textbf{Bonus} & F\textbf{ort}\linebreak{}
\textbf{Save} & R\textbf{ef}\linebreak{}
\textbf{Save} & W\textbf{ill}\linebreak{}
\textbf{Save} & S\textbf{pecial}\tabularnewline
\hline
1st & +0 & +2 & +0 & +2 & Special ability\tabularnewline
\hline
2nd & +1 & +3 & +0 & +3 & Special ability\tabularnewline
\hline
3rd & +1 & +3 & +1 & +3 & Special ability\tabularnewline
\hline
4th & +2 & +4 & +1 & +4 & Special ability\tabularnewline
\hline
5th & +2 & +4 & +1 & +4 & Special ability\tabularnewline
\hline
\end{tabular}

\vspace{12pt}
\textbf{Class Features}

All the following are Class Features of the hierophant prestige class.

\textbf{Weapon and Armor Proficiency:} Hierophants gain no proficiency with any 
weapon or armor.

\textbf{Spells and Caster Level:} Levels in the hierophant prestige class, even 
though they do not advance spell progression in the character's base class, still 
stack with the character's base spellcasting levels to determine caster level.

\textbf{Special Ability:} Every level, a hierophant gains a special ability of 
his choice from among the following.

\textit{Blast Infidel (Su): }A hierophant can use negative energy spells to their 
maximum effect on creatures with an alignment opposed to the hierophant. (See the 
table below for a list of which alignments are opposed to each alignment.) Any 
spell with a description that involves inflicting or channeling negative energy 
cast on a creature of the opposed alignment works as if under the effect of a Maximize 
Spell feat (without using a higher-level spell slot). Undead affected by this ability 
heal the maximized amount of damage.

\vspace{12pt}
\begin{tabular}{|>{\raggedright}p{102pt}|>{\raggedright}p{112pt}|}
\hline
H\textbf{ierophant Alignment } & O\textbf{pposed Alignment}\tabularnewline
\hline
Lawful good  & Chaotic evil\tabularnewline
\hline
Neutral good  & Neutral evil\tabularnewline
\hline
Chaotic good  & Lawful evil\tabularnewline
\hline
Lawful neutral  & Chaotic neutral\tabularnewline
\hline
Neutral  & Lawful good, chaotic good, lawful evil, chaotic evil*\tabularnewline
\hline
Chaotic neutral  & Lawful neutral\tabularnewline
\hline
Lawful evil  & Chaotic good\tabularnewline
\hline
Neutral evil  & Neutral good\tabularnewline
\hline
Chaotic evil  & Lawful good\tabularnewline
\hline
\multicolumn{2}{|p{215pt}|}{* A neutral hierophant chooses one of these alignments 
to be the one that he opposes, for the purposes of this special ability.}\tabularnewline
\hline
\end{tabular}

\vspace{12pt}
\textit{Divine Reach (Su): }A hierophant with this ability can use touch spells 
on targets up to 30 feet away. If the spell requires a melee touch attack, the 
hierophant must make a ranged touch attack instead. Divine reach can be selected 
a second time as a special ability, in which case the range increases to 60 feet.

\textit{Faith Healing (Su): }A hierophant can use healing spells to their maximum 
effect on creatures of the same alignment as the hierophant (including the hierophant 
himself ). Any spell with the healing descriptor cast on such creatures works as 
if under the effects of a Maximize Spell feat (without using a higher-level spell 
slot).

\textit{Gift of the Divine (Su): }Available only to hierophants with cleric levels, 
this ability allows a hierophant to transfer one or more uses of his turn undead 
ability to a willing creature. (Hierophants who rebuke undead transfer uses of 
rebuke undead instead.) The transfer lasts anywhere from 24 hours to one week (chosen 
at the time of transfer), and while the transfer is in effect, the number of turning 
attempts per day allowed to the hierophant is reduced by the number transferred. 
The recipient turns undead as a cleric of the hierophant's cleric level but uses 
her own Charisma modifier.

\textit{Mastery of Energy (Su): }Available only to hierophants with cleric levels, 
this ability allows a hierophant to channel positive or negative energy much more 
effectively, increasing his ability to affect undead. Add a +4 bonus to the hierophant's 
turning checks and turning damage rolls. This ability only affects undead, even 
if the hierophant can turn other creatures, such as with a granted power of a domain.

\textit{Metamagic Feat: }A hierophant can choose a metamagic feat in place of one 
of the special abilities described here if desired.

\textit{Power of Nature (Su): }Available only to hierophants with druid levels, 
this ability allows a hierophant to temporarily transfer one or more of his druid 
Class Features to a willing creature. The transfer lasts anywhere from 24 hours 
to one week (chosen at the time of transfer), and while the transfer is in effect, 
the hierophant cannot use the transferred power. He can transfer any of his druid 
powers except spellcasting and the ability to have an animal companion.

The druid's wild shape ability can be partially or completely transferred. The 
heirophant choses how many uses of wild shape per day to give to transfer and retains 
the rest of the uses for himself. If the hierophant can assume the form of Tiny 
or Huge animals, the recipient can as well.

As with the \textit{imbue with spell ability }spell, the hierophant remains responsible 
to his deity for any use to which the recipient puts the transferred abilities.

\textit{Spell Power: }This special ability increases a hierophant's effective caster 
level by 1 for purposes of determining level-dependent spell variables and for 
caster level checks. This ability can be selected more than once, and changes to 
effective caster level are cumulative.

\textit{Spell-Like Ability: }A hierophant who selects this special ability can 
use one of his divine spell slots to permanently prepare one of his divine spells 
as a spell-like ability that can be used twice per day. The hierophant does not 
use any components when casting the spell, although a spell that costs XP to cast 
still does so, and a spell with a costly material component instead costs him 10 
times that amount in XP.

The spell normally uses a spell slot of the spell's level (or higher, if the hierophant 
chooses to permanently attach a metamagic feat to the spell chosen). The hierophant 
can use an available higher-level spell slot to use the spell-like ability more 
than once per day. Allocating a slot three levels higher allows him to cast the 
spell four times per day, and a slot six levels higher lets him cast it six times 
per day. If selected more than one time as a special ability, this ability can 
apply to the same spell (increasing the number of times per day it can be used) 
or to a different spell.

\vspace{12pt}
HORIZON WALKER

\textbf{Hit Die:} d8.

\textbf{Requirements}

To qualify to become a horizon walker, a character must fulfill all the following 
criteria.

\textbf{Skills:} Knowledge (geography) 8 ranks.

\textbf{Feats:} Endurance.

\section*{\textbf{Class Skills}}

The horizon walker's class skills (and the key ability for each skill) are Balance 
(Dex), Climb (Str), Diplomacy (Cha), Handle Animal (Cha), Hide (Dex), Knowledge 
(geography) (Int), Listen (Wis), Move Silently (Dex), Profession (Wis), Ride (Dex), 
Speak Language (none), Spot (Wis), and Survival (Wis).  

\parindent=3pt
\textbf{Skill Points at Each Level:} 4 + Int modifier.

\vspace{12pt}
\parindent=0pt
\begin{tabular}{|>{\raggedright}p{26pt}|>{\raggedright}p{33pt}|>{\raggedright}p{23pt}|>{\raggedright}p{23pt}|>{\raggedright}p{23pt}|>{\raggedright}p{91pt}|}
\hline
\multicolumn{6}{|p{221pt}|}{T\textbf{able: The Horizon Walker}}\tabularnewline
\hline
L\textbf{evel} & B\textbf{ase}\linebreak{}
\textbf{Attack}\linebreak{}
\textbf{Bonus} & F\textbf{ort}\linebreak{}
\textbf{Save} & R\textbf{ef}\linebreak{}
\textbf{Save} & W\textbf{ill}\linebreak{}
\textbf{Save} & S\textbf{pecial}\tabularnewline
\hline
1st & +1 & +2 & +0 & +0 & Terrain mastery\tabularnewline
\hline
2nd & +2 & +3 & +0 & +0 & Terrain mastery\tabularnewline
\hline
3rd & +3 & +3 & +1 & +1 & Terrain mastery\tabularnewline
\hline
4th & +4 & +4 & +1 & +1 & Terrain mastery\tabularnewline
\hline
5th & +5 & +4 & +1 & +1 & Terrain mastery\tabularnewline
\hline
6th & +6 & +5 & +2 & +2 & Planar terrain mastery\tabularnewline
\hline
7th & +7 & +5 & +2 & +2 & Planar terrain mastery\tabularnewline
\hline
8th & +8 & +6 & +2 & +2 & Planar terrain mastery\tabularnewline
\hline
9th & +9 & +6 & +3 & +3 & Planar terrain mastery\tabularnewline
\hline
10th & +10 & +7 & +3 & +3 & Planar terrain mastery\tabularnewline
\hline
\end{tabular}

\vspace{12pt}
\textbf{Class Features}

All of the following are features of the horizon walker prestige class.

\textbf{Weapon and Armor Proficiency:} Horizon walkers gain no proficiency with 
any weapon or armor.

\textbf{Terrain Mastery:} At each level, the Horizon Walker adds a new terrain 
environment to their repertoire from those given below. Terrain mastery gives a 
horizon walker a bonus on checks involving a skill useful in that terrain, or some 
other appropriate benefit. A horizon walker also knows how to fight dangerous creatures 
typically found in that terrain, gaining a +1 insight bonus on attack rolls and 
damage rolls against creatures with that terrain mentioned in the Environment entry 
of their descriptions. The horizon walker only gains the bonus if the creature 
description specifically lists the terrain type.

Horizon walkers take their terrain mastery with them wherever they go. They retain 
their terrain mastery bonuses on skill checks, attack rolls, and damage rolls whether 
they're actually in the relevant terrain or not.

\textbf{Planar Terrain Mastery:} Planar terrain mastery functions just like terrain 
mastery, except that the horizon walker can choose one of the planar categories 
at each level. The horizon walker can take a non-planar terrain type instead, if 
she wishes.

\vspace{12pt}
\subsection*{\textbf{Terrain Mastery Benefits}}

\textbf{Aquatic:} You gain a +4 competence bonus on Swim checks, or a +10-foot 
bonus to your swim speed if you have one. You gain a +1 insight bonus on attack 
and damage rolls against aquatic creatures.

\textbf{Desert: }You resist effects that tire you. You are immune to fatigue, and 
anything that would cause you to become exhausted makes you fatigued instead. You 
gain a +1 insight bonus on attack and damage rolls against desert creatures.

\textbf{Forest:} You have a +4 competence bonus on Hide checks. You gain a +1 insight 
bonus on attack and damage rolls against forest creatures.

\textbf{Hills:} You gain a +4 competence bonus on Listen checks. You gain a +1 
insight bonus on attack and damage rolls against hills creatures.

\textbf{Marsh:} You have a +4 competence bonus on Move Silently checks. You gain 
a +1 insight bonus on attack and damage rolls against marsh creatures.

\textbf{Mountains:} You gain a +4 competence bonus on Climb checks, or a +10- foot 
bonus to your climb speed if you have one. You gain a +1 insight bonus on attack 
and damage rolls against mountain creatures.

\textbf{Plains}: You have a +4 competence bonus on Spot checks. You gain a +1 insight 
bonus on attack and damage rolls against plains creatures.

\textbf{Underground: }You have 60-foot darkvision, or 120-foot darkvision if you 
already had darkvision from another source. You gain a +1 insight bonus on attack 
and damage rolls against underground creatures.

\textbf{Fiery (Planar):} This kind of planar terrain mastery provides you with 
resistance to fire 20. You gain a +1 insight bonus on attack and damage rolls against 
outsiders and elementals with the fire subtype.

\textbf{Weightless (Planar):} You gain a +30-foot bonus to your fly speed on planes 
with no gravity or subjective gravity. You gain a +1 insight on attack and damage 
rolls against creatures native to the Astral Plane, the Elemental Plane of Air, 
and the Ethereal Plane.

\textbf{Cold (Planar):} This kind of planar terrain mastery provides you with resistance 
to cold 20. You gain a +1 insight bonus on attack and damage rolls against outsiders 
and elementals with the cold subtype.

\textbf{Shifting (Planar):} You instinctively anticipate shifts in the reality 
of the plane that bring you closer to your destination, giving you the spell-like 
ability to use \textit{dimension door }(as the spell cast at your character level) 
once every 1d4 rounds. You gain a +1 insight bonus on attack and damage rolls against 
outsiders and elementals native to a shifting plane.

\textbf{Aligned (Planar):} You have the instinctive ability to mimic the dominant 
alignment of the plane. You incur none of the penalties for having an alignment 
at odds with that of the plane, and spells and abilities that harm those of the 
opposite alignment don't affect you. You have the dominant alignment of the plane 
with regard to magic, but your behavior and any alignment-related Class Features 
you have are unaffected.

\textbf{Cavernous (Planar):} You gain tremorsense with a 30-foot range.

\textbf{Other (Planar):} If other planes are in use additional Planar Terrains 
can be created.

\vspace{12pt}
LOREMASTER

\textbf{Hit Die:} d4.

\textbf{Requirements}

To qualify to become a loremaster, a character must fulfill all the following criteria.

\textbf{Skills:} Knowledge (any two) 10 ranks in each.

\textbf{Feats:} Any three metamagic or item creation feats, plus Skill Focus (Knowledge 
[any individual Knowledge skill]).

\textbf{Spells:} Able to cast seven different divination spells, one of which must 
be 3rd level or higher.

\section*{\textbf{Class Skills}}

The loremaster's class skills (and the key ability for each skill) are Appraise 
(Int), Concentration (Con), Craft (alchemy) (Int), Decipher Script (Int), Gather 
Information (Cha), Handle Animals (Cha), Heal (Wis), Knowledge (all skills taken 
individually) (Int), Perform (Cha), Profession (Wis), Speak Language, Spellcraft 
(Int), and Use Magic Device (Cha). 

\textbf{Skill Points at Each Level:} 4 + Int modifier.

\vspace{12pt}
\begin{tabular}{|>{\raggedright}p{24pt}|>{\raggedright}p{29pt}|>{\raggedright}p{20pt}|>{\raggedright}p{20pt}|>{\raggedright}p{20pt}|>{\raggedright}p{59pt}|>{\raggedright}p{91pt}|}
\hline
\multicolumn{7}{|p{266pt}|}{T\textbf{able : The Loremaster}}\tabularnewline
\hline
\subsection*{L\textbf{evel}} & \subsection*{B\textbf{ase}}\linebreak{}
\subsection*{\textbf{Attack}}\linebreak{}
\subsection*{\textbf{Bonus}} & \subsection*{F\textbf{ort}}\linebreak{}
\subsection*{\textbf{Save}} & \subsection*{R\textbf{ef}}\linebreak{}
\subsection*{\textbf{Save}} & \subsection*{W\textbf{ill}}\linebreak{}
\subsection*{\textbf{Save}} & \subsection*{S\textbf{pecial}} & \subsection*{S\textbf{pells 
per Day}}\tabularnewline
\hline
1st & +0 & +0 & +0 & +2 & Secret & +1 level of existing class\tabularnewline
\hline
2nd & +1 & +0 & +0 & +3 & Lore & +1 level of existing class\tabularnewline
\hline
3rd & +1 & +1 & +1 & +3 & Secret & +1 level of existing class\tabularnewline
\hline
4th & +2 & +1 & +1 & +4 & Bonus language & +1 level of existing class\tabularnewline
\hline
5th & +2 & +1 & +1 & +4 & Secret & +1 level of existing class\tabularnewline
\hline
6th & +3 & +2 & +2 & +5 & Greater lore & +1 level of existing class\tabularnewline
\hline
7th & +3 & +2 & +2 & +5 & Secret & +1 level of existing class\tabularnewline
\hline
8th & +4 & +2 & +2 & +6 & Bonus language & +1 level of existing class\tabularnewline
\hline
9th & +4 & +3 & +3 & +6 & Secret & +1 level of existing class\tabularnewline
\hline
10th & +5 & +3 & +3 & +7 & True lore & +1 level of existing class\tabularnewline
\hline
\end{tabular}

\vspace{12pt}
\section*{\textbf{Class Features}}

All of the following are Class Features of the loremaster prestige class.

\textbf{Weapon and Armor Proficiency:} Loremasters gain no proficiency with any 
weapon or armor.

\textbf{Spells per Day/Spells Known: }When a new loremaster level is gained, the 
character gains new spells per day (and spells known, if applicable) as if she 
had also gained a level in a spellcasting class she belonged to before she added 
the prestige class. She does not, however, gain any other benefit a character of 
that class would have gained. This essentially means that she adds the level of 
loremaster to the level of some other spellcasting class the character has, then 
determines spells per day, spells known, and caster level accordingly.

\textbf{Secret:} At 1st level and every two levels higher than 1st (3rd, 5th, 7th, 
and 9th), the loremaster chooses one

secret from the table below. Her level plus Intelligence modifier determines the 
total number of secrets she can choose. She can't choose the same secret twice.

\textbf{Lore:} At 2nd level, a loremaster gains the ability to know legends or 
information regarding various topics, just as a bard can with bardic knowledge. 
The loremaster adds her level and her Intelligence modifier to the lore check, 
which functions otherwise exactly like a bardic knowledge check.

\textbf{Bonus Languages: }A loremaster can choose any new language at 4th and 8th 
level.

\textbf{Greater Lore (Ex):} At 6th level, a loremaster gains the ability to understand 
magic items, as with the \textit{identify }spell.

\textbf{True Lore (Ex):} At 10th level, once per day a loremaster can use her knowledge 
to gain the effect of a \textit{legend lore }spell or an \textit{analyze dweomer 
}spell.

\vspace{12pt}
\begin{tabular}{|>{\raggedright}p{50pt}|>{\raggedright}p{86pt}|>{\raggedright}p{178pt}|}
\hline
\multicolumn{3}{|p{314pt}|}{L\textbf{oremaster Secrets}}\tabularnewline
\hline
L\textbf{evel +}\linebreak{}
\textbf{Int Modifier } & S\textbf{ecret } & E\textbf{ffect}\tabularnewline
\hline
1  & Instant mastery  & 4 ranks of a skill in which the character has no ranks\tabularnewline
\hline
2  & Secret health  & +3 hit points\tabularnewline
\hline
3  & Secrets of inner strength  & +2 bonus on Will saves\tabularnewline
\hline
4  & The lore of true stamina  & +2 bonus on Fortitude saves\tabularnewline
\hline
5  & Secret knowledge of avoidance & +2 bonus on Reflex saves\tabularnewline
\hline
6  & Weapon trick  & +1 bonus on attack rolls\tabularnewline
\hline
7  & Dodge trick  & +1 dodge bonus to AC\tabularnewline
\hline
8  & Applicable knowledge  & Any one feat\tabularnewline
\hline
9  & Newfound arcana  & 1 bonus 1st-level spell*\tabularnewline
\hline
10  & More newfound arcana  & 1 bonus 2nd-level spell*\tabularnewline
\hline
\multicolumn{3}{|p{314pt}|}{* As if gained through having a high ability score.}\tabularnewline
\hline
\end{tabular}

\vspace{12pt}
MYSTIC THEURGE

\textbf{Hit Die:} d4.

\textbf{Requirements}

To qualify to become a mystic theurge, a character must fulfill all the following 
criteria.

\textbf{Skills:} Knowledge (arcana) 6 ranks, Knowledge (religion) 6 ranks.

\textbf{Spells:} Able to cast 2nd-level divine spells and 2nd-level arcane spells.

\section*{\textbf{Class Skills}}

The mystic theurge's class skills (and the key ability for each skill) are Concentration 
(Con), Craft (Int), Decipher Script (Int), Knowledge (arcana) (Int), Knowledge 
(religion) (Int), Profession (Wis), Sense Motive (Wis), and Spellcraft (Int).  

\parindent=3pt
\textbf{Skill Points at Each Level:} 2 + Int modifier.

\vspace{12pt}
\parindent=0pt
\begin{tabular}{|>{\raggedright}p{24pt}|>{\raggedright}p{29pt}|>{\raggedright}p{20pt}|>{\raggedright}p{20pt}|>{\raggedright}p{20pt}|>{\raggedright}p{162pt}|}
\hline
\multicolumn{6}{|p{278pt}|}{T\textbf{able: The Mystic Theurge}}\tabularnewline
\hline
L\textbf{evel} & B\textbf{ase}\linebreak{}
\textbf{Attack}\linebreak{}
\textbf{Bonus} & F\textbf{ort}\linebreak{}
\textbf{Save} & R\textbf{ef}\linebreak{}
\textbf{Save} & W\textbf{ill}\linebreak{}
\textbf{Save} & S\textbf{pells per Day}\tabularnewline
\hline
1st & +0 & +0 & +0 & +2 & +1 level of existing arcane spellcasting class/\linebreak{}
+1 level of existing divine spellcasting class\tabularnewline
\hline
2nd & +1 & +0 & +0 & +3 & +1 level of existing arcane spellcasting class/\linebreak{}
+1 level of existing divine spellcasting class\tabularnewline
\hline
3rd & +1 & +1 & +1 & +3 & +1 level of existing arcane spellcasting class/\linebreak{}
+1 level of existing divine spellcasting class\tabularnewline
\hline
4th & +2 & +1 & +1 & +4 & +1 level of existing arcane spellcasting class/\linebreak{}
+1 level of existing divine spellcasting class\tabularnewline
\hline
5th & +2 & +1 & +1 & +4 & +1 level of existing arcane spellcasting class/\linebreak{}
+1 level of existing divine spellcasting class\tabularnewline
\hline
6th & +3 & +2 & +2 & +5 & +1 level of existing arcane spellcasting class/\linebreak{}
+1 level of existing divine spellcasting class\tabularnewline
\hline
7th & +3 & +2 & +2 & +5 & +1 level of existing arcane spellcasting class/\linebreak{}
+1 level of existing divine spellcasting class\tabularnewline
\hline
8th & +4 & +2 & +2 & +6 & +1 level of existing arcane spellcasting class/\linebreak{}
+1 level of existing divine spellcasting class\tabularnewline
\hline
9th & +4 & +3 & +3 & +6 & +1 level of existing arcane spellcasting class/\linebreak{}
+1 level of existing divine spellcasting class\tabularnewline
\hline
10th & +5 & +3 & +3 & +7 & +1 level of existing arcane spellcasting class/\linebreak{}
+1 level of existing divine spellcasting class\tabularnewline
\hline
\end{tabular}

\vspace{12pt}
\textbf{Class Features}

All of the following are features of the mystic theurge prestige class.

\textbf{Weapon and Armor Proficiency:} Mystic theurges gain no proficiency with 
any weapon or armor.

\textbf{Spells per Day: }When a new mystic theurge level is gained, the character 
gains new spells per day as if he had also gained a level in any one arcane spellcasting 
class he belonged to before he added the prestige class and any one divine spellcasting 
class he belonged to previously. He does not, however, gain any other benefit a 
character of that class would have gained. This essentially means that he adds 
the level of mystic theurge to the level of whatever other arcane spellcasting 
class and divine spellcasting class the character has, then determines spells per 
day and caster level accordingly. If a character had more than one arcane spellcasting 
class or more than one divine spellcasting class before he became a mystic theurge, 
he must decide to which class he adds each level of mystic theurge for the purpose 
of determining spells per day.

\vspace{12pt}
SHADOWDANCER

\textbf{Hit Die:} d8.

\textbf{Requirements}

To qualify to become a shadowdancer, a character must fulfill all the following 
criteria.

\textbf{Skills:} Move Silently 8 ranks, Hide 10 ranks, Perform (dance) 5 ranks.

\textbf{Feats:} Combat Reflexes, Dodge, Mobility. 

\section*{\textbf{Class Skills}}

The shadowdancer's class skills (and the key ability for each skill) are Balance 
(Dex), Bluff (Cha), Decipher Script (Int), Diplomacy (Cha), Disguise (Cha), Escape 
Artist (Dex), Hide (Dex), Jump (Str), Listen (Wis), Move Silently (Dex), Perform 
(Cha), Profession (Wis), Search (Int), Sleight of Hand (Dex), Spot (Wis), Tumble 
(Dex), and Use Rope (Dex). 

\parindent=3pt
\textbf{Skill Points at Each Level:} 6 + Int modifier.

\vspace{12pt}
\parindent=0pt
\begin{tabular}{|>{\raggedright}p{25pt}|>{\raggedright}p{31pt}|>{\raggedright}p{21pt}|>{\raggedright}p{21pt}|>{\raggedright}p{21pt}|>{\raggedright}p{156pt}|}
\hline
\multicolumn{6}{|p{278pt}|}{T\textbf{able: The Shadowdancer}}\tabularnewline
\hline
\subsection*{L\textbf{evel}} & B\textbf{ase}\linebreak{}
\textbf{Attack}\linebreak{}
\textbf{Bonus} & F\textbf{ort}\linebreak{}
\textbf{Save} & R\textbf{ef}\linebreak{}
\textbf{Save} & W\textbf{ill}\linebreak{}
\textbf{Save} & S\textbf{pecial}\tabularnewline
\hline
1st & +0 & +0 & +2 & +0 & Hide in plain sight\tabularnewline
\hline
2nd & +1 & +0 & +3 & +0 & Evasion, darkvision, uncanny dodge\tabularnewline
\hline
3rd & +2 & +1 & +3 & +1 & S\textit{hadow illusion}, summon shadow\tabularnewline
\hline
4th & +3 & +1 & +4 & +1 & Shadow jump 20 ft.\tabularnewline
\hline
5th & +3 & +1 & +4 & +1 & Defensive roll, improved uncanny dodge\tabularnewline
\hline
6th & +4 & +2 & +5 & +2 & Shadow jump 40 ft., summon shadow\tabularnewline
\hline
7th & +5 & +2 & +5 & +2 & Slippery mind\tabularnewline
\hline
8th & +6 & +2 & +6 & +2 & Shadow jump 80 ft.\tabularnewline
\hline
9th & +6 & +3 & +6 & +3 & Summon shadow\tabularnewline
\hline
10th & +7 & +3 & +7 & +3 & Shadow jump 160 ft., improved evasion\tabularnewline
\hline
\end{tabular}

\vspace{12pt}
\textbf{Class Features}

All of the following are features of the shadowdancer prestige class.

\textbf{Weapon and Armor Proficiency:} Shadowdancers are proficient with the club, 
crossbow (hand, light, or heavy), dagger (any type), dart, mace, morningstar, quarterstaff, 
rapier, sap, shortbow (normal and composite), and short sword. Shadowdancers are 
proficient with light armor but not with shields.

\textbf{Hide in Plain Sight (Su):} A shadowdancer can use the Hide skill even while 
being observed. As long as she is within 10 feet of some sort of shadow, a shadowdancer 
can hide herself from view in the open without anything to actually hide behind. 
She cannot, however, hide in her own shadow.

\textbf{Evasion (Ex):} At 2nd level, a shadowdancer gains evasion. If exposed to 
any effect that normally allows her to attempt a Reflex saving throw for half damage, 
she takes no damage with a successful saving throw. The evasion ability can only 
be used if the shadowdancer is wearing light armor or no armor.

\textbf{Darkvision (Su): }At 2nd level, a shadowdancer can see in the dark as though 
she were permanently under the effect of a \textit{darkvision }spell.

\textbf{Uncanny Dodge (Ex):} Starting at 2nd level, a shadowdancer retains her 
Dexterity bonus to AC (if any) regardless of being caught flat-footed or struck 
by an invisible attacker. (She still loses any Dexterity bonus to AC if immobilized.)

If a character gains uncanny dodge from a second class, the character automatically 
gains improved uncanny dodge (see below).

\textit{\textbf{Shadow Illusion }}\textbf{(Sp):} When a shadowdancer reaches 3rd 
level, she can create visual illusions. This ability's effect is identical to that 
of the arcane spell \textit{silent image }and may be employed once per day.

\textbf{Summon Shadow (Su):} At 3rd level, a shadowdancer can summon a shadow, 
an undead shade. Unlike a normal shadow, this shadow's alignment matches that of 
the shadowdancer, and the creature cannot create spawn. The summoned shadow cannot 
be turned, rebuked, or commanded by any third party. This shadow serves as a companion 
to the shadowdancer and can communicate intelligibly with the shadowdancer. Every 
third level gained by the shadowdancer adds +2 HD (and the requisite base attack 
and base save bonus increases) to her shadow companion. 

\parindent=3pt
If a shadow companion is destroyed, or the shadowdancer chooses to dismiss it, 
the shadowdancer must attempt a DC 15 Fortitude save. If the saving throw fails, 
the shadowdancer loses 200 experience points per shadowdancer level. A successful 
saving throw reduces the loss by half, to 100 XP per prestige class level. The 
shadowdancer's XP total can never go below 0 as the result of a shadow's dismissal 
or destruction. A destroyed or dismissed shadow companion cannot be replaced for 
30 days.

\parindent=0pt
\textbf{Shadow Jump (Su):} At 4th level, a shadowdancer gains the ability to travel 
between shadows as if by means of a \textit{dimension door }spell. The limitation 
is that the magical transport must begin and end in an area with at least some 
shadow. A shadowdancer can jump up to a total of 20 feet each day in this way; 
this may be a single jump of 20 feet or two jumps of 10 feet each. Every two levels 
higher than 4th, the distance a shadowdancer can jump each day doubles (40 feet 
at 6th, 80 feet at 8th, and 160 feet at 10th). This amount can be split among many 
jumps, but each one, no matter how small, counts as a 10-foot increment.

\textbf{Defensive Roll (Ex):} Starting at 5th level, once per day, when a shadowdancer 
would be reduced to 0 hit points or less by damage in combat (from a weapon or 
other blow, not a spell or special ability), she can attempt to roll with the damage. 
She makes a Reflex saving throw (DC = damage dealt) and, if successful, takes only 
half damage from the blow. She must be aware of the attack and able to react to 
it in order to execute her defensive roll. If she is in a situation that would 
deny her any Dexterity bonus to AC, she can't attempt a defensive roll.

\textbf{Improved Uncanny Dodge (Ex):} At 5th level, a shadowdancer can no longer 
be flanked. This defense denies rogues the ability to use flank attacks to sneak 
attack the shadowdancer. The exception to this defense is that a rogue at least 
four levels higher than the shadowdancer can flank her (and thus sneak attack her).

If a character gains uncanny dodge (see above) from a second class the character 
automatically gains improved uncanny dodge, and the levels from those classes stack 
to determine the minimum rogue level required to flank the character.

\textbf{Slippery Mind (Ex):} At 7th level, if a shadowdancer is affected by an 
enchantment and fails her saving throw, 1 round later she can attempt her saving 
throw again. She only gets this one extra chance to succeed at her saving throw. 
If it fails as well, the spell's effects occur normally.

\textbf{Improved Evasion (Ex):} This ability, gained at 10th level, works like 
evasion (see above). A shadowdancer takes no damage at all on successful saving 
throws against attacks that allow a Reflex saving throw for half damage. What's 
more, she takes only half damage even if she fails her saving throw.

\vspace{12pt}
THAUMATURGIST

\textbf{Hit Die:} d4.

\textbf{Requirements}

To qualify to become a thaumaturgist, a character must fulfill all the following 
criteria.

\textbf{Feats:} Spell Focus (conjuration).

\textbf{Spells:} Able to cast \textit{lesser planar ally}.

\section*{\textbf{Class Skills}}

The thaumaturgist's class skills (and the key ability for each skill) are Concentration 
(Con), Craft (Int), Diplomacy (Cha), Knowledge (religion) (Int), Knowledge (the 
planes) (Int), Profession (Wis), Sense Motive (Wis), Speak Language (none), and 
Spellcraft (Int). 

\parindent=3pt
\textbf{Skill Points at Each Level:} 2 + Int modifier.

\vspace{12pt}
\parindent=0pt
\begin{tabular}{|>{\raggedright}p{20pt}|>{\raggedright}p{23pt}|>{\raggedright}p{16pt}|>{\raggedright}p{16pt}|>{\raggedright}p{16pt}|>{\raggedright}p{67pt}|>{\raggedright}p{106pt}|}
\hline
\multicolumn{7}{|p{266pt}|}{T\textbf{able: The Thaumaturgist}}\tabularnewline
\hline
L\textbf{evel} & B\textbf{ase}\linebreak{}
\textbf{Attack}\linebreak{}
\textbf{Bonus} & F\textbf{ort}\linebreak{}
\textbf{Save} & R\textbf{ef}\linebreak{}
\textbf{Save} & W\textbf{ill}\linebreak{}
\textbf{Save} & S\textbf{pecial} & S\textbf{pells per Day}\tabularnewline
\hline
1st & +0 & +0 & +0 & +2 & Improved ally & +1 level of existing spellcasting class\tabularnewline
\hline
2nd & +1 & +0 & +0 & +3 & Augment Summoning & +1 level of existing spellcasting 
class\tabularnewline
\hline
3rd & +1 & +1 & +1 & +3 & Extended summoning & +1 level of existing spellcasting 
class\tabularnewline
\hline
4th & +2 & +1 & +1 & +4 & Contingent conjuration & +1 level of existing spellcasting 
class\tabularnewline
\hline
5th & +2 & +1 & +1 & +4 & Planar cohort & +1 level of existing spellcasting class\tabularnewline
\hline
\end{tabular}

\vspace{12pt}
\textbf{Class Features}

All of the following are features of the thaumaturgist prestige class.

\textbf{Weapon and Armor Proficiency:} Thaumaturgists gain no proficiency with 
any weapon or armor.

\textbf{Spells per Day: }When a new thaumaturgist level is gained, the character 
gains new spells per day as if he had also gained a level in whatever spellcasting 
class he belonged to before he added the prestige class. He does not, however, 
gain any other benefit a character of that class would have gained. This essentially 
means that he adds the level of thaumaturgist to the level of whatever other spellcasting 
class the character has, then determines spells per day and caster level accordingly.

If a character had more than one spellcasting class before he became a thaumaturgist, 
he must decide to which class he adds each level of thaumaturgist for the purpose 
of determining spells per day.

\textbf{Improved Ally:} When a thaumaturgist casts a \textit{planar ally }spell 
(including the \textit{lesser }and \textit{greater }versions), he makes a Diplomacy 
check to convince the creature to aid him for a reduced payment. If the thaumaturgist's 
Diplomacy check adjusts the creature's attitude to helpful the creature will work 
for 50\% of the standard fee, as long as the task is one that is not against its 
nature. 

The thaumaturgist's improved ally class feature only works when the planar ally 
shares at least one aspect of alignment with the thaumaturgist.

A thaumaturgist can have only one such ally at a time, but he may bargain for tasks 
from other planar allies normally.

\textbf{Augment Summoning: }At 2nd level, a thaumaturgist gains the Augment Summoning 
feat.

\textbf{Extended Summoning: }At 3rd level and higher, all spells from the summoning 
subschool that the thaumaturgist casts have their durations doubled, as if the 
Extend Spell feat had been applied to them. The levels of the summoning spells 
don't change, however. This ability stacks with the effect of the Extend Spell 
feat, which does change the spell's level.

\textbf{Contingent Conjuration:} A 4th-level thaumaturgist can prepare a summoning 
or calling spell ahead of time to be triggered by some other event. This functions 
as described for the \textit{contingency }spell, including having the thaumaturgist 
cast the summoning or calling spell beforehand. The spell is cast instantly when 
the trigger event occurs. 

The conditions needed to bring the spell into effect must be clear, although they 
can be general. If complicated or convoluted condition as are prescribed, the contingent 
conjuration may fail when triggered. The conjuration spell occurs based solely 
on the stated conditions, regardless of whether the thaumaturgist wants it to, 
although most conjurations can be dismissed normally. A thaumaturgist can have 
only one contingent conjuration active at a time.

\textbf{Planar Cohort:} A 5th-level thaumaturgist can use any of the \textit{planar 
ally }spells to call a creature to act as his cohort. The called creature serves 
loyally and well as long as the thaumaturgist continues to advance a cause important 
to the creature. 

To call a planar cohort, the thaumaturgist must cast the relevant spell, paying 
the XP costs normally. It takes an offering of 1,000 gp x the HD of the creature 
to convince it to serve as a planar cohort, and the improved ally class feature 
can't be used to reduce or eliminate this cost. The planar cohort can't have more 
Hit Dice than the thaumaturgist has, and must have an ECL no higher than the thaumaturgist's 
character level -2.

A thaumaturgist can have only one planar cohort at a time, but he can continue 
to make agreements with other called creatures normally. A planar cohort replaces 
a thaumaturgist's existing cohort, if he has one by virtue of the Leadership feat.

\newpage

\end{document}
