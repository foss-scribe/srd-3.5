%&pdfLaTeX
% !TEX encoding = UTF-8 Unicode
\documentclass{article}
\usepackage{ifxetex}
\ifxetex
\usepackage{fontspec}
\setmainfont[Mapping=tex-text]{STIXGeneral}
\else
\usepackage[T1]{fontenc}
\usepackage[utf8]{inputenc}
\fi
\usepackage{textcomp}

\usepackage{array}
\usepackage{amssymb}
\usepackage{fancyhdr}
\renewcommand{\headrulewidth}{0pt}
\renewcommand{\footrulewidth}{0pt}

\begin{document}

This material is Open Game Content, and is licensed for public use under the terms 
of the Open Game License v1.0a.

{\LARGE{}SPELLS (D-E)}

\vspace{12pt}
Dancing Lights

Evocation [Light]

\textbf{Level:} Brd 0, Sor/Wiz 0

\textbf{Components:} V, S

\textbf{Casting Time:} 1 standard action

\textbf{Range: }Medium (100 ft. + 10 ft./level)

\textbf{Effect:} Up to four lights, all within a 10- ft.-radius area

\textbf{Duration:} 1 minute (D)

\textbf{Saving Throw:} None

\textbf{Spell Resistance:} No

Depending on the version selected, you create up to four lights that resemble lanterns 
or torches (and cast that amount of light), or up to four glowing spheres of light 
(which look like will-o'-wisps), or one faintly glowing, vaguely humanoid shape. 
The \textit{dancing lights }must stay within a 10-foot-radius area in relation 
to each other but otherwise move as you desire (no concentration required): forward 
or back, up or down, straight or turning corners, or the like. The lights can move 
up to 100 feet per round. A light winks out if the distance between you and it 
exceeds the spell's range.

\textit{Dancing lights }can be made permanent with a \textit{permanency }spell.

\vspace{12pt}
Darkness

Evocation [Darkness]

\textbf{Level:} Brd 2, Clr 2, Sor/Wiz 2

\textbf{Components:} V, M/DF

\textbf{Casting Time:} 1 standard action

\textbf{Range:} Touch

\textbf{Target:} Object touched

\textbf{Duration:} 10 min./level (D)

\textbf{Saving Throw:} None

\textbf{Spell Resistance:} No

This spell causes an object to radiate shadowy illumination out to a 20-foot radius. 
All creatures in the area gain concealment (20\% miss chance). Even creatures that 
can normally see in such conditions (such as with darkvision or low-light vision) 
have the miss chance in an area shrouded in magical \textit{darkness}.

Normal lights (torches, candles, lanterns, and so forth) are incapable of brightening 
the area, as are light spells of lower level. Higher level light spells are not 
affected by \textit{darkness.}

If \textit{darkness }is cast on a small object that is then placed inside or under 
a lightproof covering, the spell's effect is blocked until the covering is removed.

\textit{Darkness }counters or dispels any light spell of equal or lower spell level.

\textit{Arcane Material Component: }A bit of bat fur and either a drop of pitch 
or a piece of coal.

\vspace{12pt}
Darkvision

Transmutation

\textbf{Level:} Rgr 3, Sor/Wiz 2

\textbf{Components:} V, S, M

\textbf{Casting Time:} 1 standard action

\textbf{Range:} Touch

\textbf{Target:} Creature touched

\textbf{Duration:} 1 hour/level

\textbf{Saving Throw: }Will negates (harmless)

\textbf{Spell Resistance:} Yes (harmless)

The subject gains the ability to see 60 feet even in total darkness. Darkvision 
is black and white only but otherwise like normal sight. \textit{Darkvision }does 
not grant one the ability to see in magical darkness.

\textit{Darkvision }can be made permanent with a \textit{permanency }spell.

\textit{Material Component: }Either a pinch of dried carrot or an agate.

\vspace{12pt}
Daylight

Evocation [Light]

\textbf{Level:} Brd 3, Clr 3, Drd 3, Pal 3, Sor/Wiz 3

\textbf{Components:} V, S

\textbf{Casting Time:} 1 standard action

\textbf{Range:} Touch

\textbf{Target:} Object touched

\textbf{Duration:} 10 min./level (D)

\textbf{Saving Throw:} None

\textbf{Spell Resistance:} No

The object touched sheds light as bright as full daylight in a 60-foot radius, 
and dim light for an additional 60 feet beyond that. Creatures that take penalties 
in bright light also take them while within the radius of this magical light. Despite 
its name, this spell is not the equivalent of daylight for the purposes of creatures 
that are damaged or destroyed by bright light.

If \textit{daylight }is cast on a small object that is then placed inside or under 
a light- proof covering, the spell's effects are blocked until the covering is 
removed.

\textit{Daylight }brought into an area of magical darkness (or vice versa) is temporarily 
negated, so that the otherwise prevailing light conditions exist in the overlapping 
areas of effect.

\textit{Daylight }counters or dispels any darkness spell of equal or lower level, 
such as \textit{darkness.}

\vspace{12pt}
Daze

Enchantment (Compulsion) [Mind-Affecting]

\textbf{Level:} Brd 0, Sor/Wiz 0

\textbf{Components:} V, S, M

\textbf{Casting Time:} 1 standard action

\textbf{Range:} Close (25 ft. + 5 ft./2 levels)

\textbf{Target:} One humanoid creature of 4 HD or less

\textbf{Duration:} 1 round

\textbf{Saving Throw: }Will negates

\textbf{Spell Resistance:} Yes

This enchantment clouds the mind of a humanoid creature with 4 or fewer Hit Dice 
so that it takes no actions. Humanoids of 5 or more HD are not affected. A dazed 
subject is not stunned, so attackers get no special advantage against it.

\textit{Material Component: }A pinch of wool or similar substance.

\vspace{12pt}
Daze Monster

Enchantment (Compulsion) [Mind-Affecting]

\textbf{Level:} Brd 2, Sor/Wiz 2

\textbf{Range: }Medium (100 ft. + 10 ft./level)

\textbf{Target: }One living creature of 6 HD or less

This spell functions like \textit{daze, }but \textit{daze monster }can affect any 
one living creature of any type. Creatures of 7 or more HD are not affected.

\vspace{12pt}
Death Knell

Necromancy [Death, Evil]

\textbf{Level:} Clr 2, Death 2

\textbf{Components:} V, S

\textbf{Casting Time:} 1 standard action

\textbf{Range:} Touch

\textbf{Target:} Living creature touched

\textbf{Duration:} Instantaneous/10 minutes per HD of subject; see text

\textbf{Saving Throw: }Will negates

\textbf{Spell Resistance:} Yes

You draw forth the ebbing life force of a creature and use it to fuel your own 
power. Upon casting this spell, you touch a living creature that has -1 or fewer 
hit points. If the subject fails its saving throw, it dies, and you gain 1d8 temporary 
hit points and a +2 bonus to Strength. Additionally, your effective caster level 
goes up by +1, improving spell effects dependent on caster level. (This increase 
in effective caster level does not grant you access to more spells.) These effects 
last for 10 minutes per HD of the subject creature.

\vspace{12pt}
Death Ward

Necromancy

\textbf{Level:} Clr 4, Death 4, Drd 5, Pal 4

\textbf{Components:} V, S, DF

\textbf{Casting Time:} 1 standard action

\textbf{Range:} Touch

\textbf{Target:} Living creature touched

\textbf{Duration:} 1 min./level

\textbf{Saving Throw: }Will negates (harmless)

\textbf{Spell Resistance:} Yes (harmless)

The subject is immune to all death spells, magical death effects, energy drain, 
and any negative energy effects.

This spell doesn't remove negative levels that the subject has already gained, 
nor does it affect the saving throw necessary 24 hours after gaining a negative 
level.

\textit{Death ward }does not protect against other sorts of attacks even if those 
attacks might be lethal.

\vspace{12pt}
Deathwatch

Necromancy [Evil]

\textbf{Level:} Clr 1

\textbf{Components:} V, S

\textbf{Casting Time:} 1 standard action

\textbf{Range:} 30 ft.

\textbf{Area:} Cone-shaped emanation

\textbf{Duration:} 10 min./level

\textbf{Saving Throw:} None

\textbf{Spell Resistance:} No

Using the foul sight granted by the powers of unlife, you can determine the condition 
of creatures near death within the spell's range. You instantly know whether each 
creature within the area is dead, fragile (alive and wounded, with 3 or fewer hit 
points left), fighting off death (alive with 4 or more hit points), undead, or 
neither alive nor dead (such as a construct).

\textit{Deathwatch }sees through any spell or ability that allows creatures to 
feign death.

\vspace{12pt}
Deep Slumber

Enchantment (Compulsion) [Mind-Affecting]

\textbf{Level:} Brd 3, Sor/Wiz 3

\textbf{Range:} Close (25 ft. + 5 ft./2 levels)

This spell functions like \textit{sleep, }except that it affects 10 HD of creatures.

\vspace{12pt}
Deeper Darkness

Evocation [Darkness]

\textbf{Level:} Clr 3

\textbf{Duration:} One day/level (D)

This spell functions like \textit{darkness}, except that the object radiates shadowy 
illumination in a 60-foot radius and the \textit{darkness }lasts longer.

\textit{Daylight }brought into an area of \textit{deeper darkness }(or vice versa) 
is temporarily negated, so that the otherwise prevailing light conditions exist 
in the overlapping areas of effect.

\textit{Deeper darkness }counters and dispels any light spell of equal or lower 
level, including \textit{daylight }and \textit{light.}

\vspace{12pt}
Delay Poison

Conjuration (Healing)

\textbf{Level:} Brd 2, Clr 2, Drd 2, Pal 2, Rgr 1

\textbf{Components:} V, S, DF

\textbf{Casting Time:} 1 standard action

\textbf{Range:} Touch

\textbf{Target:} Creature touched

\textbf{Duration:} 1 hour/level

\textbf{Saving Throw:} Fortitude negates (harmless)

\textbf{Spell Resistance:} Yes (harmless)

The subject becomes temporarily immune to poison. Any poison in its system or any 
poison to which it is exposed during the spell's duration does not affect the subject 
until the spell's duration has expired. \textit{Delay poison }does not cure any 
damage that poison may have already done.

\vspace{12pt}
Delayed Blast Fireball

Evocation [Fire]

\textbf{Level:} Sor/Wiz 7

\textbf{Duration:} 5 rounds or less; see text

This spell functions like \textit{fireball, }except that it is more powerful and 
can detonate up to 5 rounds after the spell is cast. The burst of flame deals 1d6 
points of fire damage per caster level (maximum 20d6).

The glowing bead created by \textit{delayed blast fireball }can detonate immediately 
if you desire, or you can choose to delay the burst for as many as 5 rounds. You 
select the amount of delay upon completing the spell, and that time cannot change 
once it has been set unless someone touches the bead (see below). If you choose 
a delay, the glowing bead sits at its destination until it detonates. A creature 
can pick up and hurl the bead as a thrown weapon (range increment 10 feet). If 
a creature handles and moves the bead within 1 round of its detonation, there is 
a 25\% chance that the bead detonates while being handled.

\vspace{12pt}
Demand

Enchantment (Compulsion) [Mind-Affecting]

\textbf{Level:} Sor/Wiz 8

\textbf{Saving Throw: }Will partial

\textbf{Spell Resistance:} Yes

This spell functions like \textit{sending, }but the message can also contain a 
\textit{suggestion }(see the \textit{suggestion }spell), which the subject does 
its best to carry out. A successful Will save negates the \textit{suggestion }effect 
but not the contact itself. The \textit{demand, }if received, is understood even 
if the subject's Intelligence score is as low as 1. If the message is impossible 
or meaningless according to the circumstances that exist for the subject at the 
time the \textit{demand }is issued, the message is understood but the \textit{suggestion 
}is ineffective.

The \textit{demand'}s message to the creature must be twenty-five words or less, 
including the \textit{suggestion. }The creature can also give a short reply immediately.

\textit{Material Component: }A short piece of copper wire and some small part of 
the subject---a hair, a bit of nail, or the like.

\vspace{12pt}
Desecrate

Evocation [Evil]

\textbf{Level:} Clr 2, Evil 2

\textbf{Components:} V, S, M, DF

\textbf{Casting Time:} 1 standard action

\textbf{Range:} Close (25 ft. + 5 ft./2 levels)

\textbf{Area:} 20-ft.-radius emanation

\textbf{Duration:} 2 hours/level

\textbf{Saving Throw:} None

\textbf{Spell Resistance:} Yes

This spell imbues an area with negative energy. Each Charisma check made to turn 
undead within this area takes a -3 profane penalty, and every undead creature entering 
a \textit{desecrated }area gains a +1 profane bonus on attack rolls, damage rolls, 
and saving throws. An undead creature created within or summoned into such an area 
gains +1 hit points per HD.

If the \textit{desecrated }area contains an altar, shrine, or other permanent fixture 
dedicated to your deity or aligned higher power, the modifiers given above are 
doubled (-6 profane penalty on turning checks, +2 profane bonus and +2 hit points 
per HD for undead in the area).

Furthermore, anyone who casts \textit{animate dead }within this area may create 
as many as double the normal amount of undead (that is, 4 HD per caster level rather 
than 2 HD per caster level).

If the area contains an altar, shrine, or other permanent fixture of a deity, pantheon, 
or higher power other than your patron, the \textit{desecrate }spell instead curses 
the area, cutting off its connection with the associated deity or power. This secondary 
function, if used, does not also grant the bonuses and penalties relating to undead, 
as given above.

\textit{Desecrate }counters and dispels \textit{consecrate}.

\textit{Material Component: }A vial of unholy water and 25 gp worth (5 pounds) 
of silver dust, all of which must be sprinkled around the area.

\vspace{12pt}
Destruction

Necromancy [Death]

\textbf{Level:} Clr 7, Death 7

\textbf{Components:} V, S, F

\textbf{Casting Time:} 1 standard action

\textbf{Range:} Close (25 ft. + 5 ft./2 levels)

\textbf{Target:} One creature

\textbf{Duration:} Instantaneous

\textbf{Saving Throw:} Fortitude partial

\textbf{Spell Resistance:} Yes

This spell instantly slays the subject and consumes its remains (but not its equipment 
and possessions) utterly. If the target's Fortitude saving throw succeeds, it instead 
takes 10d6 points of damage. The only way to restore life to a character who has 
failed to save against this spell is to use \textit{true resurrection}, a carefully 
worded \textit{wish }spell followed by \textit{resurrection}, or \textit{miracle}.

\textit{Focus: }A special holy (or unholy) symbol of silver marked with verses 
of anathema (cost 500 gp).

\vspace{12pt}
Detect Animals or Plants

Divination

\textbf{Level:} Drd 1, Rgr 1

\textbf{Components:} V, S

\textbf{Casting Time:} 1 standard action

\textbf{Range:} Long (400 ft. + 40 ft./level)

\textbf{Area:} Cone-shaped emanation

\textbf{Duration:} Concentration, up to 10 min./level (D)

\textbf{Saving Throw:} None

\textbf{Spell Resistance:} No

You can detect a particular kind of animal or plant in a cone emanating out from 
you in whatever direction you face. You must think of a kind of animal or plant 
when using the spell, but you can change the animal or plant kind each round. The 
amount of information revealed depends on how long you search a particular area 
or focus on a specific kind of animal or plant.

\textit{1st Round: }Presence or absence of that kind of animal or plant in the 
area.

\textit{2nd Round: }Number of individuals of the specified kind in the area, and 
the condition of the healthiest specimen.

\textit{3rd Round: }The condition (see below) and location of each individual present. 
If an animal or plant is outside your line of sight, then you discern its direction 
but not its exact location.

\textit{Conditions: }For purposes of this spell, the categories of condition are 
as follows:

Normal: Has at least 90\% of full normal hit points, free of disease.

Fair: 30\% to 90\% of full normal hit points remaining.

Poor: Less than 30\% of full normal hit points remaining, afflicted with a disease, 
or suffering from a debilitating injury.

Weak: 0 or fewer hit points remaining, afflicted with a disease in the terminal 
stage, or crippled.

If a creature falls into more than one category, the spell indicates the weaker 
of the two.

Each round you can turn to detect a kind of animal or plant in a new area. The 
spell can penetrate barriers, but 1 foot of stone, 1 inch of common metal, a thin 
sheet of lead, or 3 feet of wood or dirt blocks it.

\vspace{12pt}
Detect Chaos

Divination

\textbf{Level:} Clr 1

This spell functions like \textit{detect evil}, except that it detects the auras 
of chaotic creatures, clerics of chaotic deities, chaotic spells, and chaotic magic 
items, and you are vulnerable to an overwhelming chaotic aura if you are lawful.

\vspace{12pt}
Detect Evil

Divination

\textbf{Level:} Clr 1

\textbf{Components:} V, S, DF

\textbf{Casting Time:} 1 standard action

\textbf{Range:} 60 ft.

\textbf{Area:} Cone-shaped emanation

\textbf{Duration:} Concentration, up to 10 min./ level (D)

\textbf{Saving Throw:} None

\textbf{Spell Resistance:} No

You can sense the presence of evil. The amount of information revealed depends 
on how long you study a particular area or subject.

\textit{1st Round: }Presence or absence of evil.

\textit{2nd Round: }Number of evil auras (creatures, objects, or spells) in the 
area and the power of the most potent evil aura present.

If you are of good alignment, and the strongest evil aura's power is overwhelming 
(see below), and the HD or level of the aura's source is at least twice your character 
level, you are stunned for 1 round and the spell ends.

\textit{3rd Round: }The power and location of each aura. If an aura is outside 
your line of sight, then you discern its direction but not its exact location.

\textit{Aura Power: }An evil aura's power depends on the type of evil creature 
or object that you're detecting and its HD, caster level, or (in the case of a 
cleric) class level; see the accompanying table. If an aura falls into more than 
one strength category, the spell indicates the stronger of the two.

\begin{tabular}{|>{\raggedright}p{124pt}|>{\raggedright}p{42pt}|>{\raggedright}p{36pt}|>{\raggedright}p{31pt}|>{\raggedright}p{55pt}|}
\hline
------------------------ & \multicolumn{4}{p{166pt}|}{ \textbf{Aura Power ------------------------}}\tabularnewline
\hline
C\textbf{reature/Object} & F\textbf{aint} & M\textbf{oderate} & S\textbf{trong} & O\textbf{verwhelming}\tabularnewline
\hline
Evil creature\textsuperscript{\textbf{1}}\textbf{ }(HD) & 10 or lower & 11-25 & 26-50 & 51 
or higher\tabularnewline
\hline
Undead (HD) & 2 or lower & 3-8 & 9-20 & 21 or higher\tabularnewline
\hline
Evil outsider (HD) & 1 or lower & 2-4 & 5-10 & 11 or higher\tabularnewline
\hline
Cleric of an evil deity \textsuperscript{\textbf{2}}\textbf{ }(class levels) & 1 & 2-4 & 5-10 & 11 
or higher\tabularnewline
\hline
Evil magic item or spell (caster level) & 2nd or lower & 3rd-8th & 9th-20th & 21st 
or higher\tabularnewline
\hline
\multicolumn{5}{|p{290pt}|}{1 Except for undead and outsiders, which have their 
own entries on the table.}\tabularnewline
\hline
\multicolumn{5}{|p{290pt}|}{2 Some characters who are not clerics may radiate an 
aura of equivalent power. The class description will indicate whether this applies.}\tabularnewline
\hline
\end{tabular}

\textit{Lingering Aura: }An evil aura lingers after its original source dissipates 
(in the case of a spell) or is destroyed (in the case of a creature or magic item). 
If \textit{detect evil }is cast and directed at such a location, the spell indicates 
an aura strength of dim (even weaker than a faint aura). How long the aura lingers 
at this dim level depends on its original power:

\begin{tabular}{|>{\raggedright}p{74pt}|>{\raggedright}p{118pt}|}
\hline
O\textbf{riginal Strength} & D\textbf{uration of Lingering Aura}\tabularnewline
\hline
Faint & 1d6 rounds\tabularnewline
\hline
Moderate  & 1d6 minutes\tabularnewline
\hline
Strong & 1d6x10 minutes\tabularnewline
\hline
Overwhelming & 1d6 days\tabularnewline
\hline
\end{tabular}

Animals, traps, poisons, and other potential perils are not evil, and as such this 
spell does not detect them.

Each round, you can turn to detect evil in a new area. The spell can penetrate 
barriers, but 1 foot of stone, 1 inch of common metal, a thin sheet of lead, or 
3 feet of wood or dirt blocks it.

\vspace{12pt}
Detect Good

Divination

\textbf{Level:} Clr 1

This spell functions like \textit{detect evil}, except that it detects the auras 
of good creatures, clerics or paladins of good deities, good spells, and good magic 
items, and you are vulnerable to an overwhelming good aura if you are evil. Healing 
potions, antidotes, and similar beneficial items are not good.

\vspace{12pt}
Detect Law

Divination

\textbf{Level:} Clr 1

This spell functions like \textit{detect evil}, except that it detects the auras 
of lawful creatures, clerics of lawful deities, lawful spells, and lawful magic 
items, and you are vulnerable to an overwhelming lawful aura if you are chaotic.

\vspace{12pt}
Detect Magic

Divination

\textbf{Level:} Brd 0, Clr 0, Drd 0, Sor/Wiz 0

\textbf{Components:} V, S

\textbf{Casting Time:} 1 standard action

\textbf{Range:} 60 ft.

\textbf{Area:} Cone-shaped emanation

\textbf{Duration:} Concentration, up to 1 min./level (D)

\textbf{Saving Throw:} None

\textbf{Spell Resistance:} No

You detect magical auras. The amount of information revealed depends on how long 
you study a particular area or subject.

\textit{1st Round: }Presence or absence of magical auras.

\textit{2nd Round: }Number of different magical auras and the power of the most 
potent aura.

\textit{3rd Round: }The strength and location of each aura. If the items or creatures 
bearing the auras are in line of sight, you can make Spellcraft skill checks to 
determine the school of magic involved in each. (Make one check per aura; DC 15 
+ spell level, or 15 + half caster level for a nonspell effect.)

Magical areas, multiple types of magic, or strong local magical emanations may 
distort or conceal weaker auras.

\textit{Aura Strength: }An aura's power depends on a spell's functioning spell 
level or an item's caster level. If an aura falls into more than one category, 
\textit{detect magic }indicates the stronger of the two.

\begin{tabular}{|>{\raggedright}p{107pt}|>{\raggedright}p{43pt}|>{\raggedright}p{36pt}|>{\raggedright}p{35pt}|>{\raggedright}p{67pt}|}
\hline
 --------------------------- & \multicolumn{4}{p{182pt}|}{ \textbf{Aura Power ---------------------------}}\tabularnewline
\hline
S\textbf{pell or Object} & F\textbf{aint} & M\textbf{oderate} & S\textbf{trong} & O\textbf{verwhelming}\tabularnewline
\hline
Functioning spell (spell level) & 3rd or lower & 4th-6th & 7th-9th & 10th+ (deity-level)\tabularnewline
\hline
Magic item (caster level) & 5th or lower & 6th-11th & 12th-20th & 21st+ (artifact)\tabularnewline
\hline
\end{tabular}

\textit{Lingering Aura: }A magical aura lingers after its original source dissipates 
(in the case of a spell) or is destroyed (in the case of a magic item). If \textit{detect 
magic }is cast and directed at such a location, the spell indicates an aura strength 
of dim (even weaker than a faint aura). How long the aura lingers at this dim level 
depends on its original power:

\begin{tabular}{|>{\raggedright}p{74pt}|>{\raggedright}p{118pt}|}
\hline
O\textbf{riginal Strength} & D\textbf{uration of Lingering Aura}\tabularnewline
\hline
Faint & 1d6 rounds\tabularnewline
\hline
Moderate & 1d6 minutes\tabularnewline
\hline
Strong & 1d6x10 minutes\tabularnewline
\hline
Overwhelming & 1d6 days\tabularnewline
\hline
\end{tabular}

Outsiders and elementals are not magical in themselves, but if they are summoned, 
the conjuration spell registers.

Each round, you can turn to detect magic in a new area. The spell can penetrate 
barriers, but 1 foot of stone, 1 inch of common metal, a thin sheet of lead, or 
3 feet of wood or dirt blocks it.

\textit{Detect magic }can be made permanent with a \textit{permanency }spell.

\vspace{12pt}
Detect Poison

Divination

\textbf{Level:} Clr 0, Drd 0, Pal 1, Rgr 1, Sor/Wiz 0

\textbf{Components:} V, S

\textbf{Casting Time:} 1 standard action

\textbf{Range:} Close (25 ft. + 5 ft./2 levels)

\textbf{Target or Area:} One creature, one object, or a 5-ft. cube

\textbf{Duration:} Instantaneous

\textbf{Saving Throw:} None

\textbf{Spell Resistance:} No

You determine whether a creature, object, or area has been poisoned or is poisonous. 
You can determine the exact type of poison with a DC 20 Wisdom check. A character 
with the Craft (alchemy) skill may try a DC 20 Craft (alchemy) check if the Wisdom 
check fails, or may try the Craft (alchemy) check prior to the Wisdom check.

The spell can penetrate barriers, but 1 foot of stone, 1 inch of common metal, 
a thin sheet of lead, or 3 feet of wood or dirt blocks it.

\vspace{12pt}
Detect Scrying

Divination

\textbf{Level:} Brd 4, Sor/Wiz 4

\textbf{Components:} V, S, M

\textbf{Casting Time:} 1 standard action

\textbf{Range:} 40 ft.

\textbf{Area:} 40-ft.-radius emanation centered on you

\textbf{Duration:} 24 hours

\textbf{Saving Throw:} None

\textbf{Spell Resistance:} No

You immediately become aware of any attempt to observe you by means of a divination 
(scrying) spell or effect. The spell's area radiates from you and moves as you 
move. You know the location of every magical sensor within the spell's area.

If the scrying attempt originates within the area, you also know its location; 
otherwise, you and the scrier immediately make opposed caster level checks (1d20 
+ caster level). If you at least match the scrier's result, you get a visual image 
of the scrier and an accurate sense of his or her direction and distance from you.

\textit{Material Component: }A small piece of mirror and a miniature brass hearing 
trumpet.

\vspace{12pt}
Detect Secret Doors

Divination

\textbf{Level:} Brd 1, Knowledge 1, Sor/Wiz 1

\textbf{Components:} V, S

\textbf{Casting Time:} 1 standard action

\textbf{Range:} 60 ft.

\textbf{Area:} Cone-shaped emanation

\textbf{Duration:} Concentration, up to 1 min./level (D)

\textbf{Saving Throw:} None

\textbf{Spell Resistance:} No

You can detect secret doors, compartments, caches, and so forth. Only passages, 
doors, or openings that have been specifically constructed to escape detection 
are detected by this spell. The amount of information revealed depends on how long 
you study a particular area or subject.

\textit{1st Round: }Presence or absence of secret doors.

\textit{2nd Round: }Number of secret doors and the location of each. If an aura 
is outside your line of sight, then you discern its direction but not its exact 
location.

\textit{Each Additional Round: }The mechanism or trigger for one particular secret 
portal closely examined by you. Each round, you can turn to detect secret doors 
in a new area. The spell can penetrate barriers, but 1 foot of stone, 1 inch of 
common metal, a thin sheet of lead, or 3 feet of wood or dirt blocks it.

\vspace{12pt}
Detect Snares and Pits

Divination

\textbf{Level:} Drd 1, Rgr 1

\textbf{Components:} V, S

\textbf{Casting Time:} 1 standard action

\textbf{Range:} 60 ft.

\textbf{Area:} Cone-shaped emanation

\textbf{Duration:} Concentration, up to 10 min./level (D)

\textbf{Saving Throw:} None

\textbf{Spell Resistance:} No

You can detect simple pits, deadfalls, and snares as well as mechanical traps constructed 
of natural materials. The spell does not detect complex traps, including trapdoor 
traps.

\textit{Detect snares and pits }does detect certain natural hazards---quicksand 
(a snare), a sinkhole (a pit), or unsafe walls of natural rock (a deadfall). However, 
it does not reveal other potentially dangerous conditions. The spell does not detect 
magic traps (except those that operate by pit, deadfall, or snaring; see the spell 
\textit{snare}), nor mechanically complex ones, nor those that have been rendered 
safe or inactive.

The amount of information revealed depends on how long you study a particular area.

\textit{1st Round: }Presence or absence of hazards.

\textit{2nd Round: }Number of hazards and the location of each. If a hazard is 
outside your line of sight, then you discern its direction but not its exact location.

\textit{Each Additional Round: }The general type and trigger for one particular 
hazard closely examined by you.

Each round, you can turn to detect snares and pits in a new area. The spell can 
penetrate barriers, but 1 foot of stone, 1 inch of common metal, a thin sheet of 
lead, or 3 feet of wood or dirt blocks it.

\vspace{12pt}
Detect Thoughts

Divination [Mind-Affecting]

\textbf{Level:} Brd 2, Knowledge 2, Sor/Wiz 2

\textbf{Components:} V, S, F/DF

\textbf{Casting Time:} 1 standard action

\textbf{Range:} 60 ft.

\textbf{Area:} Cone-shaped emanation

\textbf{Duration:} Concentration, up to 1 min./level (D)

\textbf{Saving Throw: }Will negates; see text

\textbf{Spell Resistance:} No

You detect surface thoughts. The amount of information revealed depends on how 
long you study a particular area or subject.

\textit{1st Round: }Presence or absence of thoughts (from conscious creatures with 
Intelligence scores of 1 or higher).

\textit{2nd Round: }Number of thinking minds and the Intelligence score of each. 
If the highest Intelligence is 26 or higher (and at least 10 points higher than 
your own Intelligence score), you are stunned for 1 round and the spell ends. This 
spell does not let you determine the location of the thinking minds if you can't 
see the creatures whose thoughts you are detecting.

\textit{3rd Round: }Surface thoughts of any mind in the area. A target's Will save 
prevents you from reading its thoughts, and you must cast \textit{detect thoughts 
}again to have another chance. Creatures of animal intelligence (Int 1 or 2) have 
simple, instinctual thoughts that you can pick up.

Each round, you can turn to detect thoughts in a new area. The spell can penetrate 
barriers, but 1 foot of stone, 1 inch of common metal, a thin sheet of lead, or 
3 feet of wood or dirt blocks it.

\textit{Arcane Focus: }A copper piece.

\vspace{12pt}
Detect Undead

Divination

\textbf{Level:} Clr 1, Pal 1, Sor/Wiz 1

\textbf{Components:} V, S, M/DF

\textbf{Casting Time:} 1 standard action

\textbf{Range:} 60 ft.

\textbf{Area:} Cone-shaped emanation

\textbf{Duration:} Concentration, up to 1 minute/ level (D)

\textbf{Saving Throw:} None

\textbf{Spell Resistance:} No

You can detect the aura that surrounds undead creatures. The amount of information 
revealed depends on how long you study a particular area.

\textit{1st Round: }Presence or absence of undead auras.

\textit{2nd Round: }Number of undead auras in the area and the strength of the 
strongest undead aura present. If you are of good alignment, and the strongest 
undead aura's strength is overwhelming (see below), and the creature has HD of 
at least twice your character level, you are stunned for 1 round and the spell 
ends.

\textit{3rd Round: }The strength and location of each undead aura. If an aura is 
outside your line of sight, then you discern its direction but not its exact location.

\textit{Aura Strength: }The strength of an undead aura is determined by the HD 
of the undead creature, as given on the following table:

\begin{tabular}{|>{\raggedright}p{47pt}|>{\raggedright}p{58pt}|}
\hline
\subsection*{H\textbf{D}} & \subsection*{S\textbf{trength}}\tabularnewline
\hline
1 or lower & Faint\tabularnewline
\hline
2-4 & Moderate\tabularnewline
\hline
5-10 & Strong\tabularnewline
\hline
11 or higher & Overwhelming\tabularnewline
\hline
\end{tabular}

\textit{Lingering Aura: }An undead aura lingers after its original source is destroyed. 
If \textit{detect undead }is cast and directed at such a location, the spell indicates 
an aura strength of dim (even weaker than a faint aura). How long the aura lingers 
at this dim level depends on its original power:

\begin{tabular}{|>{\raggedright}p{74pt}|>{\raggedright}p{118pt}|}
\hline
\subsection*{O\textbf{riginal Strength}} & \subsection*{D\textbf{uration of Lingering 
Aura}}\tabularnewline
\hline
Faint & 1d6 rounds\tabularnewline
\hline
Moderate & 1d6 minutes\tabularnewline
\hline
Strong & 1d6x10 minutes\tabularnewline
\hline
Overwhelming & 1d6 days\tabularnewline
\hline
\end{tabular}

Each round, you can turn to detect undead in a new area. The spell can penetrate 
barriers, but 1 foot of stone, 1 inch of common metal, a thin sheet of lead, or 
3 feet of wood or dirt blocks it.

\textit{Arcane Material Component: }A bit of earth from a grave.

\vspace{12pt}
Dictum

Evocation [Lawful, Sonic]

\textbf{Level:} Clr 7, Law 7

\textbf{Components:} V

\textbf{Casting Time:} 1 standard action

\textbf{Range:} 40 ft.

\textbf{Area:} Nonlawful creatures in a 40-ft.-radius spread centered on you

\textbf{Duration:} Instantaneous

\textbf{Saving Throw:} None or Will negates; see text

\textbf{Spell Resistance:} Yes

Any nonlawful creature within the area of a \textit{dictum }spell suffers the following 
ill effects.

\begin{tabular}{|>{\raggedright}p{86pt}|>{\raggedright}p{141pt}|}
\hline
H\textbf{D} & E\textbf{ffect}\tabularnewline
\hline
Equal to caster level & Deafened\tabularnewline
\hline
Up to caster level -1 & S\textit{lowed, }deafened\tabularnewline
\hline
Up to caster level -5 & Paralyzed, \textit{slowed, }deafened\tabularnewline
\hline
Up to caster level -10 & Killed, paralyzed, \textit{slowed}, deafened\tabularnewline
\hline
\end{tabular}

The effects are cumulative and concurrent. No saving throw is allowed against these 
effects.

\textit{Deafened: }The creature is deafened for 1d4 rounds.

\textit{Slowed: }The creature is \textit{slowed, }as by the \textit{slow }spell, 
for 2d4 rounds.

\textit{Paralyzed: }The creature is paralyzed and helpless for 1d10 minutes.

\textit{Killed: }Living creatures die. Undead creatures are destroyed.

Furthermore, if you are on your home plane when you cast this spell, nonlawful 
extraplanar creatures within the area are instantly banished back to their home 
planes. Creatures so banished cannot return for at least 24 hours. This effect 
takes place regardless of whether the creatures hear the \textit{dictum. }The banishment 
effect allows a Will save (at a -4 penalty) to negate.

Creatures whose HD exceed your caster level are unaffected by \textit{dictum.}

\vspace{12pt}
Dimension Door

Conjuration (Teleportation)

\textbf{Level:} Brd 4, Sor/Wiz 4, Travel 4

\textbf{Components:} V

\textbf{Casting Time:} 1 standard action

\textbf{Range:} Long (400 ft. + 40 ft./level)

\textbf{Target:} You and touched objects or other touched willing creatures

\textbf{Duration:} Instantaneous

\textbf{Saving Throw:} None and Will negates (object)

\textbf{Spell Resistance:} No and Yes (object)

You instantly transfer yourself from your current location to any other spot within 
range. You always arrive at exactly the spot desired---whether by simply visualizing 
the area or by stating direction. After using this spell, you can't take any other 
actions until your next turn. You can bring along objects as long as their weight 
doesn't exceed your maximum load. You may also bring one additional willing Medium 
or smaller creature (carrying gear or objects up to its maximum load) or its equivalent 
per three caster levels. A Large creature counts as two Medium creatures, a Huge 
creature counts as two Large creatures, and so forth. All creatures to be transported 
must be in contact with one another, and at least one of those creatures must be 
in contact with you.

If you arrive in a place that is already occupied by a solid body, you and each 
creature traveling with you take 1d6 points of damage and are shunted to a random 
open space on a suitable surface within 100 feet of the intended location.

If there is no free space within 100 feet, you and each creature traveling with 
you take an additional 2d6 points of damage and are shunted to a free space within 
1,000 feet. If there is no free space within 1,000 feet, you and each creature 
travelling with you take an additional 4d6 points of damage and the spell simply 
fails.

\vspace{12pt}
Dimensional Anchor

Abjuration

\textbf{Level:} Clr 4, Sor/Wiz 4

\textbf{Components:} V, S

\textbf{Casting Time:} 1 standard action

\textbf{Range: }Medium (100 ft. + 10 ft./level)

\textbf{Effect:} Ray

\textbf{Duration:} 1 min./level

\textbf{Saving Throw:} None

\textbf{Spell Resistance:} Yes (object)

A green ray springs from your outstretched hand. You must make a ranged touch attack 
to hit the target. Any creature or object struck by the ray is covered with a shimmering 
emerald field that completely blocks extradimensional travel. Forms of movement 
barred by a \textit{dimensional anchor }include \textit{astral projection, blink, 
dimension door, ethereal jaunt, etherealness, gate, maze, plane shift, shadow walk, 
teleport, }and similar spell-like or psionic abilities. The spell also prevents 
the use of a \textit{gate }or \textit{teleportation circle }for the duration of 
the spell.

A \textit{dimensional anchor }does not interfere with the movement of creatures 
already in ethereal or astral form when the spell is cast, nor does it block extradimensional 
perception or attack forms. Also, \textit{dimensional anchor }does not prevent 
summoned creatures from disappearing at the end of a summoning spell.

\vspace{12pt}
Dimensional Lock

Abjuration

\textbf{Level:} Clr 8, Sor/Wiz 8

\textbf{Components:} V, S

\textbf{Casting Time:} 1 standard action

\textbf{Range: }Medium (100 ft. + 10 ft./level)

\textbf{Area:} 20-ft.-radius emanation centered on a point in space

\textbf{Duration:} One day/level

\textbf{Saving Throw:} None

\textbf{Spell Resistance:} Yes

You create a shimmering emerald barrier that completely blocks extradimensional 
travel. Forms of movement barred include \textit{astral projection, blink, dimension 
door, ethereal jaunt, etherealness, gate, maze, plane shift, shadow walk, teleport, 
}and similar spell-like or psionic abilities. Once \textit{dimensional lock }is 
in place, extradimensional travel into or out of the area is not possible.

A \textit{dimensional lock }does not interfere with the movement of creatures already 
in ethereal or astral form when the spell is cast, nor does it block extradimensional 
perception or attack forms. Also, the spell does not prevent summoned creatures 
from disappearing at the end of a summoning spell.

\vspace{12pt}
Diminish Plants

Transmutation

\textbf{Level:} Drd 3, Rgr 3

\textbf{Components:} V, S, DF

\textbf{Casting Time:} 1 standard action

\textbf{Range:} See text

\textbf{Target or Area:} See text

\textbf{Duration:} Instantaneous

\textbf{Saving Throw:} None

\textbf{Spell Resistance:} No

This spell has two versions.

\textit{Prune Growth: }This version causes normal vegetation within long range 
(400 feet + 40 feet per level) to shrink to about one-third of their normal size, 
becoming untangled and less bushy. The affected vegetation appears to have been 
carefully pruned and trimmed.

At your option, the area can be a 100- foot-radius circle, a 150-foot-radius semicircle, 
or a 200-foot-radius quarter-circle.

You may also designate portions of the area that are not affected.

\textit{Stunt Growth: }This version targets normal plants within a range of 1/2 
mile, reducing their potential productivity over the course of the following year 
to one third below normal.

\textit{Diminish plants }counters \textit{plant growth}.

This spell has no effect on plant creatures.

\vspace{12pt}
Discern Lies

Divination

\textbf{Level:} Clr 4, Pal 3

\textbf{Components:} V, S, DF

\textbf{Casting Time:} 1 standard action

\textbf{Range:} Close (25 ft. + 5 ft./2 levels)

\textbf{Targets:} One creature/level, no two of which can be more than 30 ft. apart

\textbf{Duration:} Concentration, up to 1 round/level

\textbf{Saving Throw: }Will negates

\textbf{Spell Resistance:} No

Each round, you concentrate on one subject, who must be within range. You know 
if the subject deliberately and knowingly speaks a lie by discerning disturbances 
in its aura caused by lying. The spell does not reveal the truth, uncover unintentional 
inaccuracies, or necessarily reveal evasions.

Each round, you may concentrate on a different subject.

\vspace{12pt}
Discern Location

Divination

\textbf{Level:} Clr 8, Knowledge 8, Sor/Wiz 8

\textbf{Components:} V, S, DF

\textbf{Casting Time:} 10 minutes

\textbf{Range:} Unlimited

\textbf{Target:} One creature or object

\textbf{Duration:} Instantaneous

\textbf{Saving Throw:} None

\textbf{Spell Resistance:} No

A \textit{discern location }spell is among the most powerful means of locating 
creatures or objects. Nothing short of a \textit{mind blank }spell or the direct 
intervention of a deity keeps you from learning the exact location of a single 
individual or object. \textit{Discern location }circumvents normal means of protection 
from scrying or location. The spell reveals the name of the creature or object's 
location (place, name, business name, building name, or the like), community, county 
(or similar political division), country, continent, and the plane of existence 
where the target lies.

To find a creature with the spell, you must have seen the creature or have some 
item that once belonged to it. To find an object, you must have touched it at least 
once.

\vspace{12pt}
Disguise Self

Illusion (Glamer)

\textbf{Level:} Brd 1, Sor/Wiz 1, Trickery 1

\textbf{Components:} V, S

\textbf{Casting Time:} 1 standard action

\textbf{Range:} Personal

\textbf{Target:} You

\textbf{Duration:} 10 min./level (D)

You make yourself---including clothing, armor, weapons, and equipment---look different. 
You can seem 1 foot shorter or taller, thin, fat, or in between. You cannot change 
your body type. Otherwise, the extent of the apparent change is up to you. You 
could add or obscure a minor feature or look like an entirely different person.

The spell does not provide the abilities or mannerisms of the chosen form, nor 
does it alter the perceived tactile (touch) or audible (sound) properties of you 
or your equipment. 

If you use this spell to create a disguise, you get a +10 bonus on the Disguise 
check.

A creature that interacts with the glamer gets a Will save to recognize it as an 
illusion.

\vspace{12pt}
Disintegrate

Transmutation

\textbf{Level:} Destruction 7, Sor/Wiz 6

\textbf{Components:} V, S, M/DF

\textbf{Casting Time:} 1 standard action

\textbf{Range: }Medium (100 ft. + 10 ft./level)

\textbf{Effect:} Ray

\textbf{Duration:} Instantaneous

\textbf{Saving Throw:} Fortitude partial (object)

\textbf{Spell Resistance:} Yes

A thin, green ray springs from your pointing finger. You must make a successful 
ranged touch attack to hit. Any creature struck by the ray takes 2d6 points of 
damage per caster level (to a maximum of 40d6). Any creature reduced to 0 or fewer 
hit points by this spell is entirely disintegrated, leaving behind only a trace 
of fine dust. A disintegrated creature's equipment is unaffected.

When used against an object, the ray simply disintegrates as much as one 10- foot 
cube of nonliving matter. Thus, the spell disintegrates only part of any very large 
object or structure targeted. The ray affects even objects constructed entirely 
of force, such as \textit{forceful hand }or a \textit{wall of force}, but not magical 
effects such as a \textit{globe of invulnerability }or an \textit{antimagic field.}

A creature or object that makes a successful Fortitude save is partially affected, 
taking only 5d6 points of damage. If this damage reduces the creature or object 
to 0 or fewer hit points, it is entirely disintegrated.

Only the first creature or object struck can be affected; that is, the ray affects 
only one target per casting.

\textit{Arcane Material Component: }A lodestone and a pinch of dust.

\vspace{12pt}
Dismissal

Abjuration

\textbf{Level:} Clr 4, Sor/Wiz 5

\textbf{Components:} V, S, DF

\textbf{Casting Time:} 1 standard action

\textbf{Range:} Close (25 ft. + 5 ft./2 levels)

\textbf{Target:} One extraplanar creature

\textbf{Duration:} Instantaneous

\textbf{Saving Throw: }Will negates; see text

\textbf{Spell Resistance:} Yes

This spell forces an extraplanar creature back to its proper plane if it fails 
a special Will save (DC = spell's save DC - creature's HD + your caster level). 
If the spell is successful, the creature is instantly whisked away, but there is 
a 20\% chance of actually sending the subject to a plane other than its own.

\vspace{12pt}
Dispel Chaos

Abjuration [Lawful]

\textbf{Level:} Clr 5, Law 5, Pal 4

This spell functions like \textit{dispel evil}, except that you are surrounded 
by constant, blue, lawful energy, and the spell affects chaotic creatures and spells 
rather than evil ones.

\vspace{12pt}
Dispel Evil

Abjuration [Good]

\textbf{Level:} Clr 5, Good 5, Pal 4

\textbf{Components:} V, S, DF

\textbf{Casting Time:} 1 standard action

\textbf{Range:} Touch

\textbf{Target or Targets:} You and a touched evil creature from another plane; 
or you and an enchantment or evil spell on a touched creature or object

\textbf{Duration:} 1 round/level or until discharged, whichever comes first

\textbf{Saving Throw:} See text

\textbf{Spell Resistance:} See text

Shimmering, white, holy energy surrounds you. This power has three effects.

First, you gain a +4 deflection bonus to AC against attacks by evil creatures.

Second, on making a successful melee touch attack against an evil creature from 
another plane, you can choose to drive that creature back to its home plane. The 
creature can negate the effects with a successful Will save (spell resistance applies). 
This use discharges and ends the spell.

Third, with a touch you can automatically dispel any one enchantment spell cast 
by an evil creature or any one evil spell. \textit{Exception: }Spells that can't 
be dispelled by \textit{dispel magic }also can't be dispelled by \textit{dispel 
evil}. Saving throws and spell resistance do not apply to this effect. This use 
discharges and ends the spell.

\vspace{12pt}
Dispel Good

Abjuration [Evil]

\textbf{Level:} Clr 5, Evil 5

This spell functions like \textit{dispel evil}, except that you are surrounded 
by dark, wavering, unholy energy, and the spell affects good creatures and spells 
rather than evil ones.

\vspace{12pt}
Dispel Law

Abjuration [Chaotic]

\textbf{Level:} Chaos 5, Clr 5

This spell functions like \textit{dispel evil}, except that you are surrounded 
by flickering, yellow, chaotic energy, and the spell affects lawful creatures and 
spells rather than evil ones.

\vspace{12pt}
Dispel Magic

Abjuration

\textbf{Level:} Brd 3, Clr 3, Drd 4, Magic 3, Pal 3, Sor/Wiz 3

\textbf{Components:} V, S

\textbf{Casting Time:} 1 standard action

\textbf{Range: }Medium (100 ft. + 10 ft./level)

\textbf{Target or Area:} One spellcaster, creature, or object; or 20-ft.-radius 
burst

\textbf{Duration:} Instantaneous

\textbf{Saving Throw:} None

\textbf{Spell Resistance:} No

You can use \textit{dispel magic }to end ongoing spells that have been cast on 
a creature or object, to temporarily suppress the magical abilities of a magic 
item, to end ongoing spells (or at least their effects) within an area, or to counter 
another spellcaster's spell. A dispelled spell ends as if its duration had expired. 
Some spells, as detailed in their descriptions, can't be defeated by \textit{dispel 
magic}. \textit{Dispel magic }can dispel (but not counter) spell-like effects just 
as it does spells.

\textit{Note: }The effect of a spell with an instantaneous duration can't be dispelled, 
because the magical effect is already over before the \textit{dispel magic }can 
take effect. 

You choose to use \textit{dispel magic }in one of three ways: a targeted dispel, 
an area dispel, or a counterspell:

\textit{Targeted Dispel: }One object, creature, or spell is the target of the \textit{dispel 
magic }spell. You make a dispel check (1d20 + your caster level, maximum +10) against 
the spell or against each ongoing spell currently in effect on the object or creature. 
The DC for this dispel check is 11 + the spell's caster level. If you succeed on 
a particular check, that spell is dispelled; if you fail, that spell remains in 
effect.

If you target an object or creature that is the effect of an ongoing spell (such 
as a monster summoned by \textit{monster summoning}), you make a dispel check to 
end the spell that conjured the object or creature.

If the object that you target is a magic item, you make a dispel check against 
the item's caster level. If you succeed, all the item's magical properties are 
suppressed for 1d4 rounds, after which the item recovers on its own. A suppressed 
item becomes nonmagical for the duration of the effect. An interdimensional interface 
(such as a \textit{bag of holding}) is temporarily closed. A magic item's physical 
properties are unchanged: A suppressed magic sword is still a sword (a masterwork 
sword, in fact). Artifacts and deities are unaffected by mortal magic such as this.

You automatically succeed on your dispel check against any spell that you cast 
yourself.

\textit{Area Dispel: }When \textit{dispel magic }is used in this way, the spell 
affects everything within a 20-foot radius.

For each creature within the area that is the subject of one or more spells, you 
make a dispel check against the spell with the highest caster level. If that check 
fails, you make dispel checks against progressively weaker spells until you dispel 
one spell (which discharges the \textit{dispel magic }spell so far as that target 
is concerned) or until you fail all your checks. The creature's magic items are 
not affected.

For each object within the area that is the target of one or more spells, you make 
dispel checks as with creatures. Magic items are not affected by an area dispel.

For each ongoing area or effect spell whose point of origin is within the area 
of the \textit{dispel magic }spell, you can make a dispel check to dispel the spell.

For each ongoing spell whose area overlaps that of the \textit{dispel magic }spell, 
you can make a dispel check to end the effect, but only within the overlapping 
area.

If an object or creature that is the effect of an ongoing spell (such as a monster 
summoned by \textit{monster summoning}) is in the area, you can make a dispel check 
to end the spell that conjured that object or creature (returning it whence it 
came) in addition to attempting to dispel spells targeting the creature or object.

You may choose to automatically succeed on dispel checks against any spell that 
you have cast.

\textit{Counterspell: }When \textit{dispel magic }is used in this way, the spell 
targets a spellcaster and is cast as a counterspell. Unlike a true counterspell, 
however, \textit{dispel magic }may not work; you must make a dispel check to counter 
the other spellcaster's spell.

\vspace{12pt}
Dispel Magic, Greater

Abjuration

\textbf{Level:} Brd 5, Clr 6, Drd 6, Sor/Wiz 6

This spell functions like \textit{dispel magic}, except that the maximum caster 
level on your dispel check is +20 instead of +10.

Additionally, \textit{greater dispel magic }has a chance to dispel any effect that 
\textit{remove curse }can remove, even if \textit{dispel magic }can't dispel that 
effect.

\vspace{12pt}
Displacement

Illusion (Glamer)

\textbf{Level:} Brd 3, Sor/Wiz 3

\textbf{Components:} V, M

\textbf{Casting Time:} 1 standard action

\textbf{Range:} Touch

\textbf{Target:} Creature touched

\textbf{Duration:} 1 round/level (D)

\textbf{Saving Throw: }Will negates (harmless)

\textbf{Spell Resistance:} Yes (harmless)

The subject of this spell appears to be about 2 feet away from its true location. 
The creature benefits from a 50\% miss chance as if it had total concealment. However, 
unlike actual total concealment, \textit{displacement }does not prevent enemies 
from targeting the creature normally. \textit{True seeing }reveals its true location.

\textit{Material Component: }A small strip of leather twisted into a loop.

\vspace{12pt}
Disrupt Undead

Necromancy

\textbf{Level:} Sor/Wiz 0

\textbf{Components:} V, S

\textbf{Casting Time:} 1 standard action

\textbf{Range:} Close (25 ft. + 5 ft./2 levels)

\textbf{Effect:} Ray

\textbf{Duration:} Instantaneous

\textbf{Saving Throw:} None

\textbf{Spell Resistance:} Yes

You direct a ray of positive energy. You must make a ranged touch attack to hit, 
and if the ray hits an undead creature, it deals 1d6 points of damage to it.

\vspace{12pt}
Disrupting Weapon

Transmutation

\textbf{Level:} Clr 5

\textbf{Components:} V, S

\textbf{Casting Time:} 1 standard action

\textbf{Range:} Touch

\textbf{Targets:} One melee weapon

\textbf{Duration:} 1 round/level

\textbf{Saving Throw: }Will negates (harmless, object); see text

\textbf{Spell Resistance:} Yes (harmless, object)

This spell makes a melee weapon deadly to undead. Any undead creature with HD equal 
to or less than your caster level must succeed on a Will save or be destroyed utterly 
if struck in combat with this weapon. Spell resistance does not apply against the 
destruction effect.

\vspace{12pt}
Divination

Divination

\textbf{Level:} Clr 4, Knowledge 4

\textbf{Components:} V, S, M

\textbf{Casting Time:} 10 minutes

\textbf{Range:} Personal

\textbf{Target:} You

\textbf{Duration:} Instantaneous 

Similar to \textit{augury }but more powerful, a \textit{divination }spell can provide 
you with a useful piece of advice in reply to a question concerning a specific 
goal, event, or activity that is to occur within one week. The advice can be as 
simple as a short phrase, or it might take the form of a cryptic rhyme or omen. 
If your party doesn't act on the information, the conditions may change so that 
the information is no longer useful. The base chance for a correct \textit{divination 
}is 70\% + 1\% per caster level, to a maximum of 90\%. If the dice roll fails, 
you know the spell failed, unless specific magic yielding false information is 
at work.

As with \textit{augury}, multiple \textit{divinations }about the same topic by 
the same caster use the same dice result as the first \textit{divination }spell 
and yield the same answer each time.

\textit{Material Component: }Incense and a sacrificial offering appropriate to 
your religion, together worth at least 25 gp.

\vspace{12pt}
Divine Favor

Evocation

\textbf{Level:} Clr 1, Pal 1

\textbf{Components:} V, S, DF

\textbf{Casting Time:} 1 standard action

\textbf{Range:} Personal

\textbf{Target:} You

\textbf{Duration:} 1 minute

Calling upon the strength and wisdom of a deity, you gain a +1 luck bonus on attack 
and weapon damage rolls for every three caster levels you have (at least +1, maximum 
+6). The bonus doesn't apply to spell damage.

\vspace{12pt}
Divine Power

Evocation

\textbf{Level:} Clr 4, War 4

\textbf{Components:} V, S, DF

\textbf{Casting Time:} 1 standard action

\textbf{Range:} Personal

\textbf{Target:} You

\textbf{Duration:} 1 round/level

Calling upon the divine power of your patron, you imbue yourself with strength 
and skill in combat. Your base attack bonus becomes equal to your character level 
(which may give you additional attacks), you gain a +6 enhancement bonus to Strength, 
and you gain 1 temporary hit point per caster level.

\vspace{12pt}
Dominate Animal

Enchantment (Compulsion) [Mind-Affecting]

\textbf{Level:} Animal 3, Drd 3

\textbf{Components:} V, S

\textbf{Casting Time:} 1 round

\textbf{Range:} Close (25 ft. + 5 ft./2 levels)

\textbf{Target:} One animal

\textbf{Duration:} 1 round/level

\textbf{Saving Throw: }Will negates

\textbf{Spell Resistance:} Yes

You can enchant an animal and direct it with simple commands such as ``Attack,'' 
``Run,'' and ``Fetch.'' Suicidal or self-destructive commands (including an order 
to attack a creature two or more size categories larger than the \textit{dominated 
}animal) are simply ignored.

\textit{Dominate animal }establishes a mental link between you and the subject 
creature. The animal can be directed by silent mental command as long as it remains 
in range. You need not see the creature to control it. You do not receive direct 
sensory input from the creature, but you know what it is experiencing. Because 
you are directing the animal with your own intelligence, it may be able to undertake 
actions normally beyond its own comprehension. You need not concentrate exclusively 
on controlling the creature unless you are trying to direct it to do something 
it normally couldn't do. Changing your instructions or giving a \textit{dominated 
}creature a new command is the equivalent of redirecting a spell, so it is a move 
action.

\vspace{12pt}
Dominate Monster

Enchantment (Compulsion) [Mind-Affecting]

\textbf{Level:} Sor/Wiz 9

\textbf{Target:} One creature

This spell functions like \textit{dominate person, }except that the spell is not 
restricted by creature type.

\vspace{12pt}
Dominate Person

Enchantment (Compulsion) [Mind-Affecting]

\textbf{Level:} Brd 4, Sor/Wiz 5

\textbf{Components:} V, S

\textbf{Casting Time:} 1 round

\textbf{Range:} Close (25 ft. + 5 ft./2 levels)

\textbf{Target:} One humanoid

\textbf{Duration:} One day/level

\textbf{Saving Throw: }Will negates

\textbf{Spell Resistance:} Yes

You can control the actions of any humanoid creature through a telepathic link 
that you establish with the subject's mind.

If you and the subject have a common language, you can generally force the subject 
to perform as you desire, within the limits of its abilities. If no common language 
exists, you can communicate only basic commands, such as ``Come here,'' ``Go there,'' 
``Fight,'' and ``Stand still.'' You know what the subject is experiencing, but 
you do not receive direct sensory input from it, nor can it communicate with you 
telepathically.

Once you have given a \textit{dominated }creature a command, it continues to attempt 
to carry out that command to the exclusion of all other activities except those 
necessary for day-to-day survival (such as sleeping, eating, and so forth). Because 
of this limited range of activity, a Sense Motive check against DC 15 (rather than 
DC 25) can determine that the subject's behavior is being influenced by an enchantment 
effect (see the Sense Motive skill description).

Changing your instructions or giving a \textit{dominated }creature a new command 
is the equivalent of redirecting a spell, so it is a move action.

By concentrating fully on the spell (a standard action), you can receive full sensory 
input as interpreted by the mind of the subject, though it still can't communicate 
with you. You can't actually see through the subject's eyes, so it's not as good 
as being there yourself, but you still get a good idea of what's going on.

Subjects resist this control, and any subject forced to take actions against its 
nature receives a new saving throw with a +2 bonus. Obviously self-destructive 
orders are not carried out. Once control is established, the range at which it 
can be exercised is unlimited, as long as you and the subject are on the same plane. 
You need not see the subject to control it.

If you don't spend at least 1 round concentrating on the spell each day, the subject 
receives a new saving throw to throw off the domination.

\textit{Protection from evil }or a similar spell can prevent you from exercising 
control or using the telepathic link while the subject is so warded, but such an 
effect neither prevents the establishment of domination nor dispels it.

\vspace{12pt}
Doom

Necromancy [Fear, Mind-Affecting]

\textbf{Level:} Clr 1

\textbf{Components:} V, S, DF

\textbf{Casting Time:} 1 standard action

\textbf{Range: }Medium (100 ft. + 10 ft./level)

\textbf{Target:} One living creature

\textbf{Duration:} 1 min./level

\textbf{Saving Throw: }Will negates

\textbf{Spell Resistance:} Yes

This spell fills a single subject with a feeling of horrible dread that causes 
it to become shaken.

\vspace{12pt}
Dream

Illusion (Phantasm) [Mind-Affecting]

\textbf{Level:} Brd 5, Sor/Wiz 5

\textbf{Components:} V, S

\textbf{Casting Time:} 1 minute

\textbf{Range:} Unlimited

\textbf{Target:} One living creature touched

\textbf{Duration:} See text

\textbf{Saving Throw:} None

\textbf{Spell Resistance:} Yes

You, or a messenger touched by you, sends a phantasmal message to others in the 
form of a dream. At the beginning of the spell, you must name the recipient or 
identify him or her by some title that leaves no doubt as to identity. The messenger 
then enters a trance, appears in the intended recipient's dream, and delivers the 
message. The message can be of any length, and the recipient remembers it perfectly 
upon waking. The communication is one-way. The recipient cannot ask questions or 
offer information, nor can the messenger gain any information by observing the 
dreams of the recipient.

Once the message is delivered, the messenger's mind returns instantly to its body. 
The duration of the spell is the time required for the messenger to enter the recipient's 
dream and deliver the message.

If the recipient is awake when the spell begins, the messenger can choose to wake 
up (ending the spell) or remain in the trance. The messenger can remain in the 
trance until the recipient goes to sleep, then enter the recipient's dream and 
deliver the message as normal. A messenger that is disturbed during the trance 
comes awake, ending the spell.

Creatures who don't sleep (such as elves, but not half-elves) or don't dream cannot 
be contacted by this spell.

The messenger is unaware of its own surroundings or of the activities around it 
while in the trance. It is defenseless both physically and mentally (always fails 
any saving throw) while in the trance.

\vspace{12pt}
Eagle's Splendor

Transmutation

\textbf{Level:} Brd 2, Clr 2, Pal 2, Sor/Wiz 2

\textbf{Components:} V, S, M/DF

\textbf{Casting Time:} 1 standard action

\textbf{Range:} Touch

\textbf{Target:} Creature touched

\textbf{Duration:} 1 min./level

\textbf{Saving Throw: }Will negates (harmless)

\textbf{Spell Resistance:} Yes

The transmuted creature becomes more poised, articulate, and personally forceful. 
The spell grants a +4 enhancement bonus to Charisma, adding the usual benefits 
to Charisma-based skill checks and other uses of the Charisma modifier. Sorcerers 
and bards (and other spellcasters who rely on Charisma) affected by this spell 
do not gain any additional bonus spells for the increased Charisma, but the save 
DCs for spells they cast while under this spell's effect do increase.

\textit{Arcane Material Component: }A few feathers or a pinch of droppings from 
an eagle.

\vspace{12pt}
Eagle's Splendor, Mass

Transmutation

\textbf{Level:} Brd 6, Clr 6, Sor/Wiz 6

\textbf{Range:} Close (25 ft. + 5 ft./2 levels)

\textbf{Target:} One creature/level, no two of which can be more than 30 ft. apart

This spell functions like \textit{eagle's splendor}, except that it affects multiple 
creatures.

\vspace{12pt}
Earthquake

Evocation [Earth]

\textbf{Level:} Clr 8, Destruction 8, Drd 8, Earth 7

\textbf{Components:} V, S, DF

\textbf{Casting Time:} 1 standard action

\textbf{Range:} Long (400 ft. + 40 ft./level)

\textbf{Area:} 80-ft.-radius spread (S)

\textbf{Duration:} 1 round

\textbf{Saving Throw:} See text

\textbf{Spell Resistance:} No

When you cast \textit{earthquake, }an intense but highly localized tremor rips 
the ground. The shock knocks creatures down, collapses structures, opens cracks 
in the ground, and more. The effect lasts for 1 round, during which time creatures 
on the ground can't move or attack. A spellcaster on the ground must make a Concentration 
check (DC 20 + spell level) or lose any spell he or she tries to cast. The earthquake 
affects all terrain, vegetation, structures, and creatures in the area. The specific 
effect of an \textit{earthquake }spell depends on the nature of the terrain where 
it is cast.

\textit{Cave, Cavern, or Tunnel: }The spell collapses the roof, dealing 8d6 points 
of bludgeoning damage to any creature caught under the cave-in (Reflex DC 15 half 
) and pinning that creature beneath the rubble (see below). An \textit{earthquake 
}cast on the roof of a very large cavern could also endanger those outside the 
actual area but below the falling debris.

\textit{Cliffs: Earthquake }causes a cliff to crumble, creating a landslide that 
travels horizontally as far as it fell vertically. Any creature in the path takes 
8d6 points of bludgeoning damage (Reflex DC 15 half ) and is pinned beneath the 
rubble (see below).

\textit{Open Ground: }Each creature standing in the area must make a DC 15 Reflex 
save or fall down. Fissures open in the earth, and every creature on the ground 
has a 25\% chance to fall into one (Reflex DC 20 to avoid a fissure). At the end 
of the spell, all fissures grind shut, killing any creatures still trapped within.

\textit{Structure: }Any structure standing on open ground takes 100 points of damage, 
enough to collapse a typical wooden or masonry building, but not a structure built 
of stone or reinforced masonry. Hardness does not reduce this damage, nor is it 
halved as damage dealt to objects normally is. Any creature caught inside a collapsing 
structure takes 8d6 points of bludgeoning damage (Reflex DC 15 half ) and is pinned 
beneath the rubble (see below).

\textit{River, Lake, or Marsh: }Fissures open underneath the water, draining away 
the water from that area and forming muddy ground. Soggy marsh or swampland becomes 
quicksand for the duration of the spell, sucking down creatures and structures. 
Each creature in the area must make a DC 15 Reflex save or sink down in the mud 
and quicksand. At the end of the spell, the rest of the body of water rushes in 
to replace the drained water, possibly drowning those caught in the mud.

\textit{Pinned beneath Rubble: }Any creature pinned beneath rubble takes 1d6 points 
of nonlethal damage per minute while pinned. If a pinned character falls unconscious, 
he or she must make a DC 15 Constitution check or take 1d6 points of lethal damage 
each minute thereafter until freed or dead.

\vspace{12pt}
Elemental Swarm

Conjuration (Summoning) [see text]

\textbf{Level:} Air 9, Drd 9, Earth 9, Fire 9, Water 9

\textbf{Components:} V, S

\textbf{Casting Time:} 10 minutes

\textbf{Range: }Medium (100 ft. + 10 ft./level)

\textbf{Effect:} Two or more summoned creatures, no two of which can be more than 
30 ft. apart

\textbf{Duration:} 10 min./level (D)

\textbf{Saving Throw:} None

\textbf{Spell Resistance:} No

This spell opens a portal to an Elemental Plane and summons elementals from it. 
A druid can choose the plane (Air, Earth, Fire, or Water); a cleric opens a portal 
to the plane matching his domain.

When the spell is complete, 2d4 Large elementals appear. Ten minutes later, 1d4 
Huge elementals appear. Ten minutes after that, one greater elemental appears. 
Each elemental has maximum hit points per HD. Once these creatures appear, they 
serve you for the duration of the spell.

The elementals obey you explicitly and never attack you, even if someone else manages 
to gain control over them. You do not need to concentrate to maintain control over 
the elementals. You can dismiss them singly or in groups at any time.

When you use a summoning spell to summon an air, earth, fire, or water creature, 
it is a spell of that type.

\vspace{12pt}
Endure Elements

Abjuration

\textbf{Level:} Clr 1, Drd 1, Pal 1, Rgr 1, Sor/Wiz 1, Sun 1

\textbf{Components:} V, S

\textbf{Casting Time:} 1 standard action

\textbf{Range:} Touch

\textbf{Target:} Creature touched

\textbf{Duration:} 24 hours

\textbf{Saving Throw: }Will negates (harmless)

\textbf{Spell Resistance:} Yes (harmless)

A creature protected by \textit{endure elements }suffers no harm from being in 
a hot or cold environment. It can exist comfortably in conditions between -50 and 
140 degrees Fahrenheit without having to make Fortitude saves). The creature's 
equipment is likewise protected.

\textit{Endure elements }doesn't provide any protection from fire or cold damage, 
nor does it protect against other environmental hazards such as smoke, lack of 
air, and so forth.

\vspace{12pt}
Energy Drain

Necromancy

\textbf{Level:} Clr 9, Sor/Wiz 9

\textbf{Saving Throw:} Fortitude partial; see text for \textit{enervation}

This spell functions like \textit{enervation, }except that the creature struck 
gains 2d4 negative levels, and the negative levels last longer.

There is no saving throw to avoid gaining the negative levels, but 24 hours after 
gaining them, the subject must make a Fortitude saving throw (DC = \textit{energy 
drain }spell's save DC) for each negative level. If the save succeeds, that negative 
level is removed. If it fails, the negative level also goes away, but one of the 
subject's character levels is permanently drained.

An undead creature struck by the ray gains 2d4x5 temporary hit points for 1 hour.

\vspace{12pt}
Enervation

Necromancy

\textbf{Level:} Sor/Wiz 4

\textbf{Components:} V, S

\textbf{Casting Time:} 1 standard action

\textbf{Range:} Close (25 ft. + 5 ft./2 levels)

\textbf{Effect:} Ray of negative energy

\textbf{Duration:} Instantaneous

\textbf{Saving Throw:} None

\textbf{Spell Resistance:} Yes

You point your finger and utter the incantation, releasing a black ray of crackling 
negative energy that suppresses the life force of any living creature it strikes. 
You must make a ranged touch attack to hit. If the attack succeeds, the subject 
gains 1d4 negative levels.

If the subject has at least as many negative levels as HD, it dies. Each negative 
level gives a creature a -1 penalty on attack rolls, saving throws, skill checks, 
ability checks, and effective level (for determining the power, duration, DC, and 
other details of spells or special abilities).

Additionally, a spellcaster loses one spell or spell slot from his or her highest 
available level. Negative levels stack.

Assuming the subject survives, it regains lost levels after a number of hours equal 
to your caster level (maximum 15 hours). Usually, negative levels have a chance 
of permanently draining the victim's levels, but the negative levels from \textit{enervation 
}don't last long enough to do so.

An undead creature struck by the ray gains 1d4x5 temporary hit points for 1 hour.

\vspace{12pt}
Enlarge Person

Transmutation

\textbf{Level:} Sor/Wiz 1, Strength 1

\textbf{Components:} V, S, M

\textbf{Casting Time:} 1 round

\textbf{Range:} Close (25 ft. + 5 ft./2 levels)

\textbf{Target:} One humanoid creature

\textbf{Duration:} 1 min./level (D)

\textbf{Saving Throw:} Fortitude negates

\textbf{Spell Resistance:} Yes

This spell causes instant growth of a humanoid creature, doubling its height and 
multiplying its weight by 8. This increase changes the creature's size category 
to the next larger one. The target gains a +2 size bonus to Strength, a -2 size 
penalty to Dexterity (to a minimum of 1), and a -1 penalty on attack rolls and 
AC due to its increased size.

A humanoid creature whose size increases to Large has a space of 10 feet and a 
natural reach of 10 feet. This spell does not change the target's speed.

If insufficient room is available for the desired growth, the creature attains 
the maximum possible size and may make a Strength check (using its increased Strength) 
to burst any enclosures in the process. If it fails, it is constrained without 
harm by the materials enclosing it--- the spell cannot be used to crush a creature 
by increasing its size.

All equipment worn or carried by a creature is similarly enlarged by the spell. 
Melee and projectile weapons affected by this spell deal more damage. Other magical 
properties are not affected by this spell. Any \textit{enlarged }item that leaves 
an \textit{enlarged }creature's possession (including a projectile or thrown weapon) 
instantly returns to its normal size. This means that thrown weapons deal their 
normal damage, and projectiles deal damage based on the size of the weapon that 
fired them. Magical properties of \textit{enlarged }items are not increased by 
this spell.

Multiple magical effects that increase size do not stack,.

\textit{Enlarge person }counters and dispels \textit{reduce person}.

\textit{Enlarge person }can be made permanent with a \textit{permanency }spell.

\textit{Material Component: }A pinch of powdered iron.

\vspace{12pt}
Enlarge Person, Mass

Transmutation

\textbf{Level:} Sor/Wiz 4

\textbf{Target:} One humanoid creature/level, no two of which can be more than 
30 ft. apart

This spell functions like \textit{enlarge person}, except that it affects multiple 
creatures.

\vspace{12pt}
Entangle

Transmutation

\textbf{Level:} Drd 1, Plant 1, Rgr 1

\textbf{Components:} V, S, DF

\textbf{Casting Time:} 1 standard action

\textbf{Range:} Long (400 ft. + 40 ft./level)

\textbf{Area:} Plants in a 40-ft.-radius spread

\textbf{Duration:} 1 min./level (D)

\textbf{Saving Throw:} Reflex partial; see text

\textbf{Spell Resistance:} No

Grasses, weeds, bushes, and even trees wrap, twist, and entwine about creatures 
in the area or those that enter the area, holding them fast and causing them to 
become entangled. The creature can break free and move half its normal speed by 
using a full-round action to make a DC 20 Strength check or a DC 20 Escape Artist 
check. A creature that succeeds on a Reflex save is not entangled but can still 
move at only half speed through the area. Each round on your turn, the plants once 
again attempt to entangle all creatures that have avoided or escaped entanglement.

\textit{Note: }The effects of the spell may be altered somewhat, based on the nature 
of the entangling plants.

\vspace{12pt}
Enthrall

Enchantment (Charm) [Language Dependent, Mind-Affecting, Sonic]

\textbf{Level:} Brd 2, Clr 2

\textbf{Components:} V, S

\textbf{Casting Time:} 1 round

\textbf{Range: }Medium (100 ft. + 10 ft./level)

\textbf{Targets:} Any number of creatures

\textbf{Duration:} 1 hour or less

\textbf{Saving Throw: }Will negates; see text

\textbf{Spell Resistance:} Yes

If you have the attention of a group of creatures, you can use this spell to hold 
them spellbound. To cast the spell, you must speak or sing without interruption 
for 1 full round. Thereafter, those affected give you their undivided attention, 
ignoring their surroundings. They are considered to have an attitude of friendly 
while under the effect of the spell. Any potentially affected creature of a race 
or religion unfriendly to yours gets a +4 bonus on the saving throw.

A creature with 4 or more HD or with a Wisdom score of 16 or higher remains aware 
of its surroundings and has an attitude of indifferent. It gains a new saving throw 
if it witnesses actions that it opposes.

The effect lasts as long as you speak or sing, to a maximum of 1 hour. Those \textit{enthralled 
}by your words take no action while you speak or sing and for 1d3 rounds thereafter 
while they discuss the topic or performance. Those entering the area during the 
performance must also successfully save or become \textit{enthralled. }The speech 
ends (but the 1d3-round delay still applies) if you lose concentration or do anything 
other than speak or sing.

If those not \textit{enthralled }have unfriendly or hostile attitudes toward you, 
they can collectively make a Charisma check to try to end the spell by jeering 
and heckling. For this check, use the Charisma bonus of the creature with the highest 
Charisma in the group; others may make Charisma checks to assist. The heckling 
ends the spell if this check result beats your Charisma check result. Only one 
such challenge is allowed per use of the spell.

If any member of the audience is attacked or subjected to some other overtly hostile 
act, the spell ends and the previously \textit{enthralled }members become immediately 
unfriendly toward you. Each creature with 4 or more HD or with a Wisdom score of 
16 or higher becomes hostile.

\vspace{12pt}
Entropic Shield

Abjuration

\textbf{Level:} Clr 1, Luck 1

\textbf{Components:} V, S

\textbf{Casting Time:} 1 standard action

\textbf{Range:} Personal

\textbf{Target:} You

\textbf{Duration:} 1 min./level (D)

A magical field appears around you, glowing with a chaotic blast of multicolored 
hues. This field deflects incoming arrows, rays, and other ranged attacks. Each 
ranged attack directed at you for which the attacker must make an attack roll has 
a 20\% miss chance (similar to the effects of concealment). Other attacks that 
simply work at a distance are not affected.

\vspace{12pt}
Erase

Transmutation

\textbf{Level:} Brd 1, Sor/Wiz 1

\textbf{Components:} V, S

\textbf{Casting Time:} 1 standard action

\textbf{Range:} Close (25 ft. + 5 ft./2 levels)

\textbf{Target:} One scroll or two pages

\textbf{Duration:} Instantaneous

\textbf{Saving Throw:} See text

\textbf{Spell Resistance:} No

\textit{Erase }removes writings of either magical or mundane nature from a scroll 
or from one or two pages of paper, parchment, or similar surfaces. With this spell, 
you can remove \textit{explosive runes, }a \textit{glyph of warding, }a \textit{sepia 
snake sigil, }or an \textit{arcane mark}, but not \textit{illusory script }or a 
\textit{symbol }spell. Nonmagical writing is automatically erased if you touch 
it and no one else is holding it. Otherwise, the chance of erasing nonmagical writing 
is 90\%.

Magic writing must be touched to be erased, and you also must succeed on a caster 
level check (1d20 + caster level) against DC 15. (A natural 1 or 2 is always a 
failure on this check.) If you fail to erase \textit{explosive runes}, a \textit{glyph 
of warding}, or a \textit{sepia snake sigil}, you accidentally activate that writing 
instead.

\vspace{12pt}
Ethereal Jaunt

Transmutation

\textbf{Level:} Clr 7, Sor/Wiz 7

\textbf{Components:} V, S

\textbf{Casting Time:} 1 standard action

\textbf{Range:} Personal

\textbf{Target:} You

\textbf{Duration:} 1 round/level (D)

You become ethereal, along with your equipment. For the duration of the spell, 
you are in a place called the Ethereal Plane, which overlaps the normal, physical, 
Material Plane. When the spell expires, you return to material existence.

An ethereal creature is invisible, insubstantial, and capable of moving in any 
direction, even up or down, albeit at half normal speed. As an insubstantial creature, 
you can move through solid objects, including living creatures. An ethereal creature 
can see and hear on the Material Plane, but everything looks gray and ephemeral. 
Sight and hearing onto the Material Plane are limited to 60 feet.

Force effects and abjurations affect an ethereal creature normally. Their effects 
extend onto the Ethereal Plane from the Material Plane, but not vice versa. An 
ethereal creature can't attack material creatures, and spells you cast while ethereal 
affect only other ethereal things. Certain material creatures or objects have attacks 
or effects that work on the Ethereal Plane.

Treat other ethereal creatures and ethereal objects as if they were material. 

If you end the spell and become material while inside a material object (such as 
a solid wall), you are shunted off to the nearest open space and take 1d6 points 
of damage per 5 feet that you so travel.

\vspace{12pt}
Etherealness

Transmutation

\textbf{Level:} Clr 9, Sor/Wiz 9

\textbf{Range:} Touch; see text

\textbf{Targets:} You and one other touched creature per three levels

\textbf{Duration:} 1 min./level (D)

\textbf{Spell Resistance:} Yes

This spell functions like \textit{ethereal jaunt, }except that you and other willing 
creatures joined by linked hands (along with their equipment) become ethereal. 
Besides yourself, you can bring one creature per three caster levels to the Ethereal 
Plane. Once ethereal, the subjects need not stay together.

When the spell expires, all affected creatures on the Ethereal Plane return to 
material existence.

\vspace{12pt}
Expeditious Retreat

Transmutation

\textbf{Level:} Brd 1, Sor/Wiz 1

\textbf{Components:} V, S

\textbf{Casting Time:} 1 standard action

\textbf{Range:} Personal

\textbf{Target:} You

\textbf{Duration:} 1 min./level (D)

This spell increases your base land speed by 30 feet. (This adjustment is treated 
as an enhancement bonus.) There is no effect on other modes of movement, such as 
burrow, climb, fly, or swim. As with any effect that increases your speed, this 
spell affects your jumping distance (see the Jump skill).

\vspace{12pt}
Explosive Runes

Abjuration [Force]

\textbf{Level:} Sor/Wiz 3

\textbf{Components:} V, S

\textbf{Casting Time:} 1 standard action

\textbf{Range:} Touch

\textbf{Target:} One touched object weighing no more than 10 lb.

\textbf{Duration:} Permanent until discharged (D)

\textbf{Saving Throw:} See text

\textbf{Spell Resistance:} Yes

You trace these mystic runes upon a book, map, scroll, or similar object bearing 
written information. The \textit{runes }detonate when read, dealing 6d6 points 
of force damage. Anyone next to the \textit{runes }(close enough to read them) 
takes the full damage with no saving throw; any other creature within 10 feet of 
the \textit{runes }is entitled to a Reflex save for half damage. The object on 
which the \textit{runes }were written also takes full damage (no saving throw).

You and any characters you specifically instruct can read the protected writing 
without triggering the \textit{runes. }Likewise, you can remove the \textit{runes 
}whenever desired. Another creature can remove them with a successful \textit{dispel 
magic }or \textit{erase }spell, but attempting to dispel or erase the \textit{runes 
}and failing to do so triggers the explosion.

\textit{Note: }Magic traps such as \textit{explosive runes }are hard to detect 
and disable. A rogue (only) can use the Search skill to find the \textit{runes 
}and Disable Device to thwart them. The DC in each case is 25 + spell level, or 
28 for \textit{explosive runes}.

\vspace{12pt}
Eyebite

Necromancy [Evil]

\textbf{Level:} Brd 6, Sor/Wiz 6

\textbf{Components:} V, S

\textbf{Casting Time:} 1 standard action

\textbf{Range:} Close (25 ft. + 5 ft./2 levels)

\textbf{Target:} One living creature

\textbf{Duration:} 1 round per three levels; see text

\textbf{Saving Throw:} Fortitude negates

\textbf{Spell Resistance:} Yes

Each round, you may target a single living creature, striking it with waves of 
evil power. Depending on the target's HD, this attack has as many as three effects.

\begin{tabular}{|>{\raggedright}p{42pt}|>{\raggedright}p{119pt}|}
\hline
\subsection*{H\textbf{D}} & \subsection*{E\textbf{ffect}}\tabularnewline
\hline
10 or more & Sickened\tabularnewline
\hline
5-9 & Panicked, sickened\tabularnewline
\hline
4 or less & Comatose, panicked, sickened\tabularnewline
\hline
\end{tabular}

The effects are cumulative and concurrent.

\textit{Sickened: }Sudden pain and fever sweeps over the subject's body. A sickened 
creature takes a -2 penalty on attack rolls, weapon damage rolls, saving throws, 
skill checks, and ability checks. A creature affected by this spell remains sickened 
for 10 minutes per caster level. The effects cannot be negated by a \textit{remove 
disease }or \textit{heal }spell, but a \textit{remove curse }is effective.

\textit{Panicked: }The subject becomes panicked for 1d4 rounds. Even after the 
panic ends, the creature remains shaken for 10 minutes per caster level, and it 
automatically becomes panicked again if it comes within sight of you during that 
time. This is a fear effect.

\textit{Comatose: }The subject falls into a catatonic coma for 10 minutes per caster 
level. During this time, it cannot be awakened by any means short of dispelling 
the effect. This is not a \textit{sleep }effect, and thus elves are not immune 
to it.

The spell lasts for 1 round per three caster levels. You must spend a move action 
each round after the first to target a foe.

\newpage

\end{document}
