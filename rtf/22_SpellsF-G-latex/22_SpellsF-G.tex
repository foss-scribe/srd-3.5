%&pdfLaTeX
% !TEX encoding = UTF-8 Unicode
\documentclass{article}
\usepackage{ifxetex}
\ifxetex
\usepackage{fontspec}
\setmainfont[Mapping=tex-text]{STIXGeneral}
\else
\usepackage[T1]{fontenc}
\usepackage[utf8]{inputenc}
\fi
\usepackage{textcomp}

\usepackage{array}
\usepackage{amssymb}
\usepackage{fancyhdr}
\renewcommand{\headrulewidth}{0pt}
\renewcommand{\footrulewidth}{0pt}

\begin{document}

This material is Open Game Content, and is licensed for public use under the terms 
of the Open Game License v1.0a.

{\LARGE{}SPELLS (F-G)}

\vspace{12pt}
Fabricate

Transmutation

\textbf{Level:} Sor/Wiz 5

\textbf{Components:} V, S, M

\textbf{Casting Time:} See text

\textbf{Range:} Close (25 ft. + 5 ft./2 levels)

\textbf{Target:} Up to 10 cu. ft./level; see text

\textbf{Duration:} Instantaneous

\textbf{Saving Throw:} None

\textbf{Spell Resistance:} No

You convert material of one sort into a product that is of the same material. Creatures 
or magic items cannot be created or transmuted by the \textit{fabricate }spell. 
The quality of items made by this spell is commensurate with the quality of material 
used as the basis for the new fabrication. If you work with a mineral, the target 
is reduced to 1 cubic foot per level instead of 10 cubic feet.

You must make an appropriate Craft check to fabricate articles requiring a high 
degree of craftsmanship.

Casting requires 1 round per 10 cubic feet (or 1 cubic foot) of material to be 
affected by the spell.

\textit{Material Component: }The original material, which costs the same amount 
as the raw materials required to craft the item to be created.

\vspace{12pt}
Faerie Fire

Evocation [Light]

\textbf{Level:} Drd 1

\textbf{Components:} V, S, DF

\textbf{Casting Time:} 1 standard action

\textbf{Range:} Long (400 ft. + 40 ft./level)

\textbf{Area:} Creatures and objects within a 5-ft.-radius burst

\textbf{Duration:} 1 min./level (D)

\textbf{Saving Throw:} None

\textbf{Spell Resistance:} Yes

A pale glow surrounds and outlines the subjects. Outlined subjects shed light as 
candles. Outlined creatures do not benefit from the concealment normally provided 
by darkness (though a 2nd-level or higher magical \textit{darkness }effect functions 
normally), \textit{blur}, displacement, invisibility, or similar effects. The light 
is too dim to have any special effect on undead or dark-dwelling creatures vulnerable 
to light. The \textit{faerie fire }can be blue, green, or violet, according to 
your choice at the time of casting. The \textit{faerie fire }does not cause any 
harm to the objects or creatures thus outlined.

\vspace{12pt}
False Life

Necromancy

\textbf{Level:} Sor/Wiz 2

\textbf{Components:} V, S, M

\textbf{Casting Time:} 1 standard action

\textbf{Range:} Personal

\textbf{Target:} You

\textbf{Duration:} 1 hour/level or until discharged; see text

You harness the power of unlife to grant yourself a limited ability to avoid death. 
While this spell is in effect, you gain temporary hit points equal to 1d10 +1 per 
caster level (maximum +10).

\textit{Material Component: }A small amount of alcohol or distilled spirits, which 
you use to trace certain sigils on your body during casting. These sigils cannot 
be seen once the alcohol or spirits evaporate.

\vspace{12pt}
False Vision

Illusion (Glamer)

\textbf{Level:} Brd 5, Sor/Wiz 5, Trickery 5

\textbf{Components:} V, S, M

\textbf{Casting Time:} 1 standard action

\textbf{Range:} Touch

\textbf{Area:} 40-ft.-radius emanation

\textbf{Duration:} 1 hour/level (D)

\textbf{Saving Throw:} None

\textbf{Spell Resistance:} No

Any divination (scrying) spell used to view anything within the area of this spell 
instead receives a false image (as the \textit{major image }spell), as defined 
by you at the time of casting. As long as the duration lasts, you can concentrate 
to change the image as desired. While you aren't concentrating, the image remains 
static.

\textit{Arcane Material Component: }The ground dust of a piece of jade worth at 
least 250 gp, which is sprinkled into the air when the spell is cast.

\vspace{12pt}
Fear

Necromancy [Fear, Mind-Affecting]

\textbf{Level:} Brd 3, Sor/Wiz 4

\textbf{Components:} V, S, M

\textbf{Casting Time:} 1 standard action

\textbf{Range:} 30 ft.

\textbf{Area:} Cone-shaped burst

\textbf{Duration:} 1 round/level or 1 round; see text

\textbf{Saving Throw: }Will partial

\textbf{Spell Resistance:} Yes

An invisible cone of terror causes each living creature in the area to become panicked 
unless it succeeds on a Will save. If cornered, a panicked creature begins cowering. 
If the Will save succeeds, the creature is shaken for 1 round.

\textit{Material Component: }Either the heart of a hen or a white feather.

\vspace{12pt}
Feather Fall

Transmutation

\textbf{Level:} Brd 1, Sor/Wiz 1

\textbf{Components:} V

\textbf{Casting Time:} 1 free action

\textbf{Range:} Close (25 ft. + 5 ft./2 levels)

\textbf{Targets:} One Medium or smaller freefalling object or creature/level, no 
two of which may be more than 20 ft. apart

\textbf{Duration:} Until landing or 1 round/level

\textbf{Saving Throw: }Will negates (harmless) or Will negates (object)

\textbf{Spell Resistance:} Yes (object)

The affected creatures or objects fall slowly. \textit{Feather fall }instantly 
changes the rate at which the targets fall to a mere 60 feet per round (equivalent 
to the end of a fall from a few feet), and the subjects take no damage upon landing 
while the spell is in effect. However, when the spell duration expires, a normal 
rate of falling resumes.

The spell affects one or more Medium or smaller creatures (including gear and carried 
objects up to each creature's maximum load) or objects, or the equivalent in larger 
creatures: A Large creature or object counts as two Medium creatures or objects, 
a Huge creature or object counts as two Large creatures or objects, and so forth.

You can cast this spell with an instant utterance, quickly enough to save yourself 
if you unexpectedly fall. Casting the spell is a free action, like casting a quickened 
spell, and it counts toward the normal limit of one quickened spell per round. 
You may even cast this spell when it isn't your turn.

This spell has no special effect on ranged weapons unless they are falling quite 
a distance. If the spell is cast on a falling item the object does half normal 
damage based on its weight, with no bonus for the height of the drop.

\textit{Feather fall }works only upon free-falling objects. It does not affect 
a sword blow or a charging or flying creature.

\vspace{12pt}
Feeblemind

Enchantment (Compulsion) [Mind-Affecting]

\textbf{Level:} Sor/Wiz 5

\textbf{Components:} V, S, M

\textbf{Casting Time:} 1 standard action

\textbf{Range: }Medium (100 ft. + 10 ft./level)

\textbf{Target:} One creature

\textbf{Duration:} Instantaneous

\textbf{Saving Throw: }Will negates; see text

\textbf{Spell Resistance:} Yes

If the target creature fails a Will saving throw, its Intelligence and Charisma 
scores each drop to 1. The affected creature is unable to use Intelligence- or 
Charisma-based skills, cast spells, understand language, or communicate coherently. 
Still, it knows who its friends are and can follow them and even protect them. 
The subject remains in this state until a \textit{heal}, \textit{limited wish}, 
\textit{miracle}, or \textit{wish }spell is used to cancel the effect of the \textit{feeblemind}. 
A creature that can cast arcane spells, such as a sorcerer or a wizard, takes a 
-4 penalty on its saving throw.

\textit{Material Component: }A handful of clay, crystal, glass, or mineral spheres.

\vspace{12pt}
Find the Path

Divination

\textbf{Level:} Brd 6, Clr 6, Drd 6, Knowledge 6, Travel 6

\textbf{Components:} V, S, F

\textbf{Casting Time:} 3 rounds

\textbf{Range:} Personal or touch

\textbf{Target:} You or creature touched

\textbf{Duration:} 10 min./level

\textbf{Saving Throw:} None or Will negates (harmless)

\textbf{Spell Resistance:} No or Yes (harmless)

The recipient of this spell can find the shortest, most direct physical route to 
a specified destination, be it the way into or out of a locale. The locale can 
be outdoors, underground, or even inside a \textit{maze }spell. \textit{Find the 
path }works with respect to locations, not objects or creatures at a locale. The 
location must be on the same plane as you are at the time of casting.

The spell enables the subject to sense the correct direction that will eventually 
lead it to its destination, indicating at appropriate times the exact path to follow 
or physical actions to take. For example, the spell enables the subject to sense 
trip wires or the proper word to bypass a \textit{glyph of warding. }The spell 
ends when the destination is reached or the duration expires, whichever comes first. 
\textit{Find the path }can be used to remove the subject and its companions from 
the effect of a \textit{maze }spell in a single round.

This divination is keyed to the recipient, not its companions, and its effect does 
not predict or allow for the actions of creatures (including guardians).

\textit{Focus: }A set of divination counters of the sort you favor.

\vspace{12pt}
Find Traps

Divination

\textbf{Level:} Clr 2

\textbf{Components:} V, S

\textbf{Casting Time:} 1 standard action

\textbf{Range:} Personal

\textbf{Target:} You

\textbf{Duration:} 1 min./level

You gain intuitive insight into the workings of traps. You can use the Search skill 
to detect traps just as a rogue can. In addition, you gain an insight bonus equal 
to one-half your caster level (maximum +10) on Search checks made to find traps 
while the spell is in effect.

Note that \textit{find traps }grants no ability to disable the traps that you may 
find.

\vspace{12pt}
Finger of Death

Necromancy [Death]

\textbf{Level:} Drd 8, Sor/Wiz 7

\textbf{Components:} V, S

\textbf{Casting Time:} 1 standard action

\textbf{Range:} Close (25 ft. + 5 ft./2 levels)

\textbf{Target:} One living creature

\textbf{Duration:} Instantaneous

\textbf{Saving Throw:} Fortitude partial

\textbf{Spell Resistance:} Yes

You can slay any one living creature within range. The target is entitled to a 
Fortitude saving throw to survive the attack. If the save is successful, the creature 
instead takes 3d6 points of damage +1 point per caster level (maximum +25).

The subject might die from damage even if it succeeds on its saving throw.

\vspace{12pt}
Fire Seeds

Conjuration (Creation) [Fire]

\textbf{Level:} Drd 6, Fire 6, Sun 6

\textbf{Components:} V, S, M

\textbf{Casting Time:} 1 standard action

\textbf{Range:} Touch

\textbf{Targets:} Up to four touched acorns or up to eight touched holly berries

\textbf{Duration:} 10 min./level or until used

\textbf{Saving Throw:} None or Reflex half; see text

\textbf{Spell Resistance:} No

Depending on the version of \textit{fire seeds }you choose, you turn acorns into 
splash weapons that you or another character can throw, or you turn holly berries 
into bombs that you can detonate on command.

\textit{Acorn Grenades: }As many as four acorns turn into special splash weapons 
that can be hurled as far as 100 feet. A ranged touch attack roll is required to 
strike the intended target. Together, the acorns are capable of dealing 1d6 points 
of fire damage per caster level (maximum 20d6), divided up among the acorns as 
you wish.

Each acorn explodes upon striking any hard surface. In addition to its regular 
fire damage, it deals 1 point of splash damage per die, and it ignites any combustible 
materials within 10 feet. A creature within this area that makes a successful Reflex 
saving throw takes only half damage; a creature struck directly is not allowed 
a saving throw.

\textit{Holly Berry Bombs: }You turn as many as eight holly berries into special 
bombs. The holly berries are usually placed by hand, since they are too light to 
make effective thrown weapons (they can be tossed only 5 feet). If you are within 
200 feet and speak a word of command, each berry instantly bursts into flame, causing 
1d8 points of fire damage +1 point per caster level to every creature in a 5-foot 
radius burst and igniting any combustible materials within 5 feet. A creature in 
the area that makes a successful Reflex saving throw takes only half damage.

\textit{Material Component: }The acorns or holly berries.

\vspace{12pt}
Fire Shield

Evocation [Fire or Cold]

\textbf{Level:} Fire 5, Sor/Wiz 4, Sun 4

\textbf{Components:} V, S, M/DF

\textbf{Casting Time:} 1 standard action

\textbf{Range:} Personal

\textbf{Target:} You

\textbf{Duration:} 1 round/level (D)

This spell wreathes you in flame and causes damage to each creature that attacks 
you in melee. The flames also protect you from either cold-based or fire-based 
attacks (your choice).

Any creature striking you with its body or a handheld weapon deals normal damage, 
but at the same time the attacker takes 1d6 points of damage +1 point per caster 
level (maximum +15). This damage is either cold damage (if the \textit{shield }protects 
against fire-based attacks) or fire damage (if the \textit{shield }protects against 
cold-based attacks). If the attacker has spell resistance, it applies to this effect. 
Creatures wielding weapons with exceptional reach are not subject to this damage 
if they attack you.

When casting this spell, you appear to immolate yourself, but the flames are thin 
and wispy, giving off light equal to only half the illumination of a normal torch 
(10 feet). The color of the flames is determined randomly (50\% chance of either 
color)---blue or green if the \textit{chill shield }is cast, violet or blue if 
the \textit{warm shield }is employed. The special powers of each version are as 
follows.

\textit{Warm Shield: }The flames are warm to the touch. You take only half damage 
from cold-based attacks. If such an attack allows a Reflex save for half damage, 
you take no damage on a successful save.

\textit{Chill Shield: }The flames are cool to the touch. You take only half damage 
from fire-based attacks. If such an attack allows a Reflex save for half damage, 
you take no damage on a successful save.

\textit{Arcane Material Component: }A bit of phosphorus for the \textit{warm shield; 
}a live firefly or glowworm or the tail portions of four dead ones for the \textit{chill 
shield}.

\vspace{12pt}
Fire Storm

Evocation [Fire]

\textbf{Level:} Clr 8, Drd 7, Fire 7

\textbf{Components:} V, S

\textbf{Casting Time:} 1 round

\textbf{Range: }Medium (100 ft. + 10 ft./level)

\textbf{Area:} Two 10-ft. cubes per level (S)

\textbf{Duration:} Instantaneous

\textbf{Saving Throw:} Reflex half

\textbf{Spell Resistance:} Yes

When a \textit{fire storm }spell is cast, the whole area is shot through with sheets 
of roaring flame. The raging flames do not harm natural vegetation, ground cover, 
and any plant creatures in the area that you wish to exclude from damage. Any other 
creature within the area takes 1d6 points of fire damage per caster level (maximum 
20d6).

\vspace{12pt}
Fire Trap

Abjuration [Fire]

\textbf{Level:} Drd 2, Sor/Wiz 4

\textbf{Components:} V, S, M

\textbf{Casting Time:} 10 minutes

\textbf{Range:} Touch

\textbf{Target:} Object touched

\textbf{Duration:} Permanent until discharged (D)

\textbf{Saving Throw:} Reflex half; see text

\textbf{Spell Resistance:} Yes

\textit{Fire trap }creates a fiery explosion when an intruder opens the item that 
the trap protects. A \textit{fire trap }can ward any object that can be opened 
and closed.

When casting \textit{fire trap, }you select a point on the object as the spell's 
center. When someone other than you opens the object, a fiery explosion fills the 
area within a 5-foot radius around the spell's center. The flames deal 1d4 points 
of fire damage +1 point per caster level (maximum +20). The item protected by the 
trap is not harmed by this explosion.

A \textit{fire trapped }item cannot have a second closure or warding spell placed 
on it.

A \textit{knock }spell does not bypass a \textit{fire trap}.  An unsuccessful \textit{dispel 
magic }spell does not detonate the spell.

Underwater, this ward deals half damage and creates a large cloud of steam.

You can use the \textit{fire trapped }object without discharging it, as can any 
individual to whom the object was specifically attuned when cast. Attuning a \textit{fire 
trapped }object to an individual usually involves setting a password that you can 
share with friends.

\textit{Note: }Magic traps such as \textit{fire trap }are hard to detect and disable. 
A rogue (only) can use the Search skill to find a \textit{fire trap }and Disable 
Device to thwart it. The DC in each case is 25 + spell level (DC 27 for a druid's 
\textit{fire trap }or DC 29 for the arcane version).

\textit{Material Component: }A half-pound of gold dust (cost 25 gp) sprinkled on 
the warded object.

\vspace{12pt}
Fireball

Evocation [Fire]

\textbf{Level:} Sor/Wiz 3

\textbf{Components:} V, S, M

\textbf{Casting Time:} 1 standard action

\textbf{Range:} Long (400 ft. + 40 ft./level)

\textbf{Area:} 20-ft.-radius spread

\textbf{Duration:} Instantaneous

\textbf{Saving Throw:} Reflex half

\textbf{Spell Resistance:} Yes

A \textit{fireball }spell is an explosion of flame that detonates with a low roar 
and deals 1d6 points of fire damage per caster level (maximum 10d6) to every creature 
within the area. Unattended objects also take this damage. The explosion creates 
almost no pressure.

You point your finger and determine the range (distance and height) at which the 
\textit{fireball }is to burst. A glowing, pea-sized bead streaks from the pointing 
digit and, unless it impacts upon a material body or solid barrier prior to attaining 
the prescribed range, blossoms into the \textit{fireball }at that point. (An early 
impact results in an early detonation.) If you attempt to send the bead through 
a narrow passage, such as through an arrow slit, you must ``hit'' the opening with 
a ranged touch attack, or else the bead strikes the barrier and detonates prematurely.

The \textit{fireball }sets fire to combustibles and damages objects in the area. 
It can melt metals with low melting points, such as lead, gold, copper, silver, 
and bronze. If the damage caused to an interposing barrier shatters or breaks through 
it, the \textit{fireball }may continue beyond the barrier if the area permits; 
otherwise it stops at the barrier just as any other spell effect does.

\textit{Material Component: }A tiny ball of bat guano and sulfur.

\vspace{12pt}
Flame Arrow

Transmutation [Fire]

\textbf{Level:} Sor/Wiz 3

\textbf{Components:} V, S, M

\textbf{Casting Time:} 1 standard action

\textbf{Range:} Close (25 ft. + 5 ft./2 levels)

\textbf{Target:} Fifty projectiles, all of which must be in contact with each other 
at the time of casting

\textbf{Duration:} 10 min./level

\textbf{Saving Throw:} None

\textbf{Spell Resistance:} No

You turn ammunition (such as arrows, bolts, shuriken, and stones) into fiery projectiles. 
Each piece of ammunition deals an extra 1d6 points of fire damage to any target 
it hits. A flaming projectile can easily ignite a flammable object or structure, 
but it won't ignite a creature it strikes.

\textit{Material Component: }A drop of oil and a small piece of flint.

\vspace{12pt}
Flame Blade

Evocation [Fire]

\textbf{Level:} Drd 2

\textbf{Components:} V, S, DF

\textbf{Casting Time:} 1 standard action

\textbf{Range:} 0 ft.

\textbf{Effect:} Sword-like beam

\textbf{Duration:} 1 min./level (D)

\textbf{Saving Throw:} None

\textbf{Spell Resistance:} Yes

A 3-foot-long, blazing beam of red-hot fire springs forth from your hand. You wield 
this bladelike beam as if it were a scimitar. Attacks with the \textit{flame blade 
}are melee touch attacks. The blade deals 1d8 points of fire damage +1 point per 
two caster levels (maximum +10). Since the blade is immaterial, your Strength modifier 
does not apply to the damage. A \textit{flame blade }can ignite combustible materials 
such as parchment, straw, dry sticks, and cloth.

The spell does not function underwater.

\vspace{12pt}
Flame Strike

Evocation [Fire]

\textbf{Level:} Clr 5, Drd 4, Sun 5, War 5

\textbf{Components:} V, S, DF

\textbf{Casting Time:} 1 standard action

\textbf{Range: }Medium (100 ft. + 10 ft./level)

\textbf{Area:} Cylinder (10-ft. radius, 40 ft. high)

\textbf{Duration:} Instantaneous

\textbf{Saving Throw:} Reflex half

\textbf{Spell Resistance:} Yes

A \textit{flame strike }produces a vertical column of divine fire roaring downward. 
The spell deals 1d6 points of damage per caster level (maximum 15d6). Half the 
damage is fire damage, but the other half results directly from divine power and 
is therefore not subject to being reduced by resistance to fire-based attacks.

\vspace{12pt}
Flaming Sphere

Evocation [Fire]

\textbf{Level:} Drd 2, Sor/Wiz 2

\textbf{Components:} V, S, M/DF

\textbf{Casting Time:} 1 standard action

\textbf{Range: }Medium (100 ft. + 10 ft./level)

\textbf{Effect:} 5-ft.-diameter sphere

\textbf{Duration:} 1 round/level

\textbf{Saving Throw:} Reflex negates

\textbf{Spell Resistance:} Yes

A burning globe of fire rolls in whichever direction you point and burns those 
it strikes. It moves 30 feet per round. As part of this movement, it can ascend 
or jump up to 30 feet to strike a target. If it enters a space with a creature, 
it stops moving for the round and deals 2d6 points of fire damage to that creature, 
though a successful Reflex save negates that damage. A \textit{flaming sphere }rolls 
over barriers less than 4 feet tall. It ignites flammable substances it touches 
and illuminates the same area as a torch would.

The sphere moves as long as you actively direct it (a move action for you); otherwise, 
it merely stays at rest and burns. It can be extinguished by any means that would 
put out a normal fire of its size. The surface of the sphere has a spongy, yielding 
consistency and so does not cause damage except by its flame. It cannot push aside 
unwilling creatures or batter down large obstacles. A \textit{flaming sphere }winks 
out if it exceeds the spell's range.

\textit{Arcane Material Component: }A bit of tallow, a pinch of brimstone, and 
a dusting of powdered iron.

\vspace{12pt}
Flare

Evocation [Light]

\textbf{Level:} Brd 0, Drd 0, Sor/Wiz 0

\textbf{Components:} V

\textbf{Casting Time:} 1 standard action

\textbf{Range:} Close (25 ft. + 5 ft./2 levels)

\textbf{Effect:} Burst of light

\textbf{Duration:} Instantaneous

\textbf{Saving Throw:} Fortitude negates

\textbf{Spell Resistance:} Yes

This cantrip creates a burst of light. If you cause the light to burst directly 
in front of a single creature, that creature is dazzled for 1 minute unless it 
makes a successful Fortitude save. Sightless creatures, as well as creatures already 
dazzled, are not affected by \textit{flare}.

\vspace{12pt}
Flesh to Stone

Transmutation

\textbf{Level:} Sor/Wiz 6

\textbf{Components:} V, S, M

\textbf{Casting Time:} 1 standard action

\textbf{Range: }Medium (100 ft. + 10 ft./level)

\textbf{Target:} One creature

\textbf{Duration:} Instantaneous

\textbf{Saving Throw:} Fortitude negates

\textbf{Spell Resistance:} Yes

The subject, along with all its carried gear, turns into a mindless, inert statue. 
If the statue resulting from this spell is broken or damaged, the subject (if ever 
returned to its original state) has similar damage or deformities. The creature 
is not dead, but it does not seem to be alive either when viewed with spells such 
as \textit{deathwatch}.

Only creatures made of flesh are affected by this spell.

\textit{Material Component: }Lime, water, and earth.

\vspace{12pt}
Fly

Transmutation

\textbf{Level:} Sor/Wiz 3, Travel 3

\textbf{Components:} V, S, F/DF

\textbf{Casting Time:} 1 standard action

\textbf{Range:} Touch

\textbf{Target:} Creature touched

\textbf{Duration:} 1 min./level

\textbf{Saving Throw:} Will negates (harmless)

\textbf{Spell Resistance:} Yes (harmless)

The subject can fly at a speed of 60 feet (or 40 feet if it wears medium or heavy 
armor, or if it carries a medium or heavy load). It can ascend at half speed and 
descend at double speed, and its maneuverability is good. Using a \textit{fly }spell 
requires only as much concentration as walking, so the subject can attack or cast 
spells normally. The subject of a \textit{fly }spell can charge but not run, and 
it cannot carry aloft more weight than its maximum load, plus any armor it wears.

Should the spell duration expire while the subject is still aloft, the magic fails 
slowly. The subject floats downward 60 feet per round for 1d6 rounds. If it reaches 
the ground in that amount of time, it lands safely. If not, it falls the rest of 
the distance, taking 1d6 points of damage per 10 feet of fall. Since dispelling 
a spell effectively ends it, the subject also descends in this way if the \textit{fly 
}spell is dispelled, but not if it is negated by an \textit{antimagic field}.

\textit{Arcane Focus: }A wing feather from any bird.

\vspace{12pt}
Floating Disk

Evocation [Force]

\textbf{Level:} Sor/Wiz 1

\textbf{Components:} V, S, M

\textbf{Casting Time:} 1 standard action

\textbf{Range:} Close (25 ft. + 5 ft./2 levels)

\textbf{Effect:} 3-ft.-diameter disk of force

\textbf{Duration:} 1 hour/level

\textbf{Saving Throw:} None

\textbf{Spell Resistance:} No

You create a slightly concave, circular plane of force that follows you about and 
carries loads for you. The disk is 3 feet in diameter and 1 inch deep at its center. 
It can hold 100 pounds of weight per caster level. (If used to transport a liquid, 
its capacity is 2 gallons.) The disk floats approximately 3 feet above the ground 
at all times and remains level. It floats along horizontally within spell range 
and will accompany you at a rate of no more than your normal speed each round. 
If not otherwise directed, it maintains a constant interval of 5 feet between itself 
and you. The disk winks out of existence when the spell duration expires. The disk 
also winks out if you move beyond range or try to take the disk more than 3 feet 
away from the surface beneath it. When the disk winks out, whatever it was supporting 
falls to the surface beneath it.

\textit{Material Component: }A drop of mercury.

\vspace{12pt}
Fog Cloud

Conjuration (Creation)

\textbf{Level:} Drd 2, Sor/Wiz 2, Water 2

\textbf{Components:} V, S

\textbf{Casting Time:} 1 standard action

\textbf{Range: }Medium (100 ft. + 10 ft. level)

\textbf{Effect:} Fog spreads in 20-ft. radius, 20 ft. high

\textbf{Duration:} 10 min./level

\textbf{Saving Throw:} None

\textbf{Spell Resistance:} No

A bank of fog billows out from the point you designate. The fog obscures all sight, 
including darkvision, beyond 5 feet. A creature within 5 feet has concealment (attacks 
have a 20\% miss chance). Creatures farther away have total concealment (50\% miss 
chance, and the attacker can't use sight to locate the target).

A moderate wind (11+ mph) disperses the fog in 4 rounds; a strong wind (21+ mph) 
disperses the fog in 1 round.

The spell does not function underwater.

\vspace{12pt}
Forbiddance

Abjuration

\textbf{Level:} Clr 6

\textbf{Components:} V, S, M, DF

\textbf{Casting Time:} 6 rounds

\textbf{Range: }Medium (100 ft. + 10 ft./level)

\textbf{Area:} 60-ft. cube/level (S)

\textbf{Duration:} Permanent

\textbf{Saving Throw:} See text

\textbf{Spell Resistance:} Yes

\textit{Forbiddance }seals an area against all planar travel into or within it. 
This includes all teleportation spells (such as \textit{dimension door }and \textit{teleport), 
plane shifting, }astral travel, ethereal travel, and all summoning spells. Such 
effects simply fail automatically.

In addition, it damages entering creatures whose alignments are different from 
yours. The effect on those attempting to enter the warded area is based on their 
alignment relative to yours (see below). A creature inside the area when the spell 
is cast takes no damage unless it exits the area and attempts to reenter, at which 
time it is affected as normal.

\textit{Alignments identical: }No effect. The creature may enter the area freely 
(although not by planar travel).

\textit{Alignments different with respect to either law/chaos or good/evil: }The 
creature takes 6d6 points of damage. A successful Will save halves the damage, 
and spell resistance applies.

\textit{Alignments different with respect to both law/chaos and good/evil: }The 
creature takes 12d6 points of damage. A successful Will save halves the damage, 
and spell resistance applies.

At your option, the abjuration can include a password, in which case creatures 
of alignments different from yours can avoid the damage by speaking the password 
as they enter the area. You must select this option (and the password) at the time 
of casting.

\textit{Dispel magic }does not dispel a \textit{forbiddance }effect unless the 
dispeller's level is at least as high as your caster level.

You can't have multiple overlapping \textit{forbiddance }effects. In such a case, 
the more recent effect stops at the boundary of the older effect.

\textit{Material Component: }A sprinkling of holy water and rare incenses worth 
at least 1,500 gp, plus 1,500 gp per 60-foot cube. If a password is desired, this 
requires the burning of additional rare incenses worth at least 1,000 gp, plus 
1,000 gp per 60-foot cube.

\vspace{12pt}
Forcecage

Evocation [Force]

\textbf{Level:} Sor/Wiz 7

\textbf{Components:} V, S, M

\textbf{Casting Time:} 1 standard action

\textbf{Range:} Close (25 ft. + 5 ft./2 levels)

\textbf{Area:} Barred cage (20-ft. cube) or windowless cell (10-ft. cube)

\textbf{Duration:} 2 hours/level (D)

\textbf{Saving Throw:} None

\textbf{Spell Resistance:} No

This powerful spell brings into being an immobile, invisible cubical prison composed 
of either bars of force or solid walls of force (your choice).

Creatures within the area are caught and contained unless they are too big to fit 
inside, in which case the spell automatically fails. Teleportation and other forms 
of astral travel provide a means of escape, but the force walls or bars extend 
into the Ethereal Plane, blocking ethereal travel.

Like a \textit{wall of force }spell, a \textit{forcecage }resists \textit{dispel 
magic, }but it is vulnerable to a \textit{disintegrate }spell, and it can be destroyed 
by a \textit{sphere of annihilation }or a \textit{rod of cancellation.}

\textit{Barred Cage: }This version of the spell produces a 20-foot cube made of 
bands of force (similar to a \textit{wall of force }spell) for bars. The bands 
are a half-inch wide, with half-inch gaps between them. Any creature capable of 
passing through such a small space can escape; others are confined. You can't attack 
a creature in a barred cage with a weapon unless the weapon can fit between the 
gaps. Even against such weapons (including arrows and similar ranged attacks), 
a creature in the barred cage has cover. All spells and breath weapons can pass 
through the gaps in the bars.

\textit{Windowless Cell: }This version of the spell produces a 10-foot cube with 
no way in and no way out. Solid walls of force form its six sides.

\textit{Material Component: }Ruby dust worth 1,500 gp, which is tossed into the 
air and disappears when you cast the spell.

\vspace{12pt}
Forceful Hand

Evocation [Force]

\textbf{Level:} Sor/Wiz 6

\textbf{Components:} V, S, F

This spell functions like \textit{interposing hand, }except that the \textit{forceful 
hand }pursues and pushes away the opponent that you designate. Treat this attack 
as a bull rush with a +14 bonus on the Strength check (+8 for Strength 27, +4 for 
being Large, and a +2 bonus for charging, which it always gets). The hand always 
moves with the opponent to push that target back the full distance allowed, and 
it has no speed limit. Directing the spell to a new target is a move action.

A very strong creature could not push the hand out of its way because the latter 
would instantly reposition itself between the creature and you, but an opponent 
could push the hand up against you by successfully bull rushing it.

\textit{Focus: }A sturdy glove made of leather or heavy cloth.

\vspace{12pt}
Foresight

Divination

\textbf{Level:} Drd 9, Knowledge 9, Sor/Wiz 9

\textbf{Components:} V, S, M/DF

\textbf{Casting Time:} 1 standard action

\textbf{Range:} Personal or touch

\textbf{Target:} See text

\textbf{Duration:} 10 min./level

\textbf{Saving Throw:} None or Will negates (harmless)

\textbf{Spell Resistance:} No or Yes (harmless)

This spell grants you a powerful sixth sense in relation to yourself or another. 
Once \textit{foresight }is cast, you receive instantaneous warnings of impending 
danger or harm to the subject of the spell. You are never surprised or flat-footed. 
In addition, the spell gives you a general idea of what action you might take to 
best protect yourself and gives you a +2 insight bonus to AC and Reflex saves. 
This insight bonus is lost whenever you would lose a Dexterity bonus to AC.

When another creature is the subject of the spell, you receive warnings about that 
creature. You must communicate what you learn to the other creature for the warning 
to be useful, and the creature can be caught unprepared in the absence of such 
a warning. Shouting a warning, yanking a person back, and even telepathically communicating 
(via an appropriate spell) can all be accomplished before some danger befalls the 
subject, provided you act on the warning without delay. The subject, however, does 
not gain the insight bonus to AC and Reflex saves.

\textit{Arcane Material Component: }A hummingbird's feather.

\vspace{12pt}
Fox's Cunning

Transmutation

\textbf{Level:} Brd 2, Sor/Wiz 2

\textbf{Components:} V, S, M/DF

\textbf{Casting Time:} 1 standard action

\textbf{Range:} Touch

\textbf{Target:} Creature touched

\textbf{Duration:} 1 min./level

\textbf{Saving Throw: }Will negates (harmless)

\textbf{Spell Resistance:} Yes

The transmuted creature becomes smarter. The spell grants a +4 enhancement bonus 
to Intelligence, adding the usual benefits to Intelligence-based skill checks and 
other uses of the Intelligence modifier. Wizards (and other spellcasters who rely 
on Intelligence) affected by this spell do not gain any additional bonus spells 
for the increased Intelligence, but the save DCs for spells they cast while under 
this spell's effect do increase. This spell doesn't grant extra skill points.

\textit{Arcane Material Component: }A few hairs, or a pinch of dung, from a fox.

\vspace{12pt}
Fox's Cunning, Mass

Transmutation

\textbf{Level:} Brd 6, Sor/Wiz 6

\textbf{Range:} Close (25 ft. + 5 ft./2 levels)

\textbf{Target:} One creature/level, no two of which can be more than 30 ft. apart

This spell functions like \textit{fox's cunning}, except that it affects multiple 
creatures.

\vspace{12pt}
Freedom

Abjuration

\textbf{Level:} Sor/Wiz 9

\textbf{Components:} V, S

\textbf{Casting Time:} 1 standard action

\textbf{Range:} Close (25 ft. + 5 ft./2 levels) or see text

\textbf{Target:} One creature

\textbf{Duration:} Instantaneous

\textbf{Saving Throw: }Will negates (harmless)

\textbf{Spell Resistance:} Yes

The subject is freed from spells and effects that restrict its movement, including 
\textit{binding}, \textit{entangle}, grappling, \textit{imprisonment}, \textit{maze}, 
paralysis, \textit{petrification}, pinning, \textit{sleep}, \textit{slow}, stunning, 
\textit{temporal stasis}, and \textit{web}. To free a creature from \textit{imprisonment 
}or \textit{maze, }you must know its name and background, and you must cast this 
spell at the spot where it was entombed or banished into the \textit{maze}.

\vspace{12pt}
Freedom of Movement

Abjuration

\textbf{Level:} Brd 4, Clr 4, Drd 4, Luck 4, Rgr 4

\textbf{Components:} V, S, M, DF

\textbf{Casting Time:} 1 standard action

\textbf{Range:} Personal or touch

\textbf{Target:} You or creature touched

\textbf{Duration:} 10 min./level

\textbf{Saving Throw: }Will negates (harmless)

\textbf{Spell Resistance:} Yes (harmless)

This spell enables you or a creature you touch to move and attack normally for 
the duration of the spell, even under the influence of magic that usually impedes 
movement, such as paralysis, \textit{solid fog}, \textit{slow}, and \textit{web}. 
The subject automatically succeeds on any grapple check made to resist a grapple 
attempt, as well as on grapple checks or Escape Artist checks made to escape a 
grapple or a pin.

The spell also allows the subject to move and attack normally while underwater, 
even with slashing weapons such as axes and swords or with bludgeoning weapons 
such as flails, hammers, and maces, provided that the weapon is wielded in the 
hand rather than hurled. The \textit{freedom of movement }spell does not, however, 
allow water breathing.

\textit{Material Component: }A leather thong, bound around the arm or a similar 
appendage.

\vspace{12pt}
Freezing Sphere

Evocation [Cold]

\textbf{Level:} Sor/Wiz 6

\textbf{Components:} V, S, F

\textbf{Casting Time:} 1 standard action

\textbf{Range:} Long (400 ft. + 40 ft./level)

\textbf{Target, Effect, or Area:} See text

\textbf{Duration:} Instantaneous or 1 round/level; see text

\textbf{Saving Throw:} Reflex half; see text

\textbf{Spell Resistance:} Yes

\textit{Freezing sphere }creates a frigid globe of cold energy that streaks from 
your fingertips to the location you select, where it explodes in a 10-foot-radius 
burst, dealing 1d6 points of cold damage per caster level (maximum 15d6) to each 
creature in the area. An elemental (water) creature instead takes 1d8 points of 
cold damage per caster level (maximum 15d8).

If the \textit{freezing sphere }strikes a body of water or a liquid that is principally 
water (not including water-based creatures), it freezes the liquid to a depth of 
6 inches over an area equal to 100 square feet (a 10- foot square) per caster level 
(maximum 1,500 square feet). This ice lasts for 1 round per caster level. Creatures 
that were swimming on the surface of frozen water become trapped in the ice. Attempting 
to break free is a full-round action. A trapped creature must make a DC 25 Strength 
check or a DC 25 Escape Artist check to do so.

You can refrain from firing the globe after completing the spell, if you wish. 
Treat this as a touch spell for which you are holding the charge. You can hold 
the charge for as long as 1 round per level, at the end of which time the \textit{freezing 
sphere }bursts centered on you (and you receive no saving throw to resist its effect). 
Firing the globe in a later round is a standard action.

\textit{Focus: }A small crystal sphere.

\vspace{12pt}
Gaseous Form

Transmutation

\textbf{Level:} Air 3, Brd 3, Sor/Wiz 3

\textbf{Components:} S, M/DF

\textbf{Casting Time:} 1 standard action

\textbf{Range:} Touch

\textbf{Target: }Willing corporeal creature touched

\textbf{Duration:} 2 min./level (D)

\textbf{Saving Throw:} None

\textbf{Spell Resistance:} No

The subject and all its gear become insubstantial, misty, and translucent. Its 
material armor (including natural armor) becomes worthless, though its size, Dexterity, 
deflection bonuses, and armor bonuses from force effects still apply. The subject 
gains damage reduction 10/magic and becomes immune to poison and critical hits. 
It can't attack or cast spells with verbal, somatic, material, or focus components 
while in gaseous form. (This does not rule out the use of certain spells that the 
subject may have prepared using the feats Silent Spell, Still Spell, and Eschew 
Materials.) The subject also loses supernatural abilities while in gaseous form. 
If it has a touch spell ready to use, that spell is discharged harmlessly when 
the \textit{gaseous form }spell takes effect.

A gaseous creature can't run, but it can fly at a speed of 10 feet (maneuverability 
perfect). It can pass through small holes or narrow openings, even mere cracks, 
with all it was wearing or holding in its hands, as long as the spell persists. 
The creature is subject to the effects of wind, and it can't enter water or other 
liquid. It also can't manipulate objects or activate items, even those carried 
along with its gaseous form. Continuously active items remain active, though in 
some cases their effects may be moot.

\textit{Arcane Material Component: }A bit of gauze and a wisp of smoke.

\vspace{12pt}
Gate

Conjuration (Creation or Calling)

\textbf{Level:} Clr 9, Sor/Wiz 9

\textbf{Components:} V, S, XP; see text

\textbf{Casting Time:} 1 standard action

\textbf{Range: }Medium (100 ft. + 10 ft./level)

\textbf{Effect:} See text

\textbf{Duration:} Instantaneous or concentration (up to 1 round/level); see text

\textbf{Saving Throw:} None

\textbf{Spell Resistance:} No

Casting a \textit{gate }spell has two effects. First, it creates an interdimensional 
connection between your plane of existence and a plane you specify, allowing travel 
between those two planes in either direction.

Second, you may then call a particular individual or kind of being through the 
\textit{gate.}

The \textit{gate }itself is a circular hoop or disk from 5 to 20 feet in diameter 
(caster's choice), oriented in the direction you desire when it comes into existence 
(typically vertical and facing you). It is a two-dimensional window looking into 
the plane you specified when casting the spell, and anyone or anything that moves 
through is shunted instantly to the other side.

A \textit{gate }has a front and a back. Creatures moving through the \textit{gate 
}from the front are transported to the other plane; creatures moving through it 
from the back are not.

\textit{Planar Travel: }As a mode of planar travel, a \textit{gate }spell functions 
much like a \textit{plane shift }spell, except that the \textit{gate }opens precisely 
at the point you desire (a creation effect). Deities and other beings who rule 
a planar realm can prevent a \textit{gate }from opening in their presence or personal 
demesnes if they so desire. Travelers need not join hands with you---anyone who 
chooses to step through the portal is transported. A \textit{gate }cannot be opened 
to another point on the same plane; the spell works only for interplanar travel.

You may hold the \textit{gate }open only for a brief time (no more than 1 round 
per caster level), and you must concentrate on doing so, or else the interplanar 
connection is severed.

\textit{Calling Creatures: }The second effect of the \textit{gate }spell is to 
call an extraplanar creature to your aid (a calling effect). By naming a particular 
being or kind of being as you cast the spell, you cause the \textit{gate }to open 
in the immediate vicinity of the desired creature and pull the subject through, 
willing or unwilling. Deities and unique beings are under no compulsion to come 
through the \textit{gate, }although they may choose to do so of their own accord. 
This use of the spell creates a \textit{gate }that remains open just long enough 
to transport the called creatures. This use of the spell has an XP cost (see below).

If you choose to call a kind of creature instead of a known individual you may 
call either a single creature (of any HD) or several creatures. You can call and 
control several creatures as long as their HD total does not exceed your caster 
level. In the case of a single creature, you can control it if its HD do not exceed 
twice your caster level. A single creature with more HD than twice your caster 
level can't be controlled. Deities and unique beings cannot be controlled in any 
event. An uncontrolled being acts as it pleases, making the calling of such creatures 
rather dangerous. An uncontrolled being may return to its home plane at any time.

A controlled creature can be commanded to perform a service for you. Such services 
fall into two categories: immediate tasks and contractual service. Fighting for 
you in a single battle or taking any other actions that can be accomplished within 
1 round per caster level counts as an immediate task; you need not make any agreement 
or pay any reward for the creature's help. The creature departs at the end of the 
spell.

If you choose to exact a longer or more involved form of service from a called 
creature, you must offer some fair trade in return for that service. The service 
exacted must be reasonable with respect to the promised favor or reward; see the 
\textit{lesser planar ally }spell for appropriate rewards. (Some creatures may 
want their payment in ``livestock'' rather than in coin, which could involve complications.) 
Immediately upon completion of the service, the being is transported to your vicinity, 
and you must then and there turn over the promised reward. After this is done, 
the creature is instantly freed to return to its own plane.

Failure to fulfill the promise to the letter results in your being subjected to 
service by the creature or by its liege and master, at the very least. At worst, 
the creature or its kin may attack you.

\textit{Note: }When you use a calling spell such as \textit{gate }to call an air, 
chaotic, earth, evil, fire, good, lawful, or water creature, it becomes a spell 
of that type.

\textit{XP Cost: }1,000 XP (only for the \textit{calling creatures }function).

\vspace{12pt}
Geas/Quest

Enchantment (Compulsion) [Language-Dependent, Mind-Affecting]

\textbf{Level:} Brd 6, Clr 6, Sor/Wiz 6

\textbf{Casting Time:} 10 minutes

\textbf{Target:} One living creature

\textbf{Saving Throw:} None

This spell functions similarly to \textit{lesser geas, }except that it affects 
a creature of any HD and allows no saving throw.

Instead of taking penalties to ability scores (as with \textit{lesser geas}), the 
subject takes 3d6 points of damage each day it does not attempt to follow the \textit{geas/quest}. 
Additionally, each day it must make a Fortitude saving throw or become sickened. 
These effects end 24 hours after the creature attempts to resume the \textit{geas/ 
quest}.

A \textit{remove curse }spell ends a \textit{geas/quest }spell only if its caster 
level is at least two higher than your caster level. \textit{Break enchantment 
}does not end a \textit{geas/quest}, but \textit{limited wish}, \textit{miracle}, 
and \textit{wish }do.

Bards, sorcerers, and wizards usually refer to this spell as \textit{geas, }while 
clerics call the same spell \textit{quest}.

\vspace{12pt}
Geas, Lesser

Enchantment (Compulsion) [Language-Dependent, Mind-Affecting]

\textbf{Level:} Brd 3, Sor/Wiz 4

\textbf{Components:} V

\textbf{Casting Time:} 1 round

\textbf{Range:} Close (25 ft. + 5 ft./2 levels)

\textbf{Target:} One living creature with 7 HD or less

\textbf{Duration:} One day/level or until discharged (D)

\textbf{Saving Throw: }Will negates

\textbf{Spell Resistance:} Yes

A \textit{lesser geas }places a magical command on a creature to carry out some 
service or to refrain from some action or course of activity, as desired by you. 
The creature must have 7 or fewer Hit Dice and be able to understand you. While 
a \textit{geas }cannot compel a creature to kill itself or perform acts that would 
result in certain death, it can cause almost any other course of activity.

The \textit{geased }creature must follow the given instructions until the \textit{geas 
}is completed, no matter how long it takes.

If the instructions involve some open-ended task that the recipient cannot complete 
through his own actions the spell remains in effect for a maximum of one day per 
caster level. A clever recipient can subvert some instructions:

If the subject is prevented from obeying the \textit{lesser geas }for 24 hours, 
it takes a -2 penalty to each of its ability scores. Each day, another -2 penalty 
accumulates, up to a total of -8. No ability score can be reduced to less than 
1 by this effect. The ability score penalties are removed 24 hours after the subject 
resumes obeying the \textit{lesser geas.}

A \textit{lesser geas }(and all ability score penalties) can be ended by \textit{break 
enchantment}, \textit{limited wish}, \textit{remove curse}, \textit{miracle}, or 
\textit{wish}. \textit{Dispel magic }does not affect a \textit{lesser geas}.

\vspace{12pt}
Gentle Repose

Necromancy

\textbf{Level:} Clr 2, Sor/Wiz 3

\textbf{Components:} V, S, M/DF

\textbf{Casting Time:} 1 standard action

\textbf{Range:} Touch

\textbf{Target:} Corpse touched

\textbf{Duration:} One day/level

\textbf{Saving Throw: }Will negates (object)

\textbf{Spell Resistance:} Yes (object)

You preserve the remains of a dead creature so that they do not decay. Doing so 
effectively extends the time limit on raising that creature from the dead (see 
\textit{raise dead}). Days spent under the influence of this spell don't count 
against the time limit. Additionally, this spell makes transporting a fallen comrade 
more pleasant.

The spell also works on severed body parts and the like.

\textit{Arcane Material Component: }A pinch of salt, and a copper piece for each 
eye the corpse has (or had).

\vspace{12pt}
Ghost Sound

Illusion (Figment)

\textbf{Level:} Brd 0, Sor/Wiz 0

\textbf{Components:} V, S, M

\textbf{Casting Time:} 1 standard action

\textbf{Range:} Close (25 ft. + 5 ft./2 levels)

\textbf{Effect:} Illusory sounds

\textbf{Duration:} 1 round/level (D)

\textbf{Saving Throw: }Will disbelief (if interacted with)

\textbf{Spell Resistance:} No

\textit{Ghost sound }allows you to create a volume of sound that rises, recedes, 
approaches, or remains at a fixed place. You choose what type of sound \textit{ghost 
sound }creates when casting it and cannot thereafter change the sound's basic character.

The volume of sound created depends on your level. You can produce as much noise 
as four normal humans per caster level (maximum twenty humans). Thus, talking, 
singing, shouting, walking, marching, or running sounds can be created. The noise 
a \textit{ghost sound }spell produces can be virtually any type of sound within 
the volume limit. A horde of rats running and squeaking is about the same volume 
as eight humans running and shouting. A roaring lion is equal to the noise from 
sixteen humans, while a roaring dire tiger is equal to the noise from twenty humans.

\textit{Ghost sound }can enhance the effectiveness of a \textit{silent image }spell.

\textit{Ghost sound }can be made permanent with a \textit{permanency }spell.

\textit{Material Component: }A bit of wool or a small lump of wax.

\vspace{12pt}
Ghoul Touch

Necromancy

\textbf{Level:} Sor/Wiz 2

\textbf{Components:} V, S, M

\textbf{Casting Time:} 1 standard action

\textbf{Range:} Touch

\textbf{Target:} Living humanoid touched

\textbf{Duration:} 1d6+2 rounds

\textbf{Saving Throw:} Fortitude negates

\textbf{Spell Resistance:} Yes

Imbuing you with negative energy, this spell allows you to paralyze a single living 
humanoid for the duration of the spell with a successful melee touch attack.

Additionally, the paralyzed subject exudes a carrion stench that causes all living 
creatures (except you) in a 10-foot-radius spread to become sickened (Fortitude 
negates). A \textit{neutralize poison }spell removes the effect from a sickened 
creature, and creatures immune to poison are unaffected by the stench.

\textit{Material Component: }A small scrap of cloth taken from clothing worn by 
a ghoul, or a pinch of earth from a ghoul's lair.

\vspace{12pt}
Giant Vermin

Transmutation

\textbf{Level:} Clr 4, Drd 4

\textbf{Components:} V, S, DF

\textbf{Casting Time:} 1 standard action

\textbf{Range:} Close (25 ft. + 5 ft./2 levels)

\textbf{Targets:} Up to three vermin, no two of which can be more than 30 ft. apart

\textbf{Duration:} 1 min./level

\textbf{Saving Throw:} None

\textbf{Spell Resistance:} Yes

You turn three normal-sized centipedes, two normal-sized spiders, or a single normal-sized 
scorpion into larger forms. Only one type of vermin can be transmuted (so a single 
casting cannot affect both a centipede and a spider), and all must be grown to 
the same size. The size to which the vermin can be grown depends on your level; 
see the table below.

Any giant vermin created by this spell do not attempt to harm you, but your control 
of such creatures is limited to simple commands (``Attack,'' ``Defend,'' ``Stop,'' 
and so forth). Orders to attack a certain creature when it appears or guard against 
a particular occurrence are too complex for the vermin to understand. Unless commanded 
to do otherwise, the giant vermin attack whoever or whatever is near them.

\begin{tabular}{|>{\raggedright}p{55pt}|>{\raggedright}p{51pt}|}
\hline
\subsection*{C\textbf{aster Level}} & \subsection*{V\textbf{ermin Size}}\tabularnewline
\hline
9th or lower & Medium\tabularnewline
\hline
10th-13th & Large\tabularnewline
\hline
14th-17th & Huge\tabularnewline
\hline
18th-19th & Gargantuan\tabularnewline
\hline
20th or higher & Colossal\tabularnewline
\hline
\end{tabular}

\vspace{12pt}
Glibness

Transmutation

\textbf{Level:} Brd 3

\textbf{Components:} S

\textbf{Casting Time:} 1 standard action

\textbf{Range:} Personal

\textbf{Target:} You

\textbf{Duration:} 10 min./level (D)

Your speech becomes fluent and more believable. You gain a +30 bonus on Bluff checks 
made to convince another of the truth of your words. (This bonus doesn't apply 
to other uses of the Bluff skill, such as feinting in combat, creating a diversion 
to hide, or communicating a hidden message via innuendo.)

If a magical effect is used against you that would detect your lies or force you 
to speak the truth the user of the effect must succeed on a caster level check 
(1d20 + caster level) against a DC of 15 + your caster level to succeed. Failure 
means the effect does not detect your lies or force you to speak only the truth.

\vspace{12pt}
Glitterdust

Conjuration (Creation)

\textbf{Level:} Brd 2, Sor/Wiz 2

\textbf{Components:} V, S, M

\textbf{Casting Time:} 1 standard action

\textbf{Range: }Medium (100 ft. + 10 ft./level)

\textbf{Area:} Creatures and objects within 10-ft.-radius spread

\textbf{Duration:} 1 round/level

\textbf{Saving Throw: }Will negates (blinding only)

\textbf{Spell Resistance:} No

A cloud of golden particles covers everyone and everything in the area, causing 
creatures to become blinded and visibly outlining invisible things for the duration 
of the spell. All within the area are covered by the dust, which cannot be removed 
and continues to sparkle until it fades.

Any creature covered by the dust takes a -40 penalty on Hide checks.

\textit{Material Component: }Ground mica.

\vspace{12pt}
Globe of Invulnerability

Abjuration

\textbf{Level:} Sor/Wiz 6

This spell functions like \textit{lesser globe of invulnerability, }except that 
it also excludes 4th-level spells and spell-like effects.

\vspace{12pt}
Globe of Invulnerability, Lesser

Abjuration

\textbf{Level:} Sor/Wiz 4

\textbf{Components:} V, S, M

\textbf{Casting Time:} 1 standard action

\textbf{Range:} 10 ft.

\textbf{Area:} 10-ft.-radius spherical emanation, centered on you

\textbf{Duration:} 1 round/level (D)

\textbf{Saving Throw:} None

\textbf{Spell Resistance:} No

An immobile, faintly shimmering magical sphere surrounds you and excludes all spell 
effects of 3rd level or lower. The area or effect of any such spells does not include 
the area of the \textit{lesser globe of invulnerability}. Such spells fail to affect 
any target located within the globe. Excluded effects include spell-like abilities 
and spells or spell-like effects from items. However, any type of spell can be 
cast through or out of the magical globe. Spells of 4th level and higher are not 
affected by the globe, nor are spells already in effect when the globe is cast. 
The globe can be brought down by a targeted \textit{dispel magic }spell, but not 
by an area \textit{dispel magic. }You can leave and return to the globe without 
penalty.

Note that spell effects are not disrupted unless their effects enter the globe, 
and even then they are merely suppressed, not dispelled. 

If a given spell has more than one level depending on which character class is 
casting it, use the level appropriate to the caster to determine whether \textit{lesser 
globe of invulnerability }stops it.

\textit{Material Component: }A glass or crystal bead that shatters at the expiration 
of the spell.

\vspace{12pt}
Glyph of Warding

Abjuration

\textbf{Level:} Clr 3

\textbf{Components:} V, S, M

\textbf{Casting Time:} 10 minutes

\textbf{Range:} Touch

\textbf{Target or Area:} Object touched or up to 5 sq. ft./level

\textbf{Duration:} Permanent until discharged (D)

\textbf{Saving Throw:} See text

\textbf{Spell Resistance:} No (object) and Yes; see text

This powerful inscription harms those who enter, pass, or open the warded area 
or object. A \textit{glyph of warding }can guard a bridge or passage, ward a portal, 
trap a chest or box, and so on.

You set the conditions of the ward. Typically, any creature entering the warded 
area or opening the warded object without speaking a password (which you set when 
casting the spell) is subject to the magic it stores. Alternatively or in addition 
to a password trigger, \textit{glyphs }can be set according to physical characteristics 
(such as height or weight) or creature type, subtype, or kind. \textit{Glyphs }can 
also be set with respect to good, evil, law, or chaos, or to pass those of your 
religion. They cannot be set according to class, Hit Dice, or level. \textit{Glyphs 
}respond to invisible creatures normally but are not triggered by those who travel 
past them ethereally. Multiple \textit{glyphs }cannot be cast on the same area. 
However, if a cabinet has three drawers, each can be separately warded.

When casting the spell, you weave a tracery of faintly glowing lines around the 
warding sigil. A \textit{glyph }can be placed to conform to any shape up to the 
limitations of your total square footage. When the spell is completed, the \textit{glyph 
}and tracery become nearly invisible.

\textit{Glyphs }cannot be affected or bypassed by such means as physical or magical 
probing, though they can be dispelled. \textit{Mislead}, \textit{polymorph}, and 
\textit{nondetection }(and similar magical effects) can fool a \textit{glyph}, 
though nonmagical disguises and the like can't. \textit{Read magic }allows you 
to identify a \textit{glyph of warding }with a DC 13 Spellcraft check. Identifying 
the \textit{glyph }does not discharge it and allows you to know the basic nature 
of the \textit{glyph }(version, type of damage caused, what spell is stored).

\textit{Note: }Magic traps such as \textit{glyph of warding }are hard to detect 
and disable. A rogue (only) can use the Search skill to find the \textit{glyph 
}and Disable Device to thwart it. The DC in each case is 25 + spell level, or 28 
for \textit{glyph of warding}.

Depending on the version selected, a \textit{glyph }either blasts the intruder 
or activates a spell.

\textit{Blast Glyph: }A \textit{blast glyph }deals 1d8 points of damage per two 
caster levels (maximum 5d8) to the intruder and to all within 5 feet of him or 
her. This damage is acid, cold, fire, electricity, or sonic (caster's choice, made 
at time of casting). Each creature affected can attempt a Reflex save to take half 
damage. Spell resistance applies against this effect.

\textit{Spell Glyph: }You can store any harmful spell of 3rd level or lower that 
you know. All level-dependent features of the spell are based on your caster level 
at the time of casting the \textit{glyph. }If the spell has a target, it targets 
the intruder. If the spell has an area or an amorphous effect the area or effect 
is centered on the intruder. If the spell summons creatures, they appear as close 
as possible to the intruder and attack. Saving throws and spell resistance operate 
as normal, except that the DC is based on the level of the spell stored in the 
\textit{glyph}.

\textit{Material Component: }You trace the \textit{glyph }with incense, which must 
first be sprinkled with powdered diamond worth at least 200 gp.

\vspace{12pt}
Glyph of Warding, Greater

Abjuration

\textbf{Level:} Clr 6

This spell functions like \textit{glyph of warding, }except that a \textit{greater 
blast glyph }deals up to 10d8 points of damage, and a \textit{greater spell glyph 
}can store a spell of 6th level or lower.

\textit{Material Component: }You trace the \textit{glyph }with incense, which must 
first be sprinkled with powdered diamond worth at least 400 gp.

\vspace{12pt}
Goodberry

Transmutation

\textbf{Level:} Drd 1

\textbf{Components:} V, S, DF

\textbf{Casting Time:} 1 standard action

\textbf{Range:} Touch

\textbf{Targets:} 2d4 fresh berries touched

\textbf{Duration:} One day/level

\textbf{Saving Throw:} None

\textbf{Spell Resistance:} Yes

Casting \textit{goodberry }upon a handful of freshly picked berries makes 2d4 of 
them magical. You (as well as any other druid of 3rd or higher level) can immediately 
discern which berries are affected. Each transmuted berry provides nourishment 
as if it were a normal meal for a Medium creature. The berry also cures 1 point 
of damage when eaten, subject to a maximum of 8 points of such curing in any 24-hour 
period.

\vspace{12pt}
Good Hope

Enchantment (Compulsion) [Mind-Affecting]

\textbf{Level:} Brd 3

\textbf{Components:} V, S

\textbf{Casting Time:} 1 standard action

\textbf{Range: }Medium (100 ft. + 10 ft./level)

\textbf{Targets:} One living creature/level, no two of which may be more than 30 
ft. apart

\textbf{Duration:} 1 min./level

\textbf{Saving Throw: }Will negates (harmless)

\textbf{Spell Resistance:} Yes (harmless)

This spell instills powerful hope in the subjects. Each affected creature gains 
a +2 morale bonus on saving throws, attack rolls, ability checks, skill checks, 
and weapon damage rolls.

\textit{Good hope }counters and dispels \textit{crushing despair}.

\vspace{12pt}
Grasping Hand

Evocation [Force]

\textbf{Level:} Sor/Wiz 7, Strength 7

\textbf{Components:} V, S, F/DF

This spell functions like \textit{interposing hand, }except the hand can also grapple 
one opponent that you select. The \textit{grasping hand }gets one grapple attack 
per round.

Its attack bonus to make contact equals your caster level + your Intelligence, 
Wisdom, or Charisma modifier (for wizards, clerics, and sorcerers, respectively), 
+10 for the hand's Strength score (31), -1 for being Large. Its grapple bonus is 
this same figure, except with a +4 modifier for being Large instead of -1. The 
hand holds but does not harm creatures it grapples.

Directing the spell to a new target is a move action.

The \textit{grasping hand }can also bull rush an opponent as \textit{forceful hand 
}does, but at a +16 bonus on the Strength check (+10 for Strength 35, +4 for being 
Large, and a +2 bonus for charging, which it always gets), or interpose itself 
as \textit{interposing hand }does.

Clerics who cast this spell name it for their deities.

\textit{Arcane Focus: }A leather glove.

\vspace{12pt}
Grease

Conjuration (Creation)

\textbf{Level:} Brd 1, Sor/Wiz 1

\textbf{Components:} V, S, M

\textbf{Casting Time:} 1 standard action

\textbf{Range:} Close (25 ft. + 5 ft./2 levels)

\textbf{Target or Area: }One object or a 10-ft. square

\textbf{Duration:} 1 round/level (D)

\textbf{Saving Throw:} See text

\textbf{Spell Resistance:} No

A \textit{grease }spell covers a solid surface with a layer of slippery grease. 
Any creature in the area when the spell is cast must make a successful Reflex save 
or fall. This save is repeated on your turn each round that the creature remains 
within the area. A creature can walk within or through the area of grease at half 
normal speed with a DC 10 Balance check. Failure means it can't move that round 
(and must then make a Reflex save or fall), while failure by 5 or more means it 
falls (see the Balance skill for details).

The spell can also be used to create a greasy coating on an item. Material objects 
not in use are always affected by this spell, while an object wielded or employed 
by a creature receives a Reflex saving throw to avoid the effect. If the initial 
saving throw fails, the creature immediately drops the item. A saving throw must 
be made in each round that the creature attempts to pick up or use the \textit{greased 
}item. A creature wearing \textit{greased }armor or clothing gains a +10 circumstance 
bonus on Escape Artist checks and on grapple checks made to resist or escape a 
grapple or to escape a pin.

\textit{Material Component: }A bit of pork rind or butter.

\vspace{12pt}
Greater (Spell Name)

Any spell whose name begins with \textit{greater }is alphabetized in this chapter 
according to the second word of the spell name. Thus, the description of a \textit{greater 
}spell appears near the description of the spell on which it is based. Spell chains 
that have \textit{greater }spells in them include those based on the spells \textit{arcane 
sight, command, dispel magic, glyph of warding, invisibility, magic fang, magic 
weapon, planar ally, planar binding, prying eyes, restoration, scrying, shadow 
conjuration, shadow evocation, shout, }and \textit{teleport.}

\vspace{12pt}
Guards and Wards

Abjuration

\textbf{Level:} Sor/Wiz 6

\textbf{Components:} V, S, M, F

\textbf{Casting Time:} 30 minutes

\textbf{Range:} Anywhere within the area to be warded

\textbf{Area:} Up to 200 sq. ft./level (S)

\textbf{Duration:} 2 hours/level (D)

\textbf{Saving Throw:} See text

\textbf{Spell Resistance:} See text

This powerful spell is primarily used to defend your stronghold. The ward protects 
200 square feet per caster level. The warded area can be as much as 20 feet high, 
and shaped as you desire. You can ward several stories of a stronghold by dividing 
the area among them; you must be somewhere within the area to be warded to cast 
the spell. The spell creates the following magical effects within the warded area.

\textit{Fog: }Fog fills all corridors, obscuring all sight, including darkvision, 
beyond 5 feet. A creature within 5 feet has concealment (attacks have a 20\% miss 
chance). Creatures farther away have total concealment (50\% miss chance, and the 
attacker cannot use sight to locate the target). Saving Throw: None. Spell Resistance: 
No.

\textit{Arcane Locks: }All doors in the warded area are \textit{arcane locked. 
}Saving Throw: None. Spell Resistance: No.

\textit{Webs: }Webs fill all stairs from top to bottom. These strands are identical 
with those created by the \textit{web }spell, except that they regrow in 10 minutes 
if they are burned or torn away while the \textit{guards and wards }spell lasts. 
Saving Throw: Reflex negates; see text for \textit{web}. Spell Resistance: No.

\textit{Confusion: }Where there are choices in direction---such as a corridor intersection 
or side passage---a minor \textit{confusion-}type effect functions so as to make 
it 50\% probable that intruders believe they are going in the opposite direction 
from the one they actually chose. This is an enchantment, mind-affecting effect. 
Saving Throw: None. Spell Resistance: Yes.

\textit{Lost Doors: }One door per caster level is covered by a \textit{silent image 
}to appear as if it were a plain wall. Saving Throw: Will disbelief (if interacted 
with). Spell Resistance: No.

In addition, you can place your choice of one of the following five magical effects.

1. \textit{Dancing lights }in four corridors. You can designate a simple program 
that causes the lights to repeat as long as the \textit{guards and wards }spell 
lasts. Saving Throw: None. Spell Resistance: No.

2. A \textit{magic mouth }in two places. Saving Throw: None. Spell Resistance: 
No.

3. A \textit{stinking cloud }in two places. The vapors appear in the places you 
designate; they return within 10 minutes if dispersed by wind while the \textit{guards 
and wards }spell lasts. Saving Throw: Fortitude negates; see text for \textit{stinking 
cloud. }Spell Resistance: No.

4. A \textit{gust of wind }in one corridor or room. Saving Throw: Fortitude negates. 
Spell Resistance: Yes.

5. A \textit{suggestion }in one place. You select an area of up to 5 feet square, 
and any creature who enters or passes through the area receives the \textit{suggestion 
}mentally. Saving Throw: Will negates. Spell Resistance: Yes.

The whole warded area radiates strong magic of the abjuration school. A \textit{dispel 
magic }cast on a specific effect, if successful, removes only that effect. A successful 
\textit{Mage's disjunction }destroys the entire \textit{guards and wards }effect.

\textit{Material Component: }Burning incense, a small measure of brimstone and 
oil, a knotted string, and a small amount of blood.

\textit{Focus: }A small silver rod.

\vspace{12pt}
Guidance

Divination

\textbf{Level:} Clr 0, Drd 0

\textbf{Components:} V, S

\textbf{Casting Time:} 1 standard action

\textbf{Range:} Touch

\textbf{Target:} Creature touched

\textbf{Duration:} 1 minute or until discharged

\textbf{Saving Throw: }Will negates (harmless)

\textbf{Spell Resistance:} Yes

This spell imbues the subject with a touch of divine guidance. The creature gets 
a +1 competence bonus on a single attack roll, saving throw, or skill check. It 
must choose to use the bonus before making the roll to which it applies.

\vspace{12pt}
Gust of Wind

Evocation [Air]

\textbf{Level:} Drd 2, Sor/Wiz 2

\textbf{Components:} V, S

\textbf{Casting Time:} 1 standard action

\textbf{Range:} 60 ft.

\textbf{Effect:} Line-shaped gust of severe wind emanating out from you to the 
extreme of the range

\textbf{Duration:} 1 round

\textbf{Saving Throw:} Fortitude negates

\textbf{Spell Resistance:} Yes

This spell creates a severe blast of air (approximately 50 mph) that originates 
from you, affecting all creatures in its path.

A Tiny or smaller creature on the ground is knocked down and rolled 1d4x10 feet, 
taking 1d4 points of nonlethal damage per 10 feet. If flying, a Tiny or smaller 
creature is blown back 2d6x10 feet and takes 2d6 points of nonlethal damage due 
to battering and buffeting.

Small creatures are knocked prone by the force of the wind, or if flying are blown 
back 1d6x10 feet.

Medium creatures are unable to move forward against the force of the wind, or if 
flying are blown back 1d6x5 feet.

Large or larger creatures may move normally within a \textit{gust of wind }effect.

A \textit{gust of wind }can't move a creature beyond the limit of its range.

Any creature, regardless of size, takes a -4 penalty on ranged attacks and Listen 
checks in the area of a \textit{gust of wind}.

The force of the \textit{gust }automatically extinguishes candles, torches, and 
similar unprotected flames. It causes protected flames, such as those of lanterns, 
to dance wildly and has a 50\% chance to extinguish those lights.

In addition to the effects noted, a \textit{gust of wind }can do anything that 
a sudden blast of wind would be expected to do. It can create a stinging spray 
of sand or dust, fan a large fire, overturn delicate awnings or hangings, heel 
over a small boat, and blow gases or vapors to the edge of its range.

\textit{Gust of wind }can be made permanent with a \textit{permanency }spell.

\newpage

\end{document}
