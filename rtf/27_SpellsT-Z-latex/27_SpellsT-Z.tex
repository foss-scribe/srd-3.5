%&pdfLaTeX
% !TEX encoding = UTF-8 Unicode
\documentclass{article}
\usepackage{ifxetex}
\ifxetex
\usepackage{fontspec}
\setmainfont[Mapping=tex-text]{STIXGeneral}
\else
\usepackage[T1]{fontenc}
\usepackage[utf8]{inputenc}
\fi
\usepackage{textcomp}

\usepackage{array}
\usepackage{amssymb}
\usepackage{fancyhdr}
\renewcommand{\headrulewidth}{0pt}
\renewcommand{\footrulewidth}{0pt}

\begin{document}

This material is Open Game Content, and is licensed for public use under the terms 
of the Open Game License v1.0a.

{\LARGE{}SPELLS (T-Z)}

\vspace{12pt}
Telekinesis

Transmutation

\textbf{Level:} Sor/Wiz 5

\textbf{Components:} V, S

\textbf{Casting Time:} 1 standard action

\textbf{Range:} Long (400 ft. + 40 ft./level)

\textbf{Target or Targets:} See text

\textbf{Duration:} Concentration (up to 1 round/ level) or instantaneous; see text

\textbf{Saving Throw: }Will negates (object) or None; see text

\textbf{Spell Resistance:} Yes (object); see text

You move objects or creatures by concentrating on them. Depending on the version 
selected, the spell can provide a gentle, sustained force, perform a variety of 
combat maneuvers, or exert a single short, violent thrust.

\textit{Sustained Force: }A sustained force moves an object weighing no more than 
25 pounds per caster level (maximum 375 pounds at 15th level) up to 20 feet per 
round. A creature can negate the effect on an object it possesses with a successful 
Will save or with spell resistance.

This version of the spell can last 1 round per caster level, but it ends if you 
cease concentration. The weight can be moved vertically, horizontally, or in both 
directions. An object cannot be moved beyond your range. The spell ends if the 
object is forced beyond the range. If you cease concentration for any reason, the 
object falls or stops.

An object can be telekinetically manipulated as if with one hand. For example, 
a lever or rope can be pulled, a key can be turned, an object rotated, and so on, 
if the force required is within the weight limitation. You might even be able to 
untie simple knots, though delicate activities such as these require Intelligence 
checks.

\textit{Combat Maneuver: }Alternatively, once per round, you can use \textit{telekinesis 
}to perform a bull rush, disarm, grapple (including pin), or trip. Resolve these 
attempts as normal, except that they don't provoke attacks of opportunity, you 
use your caster level in place of your base attack bonus (for disarm and grapple), 
you use your Intelligence modifier (if a wizard) or Charisma modifier (if a sorcerer) 
in place of your Strength or Dexterity modifier, and a failed attempt doesn't allow 
a reactive attempt by the target (such as for disarm or trip). No save is allowed 
against these attempts, but spell resistance applies normally. This version of 
the spell can last 1 round per caster level, but it ends if you cease concentration.

\textit{Violent Thrust: }Alternatively, the spell energy can be spent in a single 
round. You can hurl one object or creature per caster level (maximum 15) that are 
within range and all within 10 feet of each other toward any target within 10 feet 
per level of all the objects. You can hurl up to a total weight of 25 pounds per 
caster level (maximum 375 pounds at 15th level).

You must succeed on attack rolls (one per creature or object thrown) to hit the 
target with the items, using your base attack bonus + your Intelligence modifier 
(if a wizard) or Charisma modifier (if a sorcerer). Weapons cause standard damage 
(with no Strength bonus; note that arrows or bolts deal damage as daggers of their 
size when used in this manner). Other objects cause damage ranging from 1 point 
per 25 pounds (for less dangerous objects) to 1d6 points of damage per 25 pounds 
(for hard, dense objects).

Creatures who fall within the weight capacity of the spell can be hurled, but they 
are allowed Will saves (and spell resistance) to negate the effect, as are those 
whose held possessions are targeted by the spell. If a telekinesed creature is 
hurled against a solid surface, it takes damage as if it had fallen 10 feet (1d6 
points).

\vspace{12pt}
Telekinetic Sphere

Evocation [Force]

\textbf{Level:} Sor/Wiz 8

\textbf{Components:} V, S, M

\textbf{Casting Time:} 1 standard action

\textbf{Range:} Close (25 ft. + 5 ft./2 levels)

\textbf{Effect:} 1-ft.-diameter/level sphere, centered around creatures or objects

\textbf{Duration:} 1 min./level (D)

\textbf{Saving Throw:} Reflex negates (object)

\textbf{Spell Resistance:} Yes (object)

This spell functions like \textit{resilient sphere, }with the addition that the 
creatures or objects inside the globe are nearly weightless. Anything contained 
within an \textit{telekinetic sphere }weighs only one-sixteenth of its normal weight. 
You can telekinetically lift anything in the sphere that normally weighs 5,000 
pounds or less. The telekinetic control extends from you out to medium range (100 
feet + 10 feet per caster level) after the sphere has succeeded in encapsulating 
its contents.

You can move objects or creatures in the sphere that weigh a total of 5,000 pounds 
or less by concentrating on the sphere. You can begin moving a sphere in the round 
after casting the spell. If you concentrate on doing so (a standard action), you 
can move the sphere as much as 30 feet in a round. If you cease concentrating, 
the sphere does not move in that round (if on a level surface) or descends at its 
falling rate (if aloft) until it reaches a level surface, or the spell's duration 
expires, or you begin concentrating again. If you cease concentrating (voluntarily 
or due to failing a Concentration check), you can resume concentrating on your 
next turn or any later turn during the spell's duration.

The sphere falls at a rate of only 60 feet per round, which is not fast enough 
to cause damage to the contents of the sphere.

You can move the sphere telekinetically even if you are in it.

\textit{Material Component: }A hemispherical piece of clear crystal, a matching 
hemispherical piece of gum arabic, and a pair of small bar magnets.

\vspace{12pt}
Telepathic Bond

Divination

\textbf{Level:} Sor/Wiz 5

\textbf{Components:} V, S, M

\textbf{Casting Time:} 1 standard action

\textbf{Range:} Close (25 ft. + 5 ft./2 levels)

\textbf{Targets:} You plus one willing creature per three levels, no two of which 
can be more than 30 ft. apart

\textbf{Duration:} 10 min./level (D)

\textbf{Saving Throw:} None

\textbf{Spell Resistance:} No

You forge a telepathic bond among yourself and a number of willing creatures, each 
of which must have an Intelligence score of 3 or higher. Each creature included 
in the link is linked to all the others. The creatures can communicate telepathically 
through the bond regardless of language. No special power or influence is established 
as a result of the bond. Once the bond is formed, it works over any distance (although 
not from one plane to another).

If desired, you may leave yourself out of the telepathic bond forged. This decision 
must be made at the time of casting.

\textit{Telepathic bond }can be made permanent with a \textit{permanency }spell, 
though it only bonds two creatures per casting of \textit{permanency}.

\textit{Material Component: }Pieces of eggshell from two different kinds of creatures.

\vspace{12pt}
Teleport

Conjuration (Teleportation)

\textbf{Level:} Sor/Wiz 5, Travel 5

\textbf{Components:} V

\textbf{Casting Time:} 1 standard action

\textbf{Range:} Personal and touch

\textbf{Target:} You and touched objects or other touched willing creatures

\textbf{Duration:} Instantaneous

\textbf{Saving Throw:} None and Will negates (object)

\textbf{Spell Resistance:} No and Yes (object)

This spell instantly transports you to a designated destination, which may be as 
distant as 100 miles per caster level. Interplanar travel is not possible. You 
can bring along objects as long as their weight doesn't exceed your maximum load. 
You may also bring one additional willing Medium or smaller creature (carrying 
gear or objects up to its maximum load) or its equivalent (see below) per three 
caster levels. A Large creature counts as two Medium creatures, a Huge creature 
counts as two Large creatures, and so forth. All creatures to be transported must 
be in contact with one another, and at least one of those creatures must be in 
contact with you. As with all spells where the range is personal and the target 
is you, you need not make a saving throw, nor is spell resistance applicable to 
you. Only objects held or in use (attended) by another person receive saving throws 
and spell resistance.

You must have some clear idea of the location and layout of the destination. The 
clearer your mental image, the more likely the teleportation works. Areas of strong 
physical or magical energy may make teleportation more hazardous or even impossible.

To see how well the teleportation works, roll d\% and consult the Teleport table. 
Refer to the following information for definitions of the terms on the table.

\textit{Familiarity: }``Very familiar'' is a place where you have been very often 
and where you feel at home. ``Studied carefully'' is a place you know well, either 
because you can currently see it, you've been there often, or you have used other 
means (such as \textit{scrying}) to study the place for at least one hour. ``Seen 
casually'' is a place that you have seen more than once but with which you are 
not very familiar. ``Viewed once'' is a place that you have seen once, possibly 
using magic. 

``False destination'' is a place that does not truly exist or if you are teleporting 
to an otherwise familiar location that no longer exists as such or has been so 
completely altered as to no longer be familiar to you. When traveling to a false 
destination, roll 1d20+80 to obtain results on the table, rather than rolling d\%, 
since there is no real destination for you to hope to arrive at or even be off 
target from.

\textit{On Target: }You appear where you want to be.

\textit{Off Target: }You appear safely a random distance away from the destination 
in a random direction. Distance off target is 1d10x1d10\% of the distance that 
was to be traveled. The direction off target is determined randomly

\textit{Similar Area: }You wind up in an area that's visually or thematically similar 
to the target area.

Generally, you appear in the closest similar place within range. If no such area 
exists within the spell's range, the spell simply fails instead.

\textit{Mishap: }You and anyone else teleporting with you have gotten ``scrambled.'' 
You each take 1d10 points of damage, and you reroll on the chart to see where you 
wind up. For these rerolls, roll 1d20+80. Each time ``Mishap'' comes up, the characters 
take more damage and must reroll.

\begin{tabular}{|>{\raggedright}p{111pt}|>{\raggedright}p{43pt}|>{\raggedright}p{44pt}|>{\raggedright}p{54pt}|>{\raggedright}p{31pt}|}
\hline
F\textbf{amiliarity} & O\textbf{n Target} & O\textbf{ff Target} & S\textbf{imilar 
Area} & M\textbf{ishap}\tabularnewline
\hline
Very familiar & 01-97 & 98-99 & 100--- & \tabularnewline
\hline
Studied carefully & 01-94 & 95-97 & 98-99 & 100\tabularnewline
\hline
Seen casually & 01-88 & 89-94 & 95-98 & 99-100\tabularnewline
\hline
Viewed once & 01-76 & 77-88 & 89-96 & 97-100\tabularnewline
\hline
False destination (1d20+80)--- & --- &  & 81-92 & 93-100\tabularnewline
\hline
\end{tabular}

\vspace{12pt}
Teleport Object

Conjuration (Teleportation)

\textbf{Level:} Sor/Wiz 7

\textbf{Range:} Touch

\textbf{Target:} One touched object of up to 50 lb./level and 3 cu. ft./level

\textbf{Saving Throw: }Will negates (object)

\textbf{Spell Resistance:} Yes (object)

This spell functions like \textit{teleport, }except that it teleports an object, 
not you. Creatures and magical forces cannot be teleported.

If desired, the target object can be sent to a distant location on the Ethereal 
Plane. In this case, the point from which the object was teleported remains faintly 
magical until the item is retrieved. A successful targeted \textit{dispel magic 
}spell cast on that point brings the vanished item back from the Ethereal Plane.

\vspace{12pt}
Teleport, Greater

Conjuration (Teleportation)

\textbf{Level:} Sor/Wiz 7, Travel 7

This spell functions like \textit{teleport, }except that there is no range limit 
and there is no chance you arrive off target. In addition, you need not have seen 
the destination, but in that case you must have at least a reliable description 
of the place to which you are teleporting. If you attempt to teleport with insufficient 
information (or with misleading information), you disappear and simply reappear 
in your original location. Interplanar travel is not possible.

\vspace{12pt}
Teleportation Circle

Conjuration (Teleportation)

\textbf{Level:} Sor/Wiz 9

\textbf{Components:} V, M

\textbf{Casting Time:} 10 minutes

\textbf{Range:} 0 ft.

\textbf{Effect:} 5-ft.-radius circle that teleports those who activate it

\textbf{Duration:} 10 min./level (D)

\textbf{Saving Throw:} None

\textbf{Spell Resistance:} Yes

You create a circle on the floor or other horizontal surface that teleports, as 
\textit{greater teleport, }any creature who stands on it to a designated spot. 
Once you designate the destination for the circle, you can't change it. The spell 
fails if you attempt to set the circle to teleport creatures into a solid object, 
to a place with which you are not familiar and have no clear description, or to 
another plane.

The circle itself is subtle and nearly impossible to notice. If you intend to keep 
creatures from activating it accidentally, you need to mark the circle in some 
way.

\textit{Teleportation circle }can be made permanent with a \textit{permanency }spell. 
A permanent \textit{teleportation circle }that is disabled becomes inactive for 
10 minutes, then can be triggered again as normal.

\textit{Note: }Magic traps such as \textit{teleportation circle }are hard to detect 
and disable. A rogue (only) can use the Search skill to find the circle and Disable 
Device to thwart it. The DC in each case is 25 + spell level, or 34 in the case 
of \textit{teleportation circle}.

\textit{Material Component: }Amber dust to cover the area of the circle (cost 1,000 
gp).

\vspace{12pt}
Temporal Stasis

Transmutation

\textbf{Level:} Sor/Wiz 8

\textbf{Components:} V, S, M

\textbf{Casting Time:} 1 standard action

\textbf{Range:} Touch

\textbf{Target:} Creature touched

\textbf{Duration:} Permanent

\textbf{Saving Throw:} Fortitude negates

\textbf{Spell Resistance:} Yes

You must succeed on a melee touch attack. You place the subject into a state of 
suspended animation. For the creature, time ceases to flow and its condition becomes 
fixed. The creature does not grow older. Its body functions virtually cease, and 
no force or effect can harm it. This state persists until the magic is removed 
(such as by a successful \textit{dispel magic }spell or a \textit{freedom }spell).

\textit{Material Component: }A powder composed of diamond, emerald, ruby, and sapphire 
dust with a total value of at least 5,000 gp.

\vspace{12pt}
Time Stop

Transmutation

\textbf{Level:} Sor/Wiz 9, Trickery 9

\textbf{Components:} V

\textbf{Casting Time:} 1 standard action

\textbf{Range:} Personal

\textbf{Target:} You

\textbf{Duration:} 1d4+1 rounds (apparent time); see text

This spell seems to make time cease to flow for everyone but you. In fact, you 
speed up so greatly that all other creatures seem frozen, though they are actually 
still moving at their normal speeds. You are free to act for 1d4+1 rounds of apparent 
time. Normal and magical fire, cold, gas, and the like can still harm you. While 
the \textit{time stop }is in effect, other creatures are invulnerable to your attacks 
and spells; you cannot target such creatures with any attack or spell. A spell 
that affects an area and has a duration longer than the remaining duration of the 
\textit{time stop} have their normal effects on other creatures once the \textit{time 
stop }ends. Most spellcasters use the additional time to improve their defenses, 
summon allies, or flee from combat.

You cannot move or harm items held, carried, or worn by a creature stuck in normal 
time, but you can affect any item that is not in another creature's possession.

You are undetectable while \textit{time stop }lasts. You cannot enter an area protected 
by an \textit{antimagic field }while under the effect of \textit{time stop.}

\vspace{12pt}
Tiny Hut

Evocation [Force]

\textbf{Level:} Brd 3, Sor/Wiz 3

\textbf{Components:} V, S, M

\textbf{Casting Time:} 1 standard action

\textbf{Range:} 20 ft.

\textbf{Effect:} 20-ft.-radius sphere centered on your location

\textbf{Duration:} 2 hours/level (D)

\textbf{Saving Throw:} None

\textbf{Spell Resistance:} No

You create an unmoving, opaque sphere of force of any color you desire around yourself. 
Half the sphere projects above the ground, and the lower hemisphere passes through 
the ground. As many as nine other Medium creatures can fit into the field with 
you; they can freely pass into and out of the hut without harming it. However, 
if you remove yourself from the hut, the spell ends.

The temperature inside the hut is 70° F if the exterior temperature is between 
0° and 100° F. An exterior temperature below 0° or above 100° lowers or raises 
the interior temperature on a 1-degree-for-1 basis. The hut also provides protection 
against the elements, such as rain, dust, and sandstorms. The hut withstands any 
wind of less than hurricane force, but a hurricane (75+ mph wind speed) or greater 
force destroys it.

The interior of the hut is a hemisphere. You can illuminate it dimly upon command 
or extinguish the light as desired. Although the force field is opaque from the 
outside, it is transparent from within. Missiles, weapons, and most spell effects 
can pass through the hut without affecting it, although the occupants cannot be 
seen from outside the hut (they have total concealment).

\textit{Material Component: }A small crystal bead that shatters when the spell 
duration expires or the \textit{hut }is dispelled.

\vspace{12pt}
Tongues

Divination

\textbf{Level:} Brd 2, Clr 4, Sor/Wiz 3

\textbf{Components:} V, M/DF

\textbf{Casting Time:} 1 standard action

\textbf{Range:} Touch

\textbf{Target:} Creature touched

\textbf{Duration:} 10 min./level

\textbf{Saving Throw: }Will negates (harmless)

\textbf{Spell Resistance:} No

This spell grants the creature touched the ability to speak and understand the 
language of any intelligent creature, whether it is a racial tongue or a regional 
dialect. The subject can speak only one language at a time, although it may be 
able to understand several languages. \textit{Tongues }does not enable the subject 
to speak with creatures who don't speak. The subject can make itself understood 
as far as its voice carries. This spell does not predispose any creature addressed 
toward the subject in any way.

\textit{Tongues }can be made permanent with a \textit{permanency }spell.

\textit{Arcane Material Component: }A small clay model of a ziggurat, which shatters 
when the verbal component is pronounced.

\vspace{12pt}
Touch of Fatigue

Necromancy

\textbf{Level:} Sor/Wiz 0

\textbf{Components:} V, S, M

\textbf{Casting Time:} 1 standard action

\textbf{Range:} Touch

\textbf{Target:} Creature touched

\textbf{Duration:} 1 round/level

\textbf{Saving Throw:} Fortitude negates

\textbf{Spell Resistance:} Yes

You channel negative energy through your touch, fatiguing the target. You must 
succeed on a touch attack to strike a target.

The subject is immediately fatigued for the spell's duration.

This spell has no effect on a creature that is already fatigued. Unlike with normal 
fatigue, the effect ends as soon as the spell's duration expires.

\textit{Material Component: }A drop of sweat.

\vspace{12pt}
Touch of Idiocy

Enchantment (Compulsion) [Mind-Affecting]

\textbf{Level:} Sor/Wiz 2

\textbf{Components:} V, S

\textbf{Casting Time:} 1 standard action

\textbf{Range:} Touch

\textbf{Target:} Living creature touched

\textbf{Duration:} 10 min./level

\textbf{Saving Throw:} No

\textbf{Spell Resistance:} Yes

With a touch, you reduce the target's mental faculties. Your successful melee touch 
attack applies a 1d6 penalty to the target's Intelligence, Wisdom, and Charisma 
scores. This penalty can't reduce any of these scores below 1.

This spell's effect may make it impossible for the target to cast some or all of 
its spells, if the requisite ability score drops below the minimum required to 
cast spells of that level.

\vspace{12pt}
Transformation

Transmutation

\textbf{Level:} Sor/Wiz 6

\textbf{Components:} V, S, M

\textbf{Casting Time:} 1 standard action

\textbf{Range:} Personal

\textbf{Target:} You

\textbf{Duration:} 1 round/level

You become a virtual fighting machine--- stronger, tougher, faster, and more skilled 
in combat. Your mind-set changes so that you relish combat and you can't cast spells, 
even from magic items.

You gain a +4 enhancement bonus to Strength, Dexterity, and Constitution, a +4 
natural armor bonus to AC, a +5 competence bonus on Fortitude saves, and proficiency 
with all simple and martial weapons. Your base attack bonus equals your character 
level (which may give you multiple attacks).

You lose your spellcasting ability, including your ability to use spell activation 
or spell completion magic items, just as if the spells were no longer on your class 
list.

\textit{Material Component: }A \textit{potion of bull's strength}, which you drink 
(and whose effects are subsumed by the spell effects).

\vspace{12pt}
Transmute Metal to Wood

Transmutation

\textbf{Level:} Drd 7

\textbf{Components:} V, S, DF

\textbf{Casting Time:} 1 standard action

\textbf{Range:} Long (400 ft. + 40 ft./level)

\textbf{Area:} All metal objects within a 40-ft.-radius burst

\textbf{Duration:} Instantaneous

\textbf{Saving Throw:} None

\textbf{Spell Resistance:} Yes (object; see text)

This spell enables you to change all metal objects within its area to wood. Weapons, 
armor, and other metal objects carried by creatures are affected as well. A magic 
object made of metal effectively has spell resistance equal to 20 + its caster 
level against this spell. Artifacts cannot be transmuted. Weapons converted from 
metal to wood take a -2 penalty on attack and damage rolls. The armor bonus of 
any armor converted from metal to wood is reduced by 2. Weapons changed by this 
spell splinter and break on any natural attack roll of 1 or 2, and armor changed 
by this spell loses an additional point of armor bonus every time it is struck 
with a natural attack roll of 19 or 20.

Only \textit{limited wish, miracle, wish, }or similar magic can restore a transmuted 
object to its metallic state.

\vspace{12pt}
Transmute Mud to Rock

Transmutation [Earth]

\textbf{Level:} Drd 5, Sor/Wiz 5

\textbf{Components:} V, S, M/DF

\textbf{Casting Time:} 1 standard action

\textbf{Range: }Medium (100 ft. + 10 ft./level)

\textbf{Area:} Up to two 10-ft. cubes/level (S)

\textbf{Duration:} Permanent

\textbf{Saving Throw:} See text

\textbf{Spell Resistance:} No

This spell transforms normal mud or quicksand of any depth into soft stone (sandstone 
or a similar mineral) permanently.

Any creature in the mud is allowed a Reflex save to escape before the area is hardened 
to stone.

\textit{Transmute mud to rock }counters and dispels \textit{transmute rock to mud}.

\textit{Arcane Material Component: }Sand, lime, and water.

\vspace{12pt}
Transmute Rock to Mud

Transmutation [Earth]

\textbf{Level:} Drd 5, Sor/Wiz 5

\textbf{Components:} V, S, M/DF

\textbf{Casting Time:} 1 standard action

\textbf{Range: }Medium (100 ft. + 10 ft./level)

\textbf{Area:} Up to two 10-ft. cubes/level (S)

\textbf{Duration:} Permanent; see text

\textbf{Saving Throw:} See text

\textbf{Spell Resistance:} No

This spell turns natural, uncut or unworked rock of any sort into an equal volume 
of mud. Magical stone is not affected by the spell. The depth of the mud created 
cannot exceed 10 feet. A creature unable to levitate, fly, or otherwise free itself 
from the mud sinks until hip- or chest-deep, reducing its speed to 5 feet and causing 
a -2 penalty on attack rolls and AC. Brush thrown atop the mud can support creatures 
able to climb on top of it. Creatures large enough to walk on the bottom can wade 
through the area at a speed of 5 feet.

If \textit{transmute rock to mud }is cast upon the ceiling of a cavern or tunnel, 
the mud falls to the floor and spreads out in a pool at a depth of 5 feet. The 
falling mud and the ensuing cave-in deal 8d6 points of bludgeoning damage to anyone 
caught directly beneath the area, or half damage to those who succeed on Reflex 
saves.

Castles and large stone buildings are generally immune to the effect of the spell, 
since \textit{transmute rock to mud }can't affect worked stone and doesn't reach 
deep enough to undermine such buildings' foundations. However, small buildings 
or structures often rest upon foundations shallow enough to be damaged or even 
partially toppled by this spell.

The mud remains until a successful \textit{dispel magic }or \textit{transmute mud 
to rock }spell restores its substance---but not necessarily its form. Evaporation 
turns the mud to normal dirt over a period of days. The exact time depends on exposure 
to the sun, wind, and normal drainage.

\textit{Arcane Material Component: }Clay and water.

\vspace{12pt}
Transport via Plants

Conjuration (Teleportation)

\textbf{Level:} Drd 6

\textbf{Components:} V, S

\textbf{Casting Time:} 1 standard action

\textbf{Range:} Unlimited

\textbf{Target:} You and touched objects or other touched willing creatures

\textbf{Duration:} 1 round

\textbf{Saving Throw:} None

\textbf{Spell Resistance:} No

You can enter any normal plant (Medium or larger) and pass any distance to a plant 
of the same kind in a single round, regardless of the distance separating the two. 
The entry plant must be alive. The destination plant need not be familiar to you, 
but it also must be alive. If you are uncertain of the location of a particular 
kind of destination plant, you need merely designate direction and distance and 
the \textit{transport via plants }spell moves you as close as possible to the desired 
location. If a particular destination plant is desired but the plant is not living, 
the spell fails and you are ejected from the entry plant.

You can bring along objects as long as their weight doesn't exceed your maximum 
load. You may also bring one additional willing Medium or smaller creature (carrying 
gear or objects up to its maximum load) or its equivalent per three caster levels. 
Use the following equivalents to determine the maximum number of larger creatures 
you can bring along: A Large creature counts as two Medium creatures, a Huge creature 
counts as two Large creatures, and so forth. All creatures to be transported must 
be in contact with one another, and at least one of those creatures must be in 
contact with you.

You can't use this spell to travel through plant creatures.

The destruction of an occupied plant slays you and any creatures you have brought 
along, and ejects the bodies and all carried objects from the tree.

\vspace{12pt}
Trap the Soul

Conjuration (Summoning)

\textbf{Level:} Sor/Wiz 8

\textbf{Components:} V, S, M, (F); see text

\textbf{Casting Time:} 1 standard action or see text

\textbf{Range:} Close (25 ft. + 5 ft./2 levels)

\textbf{Target:} One creature

\textbf{Duration:} Permanent; see text

\textbf{Saving Throw:} See text

\textbf{Spell Resistance:} Yes; see text

\textit{Trap the soul }forces a creature's life force (and its material body) into 
a gem. The gem holds the trapped entity indefinitely or until the gem is broken 
and the life force is released, which allows the material body to reform. If the 
trapped creature is a powerful creature from another plane it can be required to 
perform a service immediately upon being freed. Otherwise, the creature can go 
free once the gem imprisoning it is broken.

Depending on the version selected, the spell can be triggered in one of two ways.

\textit{Spell Completion: }First, the spell can be completed by speaking its final 
word as a standard action as if you were casting a regular spell at the subject. 
This allows spell resistance (if any) and a Will save to avoid the effect. If the 
creature's name is spoken as well, any spell resistance is ignored and the save 
DC increases by 2. If the save or spell resistance is successful, the gem shatters.

\textit{Trigger Object: }The second method is far more insidious, for it tricks 
the subject into accepting a trigger object inscribed with the final spell word, 
automatically placing the creature's soul in the trap. To use this method, both 
the creature's name and the trigger word must be inscribed on the trigger object 
when the gem is enspelled. A \textit{sympathy }spell can also be placed on the 
trigger object. As soon as the subject picks up or accepts the trigger object, 
its life force is automatically transferred to the gem without the benefit of spell 
resistance or a save.

\textit{Material Component: }Before the actual casting of \textit{trap the soul, 
}you must procure a gem of at least 1,000 gp value for every Hit Die possessed 
by the creature to be trapped. If the gem is not valuable enough, it shatters when 
the entrapment is attempted. (While creatures have no concept of level or Hit Dice 
as such, the value of the gem needed to trap an individual can be researched. Remember 
that this value can change over time as creatures gain more Hit Dice.)

\textit{Focus (Trigger Object Only): }If the trigger object method is used, a special 
trigger object, prepared as described above, is needed.

\vspace{12pt}
Tree Shape

Transmutation

\textbf{Level:} Drd 2, Rgr 3

\textbf{Components:} V, S, DF

\textbf{Casting Time:} 1 standard action

\textbf{Range:} Personal

\textbf{Target:} You

\textbf{Duration:} 1 hour/level (D)

By means of this spell, you are able to assume the form of a Large living tree 
or shrub or a Large dead tree trunk with a small number of limbs. The closest inspection 
cannot reveal that the tree in question is actually a magically concealed creature. 
To all normal tests you are, in fact, a tree or shrub, although a \textit{detect 
magic }spell reveals a faint transmutation on the tree. While in tree form, you 
can observe all that transpires around you just as if you were in your normal form, 
and your hit points and save bonuses remain unaffected. You gain a +10 natural 
armor bonus to AC but have an effective Dexterity score of 0 and a speed of 0 feet. 
You are immune to critical hits while in tree form. All clothing and gear carried 
or worn changes with you.

You can dismiss \textit{tree shape }as a free action (instead of as a standard 
action).

\vspace{12pt}
Tree Stride

Conjuration (Teleportation)

\textbf{Level:} Drd 5, Rgr 4

\textbf{Components:} V, S, DF

\textbf{Casting Time:} 1 standard action

\textbf{Range:} Personal

\textbf{Target:} You

\textbf{Duration:} 1 hour/level or until expended; see text

You gain the ability to enter trees and move from inside one tree to inside another 
tree. The first tree you enter and all others you enter must be of the same kind, 
must be living, and must have girth at least equal to yours. By moving into an 
oak tree (for example), you instantly know the location of all other oak trees 
within transport range (see below) and may choose whether you want to pass into 
one or simply step back out of the tree you moved into. You may choose to pass 
to any tree of the appropriate kind within the transport range as shown on the 
following table.

\begin{tabular}{|>{\raggedright}p{64pt}|>{\raggedright}p{72pt}|}
\hline
\section*{T\textbf{ype of Tree}} & \section*{T\textbf{ransport Range}}\tabularnewline
\hline
Oak, ash, yew & 3,000 feet\tabularnewline
\hline
Elm, linden & 2,000 feet\tabularnewline
\hline
Other deciduous & 1,500 feet\tabularnewline
\hline
Any coniferous & 1,000 feet\tabularnewline
\hline
All other trees & 500 feet\tabularnewline
\hline
\end{tabular}

You may move into a tree up to one time per caster level (passing from one tree 
to another counts only as moving into one tree). The spell lasts until the duration 
expires or you exit a tree. Each transport is a full-round action.

You can, at your option, remain within a tree without transporting yourself, but 
you are forced out when the spell ends. If the tree in which you are concealed 
is chopped down or burned, you are slain if you do not exit before the process 
is complete.

\vspace{12pt}
True Resurrection

Conjuration (Healing)

\textbf{Level:} Clr 9

\textbf{Casting Time:} 10 minutes

This spell functions like \textit{raise dead, }except that you can resurrect a 
creature that has been dead for as long as 10 years per caster level. This spell 
can even bring back creatures whose bodies have been destroyed, provided that you 
unambiguously identify the deceased in some fashion (reciting the deceased's time 
and place of birth or death is the most common method).

Upon completion of the spell, the creature is immediately restored to full hit 
points, vigor, and health, with no loss of level (or Constitution points) or prepared 
spells.

You can revive someone killed by a death effect or someone who has been turned 
into an undead creature and then destroyed. This spell can also resurrect elementals 
or outsiders, but it can't resurrect constructs or undead creatures.

Even \textit{true resurrection }can't restore to life a creature who has died of 
old age.

\textit{Material Component: }A sprinkle of holy water and diamonds worth a total 
of at least 25,000 gp.

\vspace{12pt}
True Seeing

Divination

\textbf{Level:} Clr 5, Drd 7, Knowledge 5, Sor/Wiz 6

\textbf{Components:} V, S, M

\textbf{Casting Time:} 1 standard action

\textbf{Range:} Touch

\textbf{Target:} Creature touched

\textbf{Duration:} 1 min./level

\textbf{Saving Throw: }Will negates (harmless)

\textbf{Spell Resistance:} Yes (harmless)

You confer on the subject the ability to see all things as they actually are. The 
subject sees through normal and magical darkness, notices secret doors hidden by 
magic, sees the exact locations of creatures or objects under \textit{blur }or 
\textit{displacement }effects, sees invisible creatures or objects normally, sees 
through illusions, and sees the true form of polymorphed, changed, or transmuted 
things. Further, the subject can focus its vision to see into the Ethereal Plane 
(but not into extradimensional spaces). The range of \textit{true seeing }conferred 
is 120 feet.

\textit{True seeing}, however, does not penetrate solid objects. It in no way confers 
X-ray vision or its equivalent. It does not negate concealment, including that 
caused by fog and the like. \textit{True seeing }does not help the viewer see through 
mundane disguises, spot creatures who are simply hiding, or notice secret doors 
hidden by mundane means. In addition, the spell effects cannot be further enhanced 
with known magic, so one cannot use \textit{true seeing }through a \textit{crystal 
ball }or in conjunction with \textit{clairaudience/clairvoyance}.

\textit{Material Component: }An ointment for the eyes that costs 250 gp and is 
made from mushroom powder, saffron, and fat.

\vspace{12pt}
True Strike

Divination

\textbf{Level:} Sor/Wiz 1

\textbf{Components:} V, F

\textbf{Casting Time:} 1 standard action

\textbf{Range:} Personal

\textbf{Target:} You

\textbf{Duration:} See text

You gain temporary, intuitive insight into the immediate future during your next 
attack. Your next single attack roll (if it is made before the end of the next 
round) gains a +20 insight bonus. Additionally, you are not affected by the miss 
chance that applies to attackers trying to strike a concealed target.

\textit{Focus: }A small wooden replica of an archery target.

\vspace{12pt}
Undeath to Death

Necromancy

\textbf{Level:} Clr 6, Sor/Wiz 6

\textbf{Components:} V, S, M/DF

\textbf{Area:} Several undead creatures within a 40-ft.-radius burst

\textbf{Saving Throw: }Will negates

This spell functions like \textit{circle of death}, except that it destroys undead 
creatures as noted above.

\textit{Material Component: }The powder of a crushed diamond worth at least 500 
gp.

\vspace{12pt}
Undetectable Alignment

Abjuration

\textbf{Level:} Brd 1, Clr 2, Pal 2

\textbf{Components:} V, S

\textbf{Casting Time:} 1 standard action

\textbf{Range:} Close (25 ft. + 5 ft./2 levels)

\textbf{Target:} One creature or object

\textbf{Duration:} 24 hours

\textbf{Saving Throw: }Will negates (object)

\textbf{Spell Resistance:} Yes (object)

An \textit{undetectable alignment }spell conceals the alignment of an object or 
a creature from all forms of divination.

\vspace{12pt}
Unhallow

Evocation [Evil]

\textbf{Level:} Clr 5, Drd 5

\textbf{Components:} V, S, M

\textbf{Casting Time:} 24 hours

\textbf{Range:} Touch

\textbf{Area:} 40-ft. radius emanating from the touched point

\textbf{Duration:} Instantaneous

\textbf{Saving Throw:} See text

\textbf{Spell Resistance:} See text

\textit{Unhallow }makes a particular site, building, or structure an unholy site. 
This has three major effects.

First, the site or structure is guarded by a \textit{magic circle against good 
}effect.

Second, all turning checks made to turn undead take a -4 penalty, and turning checks 
to rebuke undead gain a +4 profane bonus. Spell resistance does not apply to this 
effect. (This provision does not apply to the druid version of the spell.)

Finally, you may choose to fix a single spell effect to the \textit{unhallowed 
}site. The spell effect lasts for one year and functions throughout the entire 
site, regardless of its normal duration and area or effect. You may designate whether 
the effect applies to all creatures, creatures that share your faith or alignment, 
or creatures that adhere to another faith or alignment. At the end of the year, 
the chosen effect lapses, but it can be renewed or replaced simply by casting \textit{unhallow 
}again.

Spell effects that may be tied to an \textit{unhallowed }site include \textit{aid, 
bane, bless, cause fear, darkness, daylight, death ward, deeper darkness, detect 
magic, detect good, dimensional anchor, discern lies, dispel magic, endure elements, 
freedom of movement, invisibility purge, protection from energy, remove fear, resist 
energy, silence, tongues, }and \textit{zone of truth}.

Saving throws and spell resistance might apply to these spells' effects. (See the 
individual spell descriptions for details.)

An area can receive only one \textit{unhallow }spell (and its associated spell 
effect) at a time.

\textit{Unhallow }counters but does not dispel \textit{hallow}.

\textit{Material Component: }Herbs, oils, and incense worth at least 1,000 gp, 
plus 1,000 gp per level of the spell to be tied to the \textit{unhallowed }area.

\vspace{12pt}
Unholy Aura

Abjuration [Evil]

\textbf{Level:} Clr 8, Evil 8

\textbf{Components:} V, S, F

\textbf{Casting Time:} 1 standard action

\textbf{Range:} 20 ft.

\textbf{Targets:} One creature/level in a 20-ft.-radius burst centered on you

\textbf{Duration:} 1 round/level (D)

\textbf{Saving Throw:} See text

\textbf{Spell Resistance:} Yes (harmless)

A malevolent darkness surrounds the subjects, protecting them from attacks, granting 
them resistance to spells cast by good creatures, and weakening good creatures 
when they strike the subjects. This abjuration has four effects.

First, each warded creature gains a +4 deflection bonus to AC and a +4 resistance 
bonus on saves. Unlike the effect of \textit{protection from good}, this benefit 
applies against all attacks, not just against attacks by good creatures.

Second, a warded creature gains spell resistance 25 against good spells and spells 
cast by good creatures.

Third, the abjuration blocks possession and mental influence, just as \textit{protection 
from good }does.

Finally, if a good creature succeeds on a melee attack against a warded creature, 
the offending attacker takes 1d6 points of temporary Strength damage (Fortitude 
negates).

\textit{Focus: }A tiny reliquary containing some sacred relic, such as a piece 
of parchment from an unholy text. The reliquary costs at least 500 gp.

\vspace{12pt}
Unholy Blight

Evocation [Evil]

\textbf{Level:} Evil 4

\textbf{Components:} V, S

\textbf{Casting Time:} 1 standard action

\textbf{Range: }Medium (100 ft. + 10 ft./level)

\textbf{Area:} 20-ft.-radius spread

\textbf{Duration:} Instantaneous (1d4 rounds); see text

\textbf{Saving Throw: }Will partial

\textbf{Spell Resistance:} Yes

You call up unholy power to smite your enemies. The power takes the form of a cold, 
cloying miasma of greasy darkness.

Only good and neutral (not evil) creatures are harmed by the spell.

The spell deals 1d8 points of damage per two caster levels (maximum 5d8) to a good 
creature (or 1d6 per caster level, maximum 10d6, to a good outsider) and causes 
it to be sickened for 1d4 rounds. A successful Will save reduces damage to half 
and negates the sickened effect. The effects cannot be negated by \textit{remove 
disease }or \textit{heal, }but \textit{remove curse }is effective.

The spell deals only half damage to creatures who are neither evil nor good, and 
they are not sickened. Such a creature can reduce the damage in half again (down 
to one-quarter) with a successful Will save.

\vspace{12pt}
Unseen Servant

Conjuration (Creation)

\textbf{Level:} Brd 1, Sor/Wiz 1

\textbf{Components:} V, S, M

\textbf{Casting Time:} 1 standard action

\textbf{Range:} Close (25 ft. + 5 ft./2 levels)

\textbf{Effect:} One invisible, mindless, shapeless servant

\textbf{Duration:} 1 hour/level

\textbf{Saving Throw:} None

\textbf{Spell Resistance:} No

An \textit{unseen servant }is an invisible, mindless, shapeless force that performs 
simple tasks at your command. It can run and fetch things, open unstuck doors, 
and hold chairs, as well as clean and mend. The servant can perform only one activity 
at a time, but it repeats the same activity over and over again if told to do so 
as long as you remain within range. It can open only normal doors, drawers, lids, 
and the like. It has an effective Strength score of 2 (so it can lift 20 pounds 
or drag 100 pounds). It can trigger traps and such, but it can exert only 20 pounds 
of force, which is not enough to activate certain pressure plates and other devices. 
It can't perform any task that requires a skill check with a DC higher than 10 
or that requires a check using a skill that can't be used untrained. Its speed 
is 15 feet.

The servant cannot attack in any way; it is never allowed an attack roll. It cannot 
be killed, but it dissipates if it takes 6 points of damage from area attacks. 
(It gets no saves against attacks.) If you attempt to send it beyond the spell's 
range (measured from your current position), the servant ceases to exist.

\textit{Material Component: }A piece of string and a bit of wood.

\vspace{12pt}
Vampiric Touch

Necromancy

\textbf{Level:} Sor/Wiz 3

\textbf{Components:} V, S

\textbf{Casting Time:} 1 standard action

\textbf{Range:} Touch

\textbf{Target:} Living creature touched

\textbf{Duration:} Instantaneous/1 hour; see text

\textbf{Saving Throw:} None

\textbf{Spell Resistance:} Yes

You must succeed on a melee touch attack. Your touch deals 1d6 points of damage 
per two caster levels (maximum 10d6). You gain temporary hit points equal to the 
damage you deal. However, you can't gain more than the subject's current hit points 
+10, which is enough to kill the subject. The temporary hit points disappear 1 
hour later.

\vspace{12pt}
Veil

Illusion (Glamer)

\textbf{Level:} Brd 6, Sor/Wiz 6

\textbf{Components:} V, S

\textbf{Casting Time:} 1 standard action

\textbf{Range:} Long (400 ft. + 40 ft./level)

\textbf{Targets:} One or more creatures, no two of which can be more than 30 ft. 
apart

\textbf{Duration:} Concentration + 1 hour/level (D)

\textbf{Saving Throw: }Will negates; see text

\textbf{Spell Resistance:} Yes; see text

You instantly change the appearance of the subjects and then maintain that appearance 
for the spell's duration. You can make the subjects appear to be anything you wish. 
The subjects look, feel, and smell just like the creatures the spell makes them 
resemble. Affected creatures resume their normal appearances if slain. You must 
succeed on a Disguise check to duplicate the appearance of a specific individual. 
This spell gives you a +10 bonus on the check.

Unwilling targets can negate the spell's effect on them by making Will saves or 
with spell resistance. Those who interact with the subjects can attempt Will disbelief 
saves to see through the glamer, but spell resistance doesn't help.

\vspace{12pt}
Ventriloquism

Illusion (Figment)

\textbf{Level:} Brd 1, Sor/Wiz 1

\textbf{Components:} V, F

\textbf{Casting Time:} 1 standard action

\textbf{Range:} Close (25 ft. + 5 ft./2 levels)

\textbf{Effect:} Intelligible sound, usually speech

\textbf{Duration:} 1 min./level (D)

\textbf{Saving Throw: }Will disbelief (if interacted with)

\textbf{Spell Resistance:} No

You can make your voice (or any sound that you can normally make vocally) seem 
to issue from someplace else. You can speak in any language you know. With respect 
to such voices and sounds, anyone who hears the sound and rolls a successful save 
recognizes it as illusory (but still hears it).

\textit{Focus: }A parchment rolled up into a small cone.

\vspace{12pt}
Virtue

Transmutation

\textbf{Level:} Clr 0, Drd 0, Pal 1

\textbf{Components:} V, S, DF

\textbf{Casting Time:} 1 standard action

\textbf{Range:} Touch

\textbf{Target:} Creature touched

\textbf{Duration:} 1 min.

\textbf{Saving Throw:} Fortitude negates (harmless)

\textbf{Spell Resistance:} Yes (harmless)

The subject gains 1 temporary hit point.

\vspace{12pt}
Vision

Divination

\textbf{Level:} Sor/Wiz 7

\textbf{Components:} V, S, M, XP

\textbf{Casting Time:} 1 standard action

This spell functions like \textit{legend lore, }except that it works more quickly 
but produces some strain on you. You pose a question about some person, place, 
or object, then cast the spell. If the person or object is at hand or if you are 
in the place in question, you receive a vision about it by succeeding on a caster 
level check (1d20 +1 per caster level; maximum +25) against DC 20. If only detailed 
information on the person, place, or object is known, the DC is 25, and the information 
gained is incomplete. If only rumors are known, the DC is 30, and the information 
gained is vague.

\textit{XP Cost: }100 XP.

\vspace{12pt}
Wail of the Banshee

Necromancy [Death, Sonic]

\textbf{Level:} Death 9, Sor/Wiz 9

\textbf{Components:} V

\textbf{Casting Time:} 1 standard action

\textbf{Range:} Close (25 ft. + 5 ft./2 levels)

\textbf{Area:} One living creature/level within a 40-ft.-radius spread

\textbf{Duration:} Instantaneous

\textbf{Saving Throw:} Fortitude negates

\textbf{Spell Resistance:} Yes

You emit a terrible scream that kills creatures that hear it (except for yourself 
). Creatures closest to the point of origin are affected first.

\vspace{12pt}
Wall of Fire

Evocation [Fire]

\textbf{Level:} Drd 5, Fire 4, Sor/Wiz 4

\textbf{Components:} V, S, M/DF

\textbf{Casting Time:} 1 standard action

\textbf{Range: }Medium (100 ft. + 10 ft./level)

\textbf{Effect:} Opaque sheet of flame up to 20 ft. long/level or a ring of fire 
with a radius of up to 5 ft. per two levels; either form 20 ft. high

\textbf{Duration:} Concentration + 1 round/level

\textbf{Saving Throw:} None

\textbf{Spell Resistance:} Yes

An immobile, blazing curtain of shimmering violet fire springs into existence. 
One side of the wall, selected by you, sends forth waves of heat, dealing 2d4 points 
of fire damage to creatures within 10 feet and 1d4 points of fire damage to those 
past 10 feet but within 20 feet. The wall deals this damage when it appears and 
on your turn each round to all creatures in the area. In addition, the wall deals 
2d6 points of fire damage +1 point of fire damage per caster level (maximum +20) 
to any creature passing through it. The wall deals double damage to undead creatures.

If you evoke the wall so that it appears where creatures are, each creature takes 
damage as if passing through the wall. If any 5-foot length of wall takes 20 points 
of cold damage or more in 1 round, that length goes out. (Do not divide cold damage 
by 4, as normal for objects.)

\textit{Wall of fire }can be made permanent with a \textit{permanency }spell. A 
permanent \textit{wall of fire }that is extinguished by cold damage becomes inactive 
for 10 minutes, then reforms at normal strength.

\textit{Arcane Material Component: }A small piece of phosphorus.

\vspace{12pt}
Wall of Force

Evocation [Force]

\textbf{Level:} Sor/Wiz 5

\textbf{Components:} V, S, M

\textbf{Casting Time:} 1 standard action

\textbf{Range:} Close (25 ft. + 5 ft./2 levels)

\textbf{Effect: }Wall whose area is up to one 10-ft. square/level

\textbf{Duration:} 1 round /level (D)

\textbf{Saving Throw:} None

\textbf{Spell Resistance:} No

A \textit{wall of force }spell creates an invisible wall of force. The wall cannot 
move, it is immune to damage of all kinds, and it is unaffected by most spells, 
including \textit{dispel magic}. However, \textit{disintegrate }immediately destroys 
it, as does a \textit{rod of cancellation}, a \textit{sphere of annihilation, }or 
a \textit{mage's disjunction }spell. Breath weapons and spells cannot pass through 
the wall in either direction, although \textit{dimension door, teleport, }and similar 
effects can bypass the barrier. It blocks ethereal creatures as well as material 
ones (though ethereal creatures can usually get around the wall by floating under 
or over it through material floors and ceilings). Gaze attacks can operate through 
a \textit{wall of force}.

The caster can form the wall into a flat, vertical plane whose area is up to one 
10- foot square per level. The wall must be continuous and unbroken when formed. 
If its surface is broken by any object or creature, the spell fails.

\textit{Wall of force }can be made permanent with a \textit{permanency }spell.

\textit{Material Component: }A pinch of powder made from a clear gem.

\vspace{12pt}
Wall of Ice

Evocation [Cold]

\textbf{Level:} Sor/Wiz 4

\textbf{Components:} V, S, M

\textbf{Casting Time:} 1 standard action

\textbf{Range: }Medium (100 ft. + 10 ft./level)

\textbf{Effect:} Anchored plane of ice, up to one 10-ft. square/level, or hemisphere 
of ice with a radius of up to 3 ft. + 1 ft./level

\textbf{Duration:} 1 min./level

\textbf{Saving Throw:} Reflex negates; see text

\textbf{Spell Resistance:} Yes

This spell creates an anchored plane of ice or a hemisphere of ice, depending on 
the version selected. A \textit{wall of ice }cannot form in an area occupied by 
physical objects or creatures. Its surface must be smooth and unbroken when created. 
Any creature adjacent to the wall when it is created may attempt a Reflex save 
to disrupt the wall as it is being formed. A successful save indicates that the 
spell automatically fails. Fire can melt a \textit{wall of ice, }and it deals full 
damage to the wall (instead of the normal half damage taken by objects). Suddenly 
melting a \textit{wall of ice }creates a great cloud of steamy fog that lasts for 
10 minutes.

\textit{Ice Plane: }A sheet of strong, hard ice appears. The wall is 1 inch thick 
per caster level. It covers up to a 10-foot-square area per caster level (so a 
10th-level wizard can create a wall of ice 100 feet long and 10 feet high, a wall 
50 feet long and 20 feet high, or some other combination of length and height that 
does not exceed 1,000 square feet). The plane can be oriented in any fashion as 
long as it is anchored. A vertical wall need only be anchored on the floor, while 
a horizontal or slanting wall must be anchored on two opposite sides.

Each 10-foot square of wall has 3 hit points per inch of thickness. Creatures can 
hit the wall automatically. A section of wall whose hit points drop to 0 is breached. 
If a creature tries to break through the wall with a single attack, the DC for 
the Strength check is 15 + caster level.

Even when the ice has been broken through, a sheet of frigid air remains. Any creature 
stepping through it (including the one who broke through the wall) takes 1d6 points 
of cold damage +1 point per caster level (no save).

\textit{Hemisphere: }The wall takes the form of a hemisphere whose maximum radius 
is 3 feet + 1 foot per caster level. The \textit{hemisphere }is as hard to break 
through as the \textit{ice plane }form, but it does not deal damage to those who 
go through a breach.

\textit{Material Component: }A small piece of quartz or similar rock crystal.

\vspace{12pt}
Wall of Iron

Conjuration (Creation)

\textbf{Level:} Sor/Wiz 6

\textbf{Components:} V, S, M

\textbf{Casting Time:} 1 standard action

\textbf{Range: }Medium (100 ft. + 10 ft./level)

\textbf{Effect:} Iron wall whose area is up to one 5-ft. square/level; see text

\textbf{Duration:} Instantaneous

\textbf{Saving Throw:} See text

\textbf{Spell Resistance:} No

You cause a flat, vertical iron wall to spring into being. The wall inserts itself 
into any surrounding nonliving material if its area is sufficient to do so. The 
wall cannot be conjured so that it occupies the same space as a creature or another 
object. It must always be a flat plane, though you can shape its edges to fit the 
available space.

A \textit{wall of iron }is 1 inch thick per four caster levels. You can double 
the wall's area by halving its thickness. Each 5- foot square of the wall has 30 
hit points per inch of thickness and hardness 10. A section of wall whose hit points 
drop to 0 is breached. If a creature tries to break through the wall with a single 
attack, the DC for the Strength check is 25 + 2 per inch of thickness.

If you desire, the wall can be created vertically resting on a flat surface but 
not attached to the surface, so that it can be tipped over to fall on and crush 
creatures beneath it. The wall is 50\% likely to tip in either direction if left 
unpushed. Creatures can push the wall in one direction rather than letting it fall 
randomly. A creature must make a DC 40 Strength check to push the wall over. Creatures 
with room to flee the falling wall may do so by making successful Reflex saves. 
Any Large or smaller creature that fails takes 10d6 points of damage. The wall 
cannot crush Huge and larger creatures.

Like any iron wall, this wall is subject to rust, perforation, and other natural 
phenomena.

\textit{Material Component: }A small piece of sheet iron plus gold dust worth 50 
gp (1 pound of gold dust).

\vspace{12pt}
Wall of Stone

Conjuration (Creation) [Earth]

\textbf{Level:} Clr 5, Drd 6, Earth 5, Sor/Wiz 5

\textbf{Components:} V, S, M/DF

\textbf{Casting Time:} 1 standard action

\textbf{Range: }Medium (100 ft. + 10 ft./level)

\textbf{Effect:} Stone wall whose area is up to one 5-ft. square/level (S)

\textbf{Duration:} Instantaneous

\textbf{Saving Throw:} See text

\textbf{Spell Resistance:} No

This spell creates a wall of rock that merges into adjoining rock surfaces. A \textit{wall 
of stone }is 1 inch thick per four caster levels and composed of up to one 5-foot 
square per level. You can double the wall's area by halving its thickness. The 
wall cannot be conjured so that it occupies the same space as a creature or another 
object.

Unlike a \textit{wall of iron, }you can create a \textit{wall of stone }in almost 
any shape you desire. The wall created need not be vertical, nor rest upon any 
firm foundation; however, it must merge with and be solidly supported by existing 
stone. It can be used to bridge a chasm, for instance, or as a ramp. For this use, 
if the span is more than 20 feet, the wall must be arched and buttressed. This 
requirement reduces the spell's area by half. The wall can be crudely shaped to 
allow crenellations, battlements, and so forth by likewise reducing the area.

Like any other stone wall, this one can be destroyed by a \textit{disintegrate 
}spell or by normal means such as breaking and chipping. Each 5-foot square of 
the wall has 15 hit points per inch of thickness and hardness 8. A section of wall 
whose hit points drop to 0 is breached. If a creature tries to break through the 
wall with a single attack, the DC for the Strength check is 20 + 2 per inch of 
thickness.

It is possible, but difficult, to trap mobile opponents within or under a \textit{wall 
of stone}, provided the wall is shaped so it can hold the creatures. Creatures 
can avoid entrapment with successful Reflex saves.

\textit{Arcane Material Component: }A small block of granite.

\vspace{12pt}
Wall of Thorns

Conjuration (Creation)

\textbf{Level:} Drd 5, Plant 5

\textbf{Components:} V, S

\textbf{Casting Time:} 1 standard action

\textbf{Range: }Medium (100 ft. + 10 ft./level)

\textbf{Effect: }Wall of thorny brush, up to one 10-ft. cube/level (S)

\textbf{Duration:} 10 min./level (D)

\textbf{Saving Throw:} None

\textbf{Spell Resistance:} No

A \textit{wall of thorns }spell creates a barrier of very tough, pliable, tangled 
brush bearing needle-sharp thorns as long as a human's finger. Any creature forced 
into or attempting to move through a \textit{wall of thorns }takes slashing damage 
per round of movement equal to 25 minus the creature's AC. Dexterity and dodge 
bonuses to AC do not count for this calculation. (Creatures with an Armor Class 
of 25 or higher, without considering Dexterity and dodge bonuses, take no damage 
from contact with the wall.)

You can make the wall as thin as 5 feet thick, which allows you to shape the wall 
as a number of 10-by-10-by-5-foot blocks equal to twice your caster level. This 
has no effect on the damage dealt by the thorns, but any creature attempting to 
break through takes that much less time to force its way through the barrier.

Creatures can force their way slowly through the wall by making a Strength check 
as a full-round action. For every 5 points by which the check exceeds 20, a creature 
moves 5 feet (up to a maximum distance equal to its normal land speed). Of course, 
moving or attempting to move through the thorns incurs damage as described above. 
A creature trapped in the thorns can choose to remain motionless in order to avoid 
taking any more damage.

Any creature within the area of the spell when it is cast takes damage as if it 
had moved into the wall and is caught inside. In order to escape, it must attempt 
to push its way free, or it can wait until the spell ends. Creatures with the ability 
to pass through overgrown areas unhindered can pass through a \textit{wall of thorns 
}at normal speed without taking damage.

A \textit{wall of thorns }can be breached by slow work with edged weapons. Chopping 
away at the wall creates a safe passage 1 foot deep for every 10 minutes of work. 
Normal fire cannot harm the barrier, but magical fire burns it away in 10 minutes.

Despite its appearance, a \textit{wall of thorns }is not actually a living plant, 
and thus is unaffected by spells that affect plants.

\vspace{12pt}
Warp Wood

Transmutation

\textbf{Level:} Drd 2

\textbf{Components:} V, S

\textbf{Casting Time:} 1 standard action

\textbf{Range:} Close (25 ft. + 5 ft./2 levels)

\textbf{Target:} 1 Small wooden object/level, all within a 20-ft. radius

\textbf{Duration:} Instantaneous

\textbf{Saving Throw: }Will negates (object)

\textbf{Spell Resistance:} Yes (object)

You cause wood to bend and warp, permanently destroying its straightness, form, 
and strength. A warped door springs open (or becomes stuck, requiring a Strength 
check to open, at your option). A boat or ship springs a leak. Warped ranged weapons 
are useless. A warped melee weapon causes a -4 penalty on attack rolls.

You may warp one Small or smaller object or its equivalent per caster level. A 
Medium object counts as two Small objects, a Large object as four, a Huge object 
as eight, a Gargantuan object as sixteen, and a Colossal object as thirty-two.

Alternatively, you can unwarp wood (effectively warping it back to normal) with 
this spell, straightening wood that has been warped by this spell or by other means. 
\textit{Make whole}, on the other hand, does no good in repairing a warped item.

You can combine multiple consecutive \textit{warp wood }spells to warp (or unwarp) 
an object that is too large for you to warp with a single spell. 

Until the object is completely warped, it suffers no ill effects.

\vspace{12pt}
Water Breathing

Transmutation

\textbf{Level:} Clr 3, Drd 3, Sor/Wiz 3, Water 3

\textbf{Components:} V, S, M/DF

\textbf{Casting Time:} 1 standard action

\textbf{Range:} Touch

\textbf{Target:} Living creatures touched

\textbf{Duration:} 2 hours/level; see text

\textbf{Saving Throw: }Will negates (harmless)

\textbf{Spell Resistance:} Yes (harmless)

The transmuted creatures can breathe water freely. Divide the duration evenly among 
all the creatures you touch.

The spell does not make creatures unable to breathe air.

\textit{Arcane Material Component: }A short reed or piece of straw.

\vspace{12pt}
Water Walk

Transmutation [Water]

\textbf{Level:} Clr 3, Rgr 3

\textbf{Components:} V, S, DF

\textbf{Casting Time:} 1 standard action

\textbf{Range:} Touch

\textbf{Targets:} One touched creature/level

\textbf{Duration:} 10 min./level (D)

\textbf{Saving Throw: }Will negates (harmless)

\textbf{Spell Resistance:} Yes (harmless)

The transmuted creatures can tread on any liquid as if it were firm ground. Mud, 
oil, snow, quicksand, running water, ice, and even lava can be traversed easily, 
since the subjects' feet hover an inch or two above the surface. (Creatures crossing 
molten lava still take damage from the heat because they are near it.) The subjects 
can walk, run, charge, or otherwise move across the surface as if it were normal 
ground.

If the spell is cast underwater (or while the subjects are partially or wholly 
submerged in whatever liquid they are in), the subjects are borne toward the surface 
at 60 feet per round until they can stand on it.

\vspace{12pt}
Waves of Exhaustion

Necromancy

\textbf{Level:} Sor/Wiz 7

\textbf{Components:} V, S

\textbf{Casting Time:} 1 standard action

\textbf{Range:} 60 ft.

\textbf{Area:} Cone-shaped burst

\textbf{Duration:} Instantaneous

\textbf{Saving Throw:} No

\textbf{Spell Resistance:} Yes

Waves of negative energy cause all living creatures in the spell's area to become 
exhausted. This spell has no effect on a creature that is already exhausted.

\vspace{12pt}
Waves of Fatigue

Necromancy

\textbf{Level:} Sor/Wiz 5

\textbf{Components:} V, S

\textbf{Casting Time:} 1 standard action

\textbf{Range:} 30 ft.

\textbf{Area:} Cone-shaped burst

\textbf{Duration:} Instantaneous

\textbf{Saving Throw:} No

\textbf{Spell Resistance:} Yes

Waves of negative energy render all living creatures in the spell's area fatigued. 
This spell has no effect on a creature that is already fatigued.

\vspace{12pt}
Web

Conjuration (Creation)

\textbf{Level:} Sor/Wiz 2

\textbf{Components:} V, S, M

\textbf{Casting Time:} 1 standard action

\textbf{Range: }Medium (100 ft. + 10 ft./level)

\textbf{Effect: }Webs in a 20-ft.-radius spread

\textbf{Duration:} 10 min./level (D)

\textbf{Saving Throw:} Reflex negates; see text

\textbf{Spell Resistance:} No

\textit{Web }creates a many-layered mass of strong, sticky strands. These strands 
trap those caught in them. The strands are similar to spider webs but far larger 
and tougher. These masses must be anchored to two or more solid and diametrically 
opposed points or else the web collapses upon itself and disappears. Creatures 
caught within a \textit{web }become entangled among the gluey fibers. Attacking 
a creature in a web won't cause you to become entangled.

Anyone in the effect's area when the spell is cast must make a Reflex save. If 
this save succeeds, the creature is entangled, but not prevented from moving, though 
moving is more difficult than normal for being entangled (see below). If the save 
fails, the creature is entangled and can't move from its space, but can break loose 
by spending 1 round and making a DC 20 Strength check or a DC 25 Escape Artist 
check. Once loose (either by making the initial Reflex save or a later Strength 
check or Escape Artist check), a creature remains entangled, but may move through 
the \textit{web }very slowly. Each round devoted to moving allows the creature 
to make a new Strength check or Escape Artist check. The creature moves 5 feet 
for each full 5 points by which the check result exceeds 10.

If you have at least 5 feet of web between you and an opponent, it provides cover. 
If you have at least 20 feet of web between you, it provides total cover.

The strands of a \textit{web }spell are flammable. A magic \textit{flaming sword 
}can slash them away as easily as a hand brushes away cobwebs. Any fire can set 
the webs alight and burn away 5 square feet in 1 round. All creatures within flaming 
webs take 2d4 points of fire damage from the flames.

\textit{Web }can be made permanent with a \textit{permanency }spell. A permanent 
\textit{web }that is damaged (but not destroyed) regrows in 10 minutes.

\textit{Material Component: }A bit of spider web.

\vspace{12pt}
Weird

Illusion (Phantasm) [Fear, Mind-Affecting]

\textbf{Level:} Sor/Wiz 9

\textbf{Targets:} Any number of creatures, no two of which can be more than 30 
ft. apart

This spell functions like \textit{phantasmal killer}, except it can affect more 
than one creature. Only the affected creatures see the phantasmal creatures attacking 
them, though you see the attackers as shadowy shapes.

If a subject's Fortitude save succeeds, it still takes 3d6 points of damage and 
is stunned for 1 round. The subject also takes 1d4 points of temporary Strength 
damage.

\vspace{12pt}
Whirlwind

Evocation [Air]

\textbf{Level:} Air 8, Drd 8

\textbf{Components:} V, S, DF

\textbf{Casting Time:} 1 standard action

\textbf{Range:} Long (400 ft. + 40 ft./level)

\textbf{Effect:} Cyclone 10 ft. wide at base, 30 ft. wide at top, and 30 ft. tall

\textbf{Duration:} 1 round/level (D)

\textbf{Saving Throw:} Reflex negates; see text

\textbf{Spell Resistance:} Yes

This spell creates a powerful cyclone of raging wind that moves through the air, 
along the ground, or over water at a speed of 60 feet per round. You can concentrate 
on controlling the cyclone's every movement or specify a simple program. Directing 
the cyclone's movement or changing its programmed movement is a standard action 
for you. The cyclone always moves during your turn. If the cyclone exceeds the 
spell's range, it moves in a random, uncontrolled fashion for 1d3 rounds and then 
dissipates. (You can't regain control of the cyclone, even if comes back within 
range.)

Any Large or smaller creature that comes in contact with the spell effect must 
succeed on a Reflex save or take 3d6 points of damage. A Medium or smaller creature 
that fails its first save must succeed on a second one or be picked up bodily by 
the cyclone and held suspended in its powerful winds, taking 1d8 points of damage 
each round on your turn with no save allowed. You may direct the cyclone to eject 
any carried creatures whenever you wish, depositing the hapless souls wherever 
the cyclone happens to be when they are released.

\vspace{12pt}
Whispering Wind

Transmutation [Air]

\textbf{Level:} Brd 2, Sor/Wiz 2

\textbf{Components:} V, S

\textbf{Casting Time:} 1 standard action

\textbf{Range:} 1 mile/level

\textbf{Area:} 10-ft.-radius spread

\textbf{Duration:} No more than 1 hour/level or until discharged (destination is 
reached)

\textbf{Saving Throw:} None

\textbf{Spell Resistance:} No

You send a message or sound on the wind to a designated spot. The \textit{whispering 
wind }travels to a specific location within range that is familiar to you, provided 
that it can find a way to the location. A \textit{whispering wind }is as gentle 
and unnoticed as a zephyr until it reaches the location. It then delivers its whisper-quiet 
message or other sound. Note that the message is delivered regardless of whether 
anyone is present to hear it. The wind then dissipates.

You can prepare the spell to bear a message of no more than twenty-five words, 
cause the spell to deliver other sounds for 1 round, or merely have the \textit{whispering 
wind }seem to be a faint stirring of the air. You can likewise cause the \textit{whispering 
wind }to move as slowly as 1 mile per hour or as quickly as 1 mile per 10 minutes.

When the spell reaches its objective, it swirls and remains in place until the 
message is delivered. As with \textit{magic mouth, whispering wind }cannot speak 
verbal components, use command words, or activate magical effects.

\vspace{12pt}
Wind Walk

Transmutation [Air]

\textbf{Level:} Clr 6, Drd 7

\textbf{Components:} V, S, DF

\textbf{Casting Time:} 1 standard action

\textbf{Range:} Touch

\textbf{Targets:} You and one touched creature per three levels

\textbf{Duration:} 1 hour/level (D); see text

\textbf{Saving Throw:} No and Will negates (harmless)

\textbf{Spell Resistance:} No and Yes (harmless)

You alter the substance of your body to a cloudlike vapor (as the \textit{gaseous 
form }spell) and move through the air, possibly at great speed. You can take other 
creatures with you, each of which acts independently.

Normally, a \textit{wind walker }flies at a speed of 10 feet with perfect maneuverability. 
If desired by the subject, a magical wind wafts a \textit{wind walker }along at 
up to 600 feet per round (60 mph) with poor maneuverability. \textit{Wind walkers 
}are not invisible but rather appear misty and translucent. If fully clothed in 
white, they are 80\% likely to be mistaken for clouds, fog, vapors, or the like.

A \textit{wind walker }can regain its physical form as desired and later resume 
the cloud form. Each change to and from vaporous form takes 5 rounds, which counts 
toward the duration of the spell (as does any time spent in physical form). As 
noted above, you can dismiss the spell, and you can even dismiss it for individual 
wind walkers and not others.

For the last minute of the spell's duration, a \textit{wind walker }in cloud form 
automatically descends 60 feet per round (for a total of 600 feet), though it may 
descend faster if it wishes. This descent serves as a warning that the spell is 
about to end.

\vspace{12pt}
Wind Wall

Evocation [Air]

\textbf{Level:} Air 2, Clr 3, Drd 3, Rgr 2, Sor/Wiz 3

\textbf{Components:} V, S, M/DF

\textbf{Casting Time:} 1 standard action

\textbf{Range: }Medium (100 ft. + 10 ft./level)

\textbf{Effect: }Wall up to 10 ft./level long and 5 ft./level high (S)

\textbf{Duration:} 1 round/level

\textbf{Saving Throw:} None; see text

\textbf{Spell Resistance:} Yes

An invisible vertical curtain of wind appears. It is 2 feet thick and of considerable 
strength. It is a roaring blast sufficient to blow away any bird smaller than an 
eagle, or tear papers and similar materials from unsuspecting hands. (A Reflex 
save allows a creature to maintain its grasp on an object.) Tiny and Small flying 
creatures cannot pass through the barrier. Loose materials and cloth garments fly 
upward when caught in a \textit{wind wall. }Arrows and bolts are deflected upward 
and miss, while any other normal ranged weapon passing through the wall has a 30\% 
miss chance. (A giant-thrown boulder, a siege engine projectile, and other massive 
ranged weapons are not affected.) Gases, most gaseous breath weapons, and creatures 
in gaseous form cannot pass through the wall (although it is no barrier to incorporeal 
creatures).

While the wall must be vertical, you can shape it in any continuous path along 
the ground that you like. It is possible to create cylindrical or square wind walls 
to enclose specific points. 

\textit{Arcane Material Component: }A tiny fan and a feather of exotic origin.

\vspace{12pt}
Wish

Universal

\textbf{Level:} Sor/Wiz 9

\textbf{Components:} V, XP

\textbf{Casting Time:} 1 standard action

\textbf{Range:} See text

\textbf{Target, Effect, or Area:} See text

\textbf{Duration:} See text

\textbf{Saving Throw:} See text

\textbf{Spell Resistance:} Yes

\textit{Wish }is the mightiest spell a wizard or sorcerer can cast. By simply speaking 
aloud, you can alter reality to better suit you.

Even \textit{wish}, however, has its limits.

A \textit{wish }can produce any one of the following effects.• 

Duplicate any wizard or sorcerer spell of 8th level or lower, provided the spell 
is not of a school prohibited to you.• 

Duplicate any other spell of 6th level or lower, provided the spell is not of a 
school prohibited to you.• 

Duplicate any wizard or sorcerer spell of 7th level or lower even if it's of a 
prohibited school.• 

Duplicate any other spell of 5th level or lower even if it's of a prohibited school. 
• 

Undo the harmful effects of many other spells, such as \textit{geas/quest }or \textit{insanity}.• 

Create a nonmagical item of up to 25,000 gp in value.• 

Create a magic item, or add to the powers of an existing magic item.• 

Grant a creature a +1 inherent bonus to an ability score. Two to five \textit{wish 
}spells cast in immediate succession can grant a creature a +2 to +5 inherent bonus 
to an ability score (two wishes for a +2 inherent bonus, three for a +3 inherent 
bonus, and so on). Inherent bonuses are instantaneous, so they cannot be dispelled. 
\textit{Note: }An inherent bonus may not exceed +5 for a single ability score, 
and inherent bonuses to a particular ability score do not stack, so only the best 
one applies.• 

Remove injuries and afflictions. A single \textit{wish }can aid one creature per 
caster level, and all subjects are cured of the same kind of affliction. For example, 
you could heal all the damage you and your companions have taken, or remove all 
poison effects from everyone in the party, but not do both with the same \textit{wish}. 
A \textit{wish }can never restore the experience point loss from casting a spell 
or the level or Constitution loss from being raised from the dead.• 

Revive the dead. A \textit{wish }can bring a dead creature back to life by duplicating 
a \textit{resurrection }spell. A \textit{wish }can revive a dead creature whose 
body has been destroyed, but the task takes two \textit{wishes, }one to recreate 
the body and another to infuse the body with life again. A \textit{wish }cannot 
prevent a character who was brought back to life from losing an experience level.• 

Transport travelers. A \textit{wish }can lift one creature per caster level from 
anywhere on any plane and place those creatures anywhere else on any plane regardless 
of local conditions. An unwilling target gets a Will save to negate the effect, 
and spell resistance (if any) applies.• 

Undo misfortune. A \textit{wish }can undo a single recent event. The \textit{wish 
}forces a reroll of any roll made within the last round (including your last turn). 
Reality reshapes itself to accommodate the new result. For example, a \textit{wish 
}could undo an opponent's successful save, a foe's successful critical hit (either 
the attack roll or the critical roll), a friend's failed save, and so on. The reroll, 
however, may be as bad as or worse than the original roll. An unwilling target 
gets a Will save to negate the effect, and spell resistance (if any) applies.

You may try to use a \textit{wish }to produce greater effects than these, but doing 
so is dangerous. (The \textit{wish }may pervert your intent into a literal but 
undesirable fulfillment or only a partial fulfillment.)

Duplicated spells allow saves and spell resistance as normal (but save DCs are 
for 9th-level spells).

\textit{Material Component: }When a \textit{wish }duplicates a spell with a material 
component that costs more than 10,000 gp, you must provide that component.

\textit{XP Cost: }The minimum XP cost for casting \textit{wish }is 5,000 XP. When 
a \textit{wish }duplicates a spell that has an XP cost, you must pay 5,000 XP or 
that cost, whichever is more. When a \textit{wish }creates or improves a magic 
item, you must pay twice the normal XP cost for crafting or improving the item, 
plus an additional 5,000 XP.

\vspace{12pt}
Wood Shape

Transmutation

\textbf{Level:} Drd 2

\textbf{Components:} V, S, DF

\textbf{Casting Time:} 1 standard action

\textbf{Range:} Touch

\textbf{Target:} One touched piece of wood no larger than 10 cu. ft. + 1 cu. ft./level

\textbf{Duration:} Instantaneous

\textbf{Saving Throw: }Will negates (object)

\textbf{Spell Resistance:} Yes (object)

W\textit{ood shape }enables you to form one existing piece of wood into any shape 
that suits your purpose. While it is possible to make crude coffers, doors, and 
so forth, fine detail isn't possible. There is a 30\% chance that any shape that 
includes moving parts simply doesn't work.

\vspace{12pt}
Word of Chaos

Evocation [Chaotic, Sonic]

\textbf{Level:} Chaos 7, Clr 7

\textbf{Components:} V

\textbf{Casting Time:} 1 standard action

\textbf{Range:} 40 ft.

\textbf{Area:} Nonchaotic creatures in a 40-ft.- radius spread centered on you

\textbf{Duration:} Instantaneous

\textbf{Saving Throw:} None or Will negates; see text

\textbf{Spell Resistance:} Yes

Any nonchaotic creature within the area who hears the \textit{word of chaos }suffers 
the following ill effects.

The effects are cumulative and concurrent. No saving throw is allowed against these 
effects.

\textit{Deafened: }The creature is deafened for 1d4 rounds.

\textit{Stunned: }The creature is stunned for 1 round.

\textit{Confused: }The creature is \textit{confused, }as by the \textit{confusion 
}spell, for 1d10 minutes. This is a mind-affecting enchantment effect.

\textit{Killed: }Living creatures die. Undead creatures are destroyed.

\begin{tabular}{|>{\raggedright}p{86pt}|>{\raggedright}p{141pt}|}
\hline
H\textbf{D} & E\textbf{ffect}\tabularnewline
\hline
Equal to caster level & Deafened\tabularnewline
\hline
Up to caster level -1 & Stunned, deafened\tabularnewline
\hline
Up to caster level -5 & C\textit{onfused}, stunned, deafened\tabularnewline
\hline
Up to caster level -10 & Killed, \textit{confused}, stunned, deafened\tabularnewline
\hline
\end{tabular}

Furthermore, if you are on your home plane when you cast this spell, nonchaotic 
extraplanar creatures within the area are instantly banished back to their home 
planes. Creatures so banished cannot return for at least 24 hours. This effect 
takes place regardless of whether the creatures hear the \textit{word of chaos. 
}The banishment effect allows a Will save (at a -4 penalty) to negate.

Creatures whose HD exceed your caster level are unaffected by \textit{word of chaos.}

\vspace{12pt}
Word of Recall

Conjuration (Teleportation)

\textbf{Level:} Clr 6, Drd 8

\textbf{Components:} V

\textbf{Casting Time:} 1 standard action

\textbf{Range:} Unlimited

\textbf{Target:} You and touched objects or other willing creatures

\textbf{Duration:} Instantaneous

\textbf{Saving Throw:} None or Will negates (harmless, object)

\textbf{Spell Resistance:} No or Yes (harmless, object)

\textit{Word of recall }teleports you instantly back to your sanctuary when the 
word is uttered. You must designate the sanctuary when you prepare the spell, and 
it must be a very familiar place. The actual point of arrival is a designated area 
no larger than 10 feet by 10 feet. You can be transported any distance within a 
plane but cannot travel between planes. You can transport, in addition to yourself, 
any objects you carry, as long as their weight doesn't exceed your maximum load. 
You may also bring one additional willing Medium or smaller creature (carrying 
gear or objects up to its maximum load) or its equivalent per three caster levels. 
A Large creature counts as two Medium creatures, a Huge creature counts as two 
Large creatures, and so forth. All creatures to be transported must be in contact 
with one another, and at least one of those creatures must be in contact with you. 
Exceeding this limit causes the spell to fail.

An unwilling creature can't be teleported by \textit{word of recall. }Likewise, 
a creature's Will save (or spell resistance) prevents items in its possession from 
being teleported. Unattended, nonmagical objects receive no saving throw.

\vspace{12pt}
Zone of Silence

Illusion (Glamer)

\textbf{Level:} Brd 4

\textbf{Components:} V, S

\textbf{Casting Time:} 1 round

\textbf{Range:} Personal

\textbf{Area:} 5-ft.-radius emanation centered on you

\textbf{Duration:} 1 hour/level (D)

By casting \textit{zone of silence}, you manipulate sound waves in your immediate 
vicinity so that you and those within the spell's area can converse normally, yet 
no one outside can hear your voices or any other noises from within, including 
language-dependent or sonic spell effects. This effect is centered on you and moves 
with you. Anyone who enters the zone immediately becomes subject to its effects, 
but those who leave are no longer affected. Note, however, that a successful Spot 
check to read lips can still reveal what's said inside a \textit{zone of silence.}

\vspace{12pt}
Zone of Truth

Enchantment (Compulsion) [Mind-Affecting]

\textbf{Level:} Clr 2, Pal 2

\textbf{Components:} V, S, DF

\textbf{Casting Time:} 1 standard action

\textbf{Range:} Close (25 ft. + 5 ft./2 levels)

\textbf{Area:} 20-ft.-radius emanation

\textbf{Duration:} 1 min./level

\textbf{Saving Throw: }Will negates

\textbf{Spell Resistance:} Yes

Creatures within the emanation area (or those who enter it) can't speak any deliberate 
and intentional lies. Each potentially affected creature is allowed a save to avoid 
the effects when the spell is cast or when the creature first enters the emanation 
area. Affected creatures are aware of this enchantment. Therefore, they may avoid 
answering questions to which they would normally respond with a lie, or they may 
be evasive as long as they remain within the boundaries of the truth. Creatures 
who leave the area are free to speak as they choose.

\newpage

\end{document}
