%&pdfLaTeX
% !TEX encoding = UTF-8 Unicode
\documentclass{article}
\usepackage{ifxetex}
\ifxetex
\usepackage{fontspec}
\setmainfont[Mapping=tex-text]{STIXGeneral}
\else
\usepackage[T1]{fontenc}
\usepackage[utf8]{inputenc}
\fi
\usepackage{textcomp}

\usepackage{array}
\usepackage{amssymb}
\usepackage{fancyhdr}
\renewcommand{\headrulewidth}{0pt}
\renewcommand{\footrulewidth}{0pt}

\begin{document}

This material is Open Game Content, and is licensed for public use under the terms 
of the Open Game License v1.0a.

{\LARGE{}MAGIC ITEMS I (BASICS \& CREATION)}

\vspace{12pt}
Magic items are divided into categories: armor, weapons, potions, rings, rods, 
scrolls, staffs, wands, and wondrous items. In addition, some magic items are cursed 
or intelligent. Finally, a few magic items are of such rarity and power that they 
are considered to belong to a category of their own: artifacts. Artifacts are classified 
in turn as minor (extremely rare but not one-of-a-kind items) or major (each one 
unique and extremely potent).

\textbf{Armor and Shields:} Magic armor (including shields) offers improved, magical 
protection to the wearer. Some of these items confer abilities beyond a benefit 
to Armor Class. 

\textbf{Weapons:} Magic weapons are created with a variety of combat powers and 
almost always improve the attack and damage rolls of the wielder as well.

\textbf{Potions:} A potion is an elixir concocted with a spell-like effect that 
affects only the drinker.

\textbf{Rings:} A ring is a circular metal band worn on the finger (no more than 
two rings per wearer) that has a spell-like power (often a constant effect that 
affects the wearer).

\textbf{Rods: }A rod is a scepter-like item with a special power unlike that of 
any known spell.

\textbf{Scrolls:} A scroll is a spell magically inscribed onto paper or parchment 
so that it can be used later.

\textbf{Staffs:} A staff has a number of different (but often related) spell effects. 
A newly created staff has 50 charges, and each use of the staff depletes one or 
more of those charges.

\textbf{Wands:} A wand is a short stick imbued with the power to cast a specific 
spell. A newly created wand has 50 charges, and each use of the wand depletes one 
of those charges.

\textbf{Wondrous Items:} These objects include magic jewelry, tools, books, clothing, 
and much more.

\vspace{12pt}
Magic Items and Detect Magic

When \textit{detect magic }identifies a magic item's school of magic, this information 
refers to the school of the spell placed within the potion, scroll, or wand, or 
the prerequisite given for the item. The description of each item provides its 
aura strength and the school it belongs to.

If more than one spell is given as a prerequisite, use the highest-level spell. 
If no spells are included in the prerequisites, use the following default guidelines.

\begin{tabular}{|>{\raggedright}p{166pt}|>{\raggedright}p{57pt}|}
\hline
I\textbf{tem Nature} & S\textbf{chool}\tabularnewline
\hline
Armor and protection items & Abjuration\tabularnewline
\hline
Weapons or offensive items & Evocation\tabularnewline
\hline
Bonus to ability score, on skill check, etc. & Transmutation\tabularnewline
\hline
\end{tabular}

\vspace{12pt}
{\large{}USING ITEMS}

To use a magic item, it must be activated, although sometimes activation simply 
means putting a ring on your finger. Some items, once donned, function constantly. 
In most cases, using an item requires a standard action that does not provoke attacks 
of opportunity. By contrast, spell completion items are treated like spells in 
combat and do provoke attacks of opportunity.

Activating a magic item is a standard action unless the item description indicates 
otherwise. However, the casting time of a spell is the time required to activate 
the same power in an item, regardless of the type of magic item, unless the item 
description specifically states otherwise.

The four ways to activate magic items are described below.

\textbf{Spell Completion:} This is the activation method for scrolls. A scroll 
is a spell that is mostly finished. The preparation is done for the caster, so 
no preparation time is needed beforehand as with normal spellcasting. All that's 
left to do is perform the finishing parts of the spellcasting (the final gestures, 
words, and so on). To use a spell completion item safely, a character must be of 
high enough level in the right class to cast the spell already. If he can't already 
cast the spell, there's a chance he'll make a mistake. Activating a spell completion 
item is a standard action and provokes attacks of opportunity exactly as casting 
a spell does.

\textbf{Spell Trigger:} Spell trigger activation is similar to spell completion, 
but it's even simpler. No gestures or spell finishing is needed, just a special 
knowledge of spellcasting that an appropriate character would know, and a single 
word that must be spoken. Anyone with a spell on his or her spell list knows how 
to use a spell trigger item that stores that spell. (This is the case even for 
a character who can't actually cast spells, such as a 3rd-level paladin.) The user 
must still determine what spell is stored in the item before she can activate it. 
Activating a spell trigger item is a standard action and does not provoke attacks 
of opportunity.

\textbf{Command Word: }If no activation method is suggested either in the magic 
item description or by the nature of the item, assume that a command word is needed 
to activate it. Command word activation means that a character speaks the word 
and the item activates. No other special knowledge is needed.

A command word can be a real word, but when this is the case, the holder of the 
item runs the risk of activating the item accidentally by speaking the word in 
normal conversation. More often, the command word is some seemingly nonsensical 
word, or a word or phrase from an ancient language no longer in common use. Activating 
a command word magic item is a standard action and does not provoke attacks of 
opportunity.

Sometimes the command word to activate an item is written right on the item. Occasionally, 
it might be hidden within a pattern or design engraved on, carved into, or built 
into the item, or the item might bear a clue to the command word.

The Knowledge (arcana) and Knowledge (history) skills might be useful in helping 
to identify command words or deciphering clues regarding them. A successful check 
against DC 30 is needed to come up with the word itself. If that check is failed, 
succeeding on a second check (DC 25) might provide some insight into a clue.

The spells \textit{identify }and \textit{analyze dweomer }both reveal command words.

\textbf{Use Activated:} This type of item simply has to be used in order to activate 
it. A character has to drink a potion, swing a sword, interpose a shield to deflect 
a blow in combat, look through a lens, sprinkle dust, wear a ring, or don a hat. 
Use activation is generally straightforward and self-explanatory.

Many use-activated items are objects that a character wears. Continually functioning 
items are practically always items that one wears. A few must simply be in the 
character's possession (on his person). However, some items made for wearing\textit{ 
}must still be activated. Although this activation sometimes requires a command 
word (see above), usually it means mentally willing the activation to happen. The 
description of an item states whether a command word is needed in such a case.

Unless stated otherwise, activating a use-activated magic item is either a standard 
action or not an action at all and does not provoke attacks of opportunity, unless 
the use involves performing an action that provokes an attack of opportunity in 
itself. If the use of the item takes time before a magical effect occurs, then 
use activation is a standard action. If the item's activation is subsumed in its 
use and takes no extra time use activation is not an action at all.

Use activation doesn't mean that if you use an item, you automatically know what 
it can do. You must know (or at least guess) what the item can do and then use 
the item in order to activate it, unless the benefit of the item comes automatically, 
such from drinking a potion or swinging a sword.

\vspace{12pt}
{\large{}SIZE AND MAGIC ITEMS}

When an article of magic clothing or jewelry is discovered, most of the time size 
shouldn't be an issue. Many magic garments are made to be easily adjustable, or 
they adjust themselves magically to the wearer. Size should not keep characters 
of various kinds from using magic items.

There may be rare exceptions, especially with racial specific items. 

\textbf{Armor and Weapon Sizes: }Armor and weapons that are found at random have 
a 30\% chance of being Small (01-30), a 60\% chance of being Medium (31-90), and 
a 10\% chance of being any other size (91-100).

\vspace{12pt}
{\large{}MAGIC ITEMS ON THE BODY}

Many magic items need to be donned by a character who wants to employ them or benefit 
from their abilities. It's possible for a creature with a humanoid-shaped body 
to wear as many as twelve magic items at the same time. However, each of those 
items must be worn on (or over) a particular part of the body.

A humanoid-shaped body can be decked out in magic gear consisting of one item from 
each of the following groups, keyed to which place on the body the item is worn.• 

\parindent=3pt
One headband, hat, helmet, or phylactery on the head• 

One pair of eye lenses or goggles on or over the eyes• 

\parindent=7pt
One amulet, brooch, medallion, necklace, periapt, or scarab around the neck• 

\parindent=3pt
One vest, vestment, or shirt on the torso• 

One robe or suit of armor on the body (over a vest, vestment, or shirt)• 

\parindent=7pt
One belt around the waist (over a robe or suit of armor)• 

\parindent=3pt
One cloak, cape, or mantle around the shoulders (over a robe or suit of armor)• 

One pair of bracers or bracelets on the arms or wrists• 

\parindent=7pt
One glove, pair of gloves, or pair of gauntlets on the hands• 

\parindent=3pt
One ring on each hand (or two rings on one hand)• 

One pair of boots or shoes on the feet

Of course, a character may carry or possess as many items of the same type as he 
wishes. However, additional items beyond those listed above have no effect. 

Some items\textit{ }can be worn or carried without taking up space on a character's 
body. The description of an item indicates when an item has this property.

\vspace{12pt}
\parindent=0pt
{\large{}SAVING THROWS AGAINST MAGIC ITEM POWERS}

Magic items produce spells or spell-like effects. For a saving throw against a 
spell or spell-like effect from a magic item, the DC is 10 + the level of the spell 
or effect + the ability modifier of the minimum ability score needed to cast that 
level of spell. 

Staffs are an exception to the rule. Treat the saving throw as if the wielder cast 
the spell, including caster level and all modifiers to save DC. 

Most item descriptions give saving throw DCs for various effects, particularly 
when the effect has no exact spell equivalent (making its level otherwise difficult 
to determine quickly).

\vspace{12pt}
{\large{}DAMAGING MAGIC ITEMS}

A magic item doesn't need to make a saving throw unless it is unattended, it is 
specifically targeted by the effect, or its wielder rolls a natural 1 on his save. 
Magic items should always get a saving throw against spells that might deal damage 
to them--- even against attacks from which a nonmagical item would normally get 
no chance to save. Magic items use the same saving throw bonus for all saves, no 
matter what the type (Fortitude, Reflex, or Will). A magic item's saving throw 
bonus equals 2 + one-half its caster level (round down). The only exceptions to 
this are intelligent magic items, which make Will saves based on their own Wisdom 
scores.

Magic items, unless otherwise noted, take damage as nonmagical items of the same 
sort. A damaged magic item continues to function, but if it is destroyed, all its 
magical power is lost.

\vspace{12pt}
{\large{}REPAIRING MAGIC ITEMS}

Some magic items take damage over the course of an adventure. It costs no more 
to repair a magic item with the Craft skill than it does to repair its nonmagical 
counterpart. The \textit{make whole }spell also repairs a damaged---but not completely 
broken---magic item.

\vspace{12pt}
{\large{}INTELLIGENT ITEMS}

Some magic items, particularly weapons, have an intelligence all their own. Only 
permanent magic items (as opposed to those with a single use or those with charges) 
can be intelligent. (This means that potions, scrolls, and wands, among other items, 
are never intelligent.)

In general, less than 1\% of magic items have intelligence. 

\vspace{12pt}
{\large{}CURSED ITEMS}

Some items are cursed---incorrectly made, or corrupted by outside forces. Cursed 
items might be particularly dangerous to the user, or they might be normal items 
with a minor flaw, an inconvenient requirement, or an unpredictable nature. Randomly 
generated items are cursed 5\% of the time. 

\vspace{12pt}
{\large{}CHARGES, DOSES, AND MULTIPLE USES}

Many items, particularly wands and staffs, are limited in power by the number of 
charges they hold. Normally, charged items have 50 charges at most. If such an 
item is found as a random part of a treasure, roll d\% and divide by 2 to determine 
the number of charges left (round down, minimum 1). If the item has a maximum number 
of charges other than 50, roll randomly to determine how many charges are left. 

Prices listed are always for fully charged items. (When an item is created, it 
is fully charged.) For an item that's worthless when its charges run out (which 
is the case for almost all charged items), the value of the partially used item 
is proportional to the number of charges left. For an item that has usefulness 
in addition to its charges, only part of the item's value is based on the number 
of charges left.

\vspace{12pt}
{\huge{}MAGIC ITEM DESCRIPTIONS}

Each general type of magic item gets an overall description, followed by descriptions 
of specific items.

General descriptions include notes on activation, random generation, and other 
material. The AC, hardness, hit points, and break DC are given for typical examples 
of some magic items. The AC assumes that the item is unattended and includes a 
-5 penalty for the item's effective Dexterity of 0. If a creature holds the item, 
use the creature's Dexterity modifier in place of the -5 penalty.

Some individual items, notably those that simply store spells and nothing else, 
don't get full-blown descriptions. Reference the spell's description for details, 
modified by the form of the item (potion, scroll, wand, and so on). Assume that 
the spell is cast at the minimum level required to cast it

Items with full descriptions have their powers detailed, and each of the following 
topics is covered in notational form at the end of the description.• 

\parindent=3pt
Aura: Most of the time, a \textit{detect magic }spell will reveal the school of 
magic associated with a magic item and the strength of the aura an item emits. 
This information (when applicable) is given at the beginning of the item's notational 
entry. See the \textit{detect magic }spell description\textit{ }for details.• 

\parindent=0pt
Caster Level: The next item in a notational entry gives the caster level of the 
item, indicating its relative power. The caster level determines the item's saving 
throw bonus, as well as range or other level-dependent aspects of the powers of 
the item (if variable). It also determines the level that must be contended with 
should the item come under the effect of a \textit{dispel magic }spell or similar 
situation. This information is given in the form ``CL x,'' where ``CL'' is an abbreviation 
for caster level and ``x'' is an ordinal number representing the caster level itself.

For potions, scrolls, and wands, the creator can set the caster level of an item 
at any number high enough to cast the stored spell and not higher than her own 
caster level. For other magic items, the caster level is determined by the item 
itself. In this case, the creator's caster level must be as high as the item's 
caster level (and prerequisites may effectively put a higher minimum on the creator's 
level).• 

\parindent=3pt
Prerequisites: Certain requirements must be met in order for a character to create 
a magic item. These include feats, spells, and miscellaneous requirements such 
as level, alignment, and race or kind. The prerequisites for creation of an item 
are given immediately following the item's caster level.

\parindent=0pt
A spell prerequisite may be provided by a character who has prepared the spell 
(or who knows the spell, in the case of a sorcerer or bard), or through the use 
of a spell completion or spell trigger magic item or a spell-like ability that 
produces the desired spell effect. For each day that passes in the creation process, 
the creator must expend one spell completion item or one charge from a spell trigger 
item if either of those objects is used to supply a prerequisite.

It is possible for more than one character to cooperate in the creation of an item, 
with each participant providing one or more of the prerequisites. In some cases, 
cooperation may even be necessary.

If two or more characters cooperate to create an item, they must agree among themselves 
who will be considered the creator for the purpose of determinations where the 
creator's level must be known. The character designated as the creator pays the 
XP required to make the item.

Typically, a list of prerequisites includes one feat and one or more spells (or 
some other requirement in addition to the feat).

When two spells at the end of a list are separated by ``or,'' one of those spells 
is required in addition to every other spell mentioned prior to the last two.• 

\parindent=3pt
Market Price: This gold piece value, given following the word ``Price,'' represents 
the price someone should expect to pay to buy the item. The market price for an 
item that can be constructed with an item creation feat is usually equal to the 
base price plus the price for any components (material or XP).• 

Cost to Create: The next part of a notational entry is the cost in gp and XP to 
create the item, given following the word

``Cost.'' This information appears only for items with components (material or 
XP), which make their market prices higher than their base prices. The cost to 
create includes the costs derived from the base cost plus the costs of the components.

Items without components do not have a ``Cost'' entry. For them, the market price 
and the base price are the same. The cost in gp is 1/2 the market price, and the 
cost in XP is 1/25 the market price.• 

\parindent=7pt
Weight: The notational entry for many wondrous items ends with a value for the 
item's weight. When a weight figure is not given, the item has no weight worth 
noting (for purposes of determining how much of a load a character can carry).

\vspace{12pt}
\parindent=0pt
\begin{tabular}{|>{\raggedright}p{28pt}|>{\raggedright}p{34pt}|>{\raggedright}p{28pt}|>{\raggedright}p{72pt}|}
\hline
\multicolumn{4}{|p{165pt}|}{T\textbf{able: Random Magic Item Generation}}\tabularnewline
\hline
M\textbf{inor} & M\textbf{edium} & M\textbf{ajor} & I\textbf{tem}\tabularnewline
\hline
01-04 & 01-10 & 01-10 & Armor and shields\tabularnewline
\hline
05-09 & 11-20 & 11-20 & Weapons\tabularnewline
\hline
10-44 & 21-30 & 21-25 & Potions\tabularnewline
\hline
45-46 & 31-40 & 26-35 & Rings\tabularnewline
\hline
--- & 41-50 & 36-45 & Rods\tabularnewline
\hline
47-81 & 51-65 & 46-55 & Scrolls\tabularnewline
\hline
--- & 66-68 & 56-75 & Staffs\tabularnewline
\hline
82-91 & 69-83 & 76-80 & Wands\tabularnewline
\hline
92-100 & 84-100 & 81-100 & Wondrous items\tabularnewline
\hline
\end{tabular}

\vspace{12pt}
{\LARGE{}CREATING MAGIC ITEMS}

To create magic items, spellcasters use special feats. They invest time, money, 
and their own personal energy (in the form of experience points) in an item's creation.

Note that all items have prerequisites in their descriptions. These prerequisites 
must be met for the item to be created. Most of the time, they take the form of 
spells that must be known by the item's creator (although access through another 
magic item or spellcaster is allowed).

While item creation costs are handled in detail below, note that normally the two 
primary factors are the caster level of the creator and the level of the spell 
or spells put into the item. A creator can create an item at a lower caster level 
than her own, but never lower than the minimum level needed to cast the needed 
spell. Using metamagic feats, a caster can place spells in items at a higher level 
than normal.

Magic supplies for items are always half of the base price in gp and 1/25 of the 
base price in XP. For many items, the market price equals the base price.

Armor, shields, weapons, and items with a value independent of their magically 
enhanced properties add their item cost to the market price. The item cost does 
not influence the base price (which determines the cost of magic supplies and the 
experience point cost), but it does increase the final market price.

In addition, some items cast or replicate spells with costly material components 
or with XP components. For these items, the market price equals the base price 
plus an extra price for the spell component costs. Each XP in the component costs 
adds 5 gp to the market price. The cost to create these items is the magic supplies 
cost and the base XP cost (both determined by the base price) plus the costs for 
the components. Descriptions of these items include an entry that gives the total 
cost of creating the item.

The creator also needs a fairly quiet, comfortable, and well-lit place in which 
to work. Any place suitable for preparing spells is suitable for making items. 
Creating an item requires one day per 1,000 gp in the item's base price, with a 
minimum of at least one day. Potions are an exception to this rule; they always 
take just one day to brew. The character must spend the gold and XP at the beginning 
of the construction process.

The caster works for 8 hours each day. He cannot rush the process by working longer 
each day. But the days need not be consecutive, and the caster can use the rest 
of his time as he sees fit.

A character can work on only one item at a time. If a character starts work on 
a new item, all materials used and XP spent on the under-construction item are 
wasted.

The secrets of creating artifacts are long lost.

\vspace{12pt}
\begin{tabular}{|>{\raggedright}p{30pt}|>{\raggedright}p{40pt}|>{\raggedright}p{29pt}|>{\raggedright}p{34pt}|>{\raggedright}p{33pt}|>{\raggedright}p{47pt}|>{\raggedright}p{49pt}|}
\hline
\multicolumn{7}{|p{266pt}|}{T\textbf{able: Summary of Magic Item Creation Costs}}\tabularnewline
\hline
  &   &   & \multicolumn{2}{p{68pt}|}{S\textbf{pell Component Costs}} &  & \tabularnewline
\hline
M\textbf{agic Item } & F\textbf{eat } & I\textbf{tem Cost } & M\textbf{aterial}\textsuperscript{\textbf{2}}\textbf{ 
} & X\textbf{P}\textsuperscript{\textbf{3}}\textbf{ } & M\textbf{agic Supplies 
Cost } & B\textbf{ase Price}\textsuperscript{\textbf{4}}\tabularnewline
\hline
Armor  & Craft Magic Arms and Armor & Masterwork armor  & Cost x 50 (usually none) & x 
50 (usually none) \linebreak{}
x 5 gp & 1/2 the value on Table: Armor and Shields & Value on Table: Armor and 
Shields\tabularnewline
\hline
Shield   & Craft Magic Arms and Armor & Masterwork shield  & x 50 (usually none) & x$ $ 
50 (usually none) \linebreak{}
x 5 gp & 1/2 the value on Table: Armor and Shields & Value on Table: Armor and 
Shields\tabularnewline
\hline
Weapon  & Craft Magic Arms and Armor & Masterwork weapon  & x 50 (usually none) 
 & x 50 (usually none) \linebreak{}
x 5 gp & 1/2 the value on Table: Weapons  & Value on Table: Weapons\tabularnewline
\hline
Potion  & Brew Potion &  ---  & Cost (usually none)  & Cost (usually none)  & 1/2 
x 25 x level of spell x level of caster & 25 x level of spell x level of caster\tabularnewline
\hline
Ring  & Forge Ring --- &  &  x 50  & x 50 \linebreak{}
x 5 gp & Special, see Table: Estimating Magic Item Gold Price Values, below & Special, 
see Table: Estimating Magic Item Gold Price Values, below\tabularnewline
\hline
Rod  & Craft Rod  & 1\textsuperscript{\textbf{ }} & x 50 (often none) & x 50 (often 
none)  & Special, see Table: Estimating Magic Item Gold Price Values, below & Special, 
see Table: Estimating Magic Item Gold Price Values, below\tabularnewline
\hline
Scroll  & Scribe Scroll --- &   & Cost (usually none)  & Cost (usually none)  & 1/2 
x 12.5 x level of spell x level of caster & 12.5 x level of spell x level of caster\tabularnewline
\hline
Staff  & Craft Staff  & Masterwork quarterstaff (300 gp) & x 50 / (\# of charges 
used to activate spell) & x 50 x 5 gp / (\# of charges used to activate spell) & See 
Creating Staffs, below  & See Creating Staffs, below\tabularnewline
\hline
Wand  & Craft Wand --- &   & x 50  & x 50 \linebreak{}
x 5 gp  & 1/2 x 375 x level of spell x level of caster $ $ & 375 x level of spell 
x level of caster\tabularnewline
\hline
Wondrous \linebreak{}
Item  & Craft Wondrous Item  & 5\textsuperscript{\textbf{ }} & x 50 (usually none) & x 
50 (usually none)\linebreak{}
x 5 gp & Special, see Table: Estimating Magic Item Gold Price Values, below & Special, 
see Table: Estimating Magic Item Gold Price Values, below\tabularnewline
\hline
\multicolumn{7}{|p{266pt}|}{1 Rods usable as weapons must include the masterwork 
weapon cost.}\tabularnewline
\hline
\multicolumn{7}{|p{266pt}|}{2 This cost is only for spells activated by the item 
that have material or XP components. Having a spell with a costly component as 
a prerequisite does not automatically incur this cost if the item doesn't actually 
cast the spell. }\tabularnewline
\hline
\multicolumn{7}{|p{266pt}|}{3 If purchasing a staff, the buyer pays 5 x the XP 
value in gold pieces.}\tabularnewline
\hline
\multicolumn{7}{|p{266pt}|}{4 A character creating an item pays 1/25 the base price 
in experience points.}\tabularnewline
\hline
\multicolumn{7}{|p{266pt}|}{5 Some items have additional value from a masterwork 
item component.\textit{ }}\tabularnewline
\hline
\multicolumn{7}{|p{266pt}|}{A\textbf{n item's market price is the sum of the item 
cost, spell component costs, and the base price.}}\tabularnewline
\hline
\end{tabular}

\vspace{12pt}
\begin{tabular}{|>{\raggedright}p{109pt}|>{\raggedright}p{117pt}|>{\raggedright}p{87pt}|}
\hline
\multicolumn{3}{|p{314pt}|}{T\textbf{able: Estimating Magic Item Gold Piece Values}}\tabularnewline
\hline
E\textbf{ffect } & B\textbf{ase Price } & E\textbf{xample}\tabularnewline
\hline
Ability bonus (enhancement)  & Bonus squared  x 1,000 gp  & G\textit{loves of Dexterity 
+2}\tabularnewline
\hline
Armor bonus (enhancement)  & Bonus squared  x 1,000 gp  & +\textit{1 chainmail}\tabularnewline
\hline
Bonus spell  & Spell level squared  x 1,000 gp  & P\textit{earl of power}\tabularnewline
\hline
AC bonus (deflection)  & Bonus squared  x 2,000 gp  & R\textit{ing of protection 
+3}\tabularnewline
\hline
AC bonus (other)\textsuperscript{\textbf{1}}\textbf{ } & Bonus squared  x 2,500 
gp  & I\textit{oun stone, dusty rose prism}\tabularnewline
\hline
Natural armor bonus (enhancement)  & Bonus squared  x 2,000 gp  & A\textit{mulet 
of natural armor +1}\tabularnewline
\hline
Save bonus (resistance)  & Bonus squared  x 1,000 gp  & C\textit{loak of resistance 
+5}\tabularnewline
\hline
Save bonus (other)\textsuperscript{\textbf{1}}\textbf{ } & Bonus squared  x 2,000 
gp  & S\textit{tone of good luck}\tabularnewline
\hline
Skill bonus (competence)  & Bonus squared  x 100 gp  & C\textit{loak of elvenkind}\tabularnewline
\hline
Spell resistance  & 10,000 gp per point over SR 12; \linebreak{}
SR 13 minimum  & M\textit{antle of spell resistance}\tabularnewline
\hline
Weapon bonus (enhancement)  & Bonus squared  x 2,000 gp  & +\textit{1 longsword}\tabularnewline
\hline
S\textbf{pell Effect } & B\textbf{ase Price } & E\textbf{xample}\tabularnewline
\hline
Single use, spell completion  & Spell level  x caster level  x 25 gp  & Scroll 
of \textit{haste}\tabularnewline
\hline
Single use, use-activated  & Spell level  x caster level  x 50 gp  & P\textit{otion 
of cure light wounds}\tabularnewline
\hline
50 charges, spell trigger  & Spell level  x caster level  x 750 gp  & W\textit{and 
of fireball}\tabularnewline
\hline
Command word  & Spell level  x caster level  x 1,800 gp  & C\textit{ape of the 
mountebank}\tabularnewline
\hline
Use-activated or continuous  & Spell level  x caster level  x 2,000 gp\textsuperscript{\textbf{2}}\textbf{ 
} & L\textit{antern of revealing}\tabularnewline
\hline
S\textbf{pecial } & B\textbf{ase Price Adjustment } & E\textbf{xample}\tabularnewline
\hline
Charges per day  & Divide by (5 divided by charges per day)  & B\textit{oots of 
teleportation}\tabularnewline
\hline
Uncustomary space limitation\textsuperscript{\textbf{3}}\textbf{ } & Multiply entire 
cost by 1.5  & H\textit{elm of teleportation}\tabularnewline
\hline
No space limitation\textsuperscript{\textbf{4}}\textbf{ } & Multiply entire cost 
by 2  & I\textit{oun stone}\tabularnewline
\hline
Multiple different abilities  & Multiply higher item cost by 2  & H\textit{elm 
of brilliance}\tabularnewline
\hline
Charged (50 charges)  & 1/2 unlimited use base price  & R\textit{ing of the ram}\tabularnewline
\hline
C\textbf{omponent } & E\textbf{xtra Cost } & E\textbf{xample}\tabularnewline
\hline
Armor, shield, or weapon  & Add cost of masterwork item  & +\textit{1 composite 
longbow}\tabularnewline
\hline
Spell has material component cost  & Add directly into price of item per charge\textsuperscript{\textbf{5}}\textbf{ 
} & W\textit{and of stoneskin}\tabularnewline
\hline
Spell has XP cost  & Add 5 gp per 1 XP per charge\textsuperscript{\textbf{5}}\textbf{ 
} & R\textit{ing of three wishes}\tabularnewline
\hline
\multicolumn{3}{|p{314pt}|}{S\textit{pell Level: }A 0-level spell is half the value 
of a 1st-level spell for determining price.}\tabularnewline
\hline
\multicolumn{3}{|p{314pt}|}{1 Such as a luck, insight, sacred, or profane bonus.}\tabularnewline
\hline
\multicolumn{3}{|p{314pt}|}{2 If a continuous item has an effect based on a spell 
with a duration measured in rounds, multiply the cost by 4. If the duration of 
the spell is 1 minute/level, multiply the cost by 2, and if the duration is 10 
minutes/level, multiply the cost by 1.5. If the spell has a 24-hour duration or 
greater, divide the cost in half.}\tabularnewline
\hline
\multicolumn{3}{|p{314pt}|}{3 See Body Slot Affinities, below.}\tabularnewline
\hline
\multicolumn{3}{|p{314pt}|}{4 An item that does not take up one of the spaces on 
a body costs double.}\tabularnewline
\hline
\multicolumn{3}{|p{314pt}|}{5 If item is continuous or unlimited, not charged, 
determine cost as if it had 100 charges. If it has some daily limit, determine 
as if it had 50 charges.}\tabularnewline
\hline
\end{tabular}

\vspace{12pt}
MAGIC ITEM GOLD PIECE VALUES

Many factors must be considered when determining the price of new magic items. 
The easiest way to come up with a price is to match the new item to an item that 
is already priced that price as a guide. Otherwise, use the guidelines summarized 
on Table: Estimating Magic Item Gold Piece Values.

\textbf{Multiple Similar Abilities: }For items with multiple similar abilities 
that don't take up space on a character's body use the following formula: Calculate 
the price of the single most costly ability, then add 75\% of the value of the 
next most costly ability, plus one-half the value of any other abilities.

\textbf{Multiple Different Abilities: }Abilities such as an attack roll bonus or 
saving throw bonus and a spell-like function are not similar, and their values 
are simply added together to determine the cost. For items that do take up a space 
on a character's body each additional power not only has no discount but instead 
has a 50\% increase in price.

\textbf{0-Level Spells: }When multiplying spell levels to determine value, 0- level 
spells should be treated as 1/2 level.

\textbf{Other Considerations: }Once you have a final cost figure, reduce that number 
if either of the following conditions applies:---

\textit{Item Requires Skill to Use: }Some items require a specific skill to get 
them to function. This factor should reduce the cost about 10\%.---

\textit{Item Requires Specific Class or Alignment to Use: }Even more restrictive 
than requiring a skill, this limitation cuts the cost by 30\%.

Prices presented in the magic item descriptions (the gold piece value following 
the item's caster level) are the market value, which is generally twice what it 
costs the creator to make the item.

Since different classes get access to certain spells at different levels, the prices 
for two characters to make the same item might actually be different. An item is 
only worth two times what the caster of lowest possible level can make it for. 
Calculate the market price based on the lowest possible level caster, no matter 
who makes the item.

Not all items adhere to these formulas directly. The reasons for this are several. 
First and foremost, these few formulas aren't enough to truly gauge the exact differences 
between items. The price of a magic item may be modified based on its actual worth. 
The formulas only provide a starting point. The pricing of scrolls assumes that, 
whenever possible, a wizard or cleric created it. Potions and wands follow the 
formulas exactly. Staffs follow the formulas closely, and other items require at 
least some judgment calls.

\vspace{12pt}
{\large{}MASTERWORK ITEMS}

Masterwork items are extraordinarily well-made items. They are more expensive, 
but they benefit the user with improved quality. They are not magical in any way. 
However, only masterwork items may be enhanced to become magic armor and weapons. 
(Items that are not weapons or armor may or may not be masterwork items.)

\vspace{12pt}
{\large{}CREATING MAGIC ARMOR}

To create magic armor, a character needs a heat source and some iron, wood, or 
leatherworking tools. He also needs a supply of materials, the most obvious being 
the armor or the pieces of the armor to be assembled. Armor to be made into magic 
armor must be masterwork armor, and the masterwork cost is added to the base price 
to determine final market value. Additional magic supplies costs for the materials 
are subsumed in the cost for creating the magic armor---half the base price of 
the item.

Creating magic armor has a special prerequisite: The creator's caster level must 
be at least three times the enhancement bonus of the armor. If an item has both 
an enhancement bonus and a special ability, the higher of the two caster level 
requirements must be met.

Magic armor or a magic shield must have at least a +1 enhancement bonus to have 
any of the abilities listed on Table: Armor Special Abilities and Table: Shield 
Special Abilities.

If spells are involved in the prerequisites for making the armor, the creator must 
have prepared the spells to be cast (or must know the spells, in the case of a 
sorcerer or bard), must provide any material components or focuses the spells require, 
and must pay any XP costs required for the spells. The act of working on the armor 
triggers the prepared spells, making them unavailable for casting during each day 
of the armor's creation. (That is, those spell slots are expended from his currently 
prepared spells, just as if they had been cast.)

Creating some armor may entail other prerequisites beyond or other than spellcasting. 
See the individual descriptions for details.

Crafting magic armor requires one day for each 1,000 gp value of the base price.

Item Creation Feat Required: Craft Magic Arms and Armor.

\vspace{12pt}
{\large{}CREATING MAGIC WEAPONS}

To create a magic weapon, a character needs a heat source and some iron, wood, 
or leatherworking tools. She also needs a supply of materials, the most obvious 
being the weapon or the pieces of the weapon to be assembled. Only a masterwork 
weapon can become a magic weapon, and the masterwork cost is added to the total 
cost to determine final market value. Additional magic supplies costs for the materials 
are subsumed in the cost for creating the magic weapon---half the base price given 
on Table: Weapons, according to the weapon's total effective bonus.

Creating a magic weapon has a special prerequisite: The creator's caster level 
must be at least three times the enhancement bonus of the weapon. If an item has 
both an enhancement bonus and a special ability the higher of the two caster level 
requirements must be met.

A magic weapon must have at least a +1 enhancement bonus to have any of the abilities 
listed on Table: Melee Weapon Special Abilities or Table Ranged Weapon Special 
Abilities.

If spells are involved in the prerequisites for making the weapon, the creator 
must have prepared the spells to be cast (or must know the spells, in the case 
of a sorcerer or bard) but need not provide any material components or focuses 
the spells require, nor are any XP costs inherent in a prerequisite spell incurred 
in the creation of the item. The act of working on the weapon triggers the prepared 
spells, making them unavailable for casting during each day of the weapon's creation. 
(That is, those spell slots are expended from his currently prepared spells, just 
as if they had been cast.)

At the time of creation, the creator must decide if the weapon glows or not as 
a side-effect of the magic imbued within it. This decision does not affect the 
price or the creation time, but once the item is finished, the decision is binding.

Creating magic double-headed weapons is treated as creating two weapons when determining 
cost, time, XP, and special abilities.

Creating some weapons may entail other prerequisites beyond or other than spellcasting. 
See the individual descriptions for details.

Crafting a magic weapon requires one day for each 1,000 gp value of the base price.

Item Creation Feat Required: Craft Magic Arms and Armor.

\vspace{12pt}
{\large{}CREATING POTIONS}

The creator of a potion needs a level working surface and at least a few containers 
in which to mix liquids, as well as a source of heat to boil the brew. In addition, 
he needs ingredients. The costs for materials and ingredients are subsumed in the 
cost for brewing the potion---25 gp x$ $ the level of the spell x $ $the level 
of the caster.

All ingredients and materials used to brew a potion must be fresh and unused. The 
character must pay the full cost for brewing each potion. (Economies of scale do 
not apply.)

The imbiber of the potion is both the caster and the target. Spells with a range 
of personal cannot be made into potions.

The creator must have prepared the spell to be placed in the potion (or must know 
the spell, in the case of a sorcerer or bard) and must provide any material component 
or focus the spell requires.

If casting the spell would reduce the caster's XP total, he pays the XP cost upon 
beginning the brew in addition to the XP cost for making the potion itself. Material 
components are consumed when he begins working, but a focus is not. (A focus used 
in brewing a potion can be reused.) The act of brewing triggers the prepared spell, 
making it unavailable for casting until the character has rested and regained spells. 
(That is, that spell slot is expended from his currently prepared spells, just 
as if it had been cast.) Brewing a potion requires one day.

Item Creation Feat Required: Brew Potion.

\begin{tabular}{|>{\raggedright}p{50pt}|>{\raggedright}p{62pt}|>{\raggedright}p{31pt}|>{\raggedright}p{39pt}|>{\raggedright}p{46pt}|}
\hline
\multicolumn{5}{|p{230pt}|}{P\textbf{otion Base Prices (By Brewer's Class)}}\tabularnewline
\hline
S\textbf{pell Level} & C\textbf{lr, Drd, Wiz} & S\textbf{or} & B\textbf{rd} & P\textbf{al, 
Rgr*}\tabularnewline
\hline
0 & 25 gp & 25 gp & 25 gp--- & \tabularnewline
\hline
1st & 50 gp & 50 gp & 100 gp & 100 gp\tabularnewline
\hline
2nd & 300 gp & 400 gp & 400 gp & 400 gp\tabularnewline
\hline
3rd & 750 gp & 900 gp & 1,050 gp & 750 gp\tabularnewline
\hline
\multicolumn{5}{|p{230pt}|}{* Caster level is half class level.}\tabularnewline
\hline
\multicolumn{5}{|p{230pt}|}{Prices assume that the potion was made at the minimum 
caster level.}\tabularnewline
\hline
\end{tabular}

\vspace{12pt}
\begin{tabular}{|>{\raggedright}p{50pt}|>{\raggedright}p{62pt}|>{\raggedright}p{45pt}|>{\raggedright}p{45pt}|>{\raggedright}p{54pt}|}
\hline
\multicolumn{5}{|p{259pt}|}{B\textbf{ase Cost to Brew a Potion (By Brewer's Class)}}\tabularnewline
\hline
S\textbf{pell Level} & C\textbf{lr, Drd, Wiz} & S\textbf{or} & B\textbf{rd} & P\textbf{al, 
Rgr*}\tabularnewline
\hline
0 & 12 gp 5 sp\linebreak{}
+1 XP & 12 gp 5 sp\linebreak{}
+1 XP & 12 gp 5 sp\linebreak{}
+1 XP--- & \tabularnewline
\hline
1st & 25 gp\linebreak{}
+2 XP & 25 gp\linebreak{}
+2 XP & 50 gp\linebreak{}
+4 XP & 50 gp\linebreak{}
+4 XP\tabularnewline
\hline
2nd & 150 gp\linebreak{}
+12 XP & 200 gp\linebreak{}
+16 XP & 200 gp\linebreak{}
+16 XP & 200 gp\linebreak{}
+16 XP\tabularnewline
\hline
3rd & 375 gp\linebreak{}
+30 XP & 450 gp\linebreak{}
+36 XP & 525 gp\linebreak{}
+42 XP & 375 gp\linebreak{}
+30 XP\tabularnewline
\hline
\multicolumn{5}{|p{259pt}|}{* Caster level is half class level.}\tabularnewline
\hline
\multicolumn{5}{|p{259pt}|}{Costs assume that the creator makes the potion at the 
minimum caster level.}\tabularnewline
\hline
\end{tabular}

\vspace{12pt}
{\large{}CREATING RINGS}

To create a magic ring, a character needs a heat source. He also needs a supply 
of materials, the most obvious being a ring or the pieces of the ring to be assembled. 
The cost for the materials is subsumed in the cost for creating the ring. Ring 
costs are difficult to formularize. Refer to Table: Estimating Magic Item Gold 
Piece Values and use the ring prices in the ring descriptions as a guideline. Creating 
a ring generally costs half the ring's market price.

Rings that duplicate spells with costly material or XP components add in the value 
of 50 x the spell's component cost. Having a spell with a costly component as a 
prerequisite does not automatically incur this cost. The act of working on the 
ring triggers the prepared spells, making them unavailable for casting during each 
day of the ring's creation. (That is, those spell slots are expended from his currently 
prepared spells, just as if they had been cast.)

Creating some rings may entail other prerequisites beyond or other than spellcasting. 
See the individual descriptions for details.

Forging a ring requires one day for each 1,000 gp of the base price.

Item Creation Feat Required: Forge Ring.

\vspace{12pt}
{\large{}CREATING RODS}

To create a magic rod, a character needs a supply of materials, the most obvious 
being a rod or the pieces of the rod to be assembled. The cost for the materials 
is subsumed in the cost for creating the rod. Rod costs are difficult to formularize. 
Refer to Table: Estimating Magic Item Gold Piece Values and use the rod prices 
in the rod descriptions as a guideline. Creating a rod costs half the market value 
listed.

If spells are involved in the prerequisites for making the rod, the creator must 
have prepared the spells to be cast (or must know the spells, in the case of a 
sorcerer or bard) but need not provide any material components or focuses the spells 
require, nor are any XP costs inherent in a prerequisite spell incurred in the 
creation of the item. The act of working on the rod triggers the prepared spells, 
making them unavailable for casting during each day of the rod's creation. (That 
is, those spell slots are expended from his currently prepared spells, just as 
if they had been cast.)

Creating some rods may entail other prerequisites beyond or other than spellcasting. 
See the individual descriptions for details.

Crafting a rod requires one day for each 1,000 gp of the base price.

Item Creation Feat Required: Craft Rod.

\vspace{12pt}
{\large{}CREATING SCROLLS}

To create a scroll, a character needs a supply of choice writing materials, the 
cost of which is subsumed in the cost for scribing the scroll---12.5 gp x the level 
of the spell x the level of the caster.

All writing implements and materials used to scribe a scroll must be fresh and 
unused. A character must pay the full cost for scribing each spell scroll no matter 
how many times she previously has scribed the same spell.

The creator must have prepared the spell to be scribed (or must know the spell, 
in the case of a sorcerer or bard) and must provide any material component or focus 
the spell requires. If casting the spell would reduce the caster's XP total, she 
pays the cost upon beginning the scroll in addition to the XP cost for making the 
scroll itself. Likewise, a material component is consumed when she begins writing, 
but a focus is not. (A focus used in scribing a scroll can be reused.) The act 
of writing triggers the prepared spell, making it unavailable for casting until 
the character has rested and regained spells. (That is, that spell slot is expended 
from her currently prepared spells, just as if it had been cast.)

Scribing a scroll requires one day per each 1,000 gp of the base price.

Item Creation Feat Required: Scribe Scroll.

\begin{tabular}{|>{\raggedright}p{50pt}|>{\raggedright}p{62pt}|>{\raggedright}p{45pt}|>{\raggedright}p{45pt}|>{\raggedright}p{44pt}|}
\hline
\multicolumn{5}{|p{249pt}|}{S\textbf{croll Base Prices (By Scriber's Class)}}\tabularnewline
\hline
S\textbf{pell Level} & C\textbf{lr, Drd, Wiz} & S\textbf{or} & B\textbf{rd} & P\textbf{al, 
Rgr*}\tabularnewline
\hline
0 & 12 gp 5 sp & 12 gp 5 sp & 12 gp 5 sp--- & \tabularnewline
\hline
1st & 25 gp & 25 gp & 50 gp & 50 gp\tabularnewline
\hline
2nd & 150 gp & 200 gp & 200 gp & 200 gp\tabularnewline
\hline
3rd & 375 gp & 450 gp & 525 gp & 375 gp\tabularnewline
\hline
4th & 700 gp & 800 gp & 1,000 gp & 700 gp\tabularnewline
\hline
5th & 1,125 gp & 1,250 gp & 1,625 gp--- & \tabularnewline
\hline
6th & 1,650 gp & 1,800 gp & 2,400 gp--- & \tabularnewline
\hline
7th & 2,275 gp & 2,450 gp--- & --- & \tabularnewline
\hline
8th & 3,000 gp & 3,200 gp--- & --- & \tabularnewline
\hline
9th & 3,825 gp & 4,050 gp--- & --- & \tabularnewline
\hline
\multicolumn{5}{|p{249pt}|}{* Caster level is half class level.}\tabularnewline
\hline
\multicolumn{5}{|p{249pt}|}{Prices assume that the scroll was made at the minimum 
caster level.}\tabularnewline
\hline
\end{tabular}

\vspace{12pt}
\begin{tabular}{|>{\raggedright}p{50pt}|>{\raggedright}p{62pt}|>{\raggedright}p{59pt}|>{\raggedright}p{59pt}|>{\raggedright}p{50pt}|}
\hline
\multicolumn{5}{|p{284pt}|}{B\textbf{ase Magic Supplies and XP Cost to Scribe a 
Scroll (By Scriber's Class)}}\tabularnewline
\hline
S\textbf{pell Level} & C\textbf{lr, Drd, Wiz} & S\textbf{or} & B\textbf{rd} & P\textbf{al, 
Rgr*}\tabularnewline
\hline
0 & 6 gp 2 sp 5 cp\linebreak{}
+1 XP & 6 gp 2 sp 5 cp\linebreak{}
+1 XP & 6 gp 2 sp 5 cp\linebreak{}
+1 XP--- & \tabularnewline
\hline
1st & 12 gp 5 sp\linebreak{}
+1 XP & 12 gp 5 sp\linebreak{}
+1 XP & 25 gp\linebreak{}
+1 XP & 25 gp\linebreak{}
+2 XP\tabularnewline
\hline
2nd & 75 gp\linebreak{}
+6 XP & 100 gp\linebreak{}
+8 XP & 100 gp\linebreak{}
+8 XP & 100 gp\linebreak{}
+8 XP\tabularnewline
\hline
3rd & 187 gp 5 sp\linebreak{}
+15 XP & 225 gp\linebreak{}
+18 XP & 262 gp 5 sp\linebreak{}
+21 XP & 187 gp 5 sp\linebreak{}
+15 XP\tabularnewline
\hline
4th & 350 gp\linebreak{}
+28 XP & 400 gp\linebreak{}
+32 XP & 500 gp\linebreak{}
+40 XP & 350 gp\linebreak{}
+28 XP\tabularnewline
\hline
5th & 562 gp 5 sp\linebreak{}
+45 XP & 625 gp\linebreak{}
+50 XP & 812 gp 5 sp\linebreak{}
+65 XP--- & \tabularnewline
\hline
6th & 826 gp\linebreak{}
+66 XP & 900 gp\linebreak{}
+72 XP & 1,200 gp\linebreak{}
+96 XP--- & \tabularnewline
\hline
7th & 1,135 gp 5 sp\linebreak{}
+91 XP & 1,225 gp\linebreak{}
+98 XP--- & --- & \tabularnewline
\hline
8th & 1,500 gp\linebreak{}
+120 XP & 1,600 gp\linebreak{}
+128 XP--- & --- & \tabularnewline
\hline
9th & 1,912 gp 5 sp\linebreak{}
+153 XP & 2, 025 gp\linebreak{}
+162 XP--- & --- & \tabularnewline
\hline
\multicolumn{5}{|p{284pt}|}{* Caster level is half class level.}\tabularnewline
\hline
\multicolumn{5}{|p{284pt}|}{Costs assume that the creator makes the scroll at the 
minimum caster level.}\tabularnewline
\hline
\end{tabular}

\vspace{12pt}
{\large{}CREATING STAFFS}

To create a magic staff, a character needs a supply of materials, the most obvious 
being a staff or the pieces of the staff to be assembled.

The cost for the materials is subsumed in the cost for creating the staff---375 
gp x the level of the highest-level spell x the level of the caster, plus 75\% 
of the value of the next most costly ability (281.25 gp x the level of the spell 
x the level of the caster), plus one-half of the value of any other abilities (187.5 
gp x the level of the spell x the level of the caster). Staffs are always fully 
charged (50 charges) when created.

If desired, a spell can be placed into the staff at only half the normal cost, 
but then activating that particular spell costs 2 charges from the staff. The caster 
level of all spells in a staff must be the same, and no staff can have a caster 
level of less than 8th, even if all the spells in the staff are low-level spells.

The creator must have prepared the spells to be stored (or must know the spell, 
in the case of a sorcerer or bard) and must provide any focus the spells require 
as well as material and XP component costs sufficient to activate the spell a maximum 
number of times (50 divided by the number of charges one use of the spell expends). 
This is in addition to the XP cost for making the staff itself. Material components 
are consumed when he begins working, but focuses are not. (A focus used in creating 
a staff can be reused.) The act of working on the staff triggers the prepared spells, 
making them unavailable for casting during each day of the staff 's creation. (That 
is, those spell slots are expended from his currently prepared spells, just as 
if they had been cast.)

Creating a few staffs may entail other prerequisites beyond spellcasting. See the 
individual descriptions for details.

Crafting a staff requires one day for each 1,000 gp of the base price.

Item Creation Feat Required: Craft Staff.

\vspace{12pt}
{\large{}CREATING WANDS}

To create a magic wand, a character needs a small supply of materials, the most 
obvious being a baton or the pieces of the wand to be assembled. The cost for the 
materials is subsumed in the cost for creating the wand---375 gp x the level of 
the spell x the level of the caster. Wands are always fully charged (50 charges) 
when created.

The creator must have prepared the spell to be stored (or must know the spell, 
in the case of a sorcerer or bard) and must provide any focuses the spell requires. 
Fifty of each needed material component are required, one for each charge. If casting 
the spell would reduce the caster's XP total, she pays the cost (multiplied by 
50) upon beginning the wand in addition to the XP cost for making the wand itself. 
Likewise, material components are consumed when she begins working, but focuses 
are not. (A focus used in creating a wand can be reused.) The act of working on 
the wand triggers the prepared spell, making it unavailable for casting during 
each day devoted to the wand's creation. (That is, that spell slot is expended 
from her currently prepared spells, just as if it had been cast.)

Crafting a wand requires one day per each 1,000 gp of the base price.

Item Creation Feat Required: Craft Wand.

\vspace{12pt}
\begin{tabular}{|>{\raggedright}p{50pt}|>{\raggedright}p{62pt}|>{\raggedright}p{44pt}|>{\raggedright}p{44pt}|>{\raggedright}p{44pt}|}
\hline
\multicolumn{5}{|p{246pt}|}{W\textbf{and Base Prices (By Crafter's Class)}}\tabularnewline
\hline
S\textbf{pell Level} & C\textbf{lr, Drd, Wiz} & S\textbf{or} & B\textbf{rd} & P\textbf{al, 
Rgr*}\tabularnewline
\hline
0 & 375 gp & 375 gp & 375 gp--- & \tabularnewline
\hline
1st & 750 gp & 750 gp & 1,500 gp & 1,500 gp\tabularnewline
\hline
2nd & 4,500 gp & 6,000 gp & 6,000 gp & 6,000 gp\tabularnewline
\hline
3rd & 11,250 gp & 13,500 gp & 15,750 gp & 11,250 gp\tabularnewline
\hline
4th & 21,000 gp & 24,000 gp & 30,000 gp & 21,000 gp\tabularnewline
\hline
\multicolumn{5}{|p{246pt}|}{* Caster level is half class level.}\tabularnewline
\hline
\multicolumn{5}{|p{246pt}|}{Prices assume that the wand was made at the minimum 
caster level.}\tabularnewline
\hline
\end{tabular}

\vspace{12pt}
\begin{tabular}{|>{\raggedright}p{50pt}|>{\raggedright}p{62pt}|>{\raggedright}p{50pt}|>{\raggedright}p{50pt}|>{\raggedright}p{46pt}|}
\hline
\multicolumn{5}{|p{261pt}|}{B\textbf{ase Magic Supplies and XP Cost to Craft a 
Wand (By Crafter's Class)}}\tabularnewline
\hline
S\textbf{pell Level} & C\textbf{lr, Drd, Wiz} & S\textbf{or} & B\textbf{rd} & P\textbf{al, 
Rgr*}\tabularnewline
\hline
0 & 187 gp 5 sp\linebreak{}
+15 XP & 187 gp 5 sp\linebreak{}
+15 XP & 187 gp 5 sp\linebreak{}
+15 XP--- & \tabularnewline
\hline
1st & 375 gp\linebreak{}
+30 XP & 375 gp\linebreak{}
+30 XP & 750 gp\linebreak{}
+60 XP & 750 gp\linebreak{}
+60 XP\tabularnewline
\hline
2nd & 2,250 gp\linebreak{}
+180 XP & 3,000 gp\linebreak{}
+240 XP & 3,000 gp\linebreak{}
+240 XP & 3,000 gp\linebreak{}
+240 XP\tabularnewline
\hline
3rd & 5,625 gp\linebreak{}
+450 XP & 6,750 gp\linebreak{}
+540 XP & 7,875 gp\linebreak{}
+630 XP & 5,625 gp\linebreak{}
+450 XP\tabularnewline
\hline
4th & 10,500 gp\linebreak{}
+840 XP & 12,000 gp\linebreak{}
+960 XP & 15,000 gp\linebreak{}
+1200 XP & 10,500 gp\linebreak{}
+840 XP\tabularnewline
\hline
\multicolumn{5}{|p{261pt}|}{* Caster level is half class level.}\tabularnewline
\hline
\multicolumn{5}{|p{261pt}|}{Costs assume that the creator makes the wand at the 
minimum caster level.}\tabularnewline
\hline
\end{tabular}

\vspace{12pt}
{\large{}CREATING WONDROUS ITEMS}

To create a wondrous item, a character usually needs some sort of equipment or 
tools to work on the item. She also needs a supply of materials, the most obvious 
being the item itself or the pieces of the item to be assembled. The cost for the 
materials is subsumed in the cost for creating the item. Wondrous item costs are 
difficult to formularize. Refer to Table: Estimating Magic Item Gold Piece Values 
and use the item prices in the item descriptions as a guideline. Creating an item 
costs half the market value listed.

If spells are involved in the prerequisites for making the item, the creator must 
have prepared the spells to be cast (or must know the spells, in the case of a 
sorcerer or bard) but need not provide any material components or focuses the spells 
require, nor are any XP costs inherent in a prerequisite spell incurred in the 
creation of the item. The act of working on the item triggers the prepared spells, 
making them unavailable for casting during each day of the item's creation. (That 
is, those spell slots are expended from his currently prepared spells, just as 
if they had been cast.)

Creating some items may entail other prerequisites beyond or other than spellcasting. 
See the individual descriptions for details.

Crafting a wondrous item requires one day for each 1,000 gp of the base price.

Item Creation Feat Required: Craft Wondrous Item.

\vspace{12pt}
{\large{}INTELLIGENT ITEM CREATION}

To create an intelligent item, a character must have a caster level of 15th or 
higher. Time and creation cost are based on the normal item creation rules, with 
the market price values on Table: Item Intelligence, Wisdom, Charisma, and Capabilities 
treated as additions to time, gp cost, and XP cost. The item's alignment is the 
same as its creator's. Determine other features randomly, following the guidelines 
in the relevant section.

\vspace{12pt}
{\large{}ADDING NEW ABILITIES}

A creator can add new magical abilities to a magic item with no restrictions. The 
cost to do this is the same as if the item was not magical. Thus, a \textit{+1 
longsword }can be made into a \textit{+2 vorpal longsword}, with the cost to create 
it being equal to that of a \textit{+2 vorpal sword }minus the cost of a \textit{+1 
sword}.

If the item is one that occupies a specific place on a character's body the cost 
of adding any additional ability to that item increases by 50\%. For example, if 
a character adds the power to confer \textit{invisibility }to her \textit{ring 
of protection +2}, the cost of adding this ability is the same as for creating 
a \textit{ring of invisibility }multiplied by 1.5.

\vspace{12pt}
BODY SLOT AFFINITIES

Each location on the body, or body slot, has one or more affinities: a word or 
phrase that describes the general function or nature of magic items designed for 
that body slot. Body slot affinities are deliberately broad, abstract categorizations, 
because a hard-and-fast rule can't cover the great variety among wondrous items.

You can use the affinities in the list below to guide your decisions on which magic 
items should be allowed in which body slots. And when you design your own magic 
items, the affinities give you some guidance for what form a particular item should 
take.

Some body slots have different affinities for different specific items. 

\begin{tabular}{|>{\raggedright}p{110pt}|>{\raggedright}p{150pt}|}
\hline
B\textbf{ody Slot} & A\textbf{ffinity}\tabularnewline
\hline
Headband, helmet & Mental improvement, ranged attacks\tabularnewline
\hline
Hat & Interaction\tabularnewline
\hline
Phylactery & Morale, alignment\tabularnewline
\hline
Eye lenses, goggles & Vision\tabularnewline
\hline
Cloak, cape, mantle & Transformation, protection\tabularnewline
\hline
Amulet, brooch, medallion, necklace, periapt, scarab & Protection, discernment\tabularnewline
\hline
Robe  & Multiple effects\tabularnewline
\hline
Shirt  & Physical improvement\tabularnewline
\hline
Vest, vestment  & Class ability improvement\tabularnewline
\hline
Bracers  & Combat\tabularnewline
\hline
Bracelets  & Allies\tabularnewline
\hline
Gloves  & Quickness\tabularnewline
\hline
Gauntlets  & Destructive power\tabularnewline
\hline
Belt  & Physical improvement\tabularnewline
\hline
Boots  & Movement\tabularnewline
\hline
\end{tabular}

Wondrous items that don't match the affinity for a particular body slot should 
cost 50\% more than wondrous items that match the affinity.

\newpage

\end{document}
