%&pdfLaTeX
% !TEX encoding = UTF-8 Unicode
\documentclass{article}
\usepackage{ifxetex}
\ifxetex
\usepackage{fontspec}
\setmainfont[Mapping=tex-text]{STIXGeneral}
\else
\usepackage[T1]{fontenc}
\usepackage[utf8]{inputenc}
\fi
\usepackage{textcomp}

\usepackage{array}
\usepackage{amssymb}
\usepackage{fancyhdr}
\renewcommand{\headrulewidth}{0pt}
\renewcommand{\footrulewidth}{0pt}

\begin{document}

This material is Open Game Content, and is licensed for public use under the terms 
of the Open Game License v1.0a.

\subsubsection*{{\LARGE{}IMPROVING MONSTERS}}

Each of the monster entries describes a typical creature of its kind. However, 
there are several methods by which extraordinary or unique monsters can be created 
using a typical creature as the foundation: by adding character classes, increasing 
a monster's Hit Dice, or by adding a template to a monster. These methods are not 
mutually exclusive---it's possible for a monster with a template to be improved 
by both increasing its Hit Dice and adding character class levels.

\textbf{Class Levels:} Intelligent creatures that are reasonably humanoid in shape 
most commonly advance by adding class levels. Creatures that fall into this category 
have an entry of ``By character class'' in their Advancement line. When a monster 
adds a class level, that level usually represents an increase in experience and 
learned skills and capabilities.

\textbf{Increased Hit Dice:} Intelligent creatures that are not humanoid in shape, 
and nonintelligent monsters, can advance by increasing their Hit Dice. Creatures 
with increased Hit Dice are usually superior specimens of their race, bigger and 
more powerful than their run-of-the-mill fellows.

\textbf{Templates:} Both intelligent and nonintelligent creatures with an unusual 
heritage or an inflicted change in their essential nature may be modified with 
a template. Templates usually result in tougher monsters with capabilities that 
differ from those of their common kin.

Each of these three methods for improving monsters is discussed in more detail 
below.

\vspace{12pt}
{\LARGE{}ABILITY SCORE ARRAYS}

Monsters are assumed to have completely average (or standard) ability scores---a 
10 or an 11 in each ability, as modified by their racial bonuses. However, improved 
monsters are individuals and often have better than normal ability scores, and 
usually make use of either the elite array or the nonelite array of ability scores. 
Monsters who improve by adding a template, and monsters who improve by increasing 
their Hit Dice, may use any of the three arrays (standard, nonelite, or elite). 
Any monster unique enough to be improved could easily be considered elite.

\textbf{Elite Array: }The elite array is 15, 14, 13, 12, 10, 8. While the monster 
has one weakness compared to a typical member of its race, it is significantly 
better overall. The elite array is most appropriate for monsters who add levels 
in a player character class.

\textbf{Nonelite Array:} The nonelite array is 13, 12, 11, 10, 9, 8. The nonelite 
array does not necessarily make a monster better than normal, but it does customize 
the monster as an individual with strengths and weaknesses compared to a typical 
member of its race. The nonelite array is most appropriate for monsters who add 
class levels in a NPC class.

\textbf{Ability Score Improvement:} Treat monster Hit Dice the same as character 
level for determining ability score increases. This only applies to Hit Dice increases, 
monsters do not gain ability score increases for levels they ``already reached'' 
with their racial Hit Dice, since these adjustments are included in their basic 
ability scores.

\vspace{12pt}
{\LARGE{}MONSTERS AND CLASS LEVELS}

If a creature acquires a character class, it follows the rules for multiclass characters\textit{.}

The creature's Hit Dice equal the number of class levels it has plus its racial 
Hit Dice. A creature's ``monster class'' is always a favored class, and the creature 
never takes XP penalties for having it. Additional Hit Dice gained from taking 
levels in a character class never affect a creature's size.

\textbf{Humanoids and Class Levels:} Creatures with 1 or less HD replace their 
monster levels with their character levels. The monster loses the attack bonus, 
saving throw bonuses, skills, and feats granted by its 1 monster HD and gains the 
attack bonus, save bonuses, skills, feats, and other class abilities of a 1st-level 
character of the appropriate class.

\textbf{Level Adjustment and Effective Character Level:} To determine the effective 
character level (ECL) of a monster character, add its level adjustment to its racial 
Hit Dice and character class levels. The monster is considered to have experience 
points equal to the minimum needed to be a character of its ECL. 

If you choose to equip a monster with gear, use its ECL as its character level 
for purposes of determining how much equipment it can purchase. Generally, only 
monsters with an Advancement entry of ``By character class'' receive NPC gear; 
other creatures adding character levels should be treated as monsters of the appropriate 
CR and assigned treasure, not equipment.

\textbf{Feat Acquisition and Ability Score Increases:} A monster's total Hit Dice, 
not its ECL, govern its acquisition of feats and ability score increases. 

\vspace{12pt}
{\LARGE{}INCREASING HIT DICE}

As its Hit Dice increase, a creature's attack bonuses and saving throw modifiers 
might improve. It gains more feats and skills, depending on its type, as shown 
on Table: Creature Improvement by Type.

Note that if a creature acquires a character class, it improves according to its 
class, not its type.

\vspace{12pt}
\begin{tabular}{|>{\raggedright}p{61pt}|>{\raggedright}p{23pt}|>{\raggedright}p{57pt}|>{\raggedright}p{86pt}|>{\raggedright}p{62pt}|}
\hline
\multicolumn{5}{|p{290pt}|}{T\textbf{able: Creature Improvement by Type}}\tabularnewline
\hline
 & H\textbf{it Die} & A\textbf{ttack Bonus} & G\textbf{ood Saving Throws} & S\textbf{kill 
Points*}\tabularnewline
\hline
Aberration & d8 & HD x3/4 (as cleric) & Will & 2 + Int mod per HD\tabularnewline
\hline
Animal & d8 & HD x3/4 (as cleric) & Fort, Ref (and sometimes Will) & 2 + Int mod 
per HD\tabularnewline
\hline
Construct & d10 & HD x3/4 (as cleric)--- &  & 2 + Int mod per HD**\tabularnewline
\hline
Dragon & d12 & HD (as fighter) & Fort, Ref, Will & 6 + Int mod per HD\tabularnewline
\hline
Elemental & d8 & HD x3/4 (as cleric) & Ref (Air, Fire), or \linebreak{}
Fort (Earth, Water) & 2 + Int mod per HD\tabularnewline
\hline
Fey & d6 & HD x1/2 (as wizard) & Ref, Will & 6 + Int mod per HD\tabularnewline
\hline
Giant & d8 & HD x3/4 (as cleric) & Fort & 2 + Int mod per HD\tabularnewline
\hline
Humanoid & d8 & HD x3/4 (as cleric) & Varies (any one) & 2 + Int mod per HD\tabularnewline
\hline
Magical beast & d10 & HD (as fighter) & Fort, Ref & 2 + Int mod per HD\tabularnewline
\hline
Monstrous humanoid & d8 & HD (as fighter) & Ref, Will & 2 + Int mod per HD\tabularnewline
\hline
Ooze & d10 & HD x3/4 (as cleric)--- &  & 2 + Int mod per HD**\tabularnewline
\hline
Outsider & d8 & HD (as fighter) & Fort, Ref, Will & 8 + Int mod per HD\tabularnewline
\hline
Plant & d8 & HD x3/4 (as cleric) & Fort & 2 + Int mod per HD**\tabularnewline
\hline
Undead & d12 & HD x1/2 (as wizard) & Will & 4 + Int mod per HD**\tabularnewline
\hline
Vermin & d8 & HD x3/4 (as cleric) & Fort & 2 + Int mod per HD**\tabularnewline
\hline
\multicolumn{5}{|p{290pt}|}{All types have a number of feats equal to 1 + 1 per 
3 Hit Dice.}\tabularnewline
\hline
\multicolumn{5}{|p{290pt}|}{* As long as a creature has an Intelligence of at least 
1, it gains a minimum of 1 skill point per Hit Die.}\tabularnewline
\hline
\multicolumn{5}{|p{290pt}|}{** Creatures with an Intelligence score of ``---'' 
gain no skill points or feats.}\tabularnewline
\hline
\end{tabular}

\vspace{12pt}
SIZE INCREASES

A creature may become larger when its Hit Dice are increased (the new size is noted 
parenthetically in the monster's Advancement entry).

A size increase affects any special ability the creature has that is affected by 
size. Increased size also affects a creature's ability scores, AC, attack bonuses, 
and damage values as indicated on the tables below.

\vspace{12pt}
\begin{tabular}{|>{\raggedright}p{48pt}|>{\raggedright}p{47pt}|>{\raggedright}p{24pt}|>{\raggedright}p{24pt}|>{\raggedright}p{24pt}|>{\raggedright}p{65pt}|>{\raggedright}p{30pt}|}
\hline
\multicolumn{7}{|p{266pt}|}{T\textbf{able: Changes to Statistics by Size}}\tabularnewline
\hline
O\textbf{ld Size}* & N\textbf{ew Size} & \section*{S\textbf{tr}} & D\textbf{ex} & C\textbf{on} & N\textbf{atural 
Armor} & A\textbf{C/}\linebreak{}
\textbf{Attack}\tabularnewline
\hline
Fine & Diminutive & Same- & 2 & Same & Same- & 4\tabularnewline
\hline
Diminutive & Tiny & +2- & 2 & Same & Same- & 2\tabularnewline
\hline
Tiny & Small & +4- & 2 & Same & Same- & 1\tabularnewline
\hline
Small & Medium & +4- & 2 & +2 & Same- & 1\tabularnewline
\hline
Medium & Large & +8- & 2 & +4 & +2- & 1\tabularnewline
\hline
Large & Huge & +8- & 2 & +4 & +3- & 1\tabularnewline
\hline
Huge & Gargantuan & +8 & Same & +4 & +4- & 2\tabularnewline
\hline
Gargantuan & Colossal & +8 & Same & +4 & +5- & 4\tabularnewline
\hline
\multicolumn{7}{|p{266pt}|}{*Repeat the adjustment if the creature moves up more 
than one size.}\tabularnewline
\hline
\end{tabular}

\vspace{12pt}
\begin{tabular}{|>{\raggedright}p{92pt}|>{\raggedright}p{77pt}|}
\hline
\multicolumn{2}{|p{170pt}|}{T\textbf{able: Increased Damage By Size}}\tabularnewline
\hline
O\textbf{ld Damage (Each)}* & N\textbf{ew Damage}\tabularnewline
\hline
1d2 & 1d3\tabularnewline
\hline
1d3 & 1d4\tabularnewline
\hline
1d4 & 1d6\tabularnewline
\hline
1d6 & 1d8\tabularnewline
\hline
1d8 & 2d6\tabularnewline
\hline
1d10 & 2d8\tabularnewline
\hline
2d6 & 3d6\tabularnewline
\hline
2d8 & 3d8\tabularnewline
\hline
\multicolumn{2}{|p{170pt}|}{* Repeat the adjustment if the creature moves up more 
than one size category.}\tabularnewline
\hline
\end{tabular}

\vspace{12pt}
\subsection*{{\LARGE{}TEMPLATES}}

Certain creatures are created by adding a template to an existing creature. A templated 
creature can represent a freak of nature, the individual creation of a single experimenter, 
or the first generation of offspring from parents of different species.

\vspace{12pt}
ACQUIRED AND INHERITED TEMPLATES

Some templates can be added to creatures anytime. Templates such as these are referred 
to as acquired templates, indicating that the creature did not always have the 
attributes of the template.

Other templates, known as inherited templates, are part of a creature from the 
beginning of its existence. Creatures are born with these templates.

It's possible for a certain kind of template to be of either type. 

\vspace{12pt}
READING A TEMPLATE

A template's description provides a set of instructions for altering an existing 
creature, known as the base creature. The changes that a template might cause to 
each line of a creature 's statistics block are discussed below. Generally, if 
a template does not cause a change to a certain statistic, that entry is missing 
from the template description. For clarity, the entry for a statistic or attribute 
that is not changed is sometimes given as ``Same as the base creature.'' 

\textbf{Size and Type:} Templates often change a creature's type, and may change 
the creature's size.

If a template changes the base creature's type, the creature also acquires the 
augmented subtype unless the template description indicates otherwise. The augmented 
subtype is always paired with the creature's original type. Unless a template indicates 
otherwise, the new creature has the traits of the new type but the features of 
the original type.

If a template changes a creature's size, use Table: Changes to Statistics by Size 
to calculate changes to natural armor, Armor Class, attack rolls, and grapple bonus.

\textbf{Hit Dice and Hit Points:} Most templates do not change the number of Hit 
Dice a monster has, but some do. Some templates change the size of a creature's 
Hit Dice (usually by changing the creature type). A few templates change previously 
acquired Hit Dice, and continue to change Hit Dice gained with class levels, but 
most templates that change Hit Dice change only the creature's original HD and 
leave class Hit Dice unchanged.

If the Hit Dice entry in a template description is missing, Hit Dice and hit points 
do not change unless the creature's Constitution modifier changes.

\textbf{Initiative: }If a template changes the monster's Dexterity, or if it adds 
or removes the Improved Initiative feat, this entry changes.

\textbf{Speed:} If a template modifies a creature's speed, the template states 
how that happens. More commonly, a template adds a new movement mode.

\textbf{Armor Class:} If a template changes the creature's size, see Table: Changes 
to Statistics by Size to determine its new Armor Class and to see whether its natural 
armor changes. In some cases the method of determining Armor Class changes radically; 
the template description explains how to adjust the creature's AC.

\textbf{Base Attack/Grapple:} Templates usually do not change a creature's base 
attack bonus. If a template modifies a creature's base attack bonus, the template 
description states how that happens. Changes to a creature's Strength score can 
change a creature's grapple bonus, as can changes to its size.

\textbf{Attack and Full Attack:} Most templates do not change a creature's attack 
bonus or modes of attack, even when the creature's type changes (the creature's 
base attack bonus is the same as a creature of the original type). Of course, any 
change in ability scores may affect attack bonuses. If Strength or Dexterity changes, 
use the new modifier to determine attack bonuses. A change in a monster's size 
also changes its attack bonus; see Table: Changes to Statistics by Size.

\textbf{Damage:} Damage changes with Strength. If the creature uses a two-handed 
weapon or has a single natural weapon, it adds 1-1/2 times its Strength bonus to 
the damage. If it has more than a single attack then it adds its Strength bonus 
to damage rolls for the primary attack and 1/2 its Strength bonus to all secondary 
attacks.

\textbf{Space/Reach:} A template may change this entry if it changes the monster's 
size. Note that this table does not take into account special situations such as 
exceptional reach.

\textbf{Special Attacks:} A template may add or remove special attacks. The template 
description gives the details of any special attacks a template provides, including 
how to determine saving throw DCs, if applicable.

\textbf{Special Qualities:} A template may add or remove special qualities. The 
template description gives the details of any special qualities a template provides, 
including how to determine saving throw DCs, if applicable. Even if the special 
qualities entry is missing from a template description, the creature still gains 
any qualities associated with its new type.

\textbf{Base Saves:} As with attacks, changing a monster's type does not always 
change its base saving throw bonuses. You only need to adjust them for new modifiers 
for Constitution, Dexterity, or Wisdom. A template may, however, state that a monster 
has a different ``good'' saving throw.

\textbf{Abilities:} If a template changes one or more ability scores, these changes 
are noted here.

\textbf{Skills: }As with attacks, changing a monster's type does not always change 
its skill points. Most templates don't change the number of Hit Dice a creature 
has, so you don't need to adjust skills in that case unless the key abilities for 
those skills have changed, or the template gives a bonus on one or more skills, 
or unless the template gives a feat that provides a bonus on a skill check. 

Some templates change how skill points are determined, but this change usually 
only affects skill points gained after the template is applied. Treat skills listed 
in the base creature's description as class skills, as well as any new skills provided 
by the template. 

\textbf{Feats:} Since most templates do not change the number of Hit Dice a creature 
has, a template will not change the number of feats the creature has. Some templates 
grant one or more bonus feats.

\textbf{Environment:} Usually the same as the base creature.

\textbf{Organization:} Usually the same as the base creature.

\textbf{Challenge Rating:} Most templates increase the creature's Challenge Rating. 
A template might provide a modifier to be added to the base creature's CR, or it 
might specify a range of modifiers depending on the base creature's original Hit 
Dice or CR.

\textbf{Treasure: }Usually the same as the base creature.

\textbf{Alignment:} Usually the same as the base creature, unless the template 
is associated with a certain alignment.

\textbf{Advancement:} Usually the same as the base creature.

\textbf{Level Adjustment:} This entry is a modifier to the base creature's level 
adjustment. Any level adjustment is meaningless unless the creature retains a high 
enough Intelligence (minimum 3) to gain class levels after applying the template.

\vspace{12pt}
Adding More Than One Template

In theory, there's no limit to the number of templates you can add to a creature. 
To add more than one template, just apply each template one at a time. Always apply 
inherited templates before applying acquired templates. Whenever you add multiple 
templates, pay attention to the creature's type---you may add a template that makes 
the creature ineligible for other templates you might want to add.

\vspace{12pt}
{\LARGE{}ADVANCED MONSTER CHALLENGE RATING}

When adding class levels to a creature with 1 or less HD, you advance the creature 
like a character. Otherwise, use the following guidelines.

\vspace{12pt}
ADDING CLASS LEVELS

If you are advancing a monster by adding player character class levels, decide 
if the class levels directly improve the monster's existing capabilities.

When adding class levels to a creature, you should give it typical ability scores 
appropriate for that class. Most creatures are built using the standard array of 
ability scores: 11, 11, 11, 10, 10, 10, adjusted by racial modifiers. If you give 
a creature a PC class use the elite array of ability scores before racial adjustments: 
15, 14, 13, 12, 10, 8. Creatures with NPC classes use the nonelite array of 13, 
12, 11, 10, 9, 8. T

\vspace{12pt}
Associated Class Levels

Class levels that increase a monster's existing strengths are known as associated 
class levels. Each associated class level a monster has increases its CR by 1.

Barbarian, fighter, paladin, and ranger are associated classes for a creature that 
relies on its fighting ability.

Rogue and ranger are associated classes for a creature that relies on stealth to 
surprise its foes, or on skill use to give itself an advantage. 

A spellcasting class is an associated class for a creature that already has the 
ability to cast spells as a character of the class in question, since the monster's 
levels in the spellcasting class stack with its innate spellcasting ability.

\vspace{12pt}
Nonassociated Class Levels

If you add a class level that doesn't directly play to a creature's strength the 
class level is considered nonassociated, and things get a little more complicated\textit{. 
}Adding a nonassociated class level to a monster increases its CR by 1/2 per level 
until one of its nonassociated class levels equals its original Hit Dice. At that 
point, each additional level of the same class or a similar one is considered associated 
and increases the monster's CR by 1.

Levels in NPC classes are always treated as nonassociated.

\vspace{12pt}
ADDING HIT DICE

When you improve a monster by adding Hit Dice, use Table: Improved Monster CR Increase 
to determine the effect on the creature's CR. Keep in mind that many monsters that 
advance by adding Hit Dice also increase in size. Do not stack this CR increase 
with any increase from class levels. In general, once you've doubled a creature's 
CR, you should closely watch any additional increases in its abilities. Adding 
Hit Dice to a creature improves several of its abilities, and radical increases 
might not follow this progression indefinitely. Compare the monster's improved 
attack bonus, saving throw bonuses, and any DCs of its special abilities from the 
HD increase to typical characters of the appropriate level and adjust the CR accordingly.

\begin{tabular}{|>{\raggedright}p{246pt}|>{\raggedright}p{79pt}|}
\hline
\multicolumn{2}{|p{326pt}|}{\section*{T\textbf{able: Improved Monster CR Increase}}}\tabularnewline
\hline
C\textbf{reature's Original Type} & C\textbf{R Increase}\tabularnewline
\hline
Aberration, construct, elemental, fey, giant, humanoid, ooze, plant, undead, vermin & +1 
per 4 HD added\tabularnewline
\hline
Animal, magical beast, monstrous humanoid & +1 per 3 HD added\tabularnewline
\hline
Dragon, outsider, nonassociated class levels & +1 per 2 HD or 2 levels added\tabularnewline
\hline
Directly associated class levels & +1 per level added\tabularnewline
\hline
O\textbf{ther Modifiers:} & \tabularnewline
\hline
Size increased to Large or larger & +1 to CR\tabularnewline
\hline
Monster's ability scores based on elite array* & +1 to CR\tabularnewline
\hline
Monster possesses special attacks or qualities that significantly improve combat 
effectiveness & +2 to CR\tabularnewline
\hline
Monster possesses special attacks or qualities that improve combat effectiveness 
in a minor way & +1 to CR\tabularnewline
\hline
Template added & + template CR modifier\tabularnewline
\hline
\multicolumn{2}{|p{326pt}|}{* Do not apply this increase if you advance a monster 
by class levels. (Monsters advanced by class levels are assumed to use the elite 
array.)}\tabularnewline
\hline
\end{tabular}

\vspace{12pt}
INCREASING SIZE

Generally, increasing a monster's size increases its combat effectiveness. Large 
creatures gain increased Strength, reach, and other benefits. Apply this modifier 
if you increase a creature beyond Medium and in conjunction with any other increases.

Be careful, though. Monsters that benefit from a smaller size may actually lose 
effectiveness because of a size increase. Monsters that don't benefit from size 
increases don't advance in that manner for this reason.

\vspace{12pt}
ADDING SPECIAL ABILITIES

You can add any sort of spell-like, supernatural, or extraordinary ability to a 
creature. As with a class level, you should determine how much, or how little, 
this ability adds to the creature's existing repertoire. A suite of abilities that 
work together should be treated as a single modifier for this purpose. If the ability 
(or combination of abilities) significantly increases the monster's combat effectiveness, 
increase its CR by 2. Minor abilities increase the creature's CR by 1, and truly 
trivial abilities may not increase CR at all. If the special abilities a monster 
gains are not tied to a class or Hit Die increase, this CR increase stacks.

A significant special attack is one that stands a good chance of incapacitating 
or crippling a character in one round. A significant special quality is one that 
seriously diminishes the monster's vulnerability to common attacks. Do not add 
this factor twice if a monster has both special attacks and special qualities.

Make sure to ``scale'' your evaluation of these abilities by the monster's current 
CR.

\newpage

\end{document}
