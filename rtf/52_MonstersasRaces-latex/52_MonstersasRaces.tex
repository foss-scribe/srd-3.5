%&pdfLaTeX
% !TEX encoding = UTF-8 Unicode
\documentclass{article}
\usepackage{ifxetex}
\ifxetex
\usepackage{fontspec}
\setmainfont[Mapping=tex-text]{STIXGeneral}
\else
\usepackage[T1]{fontenc}
\usepackage[utf8]{inputenc}
\fi
\usepackage{textcomp}

\usepackage{array}
\usepackage{amssymb}
\usepackage{fancyhdr}
\renewcommand{\headrulewidth}{0pt}
\renewcommand{\footrulewidth}{0pt}

\begin{document}

This material is Open Game Content, and is licensed for public use under the terms 
of the Open Game License v1.0a.

\subsection*{{\LARGE{}MONSTERS AS RACES}}

\vspace{12pt}
While every monster\textit{ }has the statistics that a player would need to play 
the creature as a character, most monsters are not suitable as PCs. Creatures who 
have an Intelligence score of 2 or lower, who have no way to communicate, or who 
are so different from other PCs that they disrupt the campaign should not be used. 
 Some creatures have strange innate abilities or great physical power, and thus 
are questionable at best as characters (except in high-level campaigns). 

\vspace{12pt}
\textbf{Starting Level of a Monster PC:} Monsters suitable for play have a level 
adjustment given in their statistics. Add a monster's level adjustment to its Hit 
Dice and class levels to get the creature's effective character level, or ECL. 
Effectively, monsters with a level adjustment become multiclass character when 
they take class levels\textit{.} A creature's ``monster class'' is always a favored 
class, and the creature never takes XP penalties for having it.

\vspace{12pt}
\textbf{Humanoids and Class Levels:} Creatures with 1 or less HD replace their 
monster levels with their character levels. The monster loses the attack bonus, 
saving throw bonuses, skills, and feats granted by its 1 monster HD and gains the 
attack bonus, save bonuses, skills, feats, and other class abilities of a 1st-level 
character of the appropriate class.

Characters with more than 1 Hit Die because of their race do not get a feat for 
their first class level as members of the common races do, and they do not multiply 
the skill points for their first class level by four. Instead, they have already 
received a feat for their first Hit Die because of race, and they have already 
multiplied their racial skill points for their first Hit Die by four.

\vspace{12pt}
\textbf{Level Adjustment and Effective Character Level:} To determine the effective 
character level (ECL) of a monster character, add its level adjustment to its racial 
Hit Dice and character class levels. 

Use ECL instead of character level\textit{ }to determine how many experience points 
a monster character needs to reach its next level. Also use ECL to determine starting 
wealth for a monster character.

Monster characters treat skills mentioned in their monster entry as class skills. 

If a monster has 1 Hit Die or less, or if it is a template creature, it must start 
the game with one or more class levels, like a regular character. If a monster 
has 2 or more Hit Dice, it can start with no class levels (though it can gain them 
later).

Even if the creature is of a kind that normally advances by Hit Dice rather than 
class levels a PC can gain class levels rather than Hit Dice.

\vspace{12pt}
\subsubsection*{\textbf{Hit Dice: }The creature's Hit Dice equal the number of 
class levels it has plus its racial Hit Dice. Additional Hit Dice gained from taking 
levels in a character class never affect a creature's size like additional racial 
Hit Dice do.}

\vspace{12pt}
\textbf{Feat Acquisition and Ability Score Increases:} A monster's total Hit Dice, 
not its ECL, govern its acquisition of feats and ability score increases. 

\vspace{12pt}
\textbf{Ability Scores for Monster PCs: }While a monsters statistics give the ability 
scores for a typical creature of a certain kind, any ``monster'' creature that 
becomes an adventurer is definitely not typical. Therefore, when creating a PC 
from a creature\textit{, }check to see if the creature's entry has any ability 
scores of 10 or higher. If so, for each score, subtract 10 (if the score is even) 
or 11 (if the score is odd) to get the creature's modifier for that ability based 
on its race or kind. Generate the character's ability scores as normal, then add 
the racial ability modifiers to get their ability scores. 

\textbf{Note:} Some monsters have base ability scores other than 10 and 11.  If 
alternate scores were used this will be indicated in the monster entry.  Also, 
some monsters that make good PCs have their racial ability modifiers and other 
traits already listed in their monster entry.

\vspace{12pt}
For ability scores lower than 10, the procedure is different. First, determine 
the character's ability scores, and compare that number to the monster's average 
ability score, using either the table below that applies to Intelligence or the 
table that applies to the other five ability scores. 

The separate table for Intelligence ensures that no PC ends up with an Intelligence 
score lower than 3. This is important, because creatures with an Intelligence score 
lower than 3 are not playable characters. Creatures with any ability score lower 
than 1 are also not playable.

\vspace{12pt}
\begin{tabular}{|>{\raggedright}p{43pt}|>{\raggedright}p{42pt}|>{\raggedright}p{42pt}|>{\raggedright}p{42pt}|>{\raggedright}p{42pt}|}
\hline
\multicolumn{5}{|p{212pt}|}{M\textbf{onster PCs' Intelligence Scores}}\tabularnewline
\hline
G\textbf{enerated Score}----------- & \multicolumn{4}{p{168pt}|}{ \textbf{Monster 
Intelligence Score ------------}}\tabularnewline
\hline
 & 3 & 4-\textbf{5} & 6-\textbf{7} & 8-\textbf{9}\tabularnewline
\hline
18 & 10 & 12 & 14 & 16\tabularnewline
\hline
17 & 9 & 11 & 13 & 15\tabularnewline
\hline
16 & 8 & 10 & 12 & 14\tabularnewline
\hline
15 & 7 & 9 & 11 & 13\tabularnewline
\hline
14 & 6 & 8 & 10 & 12\tabularnewline
\hline
13 & 5 & 7 & 9 & 11\tabularnewline
\hline
12 & 4 & 6 & 8 & 10\tabularnewline
\hline
11 & 3 & 5 & 7 & 9\tabularnewline
\hline
10 & 3 & 5 & 7 & 9\tabularnewline
\hline
9 & 3 & 5 & 6 & 8\tabularnewline
\hline
8 & 3 & 4 & 6 & 8\tabularnewline
\hline
7 & 3 & 4 & 5 & 7\tabularnewline
\hline
6 & 3 & 4 & 5 & 6\tabularnewline
\hline
5 & 3 & 3 & 5 & 5\tabularnewline
\hline
4 & 3 & 3 & 4 & 4\tabularnewline
\hline
3 & 3 & 3 & 3 & 3\tabularnewline
\hline
\end{tabular}

\vspace{12pt}
\begin{tabular}{|>{\raggedright}p{43pt}|>{\raggedright}p{36pt}|>{\raggedright}p{36pt}|>{\raggedright}p{36pt}|>{\raggedright}p{36pt}|>{\raggedright}p{36pt}|}
\hline
\multicolumn{6}{|p{224pt}|}{M\textbf{onster PCs' Ability Scores}}\tabularnewline
\hline
G\textbf{enerated Score}--- & \multicolumn{5}{p{181pt}|}{M\textbf{onster Ability 
Score (Str, Dex, Con, Wis, Cha)---}}\tabularnewline
\hline
 & 1 & 2-\textbf{3} & 4-\textbf{5} & 6-\textbf{7} & 8-\textbf{9}\tabularnewline
\hline
18 & 8 & 10 & 12 & 14 & 16\tabularnewline
\hline
17 & 7 & 9 & 11 & 13 & 15\tabularnewline
\hline
16 & 6 & 8 & 10 & 12 & 14\tabularnewline
\hline
15 & 5 & 7 & 9 & 11 & 13\tabularnewline
\hline
14 & 4 & 6 & 8 & 10 & 12\tabularnewline
\hline
13 & 3 & 5 & 7 & 9 & 11\tabularnewline
\hline
12 & 2 & 4 & 6 & 8 & 10\tabularnewline
\hline
11 & 1 & 3 & 5 & 7 & 9\tabularnewline
\hline
10 & 1 & 2 & 4 & 6 & 8\tabularnewline
\hline
9 & 1 & 2 & 4 & 6 & 7\tabularnewline
\hline
8 & 1 & 2 & 4 & 5 & 6\tabularnewline
\hline
7 & 1 & 1 & 3 & 5 & 5\tabularnewline
\hline
6 & 1 & 1 & 2 & 4 & 4\tabularnewline
\hline
5 & 1 & 1 & 1 & 3 & 3\tabularnewline
\hline
4 & 1 & 1 & 1 & 2 & 2\tabularnewline
\hline
3 & 1 & 1 & 1 & 1 & 1\tabularnewline
\hline
\end{tabular}

\vspace{12pt}
\textbf{Other Statistics for Monsters: }Creatures with Hit Dice of 1 or less have 
normal, class-based Hit Dice and features. They get skills and feats appropriate 
to a 1st-level character (even if they have a level adjustment).

Those with 2 or more Hit Dice have statistics based on these Hit Dice plus Hit 
Dice for class levels (if any).

\vspace{12pt}
\textbf{Experience for Monsters: }A monster with Hit Dice of 1 or less, no level 
adjustment, and class levels uses the same tables as standard PC races when determining 
experience needed.

A monster with Hit Dice of 1 or less, a level adjustment, and class levels adds 
its class levels, Hit Die, and level adjustment together when determining experience 
needed (class level + HD + level adjustment).

A monster with more than one Hit Die, a level adjustment, and class levels adds 
its Hit Dice, class levels, and level adjustment together when determining experience 
needed (HD + level adjustment + class level).

\newpage

\end{document}
