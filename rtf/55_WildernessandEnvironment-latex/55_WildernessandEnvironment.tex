%&pdfLaTeX
% !TEX encoding = UTF-8 Unicode
\documentclass{article}
\usepackage{ifxetex}
\ifxetex
\usepackage{fontspec}
\setmainfont[Mapping=tex-text]{STIXGeneral}
\else
\usepackage[T1]{fontenc}
\usepackage[utf8]{inputenc}
\fi
\usepackage{textcomp}

\usepackage{array}
\usepackage{amssymb}
\usepackage{fancyhdr}
\renewcommand{\headrulewidth}{0pt}
\renewcommand{\footrulewidth}{0pt}

\begin{document}

This material is Open Game Content, and is licensed for public use under the terms 
of the Open Game License v1.0a.

{\LARGE{}WILDERNESS, WEATHER, \& ENVIRONMENT}

\vspace{12pt}
{\LARGE{}DUNGEONS}

TYPES OF DUNGEONS

The four basic dungeon types are defined by their current status. Many dungeons 
are variations on these basic types or combinations of more than one of them. Sometimes 
old dungeons are used again and again by different inhabitants for different purposes.

\textbf{Ruined Structure:} Once occupied, this place is now abandoned (completely 
or in part) by its original creator or creators, and other creatures have wandered 
in. Many subterranean creatures look for abandoned underground constructions in 
which to make their lairs. Any traps that might exist have probably been set off, 
but wandering beasts might very well be common.

\textbf{Occupied Structure:} This type of dungeon is still in use. Creatures (usually 
intelligent) live there, although they may not be the dungeon's creators. An occupied 
structure might be a home, a fortress, a temple, an active mine, a prison, or a 
headquarters. This type of dungeon is less likely to have traps or wandering beasts, 
and more likely to have organized guards---both on watch and on patrol. Traps or 
wandering beasts that might be encountered are usually under the control of the 
occupants. Occupied structures have furnishings to suit the inhabitants, as well 
as decorations, supplies, and the ability for occupants to move around (doors they 
can open, hallways large enough for them to pass through, and so on). The inhabitants 
might have a communication system, and they almost certainly control an access 
to the outside.

Some dungeons are partially occupied and partially empty or in ruins. In such cases, 
the occupants are typically not the original builders but instead a group of intelligent 
creatures that have set up their base, lair, or fortification within an abandoned 
dungeon.

\textbf{Safe Storage: }When people want to protect something, they might bury it 
underground. Whether the item they want to protect is a fabulous treasure, a forbidden 
artifact, or the dead body of an important figure, these valuable objects are placed 
within a dungeon and surrounded by barriers, traps, and guardians.

The safe storage type of dungeon is the most likely to have traps but the least 
likely to have wandering beasts. This type of dungeon normally is built for function 
rather than appearance, but sometimes it has ornamentation in the form of statuary 
or painted walls. This is particularly true of the tombs of important people.

Sometimes, however, a vault or a crypt is constructed in such a way as to house 
living guardians. The problem with this strategy is that something must be done 
to keep the creatures alive between intrusion attempts. Magic is usually the best 
solution to provide food and water for these creatures. Even if there's no way 
anything living can survive in a safe storage dungeon, certain monsters can still 
serve as guardians. Builders of vaults or tombs often place undead creatures or 
constructs, both of which which have no need for sustenance or rest, to guard their 
dungeons. Magic traps can attack intruders by summoning monsters into the dungeon. 
These guardians also need no sustenance, since they appear only when they're needed 
and disappear when their task is done.

\textbf{Natural Cavern Complex:} Underground caves provide homes for all sorts 
of subterranean monsters. Created naturally and connected by a labyrinthine tunnel 
system, these caverns lack any sort of pattern, order, or decoration. With no intelligent 
force behind its construction, this type of dungeon is the least likely to have 
traps or even doors.

Fungi of all sorts thrive in caves, sometimes growing in huge forests of mushrooms 
and puffballs. Subterranean predators prowl these forests, looking for those feeding 
upon the fungi. Some varieties of fungus give off a phosphorescent glow, providing 
a natural cavern complex with its own limited light source. In other areas, a \textit{daylight 
}spell or similar magical effect can provide enough light for green plants to grow.

Often, a natural cavern complex connects with another type of dungeons, the caves 
having been discovered when the manufactured dungeon was delved. A cavern complex 
can connect two otherwise unrelated dungeons, sometimes creating a strange mixed 
environment. A natural cavern complex joined with another dungeon often provides 
a route by which subterranean creatures find their way into a manufactured dungeon 
and populate it.

\vspace{12pt}
{\LARGE{}DUNGEON TERRAIN}

{\large{}WALLS}

Sometimes, masonry walls---stones piled on top of each other (usually but not always 
held in place with mortar)---divide dungeons into corridors and chambers. Dungeon 
walls can also be hewn from solid rock, leaving them with a rough, chiseled look. 
Or, dungeon walls can be the smooth, unblemished stone of a naturally occurring 
cave. Dungeon walls are difficult to break down or through, but they're generally 
easy to climb.

\vspace{12pt}
\begin{tabular}{|>{\raggedright}p{69pt}|>{\raggedright}p{65pt}|>{\raggedright}p{35pt}|>{\raggedright}p{33pt}|>{\raggedright}p{38pt}|>{\raggedright}p{36pt}|}
\hline
\multicolumn{6}{|p{278pt}|}{T\textbf{able: Walls}}\tabularnewline
\hline
W\textbf{all Type} & T\textbf{ypical Thickness } & B\textbf{reak DC} & H\textbf{ardness} & H\textbf{it 
Points}\textsuperscript{\textbf{1}} & C\textbf{limb DC}\tabularnewline
\hline
Masonry & 1 ft. & 35 & 8 & 90 hp & 15\tabularnewline
\hline
Superior masonry & 1 ft. & 35 & 8 & 90 hp & 20\tabularnewline
\hline
Reinforced masonry & 1 ft. & 45 & 8 & 180 hp & 15\tabularnewline
\hline
Hewn stone & 3 ft. & 50 & 8 & 540 hp & 22\tabularnewline
\hline
Unworked stone & 5 ft. & 65 & 8 & 900 hp & 20\tabularnewline
\hline
Iron & 3 in. & 30 & 10 & 90 hp & 25\tabularnewline
\hline
Paper & Paper-thin & 1--- &  & 1 hp & 30\tabularnewline
\hline
Wood & 6 in. & 20 & 5 & 60 hp & 21\tabularnewline
\hline
Magically treated\textsuperscript{\textbf{2}}--- &  & +20 & \ensuremath{\times}2 & \ensuremath{\times}2\textsuperscript{\textbf{3}}--- & \tabularnewline
\hline
\multicolumn{6}{|p{278pt}|}{1 Per 10-foot-by-10-foot section.}\tabularnewline
\hline
\multicolumn{6}{|p{278pt}|}{2 These modifiers can be applied to any of the other 
wall types.}\tabularnewline
\hline
\multicolumn{6}{|p{278pt}|}{3 Or an additional 50 hit points, whichever is greater.}\tabularnewline
\hline
\end{tabular}

\vspace{12pt}
\textbf{Masonry Walls:} The most common kind of dungeon wall, masonry walls are 
usually at least 1 foot thick. Often these ancient walls sport cracks and crevices, 
and sometimes dangerous slimes or small monsters live in these areas and wait for 
prey. Masonry walls stop all but the loudest noises. It takes a DC 20 Climb check 
to travel along a masonry wall.

\textbf{Superior Masonry Walls:} Sometimes masonry walls are better built (smoother, 
with tighter-fitting stones and less cracking), and occasionally these superior 
walls are covered with plaster or stucco. Covered walls often bear paintings, carved 
reliefs, or other decoration. Superior masonry walls are no more difficult to destroy 
than regular masonry walls but are more difficult to climb (DC 25).

\textbf{Hewn Stone Walls:} Such walls usually result when a chamber or passage 
is tunneled out from solid rock. The rough surface of a hewn wall frequently provides 
minuscule ledges where fungus grows and fissures where vermin, bats, and subterranean 
snakes live. When such a wall has an ``other side'' (it separates two chambers 
in the dungeon), the wall is usually at least 3 feet thick; anything thinner risks 
collapsing from the weight of all the stone overhead. It takes a DC 25 Climb check 
to climb a hewn stone wall.

\textbf{Unworked Stone Walls:} These surfaces are uneven and rarely flat. They 
are smooth to the touch but filled with tiny holes, hidden alcoves, and ledges 
at various heights. They're also usually wet or at least damp, since it's water 
that most frequently creates natural caves. When such a wall has an ``other side,'' 
the wall is usually at least 5 feet thick. It takes a DC 15 Climb check to move 
along an unworked stone wall. 

\vspace{12pt}
SPECIAL WALLS

\textbf{Reinforced Walls:}\textit{ }These are masonry walls with iron bars on one 
or both sides of the wall, or placed within the wall to strengthen it. The hardness 
of a reinforced wall remains the same, but its hit points are doubled and the Strength 
check DC to break through it is increased by 10.

\textbf{Iron Walls: }These walls are placed within dungeons around important places 
such as vaults. 

\textbf{Paper Walls:}\textit{ }Paper walls are the opposite of iron walls, placed 
as screens to block line of sight but nothing more.

\textbf{Wooden Walls:}\textit{ }Wooden walls often exist as recent additions to 
older dungeons, used to create animal pens, storage bins, or just to make a number 
of smaller rooms out of a larger one.

\textbf{Magically Treated Walls:}\textit{ }These walls are stronger than average, 
with a greater hardness, more hit points, and a higher break DC. Magic can usually 
double the hardness and hit points and can add up to 20 to the break DC. A magically 
treated wall also gains a saving throw against spells that could affect it, with 
the save bonus equaling 2 + one-half the caster level of the magic reinforcing 
the wall. Creating a magic wall requires the Craft Wondrous Item feat and the expenditure 
of 1,500 gp for each 10 foot-by-10-foot wall section.

\textbf{Walls with Arrow Slits: }Walls with arrow slits can be made of any durable 
material but are most commonly masonry, hewn stone, or wood. Such a wall allows 
defenders to fire arrows or crossbow bolts at intruders from behind the safety 
of the wall. Archers behind arrow slits have improved cover that gives them a +8 
bonus to Armor Class, a +4 bonus on Reflex saves, and the benefits of the improved 
evasion class feature.

\vspace{12pt}
FLOORS

As with walls, dungeon floors come in many types.

\textbf{Flagstone: }Like masonry walls, flagstone floors are made of fitted stones. 
They are usually cracked and only somewhat level. Slime and mold grows in these 
cracks. Sometimes water runs in rivulets between the stones or sits in stagnant 
puddles. Flagstone is the most common dungeon floor.

\textbf{Uneven Flagstone:} Over time, some floors can become so uneven that a DC 
10 Balance check is required to run or charge across the surface. Failure means 
the character can't move in this round. Floors as treacherous as this should be 
the exception, not the rule.

\textbf{Hewn Stone Floors:} Rough and uneven, hewn floors are usually covered with 
loose stones, gravel, dirt, or other debris. A DC 10 Balance check is required 
to run or charge across such a floor. Failure means the character can still act, 
but can't run or charge in this round.

\textbf{Light Rubble: }Small chunks of debris litter the ground. Light rubble adds 
2 to the DC of Balance and Tumble checks.

\textbf{Dense Rubble:} The ground is covered with debris of all sizes. It costs 
2 squares of movement to enter a square with dense rubble. Dense rubble adds 5 
to the DC of Balance and Tumble checks, and it adds 2 to the DC of Move Silently 
checks.

\textbf{Smooth Stone Floors:} Finished and sometimes even polished, smooth floors 
are found only in dungeons with capable and careful builders. 

\textbf{Natural Stone Floors:} The floor of a natural cave is as uneven as the 
walls. Caves rarely have flat surfaces of any great size. Rather, their floors 
have many levels. Some adjacent floor surfaces might vary in elevation by only 
a foot, so that moving from one to the other is no more difficult than negotiating 
a stair step, but in other places the floor might suddenly drop off or rise up 
several feet or more, requiring Climb checks to get from one surface to the other. 
Unless a path has been worn and well marked in the floor of a natural cave, it 
takes 2 squares of movement to enter a square with a natural stone floor, and the 
DC of Balance and Tumble checks increases by 5. Running and charging are impossible, 
except along paths.

\vspace{12pt}
SPECIAL FLOORS

\textbf{Slippery: }Water, ice, slime, or blood can make any of the dungeon floors 
described in this section more treacherous. Slippery floors increase the DC of 
Balance and Tumble checks by 5. 

\textbf{Grate: }A grate often covers a pit or an area lower than the main floor. 
Grates are usually made from iron, but large ones can also be made from iron-bound 
timbers. Many grates have hinges to allow access to what lies below (such grates 
can be locked like any door), while others are permanent and designed not to move. 
A typical 1-inch-thick iron grate has 25 hit points, hardness 10, and a DC of 27 
for Strength checks to break through it or tear it loose.

\textbf{Ledge: }Ledges allow creatures to walk above some lower area. They often 
circle around pits, run along underground streams, form balconies around large 
rooms, or provide a place for archers to stand while firing upon enemies below. 
Narrow ledges (12 inches wide or less) require those moving along them to make 
Balance checks. Failure results in the moving character

falling off the ledge. Ledges sometimes have railings. In such a case, characters 
gain a +5 circumstance bonus on Balance checks to move along the ledge. A character 
who is next to a railing gains a +2 circumstance bonus on his or her opposed Strength 
check to avoid being bull rushed off the edge.

Ledges can also have low walls 2 to 3 feet high along their edges. Such walls provide 
cover against attackers within 30 feet on the other side of the wall, as long as 
the target is closer to the low wall than the attacker is.

\textbf{Transparent Floor:}\textit{ }Transparent floors, made of reinforced glass 
or magic materials (even a \textit{wall of force}), allow a dangerous setting to 
be viewed safely from above. Transparent floors are sometimes placed over lava 
pools, arenas, monster dens, and torture chambers. They can be used by defenders 
to watch key areas for intruders.

\textbf{Sliding Floors:} A sliding floor is a type of trapdoor, designed to be 
moved and thus reveal something that lies beneath it. A typical sliding floor moves 
so slowly that anyone standing on one can avoid falling into the gap it creates, 
assuming there's somewhere else to go. If such a floor slides quickly enough that 
there's a chance of a character falling into whatever lies beneath---a spiked pit, 
a vat of burning oil, or a pool filled with sharks---then it's a trap.

\textbf{Trap Floors:} Some floors are designed to become suddenly dangerous. With 
the application of just the right amount of weight, or the pull of a lever somewhere 
nearby, spikes protrude from the floor, gouts of steam or flame shoot up from hidden 
holes, or the entire floor tilts. These strange floors are sometimes found in an 
arena, designed to make combats more exciting and deadly. Construct these floors 
as you would any other trap. 

\vspace{12pt}
DOORS

Doors in dungeons are much more than mere entrances and exits. Often they can be 
encounters all by themselves. 

Dungeon doors come in three basic types: wooden, stone, and iron.

\vspace{12pt}
\begin{tabular}{|>{\raggedright}p{72pt}|>{\raggedright}p{74pt}|>{\raggedright}p{38pt}|>{\raggedright}p{40pt}|>{\raggedright}p{22pt}|>{\raggedright}p{29pt}|}
\hline
\multicolumn{6}{|p{278pt}|}{T\textbf{able: Doors}}\tabularnewline
\hline
  &   &   &   & \multicolumn{2}{p{52pt}|}{B\textbf{reak DC}}\tabularnewline
\hline
D\textbf{oor Type} & T\textbf{ypical Thickness} & H\textbf{ardness} & H\textbf{it 
Points} & S\textbf{tuck} & L\textbf{ocked}\tabularnewline
\hline
Simple wooden & 1 in. & 5 & 10 hp & 13 & 15\tabularnewline
\hline
Good wooden & 1-1/2 in. & 5 & 15 hp & 16 & 18\tabularnewline
\hline
Strong wooden & 2 in. & 5 & 20 hp & 23 & 25\tabularnewline
\hline
Stone & 4 in. & 8 & 60 hp & 28 & 28\tabularnewline
\hline
Iron & 2 in. & 10 & 60 hp & 28 & 28\tabularnewline
\hline
Portcullis, wooden & 3 in & 5 & 30 hp & 25\textsuperscript{\textbf{1}} & 25\textsuperscript{\textbf{1}}\tabularnewline
\hline
Portcullis, iron & 2 in. & 10 & 60 hp & 25\textsuperscript{\textbf{1}} & 25\textsuperscript{\textbf{1}}\tabularnewline
\hline
Lock--- &  & 15 & 30 hp &  & \tabularnewline
\hline
Hinge--- &  & 10 & 30 hp &  & \tabularnewline
\hline
\multicolumn{6}{|p{278pt}|}{1 DC to lift. Use appropriate door figure for breaking.}\tabularnewline
\hline
\end{tabular}

\vspace{12pt}
\textbf{Wooden Doors: }Constructed of thick planks nailed together, sometimes bound 
with iron for strength (and to reduce swelling from dungeon dampness), wooden doors 
are the most common type. Wooden doors come in varying strengths: simple, good, 
and strong doors. Simple doors (break DC 13) are not meant to keep out motivated 
attackers. Good doors (break DC 16), while sturdy and long-lasting, are still not 
meant to take much punishment. Strong doors (break DC 23) are bound in iron and 
are a sturdy barrier to those attempting to get past them. Iron hinges fasten the 
door to its frame, and typically a circular pull-ring in the center is there to 
help open it. Sometimes, instead of a pull-ring, a door has an iron pull-bar on 
one or both sides of the door to serve as a handle. In inhabited dungeons, these 
doors are usually well maintained (not stuck) and unlocked, although important 
areas are locked up if possible.

\textbf{Stone:} Carved from solid blocks of stone, these heavy, unwieldy doors 
are often built so that they pivot when opened, although dwarves and other skilled 
craftsfolk are able to fashion hinges strong enough to hold up a stone door. Secret 
doors concealed within a stone wall are usually stone doors. Otherwise, such doors 
stand as tough barriers protecting something important beyond. Thus, they are often 
locked or barred.

\textbf{Iron: }Rusted but sturdy, iron doors in a dungeon are hinged like wooden 
doors. These doors are the toughest form of nonmagical door. They are usually locked 
or barred.

\textbf{Locks, Bars, and Seals:} Dungeon doors may be locked, trapped, reinforced, 
barred, magically sealed, or sometimes just stuck. All but the weakest characters 
can eventually knock down a door with a heavy tool such as a sledgehammer, and 
a number of spells and magic items give characters an easy way around a locked 
door.

Attempts to literally chop a door down with a slashing or bludgeoning weapon use 
the hardness and hit points given in Table: Doors. Often the easiest way to overcome 
a recalcitrant door is not by demolishing it but by breaking its lock, bar, or 
hinges. When assigning a DC to an attempt to knock a door down, use the following 
as guidelines:

\textit{DC 10 or Lower: }a door just about anyone can break open.

\textit{DC 11-15: }a door that a strong person could break with one try and an 
average person might be able to

break with one try. 

\textit{DC 16-20: }a door that almost anyone could break, given time.

\textit{DC 21-25: }a door that only a strong or very strong person has a hope of 
breaking, probably not on the first try.

\textit{DC 26 or Higher: }a door that only an exceptionally strong person has a 
hope of breaking.

For specific examples in applying these guidelines, see Table: Random Door Types. 

\textbf{Locks:} Dungeon doors are often locked, and thus the Open Lock skill comes 
in very handy. Locks are usually built into the door, either on the edge opposite 
the hinges or right in the middle of the door. Builtin locks either control an 
iron bar that juts out of the door and into the wall of its frame, or else a sliding 
iron bar or heavy wooden bar that rests behind the entire door. By contrast, padlocks 
are not built-in but usually run through two rings, one on the door and the other 
on the wall. More complex locks, such as combination locks and puzzle locks, are 
usually built into the door itself. Because such keyless locks are larger and more 
complex, they are typically only found in sturdy doors (strong wooden, stone, or 
iron doors).

The Open Lock DC to pick a lock often falls into the range of 20 to 30, although 
locks with lower or higher DCs can exist. A door can have more than one lock, each 
of which must be unlocked separately. Locks are often trapped, usually with poison 
needles that extend out to prick a rogue's finger.

Breaking a lock is sometimes quicker than breaking the whole door. If a PC wants 
to whack at a lock with a weapon, treat the typical lock as having hardness 15 
and 30 hit points. A lock can only be broken if it can be attacked separately from 
the door, which means that a built-in lock is immune to this sort of treatment. 
In an occupied dungeon, every locked door should have a key somewhere. 

A special door (see below for examples) might have a lock with no key, instead 
requiring that the right combination of nearby levers must be manipulated or the 
right symbols must be pressed on a keypad in the correct sequence to open the door.

\textbf{Stuck Doors: }Dungeons are often damp, and sometimes doors get stuck, particularly 
wooden doors. Assume that about 10\% of wooden doors and 5\% of nonwooden doors 
are stuck. These numbers can be doubled (to 20\% and 10\%, respectively) for long-abandoned 
or neglected dungeons.

\textbf{Barred Doors: }When characters try to bash down a barred door, it's the 
quality of the bar that matters, not the material the door is made of. It takes 
a DC 25 Strength check to break through a door with a wooden bar, and a DC 30 Strength 
check if the bar is made of iron. Characters can attack the door and destroy it 
instead, leaving the bar hanging in the now-open doorway.

\textbf{Magic Seals:} In addition to magic traps spells such as \textit{arcane 
lock }can discourage passage through a door. A door with an \textit{arcane lock 
}spell on it is considered locked even if it doesn't have a physical lock. It takes 
a \textit{knock }spell, a \textit{dispel magic }spell, or a successful Strength 
check  to get through such a door.

\textbf{Hinges:} Most doors have hinges. Obviously, sliding doors do not. (They 
usually have tracks or grooves instead, allowing them to slide easily to one side.)

\textit{Standard Hinges: }These hinges are metal, joining one edge of the door 
to the doorframe or wall. Remember that the door swings open toward the side with 
the hinges. (So, if the hinges are on the PCs' side, the door opens toward them; 
otherwise it opens away from them.) Adventurers can take the hinges apart one at 
a time with successful Disable Device checks (assuming the hinges are on their 
side of the door, of course). Such a task has a DC of 20 because most hinges are 
rusted or stuck. Breaking a hinge is difficult. Most have hardness 10 and 30 hit 
points. The break DC for a hinge is the same as for breaking down the door.

\textit{Nested Hinges: }These hinges are much more complex than ordinary hinges, 
and are found only in areas of excellent construction. These hinges are built into 
the wall and allow the door to swing open in either direction. PCs can't get at 
the hinges to fool with them unless they break through the doorframe or wall. Nested 
hinges are typically found on stone doors but sometimes on wooden or iron doors 
as well. 

\textit{Pivots: }Pivots aren't really hinges at all, but simple knobs jutting from 
the top and bottom of the door that fit into holes in the doorframe, allowing the 
door to spin. The advantages of pivots is that they can't be dismantled like hinges 
and they're simple to make. The disadvantage is that since the door pivots on its 
center of gravity (typically in the middle), nothing larger than half the door's 
width can fit through. Doors with pivots are usually stone and are often quite 
wide to overcome this disadvantage. Another solution is to place the pivot toward 
one side and have the door be thicker at that end and thinner toward the other 
end so that it opens more like a normal door. Secret doors in walls often turn 
on pivots, since the lack of hinges makes it easier to hide the door's presence. 
Pivots also allow objects such as bookcases to be used as secret doors.

\textbf{Secret Doors:} Disguised as a bare patch of wall (or floor, or ceiling), 
a bookcase, a fireplace, or a fountain, a secret door leads to a secret passage 
or room. Someone examining the area finds a secret door, if one exists, on a successful 
Search check (DC 20 for a typical secret door to DC 30 for a well-hidden secret 
door). Elves have a chance to detect a secret door just by casually looking at 
an area.

Many secret doors require a special method of opening, such as a hidden button 
or pressure plate. Secret doors can open like normal doors, or they may pivot, 
slide, sink, rise, or even lower like a drawbridge to permit access. Builders might 
put a secret door down low near the floor or high up in a wall, making it difficult 
to find or reach. Wizards and sorcerers have a spell, \textit{phase door}, that 
allows them to create a magic secret door that only they can use.

\textbf{Magic Doors:} Enchanted by the original builders, a door might speak to 
explorers, warning them away. It might be protected from harm, increasing its hardness 
or giving it more hit points as well as an improved saving throw bonus against 
\textit{disintegrate }and other similar spells. A magic door might not lead into 
the space revealed beyond, but instead it might be a portal to a faraway place 
or even another plane of existence. Other magic doors might require passwords or 
special keys to open them. 

\textbf{Portcullises: }These special doors consist of iron or thick, ironbound, 
wooden shafts that descend from a recess in the ceiling above an archway. Sometimes 
a portcullis has crossbars that create a grid, sometimes not. Typically raised 
by means of a winch or a capstan, a portcullis can be dropped quickly, and the 
shafts end in spikes to discourage anyone from standing underneath (or from attempting 
to dive under it as it drops). Once it is dropped, a portcullis locks, unless it 
is so large that no normal person could lift it anyway. In any event, lifting a 
typical portcullis requires a DC 25 Strength check.

\vspace{12pt}
WALLS, DOORS, AND DETECT SPELLS

Stone walls, iron walls, and iron doors are usually thick enough to block most 
\textit{detect }spells, such as \textit{detect thoughts. }Wooden walls, wooden 
doors, and stone doors are usually not thick enough to do so. However, a secret 
stone door built into a wall and as thick as the wall itself (at least 1 foot) 
does block most \textit{detect }spells.

\vspace{12pt}
{\large{}ROOMS}

Rooms in dungeons vary in shape and size. Although many are simple in construction 
and appearance, particularly interesting rooms have multiple levels joined by stairs, 
ramps, or ladders, as well as statuary, altars, pits, chasms, bridges, and more.

Underground chambers are prone to collapse, so many rooms--- particularly large 
ones---have arched ceilings or pillars to support the weight of the rock overhead.

Common dungeon rooms fall into the following broad categories. 

\textbf{Guard Post:} Intelligent, social denizens of the dungeon will generally 
have a series of adjacent rooms they consider ``theirs,'' and they'll guard the 
entrances to that common area. 

\textbf{Living Quarters: }All but the most nomadic creatures have a lair where 
they can rest, eat, and store their treasure. Living quarters commonly include 
beds (if the creature sleeps), possessions (both valuable and mundane), and some 
sort of food preparation area. Noncombatant creatures such as juveniles and the 
elderly are often found here.

\textbf{Work Area: }Most intelligent creatures do more than just guard, eat, and 
sleep, and many devote rooms to magic laboratories, workshops for weapons and armor, 
or studios for more esoteric tasks.

\textbf{Shrine:} Any creature that is particularly religious may have some place 
dedicated to worship, and others may venerate something of great historical or 
personal value. Depending on the creature's resources and piety, a shrine can be 
humble or extensive. A shrine is where PCs will likely encounter NPC clerics, and 
it's common for wounded monsters to flee to a shrine friendly to them when they 
seek healing.

\textbf{Vault:} Well protected, often by a locked iron door, a vault is a special 
room that contains treasure. There's usually only one entrance---an appropriate 
place for a trap.

\textbf{Crypt:} Although sometimes constructed like a vault, a crypt can also be 
a series of individual rooms, each with its own sarcophagus, or a long hall with 
recesses on either side---shelves to hold coffins or bodies.

Those who are worried about undead rising from the grave take the precaution of 
locking and trapping a crypt from the outside--- making the crypt easy to get into 
but difficult to leave. Those worried about tomb robbers make their crypts difficult 
to get into. Some builders do both, just to be on the safe side.

\vspace{12pt}
{\large{}CORRIDORS}

All dungeons have rooms, and most have corridors. While most corridors simply connect 
rooms, sometimes they can be encounter areas in their own right because of traps, 
guard patrols, and wandering monsters out on the hunt.

\textbf{Corridor Traps:} Because passageways in dungeons tend to be narrow, offering 
few movement options, dungeon builders like to place traps in them. In a cramped 
passageway, there's no way for intruders to move around concealed pits, falling 
stones, arrow traps, tilting floors, and sliding or rolling rocks that fill the 
entire passage. For the same reason, magic traps such as \textit{glyphs of warding 
}are effective in hallways as well.

\textbf{Mazes: }Usually, passages connect chambers in the simplest and straightest 
manner possible. Some dungeon builders, however, design a maze or a labyrinth within 
the dungeon. This sort of construction is difficult to navigate (or at least to 
navigate quickly) and, when filled with monsters or traps, can be an effective 
barrier.

A maze can be used to cut off one area of the dungeon, deflecting intruders away 
from a protected spot. Generally, though, the far side of a maze holds an important 
crypt or vault---someplace that the dungeon's regular inhabitants rarely need to 
get to.

\vspace{12pt}
{\large{}MISCELLANEOUS FEATURES}

\textbf{Stairs:} The usual way to connect different levels of a dungeon is with 
stairs. Straight stairways, spiral staircases, or stairwells with multiple landings 
between flights of stairs are all common in dungeons, as are ramps (sometimes with 
an incline so slight that it can be difficult to notice; Spot DC 15). Stairs are 
important accessways, and are sometimes guarded or trapped. Traps on stairs often 
cause intruders to slide or fall down to the bottom, where a pit, spikes, a pool 
of acid, or some other danger awaits.

\textit{Gradual Stairs: }Stairs that rise less than 5 feet for every 5 feet of 
horizontal distance they cover don't affect movement, but characters who attack 
a foe below them gain a +1 bonus on attack rolls from being on higher ground. Most 
stairs in dungeons are gradual, except for spiral stairs (see below).

\textit{Steep Stairs: }Characters moving up steep stairs (which rise at a 45- degree 
angle or steeper) must spend 2 squares of movement to enter each square of stairs. 
Characters running or charging down steep stairs must succeed on a DC 10 Balance 
check upon entering the first steep stairs square. Characters who fail stumble 
and must end their movement 1d2\ensuremath{\times}5 feet later. Characters who 
fail by 5 or more take 1d6 points of damage and fall prone in the square where 
they end their movement. Steep stairs increase the DC of Tumble checks by 5.

\textit{Spiral Stairs: }This form of steep stairs is designed to make defending 
a fortress easier. Characters gain cover against foes below them on spiral stairs 
because they can easily duck around the staircase's central support.

\textit{Railings and Low Walls: }Stairs that are open to large rooms often have 
railings or low walls. They function as described for ledges (see Special Floors).

\textbf{Bridge: }A bridge connects two higher areas separated by a lower area, 
stretching across a chasm, over a river, or above a pit. A simple bridge might 
be a single wooden plank, while an elaborate one could be made of mortared stone 
with iron supports and side rails.

\textit{Narrow Bridge: }If a bridge is particularly narrow, such as a series of 
planks laid over lava fissures, treat it as a ledge (see Special Floors). It requires 
a Balance check (DC dependent on width) to cross such a bridge.

\textit{Rope Bridge: }Constructed of wooden planks suspended from ropes, a rope 
bridge is convenient because it's portable and can be easily removed. It takes 
two full-round actions to untie one end of a rope bridge, but a DC 15 Use Rope 
check reduces the time to a move action. If only one of the two supporting ropes 
is attached, everyone on the bridge must succeed on a DC 15 Reflex save to avoid 
falling off, and thereafter must make DC 15 Climb checks to move along the remnants 
of the bridge. Rope bridges are usually 5 feet wide. The two ropes that support 
them have 8 hit points each.

\textit{Drawbridge: }Some bridges have mechanisms that allow them to be extended 
or retracted from the gap they cross. Typically,  the winch mechanism exists on 
only one side of the bridge. It takes a move action to lower a drawbridge, but 
the bridge doesn't come down until the beginning of the lowering character's next 
turn. It takes a full-round action to raise a drawbridge; the drawbridge is up 
at the end of the action. Particularly long or wide drawbridges may take more time 
to raise and lower, and some may require Strength checks to rotate the winch.

\textit{Railings and Low Walls: }Some bridges have railings or low walls along 
the sides. If a bridge does, the railing or low walls affect Balance checks and 
bull rush attempts as described for ledges (see Special Floors). Low walls likewise 
provide cover to bridge occupants.

\textbf{Chutes and Chimneys:} Stairs aren't the only way to move up and down in 
a dungeon. Sometimes a vertical shaft connects levels of a dungeon or links a dungeon 
with the surface. Chutes are usually traps that dump characters into a lower area---often 
a place featuring some dangerous situation with which they must contend.

\textbf{Pillar:} A common sight in any dungeon, pillars and columns give support 
to ceilings. The larger the room, the more likely it has pillars. As a rule of 
thumb, the deeper in the dungeon a room is, the thicker the pillars need to be 
to support the overhead weight. Pillars tend to be polished and often have carvings, 
paintings, or inscriptions upon them. 

\textit{Slender Pillar: }These pillars are only a foot or two across, so they don't 
occupy a whole square. A creature standing in the same square as a slender pillar 
gains a +2 cover bonus to Armor Class and a +1 cover bonus on Reflex saves (these 
bonuses don't stack with cover bonuses from other sources). The presence of a slender 
pillar does not otherwise affect a creature's fighting space, because it's assumed 
that the creature is using the pillar to its advantage when it can. A typical slender 
pillar has AC 4, hardness 8, and 250 hit points.

\textit{Wide Pillar: }These pillars take up an entire square and provide cover 
to anyone behind them. They have AC 3, hardness 8, and 900 hit points. A DC 20 
Climb check is sufficient to climb most pillars; the DC increases to 25 for polished 
or unusually slick ones. 

\textbf{Stalagmite/Stalactite:} These tapering natural rock columns extend from 
the floor (stalagmite) or the ceiling (stalactite). Stalagmites and stalactites 
function as slender pillars.

\textbf{Statue: }Most statues function as wide pillars, taking up a square and 
providing cover. Some statues are smaller and act as slender pillars. A DC 15 Climb 
check allows a character to climb a statue. 

\textbf{Tapestry:} Elaborately embroidered patterns or scenes on cloth, tapestries 
hang from the walls of well-appointed dungeon rooms or corridors. Crafty builders 
take advantage of tapestries to place alcoves, concealed doors, or secret switches 
behind them.

Tapestries provide total concealment (50\% miss chance) to characters behind them 
if they're hanging from the ceiling, or concealment (20\% miss chance) if they're 
flush with the wall. Climbing a big tapestry isn't particularly difficult, requiring 
a DC 15 Climb check (or DC 10 if a wall is within reach).

\textbf{Pedestal:} Anything important on display in a dungeon, from a fabulous 
treasure to a coffin, tends to rest atop a pedestal or a dais. Raising the object 
off the floor focuses attention on it (and, in practical terms, keeps it safe from 
any water or other substance that might seep onto the floor). A pedestal is often 
trapped to protect whatever sits atop it. It can conceal a secret trapdoor beneath 
itself or provide a way to reach a door in the ceiling above itself.

Only the largest pedestals take up an entire square; most provide no cover.

\textbf{Pool:} Pools of water collect naturally in low spots in dungeons (a dry 
dungeon is rare). Pools can also be wells or natural underground springs, or they 
can be intentionally created basins, cisterns, and fountains. In any event, water 
is fairly common in dungeons, harboring sightless fish and sometimes aquatic monsters. 
Pools provide water for dungeon denizens, and thus are as important an area for 
a predator to control as a watering hole aboveground in the wild.

\textit{Shallow Pool: }If a square contains a shallow pool, it has roughly 1 foot 
of standing water. It costs 2 squares of movement to move into a square with a 
shallow pool, and the DC of Tumble checks in such squares increases by 2.

\textit{Deep Pool: }These squares have at least 4 feet of standing water. It costs 
Medium or larger creatures 4 squares of movement to move into a square with a deep 
pool, or characters can swim if they wish. Small or smaller creatures must swim 
to move through a square containing a deep pool. Tumbling is impossible in a deep 
pool. The water in a deep pool provides cover for Medium or larger creatures. Smaller 
creatures gain improved cover (+8 bonus to AC, +4 bonus on Reflex saves). Medium 
or larger creatures can crouch as a move action to gain this improved cover. Creatures 
with this improved cover take a -10 penalty on attacks against creatures that aren't 
also underwater. 

Deep pool squares are usually clustered together and surrounded by a ring of shallow 
pool squares. Both shallow pools and deep pools impose a -2 circumstance penalty 
on Move Silently checks.

\textit{Special Pools: }Through accident or design, a pool can become magically 
enhanced. Rarely, a pool or a fountain may be found that has the ability to bestow 
beneficial magic on those who drink from it. However, magic pools are just as likely 
to curse the drinker. Typically, water from a magic pool loses its potency if removed 
from the pool for more than an hour or so.

Some pools have fountains. Occasionally these are merely decorative, but they often 
serve as the focus of a trap or the source of a pool's magic.

Most pools are made of water, but anything's possible in a dungeon. Pools can hold 
unsavory substances such as blood, poison, oil, or magma. And even if a pool holds 
water, it can be holy water, saltwater, or water tainted with disease.

\textbf{Elevator: }In place of or in addition to stairs, an elevator (essentially 
an oversized dumbwaiter) can take inhabitants from one dungeon level to the next. 
Such an elevator may be mechanical (using gears, pulleys, and winches) or magical 
(such as a \textit{levitate }spell cast on a movable flat surface). A mechanical 
elevator might be as small as a platform that holds one character at a time, or 
as large as an entire room that raises and lowers. A clever builder might design 
an elevator room that moves up or down without the occupants' knowledge to catch 
them in a trap, or one that appears to have moved when it actually remained still. 

A typical elevator ascends or descends 10 feet per round at the beginning of the 
operator's turn (or on initiative count 0 if it functions without regard to whether 
creatures are on it. Elevators can be enclosed, can have railings or low walls, 
or may simply be treacherous floating platforms.

\textbf{Ladders:} Whether free-standing or rungs set into a wall, a ladder requires 
a DC 0 Climb check to ascend or descend.

\textbf{Shifting Stone or Wall: }These features can cut off access to a passage 
or room, trapping adventurers in a dead end or preventing escape out of the dungeon. 
Shifting walls can force explorers to go down a dangerous path or prevent them 
from entering a special area. Not all shifting walls need be traps. For example, 
stones controlled by pressure plates, counterweights, or a secret lever can shift 
out of a wall to become a staircase leading to a hidden upper room or secret ledge.

Shifting stones and walls are generally constructed as traps with triggers and 
Search and Disable Device DCs. However they don't have Challenge Ratings because 
they're inconveniences, not deadly in and of themselves.

\textbf{Teleporters:} Sometimes useful, sometimes devious, places in a dungeon 
rigged with a teleportation effect (such as a \textit{teleportation circle}) transport 
characters to some other location in the dungeon or someplace far away. They can 
be traps, teleporting the unwary into dangerous situations, or they can be an easy 
mode of transport for those who built or live in the dungeon, good for bypassing 
barriers and traps or simply to get around more quickly. Devious dungeon designers 
might place a teleporter in a room that transports characters to another seemingly 
identical room so that they don't even know they've been teleported. A \textit{detect 
magic }spell will provide a clue to the presence of a teleporter, but direct experimentation 
or other research is the only way to discover where the teleporter leads.

\textbf{Altars:} Temples---particularly to dark gods---often exist underground. 
Usually taking the form of a stone block, an altar is the main fixture and central 
focus of such a temple. Sometimes all the other trappings of the temple are long 
gone, lost to theft, age, and decay, but the altar survives. Some altars have traps 
or powerful magic within them. Most take up one or two squares on the grid and 
provide cover to creatures behind them. 

\vspace{12pt}
{\large{}CAVE-INS AND COLLAPSES (CR 8)}

Cave-ins and collapsing tunnels are extremely dangerous. Not only do dungeon explorers 
face the danger of being crushed by tons of falling rock, even if they survive 
they may be buried beneath a pile of rubble or cut off from the only known exit. 
A cave-in buries anyone in the middle of the collapsing area, and then sliding 
debris damages anyone in the periphery of the collapse. A typical corridor subject 
to a cave-in might have a bury zone with a 15-foot radius and a 10-foot-radius 
slide zone extending beyond the bury zone. A weakened ceiling can be spotted with 
a DC 20 Knowledge (architecture and engineering) or DC 20 Craft (stonemasonry) 
check. Remember that Craft checks can be made untrained as Intelligence checks. 
A dwarf can make such a check if he simply passes within 10 feet of a weakened 
ceiling. 

A weakened ceiling may collapse when subjected to a major impact or concussion. 
A character can cause a cave-in by destroying half the pillars holding the ceiling 
up. 

Characters in the bury zone of a cave-in take 8d6 points of damage, or half that 
amount if they make a DC 15 Reflex save. They are subsequently buried. Characters 
in the slide zone take 3d6 points of damage, or no damage at all if they make a 
DC 15 Reflex save. Characters in the slide zone who fail their saves are buried.

Characters take 1d6 points of nonlethal damage per minute while buried. If such 
a character falls unconscious, he must make a DC 15 Constitution check. If it fails, 
he takes 1d6 points of lethal damage each minute thereafter until freed or dead.

Characters who aren't buried can dig out their friends. In 1 minute, using only 
her hands, a character can clear rocks and debris equal to five times her heavy 
load limit. The amount of loose stone that fills a 5-foot-by-5-foot area weighs 
one ton (2,000 pounds). Armed with an appropriate tool, such as a pick, crowbar, 
or shovel, a digger can clear loose stone twice as quickly as by hand. You may 
allow a buried character to free himself with a DC 25 Strength check.

\vspace{12pt}
Slimes, Molds, and Fungi

In a dungeon's damp, dark recesses, molds and fungi thrive. While some plants and 
fungi are monsters and other slime, mold, and fungus is just normal, innocuous 
stuff, a few varieties are dangerous dungeon encounters. For purposes of spells 
and other special effects, all slimes, molds, and fungi are treated as plants. 
Like traps, dangerous slimes and molds have CRs, and characters earn XP for encountering 
them.

A form of glistening organic sludge coats almost anything that remains in the damp 
and dark for too long. This kind of slime, though it might be repulsive, is not 
dangerous.

Molds and fungi flourish in dark, cool, damp places. While some are as inoffensive 
as the normal dungeon slime, others are quite dangerous. Mushrooms, puffballs, 
yeasts, mildew, and other sorts of bulbous, fibrous, or flat patches of fungi can 
be found throughout most dungeons. They are usually inoffensive, and some are even 
edible (though most are unappealing or odd-tasting).

\textbf{Green Slime (CR 4):} This dungeon peril is a dangerous variety of normal 
slime. Green slime devours flesh and organic materials on contact and is even capable 
of dissolving metal. Bright green, wet, and sticky, it clings to walls, floors, 
and ceilings in patches, reproducing as it consumes organic matter. It drops from 
walls and ceilings when it detects movement (and possible food) below.

A single 5-foot square of green slime deals 1d6 points of Constitution damage per 
round while it devours flesh. On the first round of contact, the slime can be scraped 
off a creature (most likely destroying the scraping device), but after that it 
must be frozen, burned, or cut away (dealing damage to the victim as well). Anything 
that deals cold or fire damage, sunlight, or a \textit{remove disease }spell destroys 
a patch of green slime. Against wood or metal, green slime deals 2d6 points of 
damage per round, ignoring metal's hardness but not that of wood. It does not harm 
stone.

\textbf{Yellow Mold (CR 6): }If disturbed, a 5-foot square of this mold bursts 
forth with a cloud of poisonous spores. All within 10 feet of the mold must make 
a DC 15 Fortitude save or take 1d6 points of Constitution damage. Another DC 15 
Fortitude save is required 1 minute later---even by those who succeeded on the 
first save---to avoid taking 2d6 points of Constitution damage. Fire destroys yellow 
mold, and sunlight renders it dormant.

\textbf{Brown Mold (CR 2):} Brown mold feeds on warmth, drawing heat from anything 
around it. It normally comes in patches 5 feet in diameter, and the temperature 
is always cold in a 30-foot radius around it. Living creatures within 5 feet of 
it take 3d6 points of nonlethal cold damage. Fire brought within 5 feet of brown 
mold causes it to instantly double in size. Cold damage, such as from a \textit{cone 
of cold, }instantly destroys it.

\textbf{Phosphorescent Fungus (No CR): }This strange underground fungus grows in 
clumps that look almost like stunted shrubbery. Drow elves cultivate it for food 
and light. It gives off a soft violet glow that illuminates underground caverns 
and passages as well as a candle does. Rare patches of fungus illuminate as well 
as a torch does.

\vspace{12pt}
{\LARGE{}WILDERNESS}

\vspace{12pt}
GETTING LOST

There are many ways to get lost in the wilderness. Following an obvious road, trail, 
or feature such as a stream or shoreline prevents any possibility of becoming lost, 
but travelers striking off cross-country may become disoriented---especially in 
conditions of poor visibility or in difficult terrain. 

\textbf{Poor Visibility:} Any time characters cannot see at least 60 feet in the 
prevailing conditions of visibility, they may become lost. Characters traveling 
through fog, snow, or a downpour might easily lose the ability to see any landmarks 
not in their immediate vicinity. Similarly, characters traveling at night may be 
at risk, too, depending on the quality of their light sources, the amount of moonlight, 
and whether they have darkvision or lowlight vision.

\textbf{Difficult Terrain:} Any character in forest, moor, hill, or mountain terrain 
may become lost if he or she moves away from a trail, road, stream, or other obvious 
path or track. Forests are especially dangerous because they obscure far-off landmarks 
and make it hard to see the sun or stars.

\textbf{Chance to Get Lost:} If conditions exist that make getting lost a possibility, 
the character leading the way must succeed on a Survival check or become lost. 
The difficulty of this check varies based on the terrain, the visibility conditions, 
and whether or not the character has a map of the area being traveled through. 
Refer to the table below and use the highest DC that applies.

\begin{tabular}{|>{\raggedright}p{82pt}|>{\raggedright}p{51pt}|>{\raggedright}p{72pt}|>{\raggedright}p{51pt}|}
\hline
\subsection*{ } & \subsection*{S\textbf{urvival DC}} & \subsection*{ } & \subsection*{S\textbf{urvival 
DC}}\tabularnewline
\hline
Moor or hill, map & 6 & Poor visibility & 12\tabularnewline
\hline
Mountain, map & 8 & Mountain, no map & 12\tabularnewline
\hline
Moor or hill, no map & 10 & Forest & 15\tabularnewline
\hline
\end{tabular}

A character with at least 5 ranks in Knowledge (geography) or Knowledge (local) 
pertaining to the area being traveled through gains a +2 bonus on this check.

Check once per hour (or portion of an hour) spent in local or overland movement 
to see if travelers have become lost. In the case of a party moving together, only 
the character leading the way makes the check.

\textbf{Effects of Being Lost:} If a party becomes lost, it is no longer certain 
of moving in the direction it intended to travel. Randomly determine the direction 
in which the party actually travels during each hour of local or overland movement. 
The characters' movement continues to be random until they blunder into a landmark 
they can't miss, or until they recognize that they are lost and make an effort 
to regain their bearings.

\textit{Recognizing that You're Lost:} Once per hour of random travel, each character 
in the party may attempt a Survival check (DC 20, -1 per hour of random travel) 
to recognize that they are no longer certain of their direction of travel. Some 
circumstances may make it obvious that the characters are lost.

\textit{Setting a New Course: }A lost party is also uncertain of determining in 
which direction it should travel in order to reach a desired objective. Determining 
the correct direction of travel once a party has become lost requires a Survival 
check (DC 15, +2 per hour of random travel). If a character fails this check, he 
chooses a random direction as the ``correct'' direction for resuming travel.

Once the characters are traveling along their new course, correct or incorrect, 
they may get lost again. If the conditions still make it possible for travelers 
to become lost, check once per hour of travel as described in Chance to Get Lost, 
above, to see if the party maintains its new course or begins to move at random 
again.

\textit{Conflicting Directions: }It's possible that several characters may attempt 
to determine the right direction to proceed after becoming lost. Make a Survival 
check for each character in secret, then tell the players whose characters succeeded 
the correct direction in which to travel, and tell the players whose characters 
failed a random direction they think is right. 

\textbf{Regaining Your Bearings:} There are several ways to become un-lost. First, 
if the characters successfully set a new course and follow it to the destination 
they're trying to reach, they're not lost anymore. Second, the characters through 
random movement might run into an unmistakable landmark. Third, if conditions suddenly 
improve---the fog lifts or the sun comes up---lost characters may attempt to set 
a new course, as described above, with a +4 bonus on the Survival check. Finally, 
magic may make their course clear.

\vspace{12pt}
FOREST TERRAIN

Forest terrain can be divided into three categories: sparse, medium, and dense. 
An immense forest could have all three categories within its borders, with more 
sparse terrain at the outer edge of the forest and dense forest at its heart. 

The table below describes in general terms how likely it is that a given square 
has a terrain element in it.

\textbf{Forest Terrain Features}

\begin{tabular}{|>{\raggedright}p{78pt}|>{\raggedright}p{44pt}|>{\raggedright}p{55pt}|>{\raggedright}p{40pt}|}
\hline
 ------------ & \multicolumn{3}{p{139pt}|}{ \textbf{Category of Forest ------------}}\tabularnewline
\hline
  & S\textbf{parse} & M\textbf{edium} & D\textbf{ense}\tabularnewline
\hline
Typical trees & 50\% & 70\% & 80\%\tabularnewline
\hline
Massive trees--- &  & 10\% & 20\%\tabularnewline
\hline
Light undergrowth & 50\% & 70\% & 50\%\tabularnewline
\hline
Heavy undergrowth--- &  & 20\% & 50\%\tabularnewline
\hline
\end{tabular}

\textbf{Trees: }The most important terrain element in a forest is the trees, obviously. 
A creature standing in the same square as a tree gains a +2 bonus to Armor Class 
and a +1 bonus on Reflex saves (these bonuses don't stack with cover bonuses from 
other sources). The presence of a tree doesn't otherwise affect a creature's fighting 
space, because it's assumed that the creature is using the tree to its advantage 
when it can. The trunk of a typical tree has AC 4, hardness 5, and 150 hp. A DC 
15 Climb check is sufficient to climb a tree. Medium and dense forests have massive 
trees as well. These trees take up an entire square and provide cover to anyone 
behind them. They have AC 3, hardness 5, and 600 hp. Like their smaller counterparts, 
it takes a DC 15 Climb check to climb them.

\textbf{Undergrowth:} Vines, roots, and short bushes cover much of the ground in 
a forest. A space covered with light undergrowth costs 2 squares of movement to 
move into, and it provides concealment. Undergrowth increases the DC of Tumble 
and Move Silently checks by 2 because the leaves and branches get in the way. Heavy 
undergrowth costs 4 squares of movement to move into, and it provides concealment 
with a 30\% miss chance (instead of the usual 20\%). It increases the DC of Tumble 
and Move Silently checks by 5. Heavy undergrowth is easy to hide in, granting a 
+5 circumstance bonus on Hide checks. Running and charging are impossible. Squares 
with undergrowth are often clustered together. Undergrowth and trees aren't mutually 
exclusive; it's common for a 5-foot square to have both a tree and undergrowth.

\textbf{Forest Canopy:} It's common for elves and other forest dwellers to live 
on raised platforms far above the surface floor. These wooden platforms generally 
have rope bridges between them. To get to the treehouses, characters generally 
ascend the trees' branches (Climb DC 15), use rope ladders (Climb DC 0), or take 
pulley elevators (which can be made to rise a number of feet equal to a Strength 
check, made each round as a full-round action). Creatures on platforms or branches 
in a forest canopy are considered to have cover when fighting creatures on the 
ground, and in medium or dense forests they have concealment as well.

\textbf{Other Forest Terrain Elements: }Fallen logs generally stand about 3 feet 
high and provide cover just as low walls do. They cost 5 feet of movement to cross. 
Forest streams are generally 5 to 10 feet wide and no more than 5 feet deep. Pathways 
wind through most forests, allowing normal movement and providing neither cover 
nor concealment. These paths are less common in dense forests, but even unexplored 
forests will have occasional game trails.

\textbf{Stealth and Detection in a Forest:} In a sparse forest, the maximum distance 
at which a Spot check for detecting the nearby presence of others can succeed is 
3d6\ensuremath{\times}10 feet. In a medium forest, this distance is 2d8\ensuremath{\times}10 
feet, and in a dense forest it is 2d6\ensuremath{\times}10 feet.

Because any square with undergrowth provides concealment, it's usually easy for 
a creature to use the Hide skill in the forest. Logs and massive trees provide 
cover, which also makes hiding possible.

The background noise in the forest makes Listen checks more difficult, increasing 
the DC of the check by 2 per 10 feet, not 1 (but note that Move Silently is also 
more difficult in undergrowth). 

\vspace{12pt}
\textbf{Forest Fires (CR 6)}

Most campfire sparks ignite nothing, but if conditions are dry, winds are strong, 
or the forest floor is dried out and flammable, a forest fire can result. Lightning 
strikes often set trees afire and start forest fires in this way. Whatever the 
cause of the fire, travelers can get caught in the conflagration.

A forest fire can be spotted from as far away as 2d6\ensuremath{\times}100 feet 
by a character who makes a Spot check, treating the fire as a Colossal creature 
(reducing the DC by 16). If all characters fail their Spot checks, the fire moves 
closer to them. They automatically see it when it closes to half the original distance.

Characters who are blinded or otherwise unable to make Spot checks can feel the 
heat of the fire (and thus automatically ``spot'' it) when it is 100 feet away.

The leading edge of a fire (the downwind side) can advance faster than a human 
can run (assume 120 feet per round for winds of moderate strength). Once a particular 
portion of the forest is ablaze, it remains so for 2d4\ensuremath{\times}10 minutes 
before dying to a smoking smolder. Characters overtaken by a forest fire may find 
the leading edge of the fire advancing away from them faster than they can keep 
up, trapping them deeper and deeper in its grasp.

Within the bounds of a forest fire, a character faces three dangers: heat damage, 
catching on fire, and smoke inhalation. 

\textbf{Heat Damage:} Getting caught within a forest fire is even worse than being 
exposed to extreme heat (see Heat Dangers). Breathing the air causes a character 
to take 1d6 points of damage per round (no save). In addition, a character must 
make a Fortitude save every 5 rounds (DC 15, +1 per previous check) or take 1d4 
points of nonlethal damage. A character who holds his breath can avoid the lethal 
damage, but not the nonlethal damage. Those wearing heavy clothing or any sort 
of armor take a -4 penalty on their saving throws. In addition, those wearing metal 
armor or coming into contact with very hot metal are affected as if by a \textit{heat 
metal }spell.

\textbf{Catching on Fire:} Characters engulfed in a forest fire are at risk of 
catching on fire when the leading edge of the fire overtakes them, and are then 
at risk once per minute thereafter (see Catching on Fire)\textit{.}

\textbf{Smoke Inhalation:} Forest fires naturally produce a great deal of smoke. 
A character who breathes heavy smoke must make a Fortitude save each round (DC 
15, +1 per previous check) or spend that round choking and coughing. A character 
who chokes for 2 consecutive rounds takes 1d6 points of nonlethal damage. Also, 
smoke obscures vision, providing concealment to characters within it.

\vspace{12pt}
MARSH TERRAIN

Two categories of marsh exist: relatively dry moors and watery swamps. Both are 
often bordered by lakes (described in Aquatic Terrain, below), which effectively 
are a third category of terrain found in marshes.

The table below describes terrain features found in marshes.

\textbf{Marsh Terrain Features}

\begin{tabular}{|>{\raggedright}p{78pt}|>{\raggedright}p{35pt}|>{\raggedright}p{46pt}|}
\hline
 --- & \multicolumn{2}{p{82pt}|}{ \textbf{Marsh Category ---}}\tabularnewline
\hline
  & M\textbf{oor} & S\textbf{wamp}\tabularnewline
\hline
Shallow bog & 20\% & 40\%\tabularnewline
\hline
Deep bog & 5\% & 20\%\tabularnewline
\hline
Light undergrowth & 30\% & 20\%\tabularnewline
\hline
Heavy undergrowth & 10\% & 20\%\tabularnewline
\hline
\end{tabular}

\textbf{Bogs:} If a square is part of a shallow bog, it has deep mud or standing 
water of about 1 foot in depth. It costs 2 squares of movement to move into a square 
with a shallow bog, and the DC of Tumble checks in such a square increases by 2. 

A square that is part of a deep bog has roughly 4 feet of standing water. It costs 
Medium or larger creatures 4 squares of movement to move into a square with a deep 
bog, or characters can swim if they wish. Small or smaller creatures must swim 
to move through a deep bog. Tumbling is impossible in a deep bog.

The water in a deep bog provides cover for Medium or larger creatures. Smaller 
creatures gain improved cover (+8 bonus to AC, +4 bonus on Reflex saves). Medium 
or larger creatures can crouch as a move action to gain this improved cover. Creatures 
with this improved cover take a -10 penalty on attacks against creatures that aren't 
underwater.

Deep bog squares are usually clustered together and surrounded by an irregular 
ring of shallow bog squares.

Both shallow and deep bogs increase the DC of Move Silently checks by 2.

\textbf{Undergrowth:} The bushes, rushes, and other tall grasses in marshes function 
as undergrowth does in a forest (see above). A square that is part of a bog does 
not also have undergrowth. 

\textbf{Quicksand:} Patches of quicksand present a deceptively solid appearance 
(appearing as undergrowth or open land) that may trap careless characters. A character 
approaching a patch of quicksand at a normal pace is entitled to a DC 8 Survival 
check to spot the danger before stepping in, but charging or running characters 
don't have a chance to detect a hidden bog before blundering in. A typical patch 
of quicksand is 20 feet in diameter; the momentum of a charging or running character 
carries him or her 1d2\ensuremath{\times}5 feet into the quicksand.

\textit{Effects of Quicksand: }Characters in quicksand must make a DC 10 Swim check 
every round to simply tread water in place, or a DC 15 Swim check to move 5 feet 
in whatever direction is desired. If a trapped character fails this check by 5 
or more, he sinks below the surface and begins to drown whenever he can no longer 
hold his breath (see the Swim skill description)\textit{.}

Characters below the surface of a bog may swim back to the surface with a successful 
Swim check (DC 15, +1 per consecutive round of being under the surface).

\textit{Rescue: }Pulling out a character trapped in quicksand can be difficult. 
A rescuer needs a branch, spear haft, rope, or similar tool that enables him to 
reach the victim with one end of it. Then he must make a DC 15 Strength check to 
successfully pull the victim, and the victim must make a DC 10 Strength check to 
hold onto the branch, pole, or rope. If the victim fails to hold on, he must make 
a DC 15 Swim check immediately to stay above the surface. If both checks succeed, 
the victim is pulled 5 feet closer to safety.

\textbf{Hedgerows:} Common in moors, hedgerows are tangles of stones, soil, and 
thorny bushes. Narrow hedgerows function as low walls, and it takes 15 feet of 
movement to cross them. Wide hedgerows are more than 5 feet tall and take up entire 
squares. They provide total cover, just as a wall does. It takes 4 squares of movement 
to move through a square with a wide hedgerow; creatures that succeed on a DC 10 
Climb check need only 2 squares of movement to move through the square.

\textbf{Other Marsh Terrain Elements:} Some marshes, particularly swamps, have 
trees just as forests do, usually clustered in small stands. Paths lead across 
many marshes, winding to avoid bog areas. As in forests, paths allow normal movement 
and don't provide the concealment that undergrowth does.

\textbf{Stealth and Detection in a Marsh:} In a moor, the maximum distance at which 
a Spot check for detecting the nearby presence of others can succeed is 6d6\ensuremath{\times}10 
feet. In a swamp, this distance is 2d8\ensuremath{\times}10 feet.

Undergrowth and deep bogs provide plentiful concealment, so it's easy to hide in 
a marsh.

A marsh imposes no penalties on Listen checks, and using the Move Silently skill 
is more difficult in both undergrowth and bogs.

\vspace{12pt}
HILLS TERRAIN

A hill can exist in most other types of terrain, but hills can also dominate the 
landscape. Hills terrain is divided into two categories: gentle hills and rugged 
hills. Hills terrain often serves as a transition zone between rugged terrain such 
as mountains and flat terrain such as plains.

\textbf{Hills Terrain Features}

\begin{tabular}{|>{\raggedright}p{74pt}|>{\raggedright}p{45pt}|>{\raggedright}p{50pt}|}
\hline
 ------ & \multicolumn{2}{p{95pt}|}{H\textbf{ills Category------}}\tabularnewline
\hline
  & G\textbf{entle Hill} & R\textbf{ugged Hill}\tabularnewline
\hline
Gradual slope & 75\% & 40\%\tabularnewline
\hline
Steep slope & 20\% & 50\%\tabularnewline
\hline
Cliff & 5\% & 10\%\tabularnewline
\hline
Light undergrowth & 15\% & 15\%\tabularnewline
\hline
\end{tabular}

\textbf{Gradual Slope:} This incline isn't steep enough to affect movement, but 
characters gain a +1 bonus on melee attacks against foes downhill from them.

\textbf{Steep Slope:} Characters moving uphill (to an adjacent square of higher 
elevation) must spend 2 squares of movement to enter each square of steep slope. 
Characters running or charging downhill (moving to an adjacent square of lower 
elevation) must succeed on a DC 10 Balance check upon entering the first steep 
slope square. Mounted characters make a DC 10 Ride check instead. Characters who 
fail this check stumble and must end their movement 1d2\ensuremath{\times}5 feet 
later. Characters who fail by 5 or more fall prone in the square where they end 
their movement. A steep slope increases the DC of Tumble checks by 2.

\textbf{Cliff:} A cliff typically requires a DC 15 Climb check to scale and is 
1d4\ensuremath{\times}10 feet tall, although the needs of your map may mandate 
a taller cliff. A cliff isn't perfectly vertical, taking up 5-foot squares if it's 
less than 30 feet tall and 10-foot squares if it's 30 feet or taller. 

\textbf{Light Undergrowth:} Sagebrush and other scrubby bushes grow on hills, athough 
they rarely cover the landscape as they do in forests and marshes. Light undergrowth 
provides concealment and increases the DC of Tumble and Move Silently checks by 
2. 

\textbf{Other Hills Terrain Elements:} Trees aren't out of place in hills terrain, 
and valleys often have active streams (5 to 10 feet wide and no more than 5 feet 
deep) or dry streambeds (treat as a trench 5 to 10 feet across) in them. If you 
add a stream or streambed, remember that water always flows downhill.

\textbf{Stealth and Detection in Hills:} In gentle hills, the maximum distance 
at which a Spot check for detecting the nearby presence of others can succeed is 
2d10\ensuremath{\times}10 feet. In rugged hills, this distance is 2d6\ensuremath{\times}10 
feet.

Hiding in hills terrain can be difficult if there isn't undergrowth around. A hilltop 
or ridge provides enough cover to hide from anyone below the hilltop or ridge.

Hills don't affect Listen or Move Silently checks. 

\vspace{12pt}
MOUNTAIN TERRAIN

The three mountain terrain categories are alpine meadows, rugged mountains, and 
forbidding mountains. As characters ascend into a mountainous area, they're likely 
to face each terrain category in turn, beginning with alpine meadows, extending 
through rugged mountains, and reaching forbidding mountains near the summit.

Mountains have an important terrain element, the rock wall, that is marked on the 
border between squares rather than taking up squares itself. 

\textbf{Mountain Terrain Features}

\begin{tabular}{|>{\raggedright}p{74pt}|>{\raggedright}p{66pt}|>{\raggedright}p{31pt}|>{\raggedright}p{47pt}|}
\hline
 ------------ & \multicolumn{3}{p{145pt}|}{ \textbf{Mountain Category ------------}}\tabularnewline
\hline
  & A\textbf{lpine Meadow} & R\textbf{ugged} & F\textbf{orbidding}\tabularnewline
\hline
Gradual slope & 50\% & 25\% & 15\%\tabularnewline
\hline
Steep slope & 40\% & 55\% & 55\%\tabularnewline
\hline
Cliff & 10\% & 15\% & 20\%\tabularnewline
\hline
Chasm--- &  & 5\% & 10\%\tabularnewline
\hline
Light undergrowth & 20\% & 10\%--- & \tabularnewline
\hline
Scree--- &  & 20\% & 30\%\tabularnewline
\hline
Dense rubble--- &  & 20\% & 30\%\tabularnewline
\hline
\end{tabular}

\textbf{Gradual and Steep Slopes:} These function as described in Hills Terrain, 
above.

\textbf{Cliff: }These terrain elements also function like their hills terrain counterparts, 
but they're typically 2d6\ensuremath{\times}10 feet tall. Cliffs taller than 80 
feet take up 20 feet of horizontal space.

\textbf{Chasm: }Usually formed by natural geological processes, chasms function 
like pits in a dungeon setting. Chasms aren't hidden, so characters won't fall 
into them by accident (although bull rushes are another story). A typical chasm 
is 2d4\ensuremath{\times}10 feet deep, at least 20 feet long, and anywhere from 
5 feet to 20 feet wide. It takes a DC 15 Climb check to climb out of a chasm. In 
forbidding mountain terrain, chasms are typically 2d8\ensuremath{\times}10 feet 
deep.

Light Undergrowth: This functions as described in Forest Terrain,

above.

\textbf{Scree:} A field of shifting gravel, scree doesn't affect speed, but it 
can be treacherous on a slope. The DC of Balance and Tumble checks increases by 
2 if there's scree on a gradual slope and by 5 if there's scree on a steep slope. 
The DC of Move silently checks increases by 2 if the scree is on a slope of any 
kind.

\textbf{Dense Rubble: }The ground is covered with rocks of all sizes. It costs 
2 squares of movement to enter a square with dense rubble. The DC of Balance and 
Tumble checks on dense rubble increases by 5, and the DC of Move Silently checks 
increases by +2. 

\textbf{Rock Wall:} A vertical plane of stone, rock walls require DC 25 Climb checks 
to ascend. A typical rock wall is 2d4\ensuremath{\times}10 feet tall in rugged 
mountains and 2d8\ensuremath{\times}10 feet tall in forbidding mountains. Rock 
walls are drawn on the edges of squares, not in the squares themselves.

\textbf{Cave Entrance:} Found in cliff and steep slope squares and next to rock 
walls, cave entrances are typically between 5 and 20 feet wide and 5 feet deep. 
Beyond the entrance, a cave could be anything from a simple chamber to the entrance 
to an elaborate dungeon. Caves used as monster lairs typically have 1d3 rooms that 
are 1d4\ensuremath{\times}10 feet across. 

\textbf{Other Mountain Terrain Features:} Most alpine meadows begin above the tree 
line, so trees and other forest elements are rare in the mountains. Mountain terrain 
can include active streams (5 to 10 feet wide and no more than 5 feet deep) and 
dry streambeds (treat as a trench 5 to 10 feet across). Particularly high-altitude 
areas tend to be colder than the lowland areas that surround them, so they may 
be covered in ice sheets (described below).

\textbf{Stealth and Detection in Mountains:} As a guideline, the maximum distance 
in mountain terrain at which a Spot check for detecting the nearby presence of 
others can succeed is 4d10\ensuremath{\times}10 feet. Certain peaks and ridgelines 
afford much better vantage points, of course, and twisting valleys and canyons 
have much shorter spotting distances. Because there's little vegetation to obstruct 
line of sight, the specifics on your map are your best guide for the range at which 
an encounter could begin. As in hills terrain, a ridge or peak provides enough 
cover to hide from anyone below the high point.

It's easier to hear faraway sounds in the mountains. The DC of Listen checks increases 
by 1 per 20 feet between listener and source, not per 10 feet.

\textbf{Avalanches (CR 7)}

The combination of high peaks and heavy snowfalls means that avalanches are a deadly 
peril in many mountainous areas. While avalanches of snow and ice are common, it's 
also possible to have an avalanche of rock and soil.

An avalanche can be spotted from as far away as 1d10\ensuremath{\times}500 feet 
downslope by a character who makes a DC 20 Spot check, treating the avalanche as 
a Colossal creature. If all characters fail their Spot checks to determine the 
encounter distance, the avalanche moves closer to them, and they automatically 
become aware of it when it closes to half the original distance. It's possible 
to hear an avalanche coming even if you can't see it. Under optimum conditions 
(no other loud noises occurring), a character who makes a DC 15 Listen check can 
hear the avalanche or landslide when it is 1d6\ensuremath{\times}500 feet away. 
This check might have a DC of 20, 25, or higher in conditions where hearing is 
difficult (such as in the middle of a thunderstorm). 

A landslide or avalanche consists of two distinct areas: the bury zone (in the 
direct path of the falling debris) and the slide zone (the area the debris spreads 
out to encompass). Characters in the bury zone always take damage from the avalanche; 
characters in the slide zone may be able to get out of the way. Characters in the 
bury zone take 8d6 points of damage, or half that amount if they make a DC 15 Reflex 
save. They are subsequently buried (see below). Characters in the slide zone take 
3d6 points of damage, or no damage if they make a DC 15 Reflex save. Those who 
fail their saves are buried. 

Buried characters take 1d6 points of nonlethal damage per minute. If a buried character 
falls unconscious, he or she must make a DC 15 Constitution check or take 1d6 points 
of lethal damage each minute thereafter until freed or dead.

The typical avalanche has a width of 1d6\ensuremath{\times}100 feet, from one edge 
of the slide zone to the opposite edge. The bury zone in the center of the avalanche 
is half as wide as the avalanche's full width.

To determine the precise location of characters in the path of an avalanche, roll 
1d6\ensuremath{\times}20; the result is the number of feet from the center of the 
path taken by the bury zone to the center of the party's location. Avalanches of 
snow and ice advance at a speed of 500 feet per round, and rock avalanches travel 
at a speed of 250 feet per round.

\subsubsection*{\textbf{Mountain Travel}}

High altitude can be extremely fatiguing---or sometimes deadly---to creatures that 
aren't used to it. Cold becomes extreme, and the lack of oxygen in the air can 
wear down even the most hardy of warriors.

\textbf{Acclimated Characters:} Creatures accustomed to high altitude generally 
fare better than lowlanders. Any creature with an Environment entry that includes 
mountains is considered native to the area, and acclimated to the high altitude. 
Characters can also acclimate themselves by living at high altitude for a month. 
Characters who spend more than two months away from the mountains must reacclimate 
themselves when they return. Undead, constructs, and other creatures that do not 
breathe are immune to altitude effects.

\textbf{Altitude Zones: }In general, mountains present three possible altitude 
bands: low pass, low peak/high pass, and high peak. 

\textit{Low Pass (lower than 5,000 feet): }Most travel in low mountains takes place 
in low passes, a zone consisting largely of alpine meadows and forests. Travelers 
may find the going difficult (which is reflected in the movement modifiers for 
traveling through mountains), but the altitude itself has no game effect.

\textit{Low Peak or High Pass (5,000 to 15,000 feet): }Ascending to the highest 
slopes of low mountains, or most normal travel through high mountains, falls into 
this category. All nonacclimated creatures labor to breathe in the thin air at 
this altitude. Characters must succeed on a Fortitude save each hour (DC 15, +1 
per previous check) or be fatigued. The fatigue ends when the character descends 
to an altitude with more air. Acclimated characters do not have to attempt the 
Fortitude save. 

\textit{High Peak (more than 15,000 feet): }The highest mountains exceed 20,000 
feet in height. At these elevations, creatures are subject to both high altitude 
fatigue (as described above) and altitude sickness, whether or not they're acclimated 
to high altitudes\textit{. }Altitude sickness represents long-term oxygen deprivation, 
and it affects mental and physical ability scores. After each 6-hour period a character 
spends at an altitude of over 15,000 feet, he must succeed on a Fortitude save 
(DC 15, +1 per previous check) or take 1 point of damage to all ability scores. 
Creatures acclimated to high altitude receive a +4 competence bonus on their saving 
throws to resist high altitude effects and altitude sickness, but eventually even 
seasoned mountaineers must abandon these dangerous elevations. 

\vspace{12pt}
DESERT TERRAIN

Desert terrain exists in warm, temperate, and cold climates, but all deserts share 
one common trait: little rain. The three categories of desert terrain are tundra 
(cold deserts), rocky desert (often temperate), and sandy desert (often warm).

Tundra differs from the other desert categories in two important ways. Because 
snow and ice cover much of the landscape, it's easy to find water. And during the 
height of summer, the permafrost thaws to a depth of a foot or so, turning the 
landscape into a vast field of mud. The muddy tundra affects movement and skill 
use as the shallow bogs described in marsh terrain, although there's little standing 
water.

The table above describes terrain elements found in each of the three desert categories. 
The terrain elements on this table are mutually exclusive; for instance, a square 
of tundra may contain either light undergrowth or an ice sheet, but not both.

\textbf{Desert Terrain Features}

\begin{tabular}{|>{\raggedright}p{74pt}|>{\raggedright}p{41pt}|>{\raggedright}p{34pt}|>{\raggedright}p{33pt}|}
\hline
 --------- & \multicolumn{3}{p{109pt}|}{ \textbf{Desert Category ---------}}\tabularnewline
\hline
  & T\textbf{undra} & R\textbf{ocky} & S\textbf{andy}\tabularnewline
\hline
Light undergrowth & 15\% & 5\% & 5\%\tabularnewline
\hline
Ice sheet & 25\%--- & --- & \tabularnewline
\hline
Light rubble & 5\% & 30\% & 10\%\tabularnewline
\hline
Dense rubble--- &  & 30\% & 5\%\tabularnewline
\hline
Sand dunes--- & --- &  & 50\%\tabularnewline
\hline
\end{tabular}

\textbf{Light Undergrowth:} Consisting of scrubby, hardy bushes and cacti, light 
undergrowth functions as described for other terrain types.

\textbf{Ice Sheet: }The ground is covered with slippery ice. It costs 2 squares 
of movement to enter a square covered by an ice sheet, and the DC of Balance and 
Tumble checks there increases by 5. A DC 10 Balance check is required to run or 
charge across an ice sheet. 

\textbf{Light Rubble:} Small rocks are strewn across the ground, making nimble 
movement more difficult more difficult. The DC of Balance and Tumble checks increases 
by 2. 

\textbf{Dense Rubble:} This terrain feature consists of more and larger stones. 
It costs 2 squares of movement to enter a square with dense rubble. The DC of Balance 
and Tumble checks increases by 5, and the DC of Move Silently checks increases 
by 2.

\textbf{Sand Dunes:} Created by the action of wind on sand, sand dunes function 
as hills that move. If the wind is strong and consistent, a sand dune can move 
several hundred feet in a week's time. Sand dunes can cover hundreds of squares. 
They always have a gentle slope pointing in the direction of the prevailing wind 
and a steep slope on the leeward side.

\textbf{Other Desert Terrain Features:} Tundra is sometimes bordered by forests, 
and the occasional tree isn't out of place in the cold wastes. Rocky deserts have 
towers and mesas consisting of flat ground surrounded on all sides by cliffs and 
steep slopes (described in Mountain Terrain, above). Sandy deserts sometimes have 
quicksand; this functions as described in Marsh Terrain, above, although desert 
quicksand is a waterless mixture of fine sand and dust. All desert terrain is crisscrossed 
with dry streambeds (treat as trenches 5 to 15 feet wide) that fill with water 
on the rare occasions when rain falls.

\textbf{Stealth and Detection in the Desert:} In general, the maximum distance 
in desert terrain at which a Spot check for detecting the nearby presence of others 
can succeed is 6d6\ensuremath{\times}20 feet; beyond this distance, elevation changes 
and heat distortion in warm deserts makes spotting impossible. The presence of 
dunes in sandy deserts limits spotting distance to 6d6\ensuremath{\times}10 feet. 

The desert imposes neither bonuses nor penalties on Listen or Spot checks. The 
scarcity of undergrowth or other elements that offer concealment or cover makes 
hiding more difficult.

\subsubsection*{\textbf{Sandstorms}}

A sandstorm reduces visibility to 1d10\ensuremath{\times}5 feet and provides a 
-4 penalty on Listen, Search, and Spot checks. A sandstorm deals 1d3 points of 
nonlethal damage per hour to any creatures caught in the open, and leaves a thin 
coating of sand in its wake. Driving sand creeps in through all but the most secure 
seals and seams, to chafe skin and contaminate carried gear. 

\vspace{12pt}
PLAINS TERRAIN

Plains come in three categories: farms, grasslands, and battlefields. Farms are 
common in settled areas, of course, while grasslands represent untamed plains. 
The battlefields where large armies clash are temporary places, usually reclaimed 
by natural vegetation or the farmer's plow. Battlefields represent a third terrain 
category because adventurers tend to spend a lot of time there, not because they're 
particularly prevalent.

The table below shows the proportions of terrain elements in the different categories 
of plains. On a farm, light undergrowth represents most mature grain crops, so 
farms growing vegetable crops will have less light undergrowth, as will all farms 
during the time between harvest and a few months after planting.

The terrain elements in the table below are mutually exclusive.

\textbf{Plains Terrain Features}

\begin{tabular}{|>{\raggedright}p{78pt}|>{\raggedright}p{22pt}|>{\raggedright}p{42pt}|>{\raggedright}p{43pt}|}
\hline
 --------- & \multicolumn{3}{p{108pt}|}{ \textbf{Plains Category ---------}}\tabularnewline
\hline
  & F\textbf{arm} & G\textbf{rassland} & B\textbf{attlefield}\tabularnewline
\hline
Light undergrowth & 40\% & 20\% & 10\%\tabularnewline
\hline
Heavy undergrowth--- &  & 10\%--- & \tabularnewline
\hline
Light rubble--- & --- &  & 10\%\tabularnewline
\hline
Trench & 5\%--- &  & 5\%\tabularnewline
\hline
Berm--- & --- &  & 5\%\tabularnewline
\hline
\end{tabular}

\textbf{Undergrowth: }Whether they're crops or natural vegetation, the tall grasses 
of the plains function like light undergrowth in a forest. Particularly thick bushes 
form patches of heavy undergrowth that dot the landscape in grasslands.

\textbf{Light Rubble:} On the battlefield, light rubble usually represents something 
that was destroyed: the ruins of a building or the scattered remnants of a stone 
wall, for example. It functions as described in the desert terrain section above.

\textbf{Trench:} Often dug before a battle to protect soldiers, a trench functions 
as a low wall, except that it provides no cover against adjacent foes. It costs 
2 squares of movement to leave a trench, but it costs nothing extra to enter one. 
Creatures outside a trench who make a melee attack against a creature inside the 
trench gain a +1 bonus on melee attacks because they have higher ground. In farm 
terrain, trenches are generally irrigation ditches.

\textbf{Berm: }A common defensive structure, a berm is a low, earthen wall that 
slows movement and provides a measure of cover. Put a berm on the map by drawing 
two adjacent rows of steep slope (described in Hills Terrain, above), with the 
edges of the berm on the downhill side. Thus, a character crossing a two-square 
berm will travel uphill for 1 square, then downhill for 1 square. Two square berms 
provide cover as low walls for anyone standing behind them. Larger berms provide 
the low wall benefit for anyone standing 1 square downhill from the top of the 
berm. 

\textbf{Fences:} Wooden fences are generally used to contain livestock or impede 
oncoming soldiers. It costs an extra square of movement to cross a wooden fence. 
A stone fence provides a measure of cover as well, functioning as low walls. Mounted 
characters can cross a fence without slowing their movement if they succeed on 
a DC 15 Ride check. If the check fails, the steed crosses the fence, but the rider 
falls out of the saddle.

\textbf{Other Plains Terrain Features:} Occasional trees dot the landscape in many 
plains, although on battlefields they're often felled to provide raw material for 
siege engines (described in Urban Features). Hedgerows (described in Marsh Terrain) 
are found in plains as well. Streams, generally 5 to 20 feet wide and 5 to 10 feet 
deep, are commonplace.

\textbf{Stealth and Detection in Plains:} In plains terrain, the maximum distance 
at which a Spot check for detecting the nearby presence of others can succeed is 
6d6\ensuremath{\times}40 feet, although the specifics of your map may restrict 
line of sight. Plains terrain provides no bonuses or penalties on Listen and Spot 
checks. Cover and concealment are not uncommon, so a good place of refuge is often 
nearby, if not right at hand.

\vspace{12pt}
AQUATIC TERRAIN

Aquatic terrain is the least hospitable to most PCs, because they can't breathe 
there. Aquatic terrain doesn't offer the variety that land terrain does. The ocean 
floor holds many marvels, including undersea analogues of any of the terrain elements 
described earlier in this section. But if characters find themselves in the water 
because they were bull rushed off the deck of a pirate ship, the tall kelp beds 
hundreds of feet below them don't matter. Accordingly, these rules simply divide 
aquatic terrain into two categories: flowing water (such as streams and rivers) 
and nonflowing water (such as lakes and oceans).

\textbf{Flowing Water:} Large, placid rivers move at only a few miles per hour, 
so they function as still water for most purposes. But some rivers and streams 
are swifter; anything floating in them moves downstream at a speed of 10 to 40 
feet per round. The fastest rapids send swimmers bobbing downstream at 60 to 90 
feet per round. Fast rivers are always at least rough water (Swim DC 15), and whitewater 
rapids are stormy water (Swim DC 20). If a character is in moving water, move her 
downstream the indicated distance at the end of her turn. A character trying to 
maintain her position relative to the riverbank can spend some or all of her turn 
swimming upstream.

\textit{Swept Away: }Characters swept away by a river moving 60 feet per round 
or faster must make DC 20 Swim checks every round to avoid going under. If a character 
gets a check result of 5 or more over the minimum necessary, he arrests his motion 
by catching a rock, tree limb, or bottom snag---he is no longer being carried along 
by the flow of the water. Escaping the rapids by reaching the bank requires three 
DC 20 Swim checks in a row. Characters arrested by a rock, limb, or snag can't 
escape under their own power unless they strike out into the water and attempt 
to swim their way clear. Other characters can rescue them as if they were trapped 
in quicksand (described in Marsh Terrain, above). 

\textbf{Nonflowing Water:} Lakes and oceans simply require a swim speed or successful 
Swim checks to move through (DC 10 in calm water, DC 15 in rough water, DC 20 in 
stormy water). Characters need a way to breathe if they're underwater; failing 
that, they risk drowning. When underwater, characters can move in any direction 
as if they were flying with perfect maneuverability.

\textbf{Stealth and Detection Underwater: }How far you can see underwater depends 
on the water's clarity. As a guideline, creatures can see 4d8\ensuremath{\times}10 
feet if the water is clear, and 1d8\ensuremath{\times}10 feet if it's murky. Moving 
water is always murky, unless it's in a particularly large, slow-moving river.

It's hard to find cover or concealment to hide underwater (except along the seafloor). 
Listen and Move Silently checks function normally underwater.

\textit{Invisibility: }An invisible creature displaces water and leaves a visible, 
body-shaped ``bubble'' where the water was displaced. The creature still has concealment 
(20\% miss chance), but not total concealment (50\% miss chance).

\subsubsection*{\textbf{Underwater Combat}}

Land-based creatures can have considerable difficulty when fighting in water. Water 
affects a creature's Armor Class, attack rolls, damage, and movement. In some cases 
a creature's opponents may get a bonus on attacks. The effects are summarized in 
the accompanying table. They apply whenever a character is swimming, walking in 
chestdeep water, or walking along the bottom. 

\textbf{Ranged Attacks Underwater:} Thrown weapons are ineffective underwater, 
even when launched from land. Attacks with other ranged weapons take a -2 penalty 
on attack rolls for every 5 feet of water they pass through, in addition to the 
normal penalties for range. 

\textbf{Attacks from Land:} Characters swimming, floating, or treading water on 
the surface, or wading in water at least chest deep, have improved cover (+8 bonus 
to AC, +4 bonus on Reflex saves) from opponents on land. Landbound opponents who 
have \textit{freedom of movement }effects ignore this cover when making melee attacks 
against targets in the water. A completely submerged creature has total cover against 
opponents on land unless those opponents have \textit{freedom of movement }effects. 
Magical effects are unaffected except for those that require attack rolls (which 
are treated like any other effects) and fire effects.

\textbf{Fire:} Nonmagical fire (including alchemist's fire) does not burn underwater. 
Spells or spell-like effects with the fire descriptor are ineffective underwater 
unless the caster makes a Spellcraft check (DC 20 + spell level). If the check 
succeeds, the spell creates a bubble of steam instead of its usual fiery effect, 
but otherwise the spell works as described. A supernatural fire effect is ineffective 
underwater unless its description states otherwise. The surface of a body of water 
blocks line of effect for any fire spell. If the caster has made a Spellcraft check 
to make the fire spell usable underwater, the surface still blocks the spell's 
line of effect.

\begin{tabular}{|>{\raggedright}p{63pt}|>{\raggedright}p{73pt}|>{\raggedright}p{56pt}|>{\raggedright}p{49pt}|>{\raggedright}p{47pt}|}
\hline
\multicolumn{5}{|p{290pt}|}{\subsection*{T\textbf{able: Combat Adjustments Underwater}}}\tabularnewline
\hline
 --------------- & \multicolumn{2}{p{130pt}|}{ \textbf{Attack/Damage ---------------}} &  & \tabularnewline
\hline
C\textbf{ondition} & S\textbf{lashing or Bludgeoning} & T\textbf{ail} & M\textbf{ovement} & O\textbf{ff 
Balance?}\textsuperscript{\textbf{4}}\tabularnewline
\hline
F\textit{reedom of movement} & normal/normal & normal/normal & normal & No\tabularnewline
\hline
Has a swim speed- & 2/half & normal & normal & No\tabularnewline
\hline
Successful Swim check- & 2/half\textsuperscript{\textbf{1}}- & 2/half & quarter 
or half\textsuperscript{\textbf{2}} & No\tabularnewline
\hline
Firm footing\textsuperscript{\textbf{3}}- & 2/half- & 2/half & half & No\tabularnewline
\hline
None of the above- & 2/half- & 2/half & normal & Yes\tabularnewline
\hline
\multicolumn{5}{|p{290pt}|}{1 A creature without a \textit{freedom of movement 
}effects or a swim speed makes grapple checks underwater at a -2 penalty, but deals 
damage normally when grappling.}\tabularnewline
\hline
\multicolumn{5}{|p{290pt}|}{2 A successful Swim check lets a creature move one-quarter 
its speed as a move action or one-half its speed as a full-round action.}\tabularnewline
\hline
\multicolumn{5}{|p{290pt}|}{3 Creatures have firm footing when walking along the 
bottom, braced against a ship's hull, or the like. A creature can only walk along 
the bottom if it wears or carries enough gear to weigh itself down---at least 16 
pounds for Medium creatures, twice that for each size category larger than Medium, 
and half that for each size category smaller than Medium. }\tabularnewline
\hline
\multicolumn{5}{|p{290pt}|}{4 Creatures flailing about in the water (usually because 
they failed their Swim checks) have a hard time fighting effectively. An off-balance 
creature loses its Dexterity bonus to Armor Class, and opponents gain a +2 bonus 
on attacks against it. }\tabularnewline
\hline
\end{tabular}

\subsubsection*{\textbf{Floods}}

In many wilderness areas, river floods are a common occurrence.

In spring, an enormous snowmelt can engorge the streams and rivers it feeds. Other 
catastrophic events such as massive rainstorms or the destruction of a dam can 
create floods as well.

During a flood, rivers become wider, deeper, and swifter. Assume that a river rises 
by 1d10+10 feet during the spring flood, and its width increases by a factor of 
1d4\ensuremath{\times}50\%. Fords may disappear for days, bridges may be swept 
away, and even ferries might not be able to manage the crossing of a flooded river. 
A river in flood makes Swim checks one category harder (calm water becomes rough, 
and rough water becomes stormy). Rivers also become 50\% swifter.

\vspace{12pt}
URBAN ADVENTURES

At first glance, a city is much like a dungeon, made up of walls, doors, rooms, 
and corridors. Adventures that take place in cities have two salient differences 
from their dungeon counterparts, however. Characters have greater access to resources, 
and they must contend with law enforcement.

\textbf{Access to Resources:} Unlike in dungeons and the wilderness, characters 
can buy and sell gear quickly in a city. A large city or metropolis probably has 
high-level NPCs and experts in obscure fields of knowledge who can provide assistance 
and decipher clues. And when the PCs are battered and bruised, they can retreat 
to the comfort of a room at the inn.

The freedom to retreat and ready access to the marketplace means that the players 
have a greater degree of control over the pacing of an urban adventure. 

\parindent=3pt
\textbf{Law Enforcement:} The other key distinctions between adventuring in a city 
and delving into a dungeon is that a dungeon is, almost by definition, a lawless 
place where the only law is that of the jungle: Kill or be killed. A city, on the 
other hand, is held together by a code of laws, many of which are explicitly designed 
to prevent the sort of behavior that adventurers engage in all the time: killing 
and looting. Even so, most cities' laws recognize monsters as a threat to the stability 
the city relies on, and prohibitions about murder rarely apply to monsters such 
as aberrations or evil outsiders. Most evil humanoids, however, are typically protected 
by the same laws that protect all the citizens of the city. Having an evil alignment 
is not a crime (except in some severely theocratic cities, perhaps, with the magical 
power to back up the law); only evil deeds are against the law. Even when adventurers 
encounter an evildoer in the act of perpetrating some heinous evil upon the populace 
of the city, the law tends to frown on the sort of vigilante justice that leaves 
the evildoer dead or otherwise unable to testify at a trial.

\parindent=0pt
\textbf{Weapon And Spell Restrictions}

Different cities have different laws about such issues as carrying weapons in public 
and restricting spellcasters.

The city's laws may not affect all characters equally. A monk isn't hampered at 
all by a law about peace-bonding weapons, but a cleric is reduced to a fraction 
of his power if all holy symbols are confiscated at the city's gates.

\subsubsection*{\textbf{Urban Features}}

Walls, doors, poor lighting, and uneven footing: In many ways a city is much like 
a dungeon. Some new considerations for an urban setting are covered below.

\vspace{12pt}
\subsubsection*{\textbf{Walls and Gates}}

Many cities are surrounded by walls. A typical small city wall is a fortified stone 
wall 5 feet thick and 20 feet high. Such a wall is fairly smooth, requiring a DC 
30 Climb check to scale. The walls are crenellated on one side to provide a low 
wall for the guards atop it, and there is just barely room for guards to walk along 
the top of the wall. A typical small city wall has AC 3, hardness 8, and 450 hp 
per 10-foot section.

A typical large city wall is 10 feet thick and 30 feet high, with crenellations 
on both sides for the guards on top of the wall. It is likewise smooth, requiring 
a DC 30 Climb check to scale. Such a wall has AC 3, hardness 8, and 720 hp per 
10-foot section.

A typical metropolis wall is 15 feet thick and 40 feet tall. It has crenellations 
on both sides and often has a tunnel and small rooms running through its interior. 
Metropolis walls have AC 3, hardness 8, and 1,170 hp per 10- foot section.

Unlike smaller cities, metropolises often have interior walls as well as surrounding 
walls---either old walls that the city has outgrown, or walls dividing individual 
districts from each other. Sometimes these walls are as large and thick as the 
outer walls, but more often they have the characteristics of a large city's or 
small city's walls.

\textbf{Watch Towers:} Some city walls are adorned with watch towers set at irregular 
intervals. Few cities have enough guards to keep someone constantly stationed at 
every tower, unless the city is expecting attack from outside. The towers provide 
a superior view of the surrounding countryside as well as a point of defense against 
invaders.

Watch towers are typically 10 feet higher than the wall they adjoin, and their 
diameter is 5 times the thickness of the wall. Arrow slits line the outer sides 
of the upper stories of a tower, and the top is crenellated like the surrounding 
walls are. In a small tower (25 feet in diameter adjoining a 5-foot-thick wall), 
a simple ladder typically connect the tower's stories and the roof. In a larger 
tower, stairs serve that purpose. 

Heavy wooden doors, reinforced with iron and bearing good locks (Open Lock DC 30), 
block entry to a tower, unless the tower is in regular use. As a rule, the captain 
of the guard keeps the key to the tower secured on her person, and a second copy 
is in the city's inner fortress or barracks.

\textbf{Gates:} A typical city gate is a gatehouse with two portcullises and murder 
holes above the space between them. In towns and some small cities, the primary 
entry is through iron double doors set into the city wall.

Gates are usually open during the day and locked or barred at night. Usually, one 
gate lets in travelers after sunset and is staffed by guards who will open it for 
someone who seems honest, presents proper papers, or offers a large enough bribe 
(depending on the city and the guards).

\vspace{12pt}
\subsubsection*{\textbf{Guards and Soldiers}}

A city typically has full-time military personnel equal to 1\% of its adult population, 
in addition to militia or conscript soldiers equal to 5\% of the population. The 
full-time soldiers are city guards responsible for maintaining order within the 
city, similar to the role of modern police, and (to a lesser extent) for defending 
the city from outside assault. Conscript soldiers are called up to serve in case 
of an attack on the city.

A typical city guard force works on three eight-hour shifts, with 30\% of the force 
on a day shift (8 {\footnotesize{}A}.{\footnotesize{}M}. to 4 {\footnotesize{}P}.{\footnotesize{}M}.), 
35\% on an evening shift (4 {\footnotesize{}P}.{\footnotesize{}M}. to 12 {\footnotesize{}A}.{\footnotesize{}M}.), 
and 35\% on a night shift (12 {\footnotesize{}A}.{\footnotesize{}M}. to 8 {\footnotesize{}A}.{\footnotesize{}M}.). 
At any given time, 80\% of the guards on duty are on the streets patrolling, while 
the remaining 20\% are stationed at various posts throughout the city, where they 
can respond to nearby alarms. At least one such guard post is present within each 
neighborhood of a city (each neighborhood consisting of several districts).

The majority of a city guard force is made up of warriors, mostly 1st level. Officers 
include higher-level warriors, fighters, a fair number of clerics, and wizards 
or sorcerers, as well as multiclass fighter/spellcasters.

\vspace{12pt}
\subsubsection*{\textbf{Siege Engines}}

Siege engines are large weapons, temporary structures, or pieces of equipment traditionally 
used in besieging a castle or fortress.

\begin{tabular}{|>{\raggedright}p{54pt}|>{\raggedright}p{29pt}|>{\raggedright}p{29pt}|>{\raggedright}p{27pt}|>{\raggedright}p{88pt}|>{\raggedright}p{49pt}|}
\hline
\multicolumn{6}{|p{278pt}|}{T\textbf{able: Siege Engines}}\tabularnewline
\hline
I\textbf{tem } & C\textbf{ost } & D\textbf{amage} & C\textbf{ritical} & R\textbf{ange 
Increment} & T\textbf{ypical Crew}\tabularnewline
\hline
Catapult, heavy & 800 gp & 6d6--- &  & 200 ft. (100 ft. minimum) & 4\tabularnewline
\hline
Catapult, light  & 550 gp  & 4d6--- &  & 150 ft. (100 ft. minimum) & 2\tabularnewline
\hline
Ballista  & 500 gp  & 3d8 & 19-20 & 120 ft. & 1\tabularnewline
\hline
Ram  & 1,000 gp  & 3d6*--- & --- &  & 10\tabularnewline
\hline
Siege tower  & 2,000 gp --- & --- & --- &  & 20\tabularnewline
\hline
\multicolumn{6}{|p{278pt}|}{* See description for special rules.}\tabularnewline
\hline
\end{tabular}

\vspace{12pt}
\begin{tabular}{|>{\raggedright}p{191pt}|>{\raggedright}p{135pt}|}
\hline
\multicolumn{2}{|p{326pt}|}{C\textbf{atapult Attack Modifiers}}\tabularnewline
\hline
C\textbf{ondition } & M\textbf{odifier}\tabularnewline
\hline
No line of sight to target square- & 6\tabularnewline
\hline
Successive shots (crew can see where most recent misses landed)  & Cumulative +2 
per previous miss (maximum +10)\tabularnewline
\hline
Successive shots (crew can't see where most recent misses landed, but observer 
is providing feedback) & Cumulative +1 per previous miss (maximum +5)\tabularnewline
\hline
\end{tabular}

\vspace{12pt}
\textbf{Catapult, Heavy: }A heavy catapult is a massive engine capable of throwing 
rocks or heavy objects with great force. Because the catapult throws its payload 
in a high arc, it can hit squares out of its line of sight. To fire a heavy catapult, 
the crew chief makes a special check against DC 15 using only his base attack bonus, 
Intelligence modifier, range increment penalty, and the appropriate modifiers from 
the lower section of Table 3-26. If the check succeeds, the catapult stone hits 
the square the catapult was aimed at, dealing the indicated damage to any object 
or character in the square. Characters who succeed on a DC 15 Reflex save take 
half damage. Once a catapult stone hits a square, subsequent shots hit the same 
square unless the catapult is reaimed or the wind changes direction or speed.

If a catapult stone misses, roll 1d8 to determine where it lands. This determines 
the misdirection of the throw, with 1 being back toward the catapult and 2 through 
8 counting clockwise around the target square. Then, count 3 squares away from 
the target square for every range increment of the attack.

Loading a catapult requires a series of full-round actions. It takes a DC 15 Strength 
check to winch the throwing arm down; most catapults have wheels to allow up to 
two crew members to use the aid another action, assisting the main winch operator. 
A DC 15 Profession (siege engineer) check latches the arm into place, and then 
another DC 15 Profession (siege engineer) check loads the catapult ammunition. 
It takes four full-round actions to reaim a heavy catapult (multiple crew members 
can perform these full-round actions in the same round, so it would take a crew 
of four only 1 round to reaim the catapult).

A heavy catapult takes up a space 15 feet across.

\textbf{Catapult, Light: }This is a smaller, lighter version of the heavy catapult. 
It functions as the heavy catapult, except that it takes a DC 10 Strength check 
to winch the arm into place, and only two full-round actions are required to reaim 
the catapult.

A light catapult takes up a space 10 feet across.

\textbf{Ballista:} A ballista is essentially a Huge heavy crossbow fixed in place. 
Its size makes it hard for most creatures to aim it\textit{. }Thus, a Medium creature 
takes a -4 penalty on attack rolls when using a ballista, and a Small creature 
takes a -6 penalty. It takes a creature smaller than Large two full-round actions 
to reload the ballista after firing.

A ballista takes up a space 5 feet across.

\textbf{Ram: }This heavy pole is sometimes suspended from a movable scaffold that 
allows the crew to swing it back and forth against objects. As a full-round action, 
the character closest to the front of the ram makes an attack roll against the 
AC of the construction, applying the -4 penalty for lack of proficiency. (It's 
not possible to be proficient with this device.) In addition to the damage given 
on Table: Siege Engines, up to nine other characters holding the ram can add their 
Strength modifier to the ram's damage, if they devote an attack action to doing 
so. It takes at least one Huge or larger creature, two Large creatures, four Medium-size 
creatures, or eight Small creatures to swing a ram. (Tiny or smaller creatures 
can't use a ram.)

A ram is typically 30 feet long. In a battle, the creatures wielding the ram stand 
in two adjacent columns of equal length, with the ram between them.

\textbf{Siege Tower: }This device is a massive wooden tower on wheels or rollers 
that can be rolled up against a wall to allow attackers to scale the tower and 
thus to get to the top of the wall with cover. The wooden walls are usually 1 foot 
thick.

A typical siege tower takes up a space 15 feet across. The creatures inside push 
it at a speed of 10 feet (and a siege tower can't run). The eight creatures pushing 
on the ground floor have total cover, and those on higher floors get improved cover 
and can fire through arrow slits.

\vspace{12pt}
\subsubsection*{\textbf{City Streets}}

Typical city streets are narrow and twisting. Most streets average 15 to 20 feet 
wide [(1d4+1)\ensuremath{\times}5 feet)], while alleys range from 10 feet wide 
to only 5 feet. Cobblestones in good condition allow normal movement, but ones 
in poor repair and heavily rutted dirt streets are considered light rubble, increasing 
the DC of Balance and Tumble checks by 2.

Some cities have no larger thoroughfares, particularly cities that gradually grew 
from small settlements to larger cities. Cities that are planned, or perhaps have 
suffered a major fire that allowed authorities to construct new roads through formerly 
inhabited areas, might have a few larger streets through town. These main roads 
are 25 feet wide---offering room for wagons to pass each other---with 5-foot-wide 
sidewalks on either side.

\textbf{Crowds: }Urban streets are often full of people going about their daily 
lives. In most cases, it isn't necessary to put every 1st-level commoner on the 
map when a fight breaks out on the city's main thoroughfare. Instead just indicate 
which squares on the map contain crowds. If crowds see something obviously dangerous, 
they'll move away at 30 feet per round at initiative count 0. It takes 2 squares 
of movement to enter a square with crowds. The crowds provide cover for anyone 
who does so, enabling a Hide check and providing a bonus to Armor Class and on 
Reflex saves.

\textit{Directing Crowds: }It takes a DC 15 Diplomacy check or DC 20 Intimidate 
check to convince a crowd to move in a particular direction, and the crowd must 
be able to hear or see the character making the attempt. It takes a full-round 
action to make the Diplomacy check, but only a free action to make the Intimidate 
check.

If two or more characters are trying to direct a crowd in different directions, 
they make opposed Diplomacy or Intimidate checks to determine whom the crowd listens 
to. The crowd ignores everyone if none of the characters' check results beat the 
DCs given above.

\vspace{12pt}
\subsubsection*{\textbf{Above and beneath the Streets}}

\textbf{Rooftops: }Getting to a roof usually requires climbing a wall (see the 
Walls section), unless the character can reach a roof by jumping down from a higher 
window, balcony, or bridge. Flat roofs, common only in warm climates (accumulated 
snow can cause a flat roof to collapse), are easy to run across. Moving along the 
peak of a roof requires a DC 20 Balance check. Moving on an angled roof surface 
without changing altitude (moving parallel to the peak, in other words) requires 
a DC 15 Balance check. Moving up and down across the peak of a roof requires a 
DC 10 Balance check.

Eventually a character runs out of roof, requiring a long jump across to the next 
roof or down to the ground. The distance to the next closest roof is usually 1d3\ensuremath{\times}5 
feet horizontally, but the roof across the gap is equally likely to be 5 feet higher, 
5 feet lower, or the same height. Use the guidelines in the Jump skill\textit{ 
}(a horizontal jump's peak height is one-fourth of the horizontal distance) to 
determine whether a character can make a jump.

\textbf{Sewers:} To get into the sewers, most characters open a grate (a full-round 
action) and jump down 10 feet. Sewers are built exactly like dungeons, except that 
they're much more likely to have floors that are slippery or covered with water. 
Sewers are also similar to dungeons in terms of creatures liable to be encountered 
therein. Some cities were built atop the ruins of older civilizations, so their 
sewers sometimes lead to treasures and dangers from a bygone age.

\vspace{12pt}
\subsubsection*{\textbf{City Buildings}}

Most city buildings fall into three categories. The majority of buildings in the 
city are two to five stories high, built side by side to form long rows separated 
by secondary or main streets. These row houses usually have businesses on the ground 
floor, with offices or apartments above.

Inns, successful businesses, and large warehouses---as well as millers, tanners, 
and other businesses that require extra space--- are generally large, free-standing 
buildings with up to five stories. 

Finally, small residences, shops, warehouses, or storage sheds are simple, one-story 
wooden buildings, especially if they're in poorer neighborhoods.

Most city buildings are made of a combination of stone or clay brick (on the lower 
one or two stories) and timbers (for the upper stories, interior walls, and floors). 
Roofs are a mixture of boards, thatch, and slates, sealed with pitch. A typical 
lower-story wall is 1 foot thick, with AC 3, hardness 8, 90 hp, and a Climb DC 
of 25. Upper-story walls are 6 inches thick, with AC 3, hardness 5, 60 hp, and 
a Climb DC of 21. Exterior doors on most buildings are good wooden doors that are 
usually kept locked, except on public buildings such as shops and taverns.

\vspace{12pt}
\subsubsection*{\textbf{Buying Buildings}}

Characters might want to buy their own buildings or even construct

their own castle. Use the prices in Table: Buildings directly, or as a guide when 
for extrapolating costs for more exotic structures.

\begin{tabular}{|>{\raggedright}p{68pt}|>{\raggedright}p{51pt}|}
\hline
\multicolumn{2}{|p{119pt}|}{T\textbf{able: Buildings}}\tabularnewline
\hline
I\textbf{tem} & C\textbf{ost}\tabularnewline
\hline
Simple house & 1,000 gp\tabularnewline
\hline
Grand house & 5,000 gp\tabularnewline
\hline
Mansion & 100,000 gp\tabularnewline
\hline
Tower & 50,000 gp\tabularnewline
\hline
Keep & 150,000 gp\tabularnewline
\hline
Castle & 500,000 gp\tabularnewline
\hline
Huge castle & 1,000,000 gp\tabularnewline
\hline
Moat with bridge & 50,000 gp\tabularnewline
\hline
\end{tabular}

\textit{Simple House: }This one- to three-room house is made of wood and has a 
thatched roof.

\textit{Grand House: }This four- to ten-room house is made of wood and has a thatched 
roof.

\textit{Mansion: }This ten- to twenty-room residence has two or three stories and 
is made of wood and brick. It has a slate roof.

\textit{Tower: }This round or square, three-level tower is made of stone.

\textit{Keep: }This fortified stone building has fifteen to twenty-five rooms.

\textit{Castle: }A castle is a keep surrounded by a 15-foot stone wall with four 
towers. The wall is 10 feet thick.

\textit{Huge Castle: }A huge castle is a particularly large keep with numerous 
associated buildings (stables, forge, granaries, and so on) and an elaborate 20-foot-high 
wall that creates bailey and courtyard areas. The wall has six towers and is 10 
feet thick.

\textit{Moat with Bridge: }The moat is 15 feet deep and 30 feet wide. The bridge 
may be a wooden drawbridge or a permanent stone structure.

\vspace{12pt}
\subsubsection*{\textbf{City Lights}}

If a city has main thoroughfares, they are lined with lanterns hanging at a height 
of 7 feet from building awnings. These lanterns are spaced 60 feet apart, so their 
illumination is all but continuous. Secondary streets and alleys are not lit; it 
is common for citizens to hire lantern-bearers when going out after dark.

Alleys can be dark places even in daylight, thanks to the shadows of the tall buildings 
that surround them. A dark alley in daylight is rarely dark enough to afford true 
concealment, but it can lend a +2 circumstance bonus on Hide checks.

\vspace{12pt}
WEATHER

Sometimes weather can play an important role in an adventure.

Table: Random Weather is an appropriate weather table for general use, and can 
be used as a basis for a local weather tables. Terms on that table are defined 
as follows.

\textbf{Calm:} Wind speeds are light (0 to 10 mph).

\textbf{Cold:} Between 0° and 40° Fahrenheit during the day, 10 to 20 degrees 
colder at night.

\textbf{Cold Snap: }Lowers temperature by -10° F.

\textbf{Downpour:} Treat as rain (see Precipitation, below), but conceals as fog. 
Can create floods (see above). A downpour lasts for 2d4 hours.

\textbf{Heat Wave:} Raises temperature by +10° F.

\textbf{Hot:} Between 85° and 110° Fahrenheit during the day, 10 to 20 degrees 
colder at night.

\textbf{Moderate:} Between 40° and 60° Fahrenheit during the day, 10 to 20 degrees 
colder at night.

\textbf{Powerful Storm }\textit{\textbf{(Windstorm/Blizzard/Hurricane/Tornado): 
}}Wind speeds are over 50 mph (see Table: Wind Effects). In addition, blizzards 
are accompanied by heavy snow (1d3 feet), and hurricanes are accompanied by downpours 
(see above). Windstorms last for 1d6 hours. Blizzards last for 1d3 days. Hurricanes 
can last for up to a week, but their major impact on characters will come in a 
24-to-48-hour period when the center of the storm moves through their area. Tornadoes 
are very short-lived (1d6\ensuremath{\times}10 minutes), typically forming as part 
of a thunderstorm system. 

\textbf{Precipitation:} Roll d\% to determine whether the precipitation is fog 
(01-30), rain/snow (31-90), or sleet/hail (91-00). Snow and sleet occur only when 
the temperature is 30° Fahrenheit or below. Most precipitation lasts for 2d4 hours. 
By contrast, hail lasts for only 1d20 minutes but usually accompanies 1d4 hours 
of rain.

\textbf{Storm }\textit{\textbf{(Duststorm/Snowstorm/Thunderstorm):}}\textit{ }Wind 
speeds are severe (30 to 50 mph) and visibility is cut by three-quarters. Storms 
last for 2d4-1 hours. See Storms, below, for more details. 

\textbf{Warm:} Between 60° and 85° Fahrenheit during the day, 10 to 20 degrees 
colder at night.

\textbf{Windy:} Wind speeds are moderate to strong (10 to 30 mph); see Table: Wind 
Effects on the following page.

\begin{tabular}{|>{\raggedright}p{20pt}|>{\raggedright}p{33pt}|>{\raggedright}p{104pt}|>{\raggedright}p{104pt}|>{\raggedright}p{27pt}|}
\hline
\multicolumn{5}{|p{290pt}|}{T\textbf{able: Random Weather }}\tabularnewline
\hline
d\textbf{\%} & W\textbf{eather} & C\textbf{old Climate} & T\textbf{emperate Climate}\textsuperscript{\textbf{1}} & D\textbf{esert}\tabularnewline
\hline
01-70 & Normal weather & Cold, calm & Normal for season\textsuperscript{\textbf{2}} & Hot, 
calm\tabularnewline
\hline
71-80 & Abnormal weather & Heat wave (01-30) or cold snap (31-100) & Heat wave 
(01-50) or cold snap (51-100) & Hot, windy\tabularnewline
\hline
81-90 & Inclement weather & Precipitation (snow) & Precipitation (normal for season) & Hot, 
windy\tabularnewline
\hline
91-99 & Storm & Snowstorm & Thunderstorm, snowstorm\textsuperscript{\textbf{3}} & Duststorm\tabularnewline
\hline
100 & Powerful storm & Blizzard & Windstorm, blizzard\textsuperscript{\textbf{4}}, 
hurricane, tornado & Downpour\tabularnewline
\hline
\multicolumn{5}{|p{290pt}|}{1 Temperate includes forest, hills, marsh, mountains, 
plains, and warm aquatic.}\tabularnewline
\hline
\multicolumn{5}{|p{290pt}|}{2 Winter is cold, summer is warm, spring and autumn 
are temperate. Marsh regions are slightly warmer in winter.}\tabularnewline
\hline
\end{tabular}

\subsubsection*{\textbf{Rain, Snow, Sleet, and Hail}}

Bad weather frequently slows or halts travel and makes it virtually impossible 
to navigate from one spot to another. Torrential downpours and blizzards obscure 
vision as effectively as a dense fog.

Most precipitation is rain, but in cold conditions it can manifest as snow, sleet, 
or hail. Precipitation of any kind followed by a cold snap in which the temperature 
dips from above freezing to 30° F or below may produce ice. \textit{Rain: }Rain 
reduces visibility ranges by half, resulting in a -4 penalty on Spot and Search 
checks. It has the same effect on flames, ranged weapon attacks, and Listen checks 
as severe wind.

\textit{Snow: }Falling snow has the same effects on visibility, ranged weapon attacks, 
and skill checks as rain, and it costs 2 squares of movement to enter a snow-covered 
square. A day of snowfall leaves 1d6 inches of snow on the ground.

\textit{Heavy Snow: }Heavy snow has the same effects as normal snowfall, but also 
restricts visibility as fog does (see Fog, below). A day of heavy snow leaves 1d4 
feet of snow on the ground, and it costs 4 squares of movement to enter a square 
covered with heavy snow. Heavy snow accompanied by strong or severe winds may result 
in snowdrifts 1d4\ensuremath{\times}5 feet deep, especially in and around objects 
big enough to deflect the wind---a cabin or a large tent, for instance. There is 
a 10\% chance that a heavy snowfall is accompanied by lightning (see Thunderstorm, 
below). Snow has the same effect on flames as moderate wind.

\textit{Sleet: }Essentially frozen rain, sleet has the same effect as rain while 
falling (except that its chance to extinguish protected flames is 75\%) and the 
same effect as snow once on the ground. 

\textit{Hail: }Hail does not reduce visibility, but the sound of falling hail makes 
Listen checks more difficult (-4 penalty). Sometimes (5\% chance) hail can become 
large enough to deal 1 point of lethal damage (per storm) to anything in the open. 
Once on the ground, hail has the same effect on movement as snow.

\subsubsection*{\textbf{Storms}}

The combined effects of precipitation (or dust) and wind that accompany all storms 
reduce visibility ranges by three quarters, imposing a -8 penalty on Spot, Search, 
and Listen checks. Storms make ranged weapon attacks impossible, except for those 
using siege weapons, which have a -4 penalty on attack rolls. They automatically 
extinguish candles, torches, and similar unprotected flames. They cause protected 
flames, such as those of lanterns, to dance wildly and have a 50\% chance to extinguish 
these lights. See Table: Wind Effects for possible consequences to creatures caught 
outside without shelter during such a storm. Storms are divided into the following 
three types. 

\textit{Duststorm (CR 3): }These desert storms differ from other storms in that 
they have no precipitation. Instead, a duststorm blows fine grains of sand that 
obscure vision, smother unprotected flames, and can even choke protected flames 
(50\% chance). Most duststorms are accompanied by severe winds and leave behind 
a deposit of 1d6 inches of sand. However, there is a 10\% chance for a greater 
duststorm to be accompanied by windstorm-magnitude winds (see Table: Wind Effects). 
These greater duststorms deal 1d3 points of nonlethal damage each round to anyone 
caught out in the open without shelter and also pose a choking hazard (see Drowning---except 
that a character with a scarf or similar protection across her mouth and nose does 
not begin to choke until after a number of rounds equal to 10 \ensuremath{\times} 
her Constitution score). Greater duststorms leave 2d3-1 feet of fine sand in their 
wake.

\textit{Snowstorm: }In addition to the wind and precipitation common to other storms, 
snowstorms leave 1d6 inches of snow on the ground afterward. 

\textit{Thunderstorm: }In addition to wind and precipitation (usually rain, but 
sometimes also hail), thunderstorms are accompanied by lightning that can pose 
a hazard to characters without proper shelter (especially those in metal armor). 
As a rule of thumb, assume one bolt per minute for a 1-hour period at the center 
of the storm. Each bolt causes electricity damage equal to 1d10 eight-sided dice. 
One in ten thunderstorms is accompanied by a tornado (see below). 

\textbf{Powerful Storms:} Very high winds and torrential precipitation reduce visibility 
to zero, making Spot, Search, and Listen checks and all ranged weapon attacks impossible. 
Unprotected flames are automatically extinguished, and protected flames have a 
75\% chance of being doused. Creatures caught in the area must make a DC 20 Fortitude 
save or face the effects based on the size of the creature (see Table: Wind Effects). 
Powerful storms are divided into the

following four types.

\textit{Windstorm: }While accompanied by little or no precipitation, windstorms 
can cause considerable damage simply through the force of their wind.

\textit{Blizzard: }The combination of high winds, heavy snow (typically 1d3 feet), 
and bitter cold make blizzards deadly for all who are unprepared for them.

\textit{Hurricane: }In addition to very high winds and heavy rain, hurricanes are 
accompanied by floods. Most adventuring activity is impossible under such conditions.

\textit{Tornado: }One in ten thunderstorms is accompanied by a tornado.

\subsubsection*{\textbf{Fog}}

Whether in the form of a low-lying cloud or a mist rising from the ground, fog 
obscures all sight, including darkvision, beyond 5 feet. Creatures 5 feet away 
have concealment (attacks by or against them have a 20\% miss chance).

\subsubsection*{\textbf{Winds}}

The wind can create a stinging spray of sand or dust, fan a large fire, heel over 
a small boat, and blow gases or vapors away. If powerful enough, it can even knock 
characters down (see Table: Wind Effects), interfere with ranged attacks, or impose 
penalties on some skill checks.

\textit{Light Wind: }A gentle breeze, having little or no game effect.

\textit{Moderate Wind: }A steady wind with a 50\% chance of extinguishing small, 
unprotected flames, such as candles.

\textit{Strong Wind: }Gusts that automatically extinguish unprotected flames (candles, 
torches, and the like). Such gusts impose a -2 penalty on ranged attack rolls and 
on Listen checks.

\textit{Severe Wind: }In addition to automatically extinguishing any unprotected 
flames, winds of this magnitude cause protected flames (such as those of lanterns) 
to dance wildly and have a 50\% chance of extinguishing these lights. Ranged weapon 
attacks and Listen checks are at a -4 penalty. This is the velocity of wind produced 
by a \textit{gust of wind }spell.

\textit{Windstorm: }Powerful enough to bring down branches if not whole trees, 
windstorms automatically extinguish unprotected flames and have a 75\% chance of 
blowing out protected flames, such as those of lanterns. Ranged weapon attacks 
are impossible, and even siege weapons have a -4 penalty on attack rolls. Listen 
checks are at a -8 penalty due to the howling of the wind. 

\textit{Hurricane-Force Wind: }All flames are extinguished. Ranged attacks are 
impossible (except with siege weapons, which have a -8 penalty on attack rolls). 
Listen checks are impossible: All characters can hear is the roaring of the wind. 
Hurricane-force winds often fell trees.

\textit{Tornado (CR 10): }All flames are extinguished. All ranged attacks are impossible 
(even with siege weapons), as are Listen checks. Instead of being blown away (see 
Table: Wind Effects), characters in close proximity to a tornado who fail their 
Fortitude saves are sucked toward the tornado. Those who come in contact with the 
actual funnel cloud are picked up and whirled around for 1d10 rounds, taking 6d6 
points of damage per round, before being violently expelled (falling damage may 
apply). While a tornado's rotational speed can be as great as 300 mph, the funnel 
itself moves forward at an average of 30 mph (roughly 250 feet per round). A tornado 
uproots trees, destroys buildings, and causes other similar forms of major destruction.

\vspace{12pt}
\begin{tabular}{|>{\raggedright}p{43pt}|>{\raggedright}p{41pt}|>{\raggedright}p{67pt}|>{\raggedright}p{68pt}|>{\raggedright}p{41pt}|>{\raggedright}p{15pt}|}
\hline
\multicolumn{6}{|p{278pt}|}{\subsection*{T\textbf{able: Wind Effects}}}\tabularnewline
\hline
W\textbf{ind Force} & W\textbf{ind Speed} & R\textbf{anged Attacks Normal/Siege 
Weapons}\textsuperscript{\textbf{1}} & C\textbf{reature Size}\textsuperscript{\textbf{2}}\textbf{ 
} & W\textbf{ind Effect on Creatures} & F\textbf{ort Save DC}\tabularnewline
\hline
Light & 0-10 mph--- & /--- & Any & None--- & \tabularnewline
\hline
Moderate & 11-20 mph--- & /--- & Any & None--- & \tabularnewline
\hline
Strong & 21-30 mph- & 2/--- & Tiny or smaller & Knocked down & 10\tabularnewline
\hline
 &  &  & Small or larger & None & \tabularnewline
\hline
Severe & 31-50 mph- & 4/--- & Tiny & Blown away & 15\tabularnewline
\hline
 &  &  & Small & Knocked down & \tabularnewline
\hline
 &  &  & Medium & Checked & \tabularnewline
\hline
 &  &  & Large or larger & None & \tabularnewline
\hline
Windstorm & 51-74 mph & Impossible/-4 & Small or smaller & Blown away & 18\tabularnewline
\hline
 &  &  & Medium & Knocked down & \tabularnewline
\hline
 &  &  & Large or Huge & Checked & \tabularnewline
\hline
 &  &  & Gargantuan or Colossal & None & \tabularnewline
\hline
Hurricane & 75-174 mph & Impossible/-8 & Medium or smaller & Blown away & 20\tabularnewline
\hline
 &  &  & Large & Knocked down & \tabularnewline
\hline
 &  &  & Huge & Checked & \tabularnewline
\hline
 &  &  & Gargantuan or Colossal & None & \tabularnewline
\hline
Tornado & 175-300 mph & Impossible/impossible & Large or smaller & Blown away & 30\tabularnewline
\hline
 &  &  & Huge & Knocked down & \tabularnewline
\hline
 &  &  & Gargantuan or Colossal & Checked & \tabularnewline
\hline
\multicolumn{6}{|p{278pt}|}{1 The siege weapon category includes ballista and catapult 
attacks as well as boulders tossed by giants.}\tabularnewline
\hline
\multicolumn{6}{|p{278pt}|}{2 Flying or airborne creatures are treated as one size 
category smaller than their actual size, so an airborne Gargantuan dragon is treated 
as Huge for purposes of wind effects.}\tabularnewline
\hline
\multicolumn{6}{|p{278pt}|}{C\textit{hecked: }Creatures are unable to move forward 
against the force of the wind. Flying creatures are blown back 1d6\ensuremath{\times}5 
feet.}\tabularnewline
\hline
\multicolumn{6}{|p{278pt}|}{K\textit{nocked Down: }Creatures are knocked prone 
by the force of the wind. Flying creatures are instead blown back 1d6\ensuremath{\times}10 
feet.}\tabularnewline
\hline
\multicolumn{6}{|p{278pt}|}{B\textit{lown Away: }Creatures on the ground are knocked 
prone and rolled 1d4\ensuremath{\times}10 feet, taking 1d4 points of nonlethal 
damage per 10 feet. Flying creatures are blown back 2d6\ensuremath{\times}10 feet 
and take 2d6 points of nonlethal damage due to battering and buffeting.}\tabularnewline
\hline
\end{tabular}

\vspace{12pt}
{\LARGE{}THE ENVIRONMENT}

Environmental hazards specific to one kind of terrain (such as an avalanche, which 
occurs in the mountains) are described in Wilderness, above. Environmental hazards 
common to more than one setting are detailed below.

\vspace{12pt}
ACID EFFECTS

Corrosive acids deals 1d6 points of damage per round of exposure except in the 
case of total immersion (such as into a vat of acid), which deals 10d6 points of 
damage per round. An attack with acid, such as from a hurled vial or a monster's 
spittle, counts as a round of exposure.

The fumes from most acids are inhaled poisons. Those who come close enough to a 
large body of acid to dunk a creature in it must make a DC 13 Fortitude save or 
take 1 point of Constitution damage. All such characters must make a second save 
1 minute later or take another 1d4 points of Constitution damage.

Creatures immune to acid's caustic properties might still drown in it if they are 
totally immersed (see Drowning).

\vspace{12pt}
COLD DANGERS

Cold and exposure deal nonlethal damage to the victim. This nonlethal damage cannot 
be recovered until the character gets out of the cold and warms up again. Once 
a character is rendered unconscious through the accumulation of nonlethal damage, 
the cold and exposure begins to deal lethal damage at the same rate.

An unprotected character in cold weather (below 40° F) must make a Fortitude save 
each hour (DC 15, + 1 per previous check) or take 1d6 points of nonlethal damage. 
A character who has the Survival skill may receive a bonus on this saving throw 
and may be able to apply this bonus to other characters as well (see the skill 
Description).

In conditions of severe cold or exposure (below 0° F), an unprotected character 
must make a Fortitude save once every 10 minutes (DC 15, +1 per previous check), 
taking 1d6 points of nonlethal damage on each failed save. A character who has 
the Survival skill may receive a bonus on this saving throw and may be able to 
apply this bonus to other characters as well (see the skill description). Characters 
wearing winter clothing only need check once per hour for cold and exposure damage.

A character who takes any nonlethal damage from cold or exposure is beset by frostbite 
or hypothermia (treat her as fatigued). These penalties end when the character 
recovers the nonlethal damage she took from the cold and exposure.

Extreme cold (below -20° F) deals 1d6 points of lethal damage per minute (no save). 
In addition, a character must make a Fortitude save (DC 15, +1 per previous check) 
or take 1d4 points of nonlethal damage. Those wearing metal armor or coming into 
contact with very cold metal are affected as if by a \textit{chill metal }spell.

\subsubsection*{\textbf{Ice Effects}}

Characters walking on ice must spend 2 squares of movement to enter a square covered 
by ice, and the DC for Balance and Tumble checks increases by +5. Characters in 
prolonged contact with ice may run the risk of taking damage from severe cold (see 
above).

\vspace{12pt}
DARKNESS

Darkvision allows many characters and monsters to see perfectly well without any 
light at all, but characters with normal vision (or low-light vision, for that 
matter) can be rendered completely blind by putting out the lights. Torches or 
lanterns can be blown out by sudden gusts of subterranean wind, magical light sources 
can be dispelled or countered, or magical traps might create fields of impenetrable 
darkness.

In many cases, some characters or monsters might be able to see, while others are 
blinded. For purposes of the following points, a blinded creature is one who simply 
can't see through the surrounding darkness.---

Creatures blinded by darkness lose the ability to deal extra damage due to precision 
(for example, a sneak attack).---

Blinded creatures are hampered in their movement, and pay 2 squares of movement 
per square moved into (double normal cost). Blinded creatures can't run or charge.---

All opponents have total concealment from a blinded creature, so the blinded creature 
has a 50\% miss chance in combat. A blinded creature must first pinpoint the location 
of an opponent in order to attack the right square; if the blinded creature launches 
an attack without pinpointing its foe, it attacks a random square within its reach. 
For ranged attacks or spells against a foe whose location is not pinpointed, roll 
to determine which adjacent square the blinded creature is facing; its attack is 
directed at the closest target that lies in that direction.---

A blinded creature loses its Dexterity adjustment to AC and takes a -2 penalty 
to AC.---

A blinded creature takes a -4 penalty on Search checks and most Strength- and Dexterity-based 
skill checks, including any with an armor check penalty. A creature blinded by 
darkness automatically fails any skill check relying on vision.---

Creatures blinded by darkness cannot use gaze attacks and are immune to gaze attacks.

A creature blinded by darkness can make a Listen check as a free action each round 
in order to locate foes (DC equal to opponents' Move Silently checks). A successful 
check lets a blinded character hear an unseen creature ``over there somewhere.'' 
It's almost impossible to pinpoint the location of an unseen creature. A Listen 
check that beats the DC by 20 reveals the unseen creature's square (but the unseen 
creature still has total concealment from the blinded creature).---

A blinded creature can grope about to find unseen creatures. A character can make 
a touch attack with his hands or a weapon into two adjacent squares using a standard 
action. If an unseen target is in the designated square, there is a 50\% miss chance 
on the touch attack. If successful, the groping character deals no damage but has 
pinpointed the unseen creature's current location. (If the unseen creature moves, 
its location is once again unknown.)---

If a blinded creature is struck by an unseen foe, the blinded character pinpoints 
the location of the creature that struck him (until the unseen creature moves, 
of course). The only exception is if the unseen creature has a reach greater than 
5 feet (in which case the blinded character knows the location of the unseen opponent, 
but has not pinpointed him) or uses a ranged attack (in which case, the blinded 
character knows the general direction of the foe, but not his location).---

A creature with the scent ability automatically pinpoints unseen creatures within 
5 feet of its location.

\vspace{12pt}
FALLING

\textbf{Falling Damage:} The basic rule is simple: 1d6 points of damage per 10 
feet fallen, to a maximum of 20d6.

If a character deliberately jumps instead of merely slipping or falling, the damage 
is the same but the first 1d6 is nonlethal damage. A DC 15 Jump check or DC 15 
Tumble check allows the character to avoid any damage from the first 10 feet fallen 
and converts any damage from the second 10 feet to nonlethal damage. Thus, a character 
who slips from a ledge 30 feet up takes 3d6 damage. If the same character deliberately 
jumped, he takes 1d6 points of nonlethal damage and 2d6 points of lethal damage. 
And if the character leaps down with a successful Jump or Tumble check, he takes 
only 1d6 points of nonlethal damage and 1d6 points of lethal damage from the plunge.

Falls onto yielding surfaces (soft ground, mud) also convert the first 1d6 of damage 
to nonlethal damage. This reduction is cumulative with reduced damage due to deliberate 
jumps and the Jump skill.

\textbf{Falling into Water:} Falls into water are handled somewhat differently. 
If the water is at least 10 feet deep, the first 20 feet of falling do no damage. 
The next 20 feet do nonlethal damage (1d3 per 10-foot increment). Beyond that, 
falling damage is lethal damage (1d6 per additional 10-foot increment).

Characters who deliberately dive into water take no damage on a successful DC 15 
Swim check or DC 15 Tumble check, so long as the water is at least 10 feet deep 
for every 30 feet fallen. However, the DC of the check increases by 5 for every 
50 feet of the dive. 

\vspace{12pt}
FALLING OBJECTS

Just as characters take damage when they fall more than 10 feet, so too do they 
take damage when they are hit by falling objects.

Objects that fall upon characters deal damage based on their weight and the distance 
they have fallen.

For each 200 pounds of an object's weight, the object deals 1d6 points of damage, 
provided it falls at least 10 feet. Distance also comes into play, adding an additional 
1d6 points of damage for every 10-foot increment it falls beyond the first (to 
a maximum of 20d6 points of damage).

Objects smaller than 200 pounds also deal damage when dropped, but they must fall 
farther to deal the same damage. Use Table: Damage from Falling Objects to see 
how far an object of a given weight must drop to deal 1d6 points of damage.

\begin{tabular}{|>{\raggedright}p{68pt}|>{\raggedright}p{75pt}|}
\hline
\multicolumn{2}{|p{143pt}|}{T\textbf{able: Damage from Falling Objects}}\tabularnewline
\hline
O\textbf{bject Weight} & F\textbf{alling Distance}\tabularnewline
\hline
200-101 lb. & 20 ft.\tabularnewline
\hline
100-51 lb. & 30 ft.\tabularnewline
\hline
50-31 lb. & 40 ft.\tabularnewline
\hline
30-11 lb. & 50 ft.\tabularnewline
\hline
10-6 lb. & 60 ft.\tabularnewline
\hline
5-1 lb. & 70 ft.\tabularnewline
\hline
\end{tabular}

For each additional increment an object falls, it deals an additional 1d6 points 
of damage.

Objects weighing less than 1 pound do not deal damage to those they land upon, 
no matter how far they have fallen.

\vspace{12pt}
HEAT DANGERS

Heat deals nonlethal damage that cannot be recovered until the character gets cooled 
off (reaches shade, survives until nightfall, gets doused in water, is targeted 
by \textit{endure elements}, and so forth). Once rendered unconscious through the 
accumulation of nonlethal damage, the character begins to take lethal damage at 
the same rate.

A character in very hot conditions (above 90° F) must make a Fortitude saving 
throw each hour (DC 15, +1 for each previous check) or take 1d4 points of nonlethal 
damage. Characters wearing heavy clothing or armor of any sort take a -4 penalty 
on their saves. A character with the Survival skill may receive a bonus on this 
saving throw and may be able to apply this bonus to other characters as well (see 
the skill description). Characters reduced to unconsciousness begin taking lethal 
damage (1d4 points per hour).

In severe heat (above 110° F), a character must make a Fortitude save once every 
10 minutes (DC 15, +1 for each previous check) or take 1d4 points of nonlethal 
damage. Characters wearing heavy clothing or armor of any sort take a -4 penalty 
on their saves. A character with the Survival skill may receive a bonus on this 
saving throw and may be able to apply this bonus to other characters as well. Characters 
reduced to unconsciousness begin taking lethal damage (1d4 points per each 10-minute 
period).

A character who takes any nonlethal damage from heat exposure now suffers from 
heatstroke and is fatigued.

These penalties end when the character recovers the nonlethal damage she took from 
the heat.

Extreme heat (air temperature over 140° F, fire, boiling water, lava) deals lethal 
damage. Breathing air in these temperatures deals 1d6 points of damage per minute 
(no save). In addition, a character must make a Fortitude save every 5 minutes 
(DC 15, +1 per previous check) or take 1d4 points of nonlethal damage. Those wearing 
heavy clothing or any sort of armor take a -4 penalty on their saves. In addition, 
those wearing metal armor or coming into contact with very hot metal are affected 
as if by a \textit{heat metal }spell.

Boiling water deals 1d6 points of scalding damage, unless the character is fully 
immersed, in which case it deals 10d6 points of damage per round of exposure.

\subsubsection*{\textbf{Catching on Fire}}

Characters exposed to burning oil, bonfires, and noninstantaneous magic fires\textit{ 
}might find their clothes, hair, or equipment on fire. Spells with an instantaneous 
duration\textit{ }don't normally set a character on fire, since the heat and flame 
from these come and go in a flash.

Characters at risk of catching fire are allowed a DC 15 Reflex save to avoid this 
fate. If a character's clothes or hair catch fire, he takes 1d6 points of damage 
immediately. In each subsequent round, the burning character must make another 
Reflex saving throw. Failure means he takes another 1d6 points of damage that round. 
Success means that the fire has gone out. (That is, once he succeeds on his saving 
throw, he's no longer on fire.)

A character on fire may automatically extinguish the flames by jumping into enough 
water to douse himself. If no body of water is at hand, rolling on the ground or 
smothering the fire with cloaks or the like permits the character another save 
with a +4 bonus.

Those unlucky enough to have their clothes or equipment catch fire must make DC 
15 Reflex saves for each item. Flammable items that fail take the same amount of 
damage as the character.

\subsubsection*{\textbf{Lava Effects}}

Lava or magma deals 2d6 points of damage per round of exposure, except in the case 
of total immersion (such as when a character falls into the crater of an active 
volcano), which deals 20d6 points of damage per round.

Damage from magma continues for 1d3 rounds after exposure ceases, but this additional 
damage is only half of that dealt during actual contact (that is, 1d6 or 10d6 points 
per round).

An immunity or resistance to fire serves as an immunity to lava or magma. However, 
a creature immune to fire might still drown if completely immersed in lava (see 
Drowning, below).

\vspace{12pt}
SMOKE EFFECTS

A character who breathes heavy smoke must make a Fortitude save each round (DC 
15, +1 per previous check) or spend that round choking and coughing. A character 
who chokes for 2 consecutive rounds takes 1d6 points of nonlethal damage.

Smoke obscures vision, giving concealment (20\% miss chance) to characters within 
it.

\vspace{12pt}
STARVATION AND THIRST

Characters might find themselves without food or water and with no means to obtain 
them. In normal climates, Medium characters need at least a gallon of fluids and 
about a pound of decent food per day to avoid starvation. (Small characters need 
half as much.) In very hot climates, characters need two or three times as much 
water to avoid dehydration.

A character can go without water for 1 day plus a number of hours equal to his 
Constitution score. After this time, the character must make a Constitution check 
each hour (DC 10, +1 for each previous check) or take 1d6 points of nonlethal damage.

A character can go without food for 3 days, in growing discomfort. After this time, 
the character must make a Constitution check each day (DC 10, +1 for each previous 
check) or take 1d6 points of nonlethal damage.

Characters who have taken nonlethal damage from lack of food or water are fatigued. 
Nonlethal damage from thirst or starvation cannot be recovered until the character 
gets food or water, as needed---not even magic that restores hit points heals this 
damage.

\vspace{12pt}
SUFFOCATION

A character who has no air to breathe can hold her breath for 2 rounds per point 
of Constitution. After this period of time, the character must make a DC 10 Constitution 
check in order to continue holding her breath. The save must be repeated each round, 
with the DC increasing by +1 for each previous success.

When the character fails one of these Constitution checks, she begins to suffocate. 
In the first round, she falls unconscious (0 hit points). In the following round, 
she drops to -1 hit points and is dying. In the third round, she suffocates.

\textbf{Slow Suffocation:} A Medium character can breathe easily for 6 hours in 
a sealed chamber measuring 10 feet on a side. After that time, the character takes 
1d6 points of nonlethal damage every 15 minutes. Each additional Medium character 
or significant fire source (a torch, for example) proportionally reduces the time 
the air will last.

Small characters consume half as much air as Medium characters. A larger volume 
of air, of course, lasts for a longer time. 

\vspace{12pt}
WATER DANGERS

Any character can wade in relatively calm water that isn't over his head, no check 
required. Similarly, swimming in calm water only requires skill checks with a DC 
of 10. Trained swimmers can just take 10. (Remember, however, that armor or heavy 
gear makes any attempt at swimming much more difficult. See the Swim skill description\textit{.})

By contrast, fast-moving water is much more dangerous. On a successful DC 15 Swim 
check or a DC 15 Strength check, it deals 1d3 points of nonlethal damage per round 
(1d6 points of lethal damage if flowing over rocks and cascades). On a failed check, 
the character must make another check that round to avoid going under.

Very deep water is not only generally pitch black, posing a navigational hazard, 
but worse, it deals water pressure damage of 1d6 points per minute for every 100 
feet the character is below the surface. A successful Fortitude save (DC 15, +1 
for each previous check) means the diver takes no damage in that minute. Very cold 
water deals 1d6 points of nonlethal damage from hypothermia per minute of exposure.

\subsubsection*{\textbf{Drowning}}

Any character can hold her breath for a number of rounds equal to twice her Constitution 
score. After this period of time, the character must make a DC 10 Constitution 
check every round in order to continue holding her breath. Each round, the DC increases 
by 1. 

When the character finally fails her Constitution check, she begins to drown. In 
the first round, she falls unconscious (0 hp). In the following round, she drops 
to -1 hit points and is dying. In the third round, she drowns.

It is possible to drown in substances other than water, such as sand, quicksand, 
fine dust, and silos full of grain.

\newpage

\end{document}
