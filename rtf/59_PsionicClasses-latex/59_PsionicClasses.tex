%&pdfLaTeX
% !TEX encoding = UTF-8 Unicode
\documentclass{article}
\usepackage{ifxetex}
\ifxetex
\usepackage{fontspec}
\setmainfont[Mapping=tex-text]{STIXGeneral}
\else
\usepackage[T1]{fontenc}
\usepackage[utf8]{inputenc}
\fi
\usepackage{textcomp}

\usepackage{array}
\usepackage{amssymb}
\usepackage{fancyhdr}
\renewcommand{\headrulewidth}{0pt}
\renewcommand{\footrulewidth}{0pt}

\begin{document}

This material is Open Game Content, and is licensed for public use under the terms 
of the Open Game License v1.0a.

\subsection*{{\LARGE{}PSIONIC CLASSES}}

\vspace{12pt}
\subsubsection*{THE POWER POINT RESERVE}

Psionic characters fuel their abilities through a pool, or reserve, of power points. 
Your power point reserve is equal to your base power points gained from your class, 
bonus power points from a high key ability score (see Abilities and Manifesters, 
below), and any additional bonus power points from sources such as your character 
race and feat selections.

\section*{\textbf{Multiclass Psionic Characters}}

If you have levels in more than one psionic class, you combine your power points 
from each class to make up your reserve. You can use these power points to manifest 
powers from any psionic class you have. 

While you maintain a single reserve of power points from your class, race, and 
feat selections, you are still limited by the manifester level you have achieved 
with each power you know. 

\vspace{12pt}
\subsubsection*{ABILITIES AND MANIFESTERS}

The ability that your powers depend on---your key ability score as a manifester---is 
related to what psionic class (or classes) you have levels in: Intelligence (psion), 
Wisdom (psychic warrior), and Charisma (wilder). The modifier for this ability 
is referred to as your key ability modifier. If your character's key ability score 
is 9 or lower, you can't manifest powers from that psionic class.

Just as a high Intelligence score grants bonus spells to a wizard and a high Wisdom 
score grants bonus spells to a cleric, a character who manifests powers (psions, 
psychic warriors, and wilders) gains bonus power points according to his key ability 
score. Refer to Table: Ability Modifiers and Bonus Power Points.

\textbf{How To Determine Bonus Power Points:} Your key ability score grants you 
additional power points equal to your key ability modifier x your manifester level 
x1/2. Table: Ability Modifiers and Bonus Power Points shows these calculations 
for class levels 1st through 20th and key ability scores from 10 to 41.

\vspace{12pt}
\begin{tabular}{|>{\raggedright}p{11pt}|>{\raggedright}p{3pt}|>{\raggedright}p{4pt}|>{\raggedright}p{3pt}|>{\raggedright}p{3pt}|>{\raggedright}p{3pt}|>{\raggedright}p{3pt}|>{\raggedright}p{3pt}|>{\raggedright}p{3pt}|>{\raggedright}p{3pt}|>{\raggedright}p{4pt}|>{\raggedright}p{4pt}|>{\raggedright}p{4pt}|>{\raggedright}p{4pt}|>{\raggedright}p{4pt}|>{\raggedright}p{4pt}|>{\raggedright}p{4pt}|>{\raggedright}p{4pt}|>{\raggedright}p{4pt}|>{\raggedright}p{4pt}|>{\raggedright}p{4pt}|}
\hline
\multicolumn{21}{|p{98pt}|}{T\textbf{able: Ability Modifiers and Bonus Power Points}}\tabularnewline
\hline
A\textbf{bility Score}----------------------------------------------- & \multicolumn{20}{p{86pt}|}{ 
\textbf{Bonus Power Points (by Class Level) -------------------------------------------}}\tabularnewline
\hline
 & 1\textbf{st} & 2\textbf{nd} & 3\textbf{rd} & 4\textbf{th} & 5\textbf{th} & 6\textbf{th} & 7\textbf{th} & 8\textbf{th} & 9\textbf{th} & 1\textbf{0th} & 1\textbf{1th} & 1\textbf{2th} & 1\textbf{3th} & 1\textbf{4th} & 1\textbf{5th} & 1\textbf{6th} & 1\textbf{7th} & 1\textbf{8th} & 1\textbf{9th} & 2\textbf{0th}\tabularnewline
\hline
10-11 & 0 & 0 & 0 & 0 & 0 & 0 & 0 & 0 & 0 & 0 & 0 & 0 & 0 & 0 & 0 & 0 & 0 & 0 & 0 & 0\tabularnewline
\hline
12-13 & 0 & 1 & 1 & 2 & 2 & 3 & 3 & 4 & 4 & 5 & 5 & 6 & 6 & 7 & 7 & 8 & 8 & 9 & 9 & 10\tabularnewline
\hline
14-15 & 1 & 2 & 3 & 4 & 5 & 6 & 7 & 8 & 9 & 10 & 11 & 12 & 13 & 14 & 15 & 16 & 17 & 18 & 19 & 20\tabularnewline
\hline
16-17 & 1 & 3 & 4 & 6 & 7 & 9 & 10 & 12 & 13 & 15 & 16 & 18 & 19 & 21 & 22 & 24 & 25 & 27 & 28 & 30\tabularnewline
\hline
18-19 & 2 & 4 & 6 & 8 & 10 & 12 & 14 & 16 & 18 & 20 & 22 & 24 & 26 & 28 & 30 & 32 & 34 & 36 & 38 & 40\tabularnewline
\hline
20-21 & 2 & 5 & 7 & 10 & 12 & 15 & 17 & 20 & 22 & 25 & 27 & 30 & 32 & 35 & 37 & 40 & 42 & 45 & 47 & 50\tabularnewline
\hline
22-23 & 3 & 6 & 9 & 12 & 15 & 18 & 21 & 24 & 27 & 30 & 33 & 36 & 39 & 42 & 45 & 48 & 51 & 54 & 57 & 60\tabularnewline
\hline
24-25 & 3 & 7 & 10 & 14 & 17 & 21 & 24 & 28 & 31 & 35 & 38 & 42 & 45 & 49 & 52 & 56 & 59 & 63 & 66 & 70\tabularnewline
\hline
26-27 & 4 & 8 & 12 & 16 & 20 & 24 & 28 & 32 & 36 & 40 & 44 & 48 & 52 & 56 & 60 & 64 & 68 & 72 & 76 & 80\tabularnewline
\hline
28-29 & 4 & 9 & 13 & 18 & 22 & 27 & 31 & 36 & 40 & 45 & 49 & 54 & 58 & 63 & 67 & 72 & 76 & 81 & 85 & 90\tabularnewline
\hline
30-31 & 5 & 10 & 15 & 20 & 25 & 30 & 35 & 40 & 45 & 50 & 55 & 60 & 65 & 70 & 75 & 80 & 85 & 90 & 95 & 100\tabularnewline
\hline
32-33 & 5 & 11 & 16 & 22 & 27 & 33 & 38 & 44 & 49 & 55 & 60 & 66 & 71 & 77 & 82 & 88 & 93 & 99 & 104 & 110\tabularnewline
\hline
34-35 & 6 & 12 & 18 & 24 & 30 & 36 & 42 & 48 & 54 & 60 & 66 & 72 & 78 & 84 & 90 & 96 & 102 & 108 & 114 & 120\tabularnewline
\hline
36-37 & 6 & 13 & 19 & 26 & 32 & 39 & 45 & 52 & 58 & 65 & 71 & 78 & 84 & 91 & 97 & 104 & 110 & 117 & 123 & 130\tabularnewline
\hline
38-39 & 7 & 14 & 21 & 28 & 35 & 42 & 49 & 56 & 63 & 70 & 77 & 84 & 91 & 98 & 105 & 112 & 119 & 126 & 133 & 140\tabularnewline
\hline
40-41 & 7 & 15 & 22 & 30 & 37 & 45 & 52 & 60 & 67 & 75 & 82 & 90 & 97 & 105 & 112 & 120 & 127 & 135 & 142 & 150\tabularnewline
\hline
\end{tabular}

\vspace{24pt}
RANDOM STARTING GOLD

\begin{tabular}{|>{\raggedright}p{61pt}|>{\raggedright}p{78pt}|}
\hline
\multicolumn{2}{|p{140pt}|}{T\textbf{able: Random Starting Gold}}\tabularnewline
\hline
C\textbf{lass} & A\textbf{mount (Average)}\tabularnewline
\hline
Psion & 3d4 x10 (75 gp)\tabularnewline
\hline
Psychic warrior & 5d4 x10 (125 gp)\tabularnewline
\hline
Soulknife & 5d4 x10 (125 gp)\tabularnewline
\hline
Wilder & 4d4 x10 (100 gp)\tabularnewline
\hline
\end{tabular}

\vspace{12pt}
{\LARGE{}PSION}

\textbf{Alignment:} Any.

\textbf{Hit Die:} d4.

\vspace{12pt}
\section*{\textbf{Class Skills}}

The psion's class skills (and the key ability for each skill) are Concentration* 
(Con), Craft (Int), Knowledge (all skills, taken individually)* (Int), Profession 
(Wis), and Psicraft* (Int). In addition, a psion gains access to additional class 
skills based on his discipline:

\textit{Seer (Clairsentience): }Gather Information (Cha), Listen (Wis), and Spot 
(Wis).

\textit{Shaper (Metacreativity): }Bluff (Cha), Disguise (Cha), and Use Psionic 
Device* (Cha).

\textit{Kineticist (Psychokinesis): }Autohypnosis* (Wis), Disable Device (Dex), 
and Intimidate (Cha).

\textit{Egoist (Psychometabolism): }Autohypnosis* (Wis), Balance (Dex) and Heal 
(Wis).

\textit{Nomad (Psychoportation): }Climb (Str), Jump (Str), Ride (Dex), Survival 
(Wis), and Swim (Str).

\textit{Telepath (Telepathy): }Bluff (Cha), Diplomacy (Cha), Gather Information 
(Cha), and Sense Motive (Wis).

*New skill or expanded use of existing skill.

\textbf{Skill Points at 1st Level:} (2 + Int modifier) x4.

\textbf{Skill Points at Each Additional Level:} 2 + Int modifier.

\vspace{12pt}
\begin{tabular}{|>{\raggedright}p{17pt}|>{\raggedright}p{19pt}|>{\raggedright}p{12pt}|>{\raggedright}p{12pt}|>{\raggedright}p{14pt}|>{\raggedright}p{61pt}|>{\raggedright}p{26pt}|>{\raggedright}p{24pt}|>{\raggedright}p{52pt}|}
\hline
\multicolumn{9}{|p{242pt}|}{\section*{T\textbf{able: The Psion}}}\tabularnewline
\hline
L\textbf{evel } & B\textbf{ase Attack Bonus } & F\textbf{ort Save} & R\textbf{ef 
Save} & W\textbf{ill Save} & S\textbf{pecial } & P\textbf{ower Points/}\linebreak{}
\textbf{Day} & P\textbf{owers Known} & M\textbf{aximum Power Level Known}\tabularnewline
\hline
1st & +0 & +0 & +0 & +2 & Bonus feat, discipline & 2 & 3 & 1st\tabularnewline
\hline
2nd & +1 & +0 & +0 & +3--- &  & 6 & 5 & 1st\tabularnewline
\hline
3rd & +1 & +1 & +1 & +3--- &  & 11 & 7 & 2nd\tabularnewline
\hline
4th & +2 & +1 & +1 & +4--- &  & 17 & 9 & 2nd\tabularnewline
\hline
5th & +2 & +1 & +1 & +4 & Bonus feat & 25 & 11 & 3rd\tabularnewline
\hline
6th & +3 & +2 & +2 & +5--- &  & 35 & 13 & 3rd\tabularnewline
\hline
7th & +3 & +2 & +2 & +5--- &  & 46 & 15 & 4th\tabularnewline
\hline
8th & +4 & +2 & +2 & +6--- &  & 58 & 17 & 4th\tabularnewline
\hline
9th & +4 & +3 & +3 & +6--- &  & 72 & 19 & 5th\tabularnewline
\hline
10th & +5 & +3 & +3 & +7 & Bonus feat & 88 & 21 & 5th\tabularnewline
\hline
11th & +5 & +3 & +3 & +7--- &  & 106 & 22 & 6th\tabularnewline
\hline
12th & +6/+1 & +4 & +4 & +8--- &  & 126 & 24 & 6th\tabularnewline
\hline
13th & +6/+1 & +4 & +4 & +8--- &  & 147 & 25 & 7th\tabularnewline
\hline
14th & +7/+2 & +4 & +4 & +9--- &  & 170 & 27 & 7th\tabularnewline
\hline
15th & +7/+2 & +5 & +5 & +9 & Bonus feat & 195 & 28 & 8th\tabularnewline
\hline
16th & +8/+3 & +5 & +5 & +10--- &  & 221 & 30 & 8th\tabularnewline
\hline
17th & +8/+3 & +5 & +5 & +10--- &  & 250 & 31 & 9th\tabularnewline
\hline
18th & +9/+4 & +6 & +6 & +11--- &  & 280 & 33 & 9th\tabularnewline
\hline
19th & +9/+4 & +6 & +6 & +11--- &  & 311 & 34 & 9th\tabularnewline
\hline
20th & +10/+5 & +6 & +6 & +12 & Bonus feat & 343 & 36 & 9th\tabularnewline
\hline
\end{tabular}

\vspace{12pt}
\section*{\textbf{Class Features}}

All the following are class features of the psion.

\textbf{Weapon and Armor Proficiency: }Psions are proficient with the club, dagger, 
heavy crossbow, light crossbow, quarterstaff, and shortspear. They are not proficient 
with any type of armor or shield. Armor does not, however, interfere with the manifestation 
of powers.

\textbf{Power Points/Day: }A psion's ability to manifest powers is limited by the 
power points he has available. His base daily allotment of power points is given 
on Table: The Psion. In addition, he receives bonus power points per day if he 
has a high Intelligence score (see Table: Ability Modifiers and Bonus Power Points). 
His race may also provide bonus power points per day, as may certain feats and 
items.

\textbf{Discipline: }Every psion must decide at 1st level which psionic discipline 
he will specialize in. Choosing a discipline provides a psion with access to the 
class skills associated with that discipline (see above), as well as the powers 
restricted to that discipline. However, choosing a discipline also means that the 
psion cannot learn powers that are restricted to other disciplines. He can't even 
use such powers by employing psionic items.

\textbf{Powers Known:} A psion begins play knowing three psion powers of your choice. 
Each time he achieves a new level, he unlocks the knowledge of new powers.

Choose the powers known from the psion power list, or from the list of powers of 
your chosen discipline. You cannot choose powers from disciplines other than your 
chosen discipline. (\textit{Exception: }The feats Expanded Knowledge and Epic Expanded 
Knowledge do allow a psion to learn powers from the lists of other disciplines 
or even other classes.) A psion can manifest any power that has a power point cost 
equal to or lower than his manifester level.

The number of times a psion can manifest powers in a day is limited only by his 
daily power points. 

A psion simply knows his powers; they are ingrained in his mind. He does not need 
to prepare them (in the way that some spellcasters prepare their spells), though 
he must get a good night's sleep each day to regain all his spent power points.

The Difficulty Class for saving throws against psion powers is 10 + the power's 
level + the psion's Intelligence modifier. \textbf{Maximum Power Level Known:} 
A psion begins play with the ability to learn 1st-level powers. As he attains higher 
levels, a psion may gain the ability to master more complex powers.

To learn or manifest a power, a psion must have an Intelligence score of at least 
10 + the power's level.

\textbf{Bonus Feats:} A psion gains a bonus feat at 1st level, 5th level, 10th 
level, 15th level, and 20th level. This feat must be a psionic feat, a metapsionic 
feat, or a psionic item creation feat.

These bonus feats are in addition to the feats that a character of any class gains 
every three levels. A psion is not limited to psionic feats, metapsionic feats, 
and psionic item creation feats when choosing these other feats.

\vspace{12pt}
\section*{PSIONIC DISCIPLINES}

A discipline is one of six groupings of powers, each defined by a common theme. 
The six disciplines are clairsentience, metacreativity, psychokinesis, psychometabolism, 
psychoportation, and telepathy.

\textbf{Clairsentience:} A psion who chooses clairsentience is known as a seer. 
Seers can learn precognitive powers to aid their comrades in combat, as well as 
powers that permit them to gather information in many different ways.

\textbf{Metacreativity:} A psion specializing in metacreativity is known as a shaper. 
This discipline includes powers that draw ectoplasm or matter from the Astral Plane, 
creating semisolid and solid items such as armor, weapons, or animated constructs 
to do battle at the shaper's command.

\textbf{Psychokinesis:} Psions who specialize in psychokinesis are known as kineticists. 
They are the masters of powers that manipulate and transform matter and energy. 
Kineticists can attack with devastating blasts of energy.

\textbf{Psychometabolism: }A psion who specializes in psychometabolism is known 
as an egoist. This discipline consists of powers that alter the psion's psychobiology, 
or that of creatures near him. An egoist can both heal and transform himself into 
a fearsome fighter.

\textbf{Psychoportation: }A psion who relies on psychoportation powers is known 
as a nomad. Nomads can wield powers that propel or displace objects in space or 
time.

\textbf{Telepathy:} A psion who chooses the discipline of telepathy is known as 
a telepath. He is the master of powers that allow mental contact and control of 
other sentient creatures. A telepath can deceive or destroy the minds of his enemies 
with ease.

\vspace{12pt}
PSICRYSTALS

A psicrystal is a fragment of a psionic character's personality, brought into physical 
form and a semblance of life (via the Psicrystal Affinity feat). A psicrystal appears 
as a crystalline construct about the size of a human hand.

Because it is an extension of its creator's personality, a character's psicrystal 
is in some ways a part of him. That's why, for example, a psionic character can 
manifest a personal range power on his psicrystal even though normally he can manifest 
such a power only on himself.

A psicrystal is treated as a construct for the purposes of all effects that depend 
on its type.

A psicrystal grants special abilities to its owner, as shown on the Psicrystal 
Special Abilities table below. In addition, a psicrystal has a personality (being 
a fragment of the owner's personality), which gives its owner a bonus on certain 
types of checks or saving throws, as given on the Psicrystal Personalities table 
below. These special abilities and bonuses apply only when the owner and the psicrystal 
are within 1 mile of each other.

Psicrystal abilities are based on the owner's levels in psionic classes. Levels 
from other classes do not count toward the owner's level for purposes of psicrystal 
abilities.

A psicrystal can speak one language of its owner's choice (so long as it is a language 
the owner knows). A psicrystal can understand all other languages known by its 
owner, but cannot speak them. This is a supernatural ability.

\textbf{Psicrystal Basics: }Use the statistics for a psicrystal, but make the following 
changes.

\textit{Saving Throws: }A psicrystal uses its owner's base saving throw bonuses 
and ability modifiers on saves, though it doesn't enjoy any other bonuses its owner 
might have (from magic items or feats, for example).

\textit{Abilities: }When its self-propulsion ability is not activated, a psicrystal 
has no Strength score and no Dexterity score. 

\textit{Skills: }A psicrystal has the same skill ranks as its owner, except that 
it has a minimum of 4 ranks each in Spot, Listen, Move Silently, and Search. (Even 
if its owner has no ranks in these skills, a psicrystal has 4 ranks in each.) A 
psicrystal uses its own ability modifiers on skill checks.

\vspace{12pt}
\begin{tabular}{|>{\raggedright}p{38pt}|>{\raggedright}p{47pt}|>{\raggedright}p{31pt}|>{\raggedright}p{185pt}|}
\hline
\multicolumn{4}{|p{302pt}|}{P\textbf{sicrystal Special Abilities}}\tabularnewline
\hline
O\textbf{wner Level} & N\textbf{atural Armor Adj.} & I\textbf{nt Adj.} & \section*{S\textbf{pecial}}\tabularnewline
\hline
1st-2nd & +0 & +0 & Alertness, improved evasion, personality, self-propulsion, 
share powers, sighted, telepathic link\tabularnewline
\hline
3rd-4th & +1 & +1 & Deliver touch powers\tabularnewline
\hline
5th-6th & +2 & +2 & Telepathic speech\tabularnewline
\hline
7th-8th & +3 & +3--- & \tabularnewline
\hline
9th-10th & +4 & +4 & Flight\tabularnewline
\hline
11th-12th & +5 & +5 & Power resistance\tabularnewline
\hline
13th-14th & +6 & +6 & Sight link\tabularnewline
\hline
15th-16th & +7 & +7 & Channel power\tabularnewline
\hline
17th-18th & +8 & +8 & \tabularnewline
\hline
19th-20th & +9 & +9 & \tabularnewline
\hline
\end{tabular}

\vspace{12pt}
\textbf{Psicrystal Ability Descriptions:} All psicrystals have special abilities 
(or impart abilities to their owners) depending on the level of the owner, as shown 
on the table above. The abilities on the table are cumulative.

\textit{Natural Armor Adj. (Ex): }This number noted here is an improvement to the 
psicrystal's natural armor bonus (normally 0). It represents a psicrystal's preternatural 
durability.

\textit{Intelligence Adj. (Ex): }Add this value to the psicrystal's Intelligence 
score. Psicrystals are as smart as people (though not necessarily as smart as smart 
people).

\textit{Alertness (Ex): }The presence of a psicrystal sharpens its master's senses. 
While a psicrystal is within arm's reach (adjacent to or in the same square as 
its owner), its owner gains the Alertness feat.

\textit{Improved Evasion (Ex): }If a psicrystal is subjected to an attack that 
normally allows a Reflex saving throw for half damage, it takes no damage if it 
makes a successful saving throw and half damage even if the saving throw fails.

\textit{Personality (Ex): }Every psicrystal has a personality. See Psicrystal Personality, 
below.

\textit{Self-Propulsion (Su): }As a standard action, its owner can will a psicrystal 
to form spidery, ectoplasmic legs that grant the psicrystal a land speed of 30 
feet and a climb speed of 20 feet. The legs fade into nothingness after one day 
(or sooner, if the owner desires).

\textit{Share Powers (Su): }At the owner's option, he can have any power (but not 
any psi-like ability) he manifests on himself also affect his psicrystal. The psicrystal 
must be within 5 feet of him at the time of the manifestation to receive the benefit. 
If the power has a duration other than instantaneous, it stops affecting the psicrystal 
if it moves farther than 5 feet away, and will not affect the psicrystal again, 
even if it returns to its owner before the duration expires.

Additionally, the owner can manifest a power with a target of ``You'' on his psicrystal 
(as a touch range power) instead of on himself. The owner and psicrystal cannot 
share powers if the powers normally do not affect creatures of the psicrystal's 
type (construct).

\textit{Sighted (Ex): }Although it has no physical sensory organs, a psicrystal 
can telepathically sense its environment as well as a creature with normal vision 
and hearing. Darkness (even supernatural darkness) is irrelevant, as are areas 
of supernatural silence, though a psicrystal still can't discern invisible or ethereal 
beings. A psicrystal's sighted range is 40 feet.

\textit{Telepathic Link (Su): }The owner has a telepathic link with his psicrystal 
out to a distance of up to 1 mile. The owner cannot see through the psicrystal's 
senses, but the two of them can communicate telepathically as if the psicrystal 
were the target of a \textit{mindlink }power manifested by the owner. For instance, 
a psicrystal placed in a distant room could relay the activities occurring in that 
room.

Because of the telepathic link between a psicrystal and its owner, the owner has 
the same connection to an item or place that the psicrystal does. For instance, 
if his psicrystal has seen a room, the owner can teleport into that room as if 
he has seen it too.

\textit{Deliver Touch Powers (Su): }If the owner is 3rd level or higher, his psicrystal 
can deliver touch powers for him. If the owner and psicrystal are in contact at 
the time the owner manifests a touch power, he can designate his psicrystal as 
the ``toucher.'' The psicrystal can then deliver the touch power just as the owner 
could. As usual, if the owner manifests another power before the touch is delivered, 
the touch power dissipates.

\textit{Telepathic Speech (Ex): }If the owner is 5th level or higher, the psicrystal 
can communicate telepathically with any creature that has a language and is within 
30 feet of the psicrystal, while the psicrystal is also within 1 mile of the owner.

\textit{Flight (Su): }If the owner is 9th level or higher, he can, as a standard 
action, will his psicrystal to fly at a speed of 50 feet (poor). The psicrystal 
drifts gently to the ground after one day (or sooner, if the owner desires).

\textit{Power Resistance (Ex): }If the owner is 11th level or higher, the psicrystal 
gains power resistance equal to the owner's level + 5. To affect the psicrystal 
with a power, another manifester must get a result on a manifester level check 
that equals or exceeds the psicrystal's power resistance.

\textit{Sight Link (Sp): }If the owner is 13th level or higher, the character can 
remote view the psicrystal (as if manifesting the \textit{remote view }power) once 
per day. 

\textit{Channel Power (Sp): }If the owner is 15th level or higher, he can manifest 
powers through the psicrystal to a distance of up to 1 mile. The psicrystal is 
treated as the power's originator, and all ranges are calculated from its location.

When channeling a power through his psicrystal, the owner manifests the power by 
paying its power point cost. He is still subject to attacks of opportunity and 
other hazards of manifesting a power, if applicable (for instance, he becomes visible 
when manifesting an offensive power if \textit{invisible}, as does the psicrystal).

\vspace{12pt}
\textbf{Psicrystal Personality (Ex):} Each psicrystal has a distinct personality, 
chosen by its owner at the time of its creation from among those given on the following 
table. At 1st level, its owner typically gets a feel for a psicrystal's personality 
only through occasional impulses, but as the owner increases in level the psicrystal's 
personality becomes more pronounced. At higher levels, it is not uncommon for a 
psicrystal to constantly ply its owner with observations and advice, often severely 
slanted toward the psicrystal's particular worldview. The owner always sees a bit 
of himself in his psicrystal, even if magnified and therefore distorted.

\vspace{12pt}
\begin{tabular}{|>{\raggedright}p{62pt}|>{\raggedright}p{180pt}|}
\hline
\multicolumn{2}{|p{242pt}|}{P\textbf{sicrystal Personalities}}\tabularnewline
\hline
P\textbf{ersonality} & B\textbf{enefit to Owner}\tabularnewline
\hline
Artiste & +3 bonus on Craft checks\tabularnewline
\hline
Bully & +3 bonus on Intimidate checks\tabularnewline
\hline
Coward & +3 bonus on Hide checks\tabularnewline
\hline
Friendly & +3 bonus on Diplomacy checks\tabularnewline
\hline
Hero & +2 bonus on Fortitude saves\tabularnewline
\hline
Liar & +3 bonus on Bluff checks\tabularnewline
\hline
Meticulous & +3 bonus on Search checks\tabularnewline
\hline
Nimble & +2 bonus on Initiative checks\tabularnewline
\hline
Observant & +3 bonus on Spot checks\tabularnewline
\hline
Poised & +3 bonus on Balance checks\tabularnewline
\hline
Resolved & +2 bonus on Will saves\tabularnewline
\hline
Sage & +3 bonus on checks involving any one Knowledge skill owner already knows; 
once chosen, this does not vary\tabularnewline
\hline
Single-minded & +3 bonus on Concentration checks\tabularnewline
\hline
Sneaky & +3 bonus on Move Silently checks\tabularnewline
\hline
Sympathetic & +3 bonus on Sense Motive checks\tabularnewline
\hline
\end{tabular}

\vspace{12pt}
{\LARGE{}PSYCHIC WARRIOR}

\textbf{Alignment:} Any.

\textbf{Hit Die:} d8.

\vspace{12pt}
\section*{\textbf{Class Skills}}

The psychic warrior's class skills (and the key ability for each skill) are Autohypnosis* 
(Wis), Climb (Str), Concentration* (Con), Craft (Int), Jump (Str), Knowledge (psionics)* 
(Int), Profession (Wis), Ride (Dex), Search (Int), and Swim (Str).

*New skill or expanded use of existing skill.

\textbf{Skill Points at 1st Level:} (2 + Int modifier) x4.

\textbf{Skill Points at Each Additional Level:} 2 + Int modifier.

\vspace{12pt}
\begin{tabular}{|>{\raggedright}p{22pt}|>{\raggedright}p{36pt}|>{\raggedright}p{17pt}|>{\raggedright}p{14pt}|>{\raggedright}p{14pt}|>{\raggedright}p{38pt}|>{\raggedright}p{34pt}|>{\raggedright}p{22pt}|>{\raggedright}p{39pt}|}
\hline
\multicolumn{9}{|p{242pt}|}{T\textbf{able: The Psychic Warrior}}\tabularnewline
\hline
L\textbf{evel} & B\textbf{ase Attack Bonus } & F\textbf{ort Save} & R\textbf{ef 
Save} & W\textbf{ill Save} & S\textbf{pecial} & P\textbf{ower Points/Day} & P\textbf{owers 
Known} & M\textbf{aximum Power Level Known}\tabularnewline
\hline
1st & +0 & +2 & +0 & +0 & Bonus feat & 0* & 1 & 1st\tabularnewline
\hline
2nd & +1 & +3 & +0 & +0 & Bonus feat & 1 & 2 & 1st\tabularnewline
\hline
3rd & +2 & +3 & +1 & +1--- &  & 3 & 3 & 1st\tabularnewline
\hline
4th & +3 & +4 & +1 & +1--- &  & 5 & 4 & 2nd\tabularnewline
\hline
5th & +3 & +4 & +1 & +1 & Bonus feat & 7 & 5 & 2nd\tabularnewline
\hline
6th & +4 & +5 & +2 & +2--- &  & 11 & 6 & 2nd\tabularnewline
\hline
7th & +5 & +5 & +2 & +2--- &  & 15 & 7 & 3rd\tabularnewline
\hline
8th & +6/+1 & +6 & +2 & +2 & Bonus feat & 19 & 8 & 3rd\tabularnewline
\hline
9th & +6/+1 & +6 & +3 & +3--- &  & 23 & 9 & 3rd\tabularnewline
\hline
10th & +7/+2 & +7 & +3 & +3--- &  & 27 & 10 & 4th\tabularnewline
\hline
11th & +8/+3 & +7 & +3 & +3 & Bonus feat & 35 & 11 & 4th\tabularnewline
\hline
12th & +9/+4 & +8 & +4 & +4--- &  & 43 & 12 & 4th\tabularnewline
\hline
13th & +9/+4 & +8 & +4 & +4--- &  & 51 & 13 & 5th\tabularnewline
\hline
14th & +10/+5 & +9 & +4 & +4 & Bonus feat & 59 & 14 & 5th\tabularnewline
\hline
15th & +11/+6/+1 & +9 & +5 & +5--- &  & 67 & 15 & 5th\tabularnewline
\hline
16th & +12/+7/+2 & +10 & +5 & +5--- &  & 79 & 16 & 6th\tabularnewline
\hline
17th & +12/+7/+2 & +10 & +5 & +5 & Bonus feat & 91 & 17 & 6th\tabularnewline
\hline
18th & +13/+8/+3 & +11 & +6 & +6--- &  & 103 & 18 & 6th\tabularnewline
\hline
19th & +14/+9/+4 & +11 & +6 & +6--- &  & 115 & 19 & 6th\tabularnewline
\hline
20th & +15/+10/+5 & +12 & +6 & +6 & Bonus feat & 127 & 20 & 6th\tabularnewline
\hline
\multicolumn{9}{|p{242pt}|}{*The psychic warrior gains no power points from his 
class at 1st level. However, he does add any bonus power points he gains from a 
high Wisdom score, his race, and feats or other sources to his reserve. He can 
use these points (if any) to manifest his power.}\tabularnewline
\hline
\end{tabular}

\vspace{12pt}
\section*{\textbf{Class Features}}

All the following are class features of the psychic warrior.

\textbf{Weapon and Armor Proficiency:} Psychic warriors are proficient with all 
simple and martial weapons, with all types of armor (heavy, medium, and light), 
and with shields (except tower shields).

\textbf{Power Points/Day: }A psychic warrior's ability to manifest powers is limited 
by the power points he has available. His base daily allotment of power points 
is given on Table: The Psychic Warrior. In addition, he receives bonus power points 
per day if he has a high Wisdom score (see Table: Ability Modifiers and Bonus Power 
Points). His race may also provide bonus power points per day, as may certain feats 
and items. A 1st-level psychic warrior gains no power points for his class level, 
but he gains bonus power points (if he is entitled to any), and can manifest the 
single power he knows with those power points.

\textbf{Powers Known: }A psychic warrior begins play knowing one psychic warrior 
power of your choice. Each time he achieves a new level, he unlocks the knowledge 
of a new power.

Choose the powers known from the psychic warrior power list. (\textit{Exception: 
}The feats Expanded Knowledge and Epic Expanded Knowledge do allow a psychic warrior 
to learn powers from the lists of other classes.) A psychic warrior can manifest 
any power that has a power point cost equal to or lower than his manifester level.

The total number of powers a psychic warrior can manifest in a day is limited only 
by his daily power points.

A psychic warrior simply knows his powers; they are ingrained in his mind. He does 
not need to prepare them (in the way that some spellcasters prepare their spells), 
though he must get a good night's sleep each day to regain all his spent power 
points.

The Difficulty Class for saving throws against psychic warrior powers is 10 + the 
power's level + the psychic warrior's Wisdom modifier.

\textbf{Maximum Power Level Known:} A psychic warrior begins play with the ability 
to learn 1st-level powers. As he attains higher levels, he may gain the ability 
to master more complex powers.

To learn or manifest a power, a psychic warrior must have a Wisdom score of at 
least 10 + the power's level.

\textbf{Bonus Feats: }At 1st level, a psychic warrior gets a bonus combat-oriented 
feat in addition to the feat that any 1st level character gets and the bonus feat 
granted to a human character. The psychic warrior gains an additional bonus feat 
at 2nd level and every three levels thereafter (5th, 8th, 11th, 14th, 17th, and 
20th). These bonus feats must be drawn from the feats noted as fighter bonus feats 
or psionic feats. The psychic warrior must still meet all prerequisites for the 
bonus feat, including ability score and base attack bonus minimums as well as class 
requirements. A psychic warrior cannot choose feats that specifically require levels 
in the fighter class unless he is a multiclass character with the requisite levels 
in the fighter class.

These bonus feats are in addition to the feats that a character of any class gains 
every three levels. A psychic warrior is not limited to fighter bonus feats and 
psionic feats when choosing these other feats.

\vspace{12pt}
{\LARGE{}SOULKNIFE}

\textbf{Alignment:} Any.

\textbf{Hit Die:} d10.

\vspace{12pt}
\section*{\textbf{Class Skills}}

The soulknife's class skills (and the key ability for each skill) are Autohypnosis* 
(Wis), Climb (Str), Concentration* (Con), Craft (Int), Hide (Dex), Jump (Str), 
Knowledge (psionics)* (Int), Listen (Wis), Move Silently (Dex), Profession (Wis), 
Spot (Wis), and Tumble (Dex).

*New skill or expanded use of existing skill.

\textbf{Skill Points at 1st Level:} (4 + Int modifier) x4.

\textbf{Skill Points at Each Additional Level:} 4 + Int modifier.

\vspace{12pt}
\begin{tabular}{|>{\raggedright}p{18pt}|>{\raggedright}p{35pt}|>{\raggedright}p{16pt}|>{\raggedright}p{17pt}|>{\raggedright}p{23pt}|>{\raggedright}p{167pt}|}
\hline
\multicolumn{6}{|p{278pt}|}{T\textbf{able: The Soulknife}}\tabularnewline
\hline
L\textbf{evel} & B\textbf{ase Attack Bonus} & F\textbf{ort Save} & R\textbf{ef 
Save} & W\textbf{ill Save} & S\textbf{pecial}\tabularnewline
\hline
1st & +0 & +0 & +2 & +2 & Mind blade, Weapon Focus (mind blade), Wild Talent\tabularnewline
\hline
2nd & +1 & +0 & +3 & +3 & Throw mind blade\tabularnewline
\hline
3rd & +2 & +1 & +3 & +3 & Psychic strike +1d8\tabularnewline
\hline
4th & +3 & +1 & +4 & +4 & +\textit{1 mind blade}\tabularnewline
\hline
5th & +3 & +1 & +4 & +4 & Free draw, shape mind blade\tabularnewline
\hline
6th & +4 & +2 & +5 & +5 & Mind blade enhancement +1, Speed of Thought\tabularnewline
\hline
7th & +5 & +2 & +5 & +5 & Psychic strike +2d8\tabularnewline
\hline
8th & +6/+1 & +2 & +6 & +6 & +\textit{2 mind blade}\tabularnewline
\hline
9th & +6/+1 & +3 & +6 & +6 & Bladewind, Greater Weapon Focus (mind blade)\tabularnewline
\hline
10th & +7/+2 & +3 & +7 & +7 & Mind blade enhancement +2\tabularnewline
\hline
11th & +8/+3 & +3 & +7 & +7 & Psychic strike +3d8\tabularnewline
\hline
12th & +9/+4 & +4 & +8 & +8 & +\textit{3 mind blade}\tabularnewline
\hline
13th & +9/+4 & +4 & +8 & +8 & Knife to the soul\tabularnewline
\hline
14th & +10/+5 & +4 & +9 & +9 & Mind blade enhancement +3\tabularnewline
\hline
15th & +11/+6 & +5 & +9 & +9 & Psychic strike +4d8\tabularnewline
\hline
16th & +12/+7 & +5 & +10 & +10 & +\textit{4 mind blade}\tabularnewline
\hline
17th & +12/+7 & +5 & +10 & +10 & Multiple throw\tabularnewline
\hline
18th & +13/+8 & +6 & +11 & +11 & Mind blade enhancement +4\tabularnewline
\hline
19th & +14/+9 & +6 & +11 & +11 & Psychic strike +5d8\tabularnewline
\hline
20th & +15/+10/+5 & +6 & +12 & +12 & +\textit{5 mind blade}\tabularnewline
\hline
\end{tabular}

\vspace{12pt}
\section*{\textbf{Class Features}}

All the following are class features of the soulknife.

\textbf{Weapon and Armor Proficiency:} Soulknives are proficient with all simple 
weapons, with their own mind blades, and with light armor and shields (except tower 
shields).

\textbf{Mind Blade (Su):} As a move action, a soulknife can create a semisolid 
blade composed of psychic energy distilled from his own mind. The blade is identical 
in all ways (except visually) to a short sword of a size appropriate for its wielder. 
For instance, a Medium soulknife materializes a Medium mind blade that he can wield 
as a light weapon, and the blade deals 1d6 points of damage (crit 19-20/x2). Soulknives 
who are smaller or larger than Medium create mind blades identical to short swords 
appropriate for their size, with a corresponding change to the blade's damage. 
The wielder of a mind blade gains the usual benefits to his attack roll and damage 
roll from a high Strength bonus.

The blade can be broken (it has hardness 10 and 10 hit points); however, a soulknife 
can simply create another on his next move action. The moment he relinquishes his 
grip on his blade, it dissipates (unless he intends to throw it; see below). A 
mind blade is considered a magic weapon for the purpose of overcoming damage reduction.

A soulknife can use feats such as Power Attack or Combat Expertise in conjunction 
with the mind blade just as if it were a normal weapon. He can also choose mind 
blade for feats requiring a specific weapon choice, such as Weapon Specialization. 
Powers or spells that upgrade weapons can be used on a mind blade.

A soulknife's mind blade improves as the character gains higher levels. At 4th 
level and every four levels thereafter, the mind blade gains a cumulative +1 enhancement 
bonus on attack rolls and damage rolls (+2 at 8th level, +3 at 12th level, +4 at 
16th level, and +5 at 20th level).

Even in places where psionic effects do not normally function (such as within a 
\textit{null psionics field}), a soulknife can attempt to sustain his mind blade 
by making a DC 20 Will save. On a successful save, the soulknife maintains his 
mind blade for a number of rounds equal to his class level before he needs to check 
again. On an unsuccessful attempt, the mind blade vanishes. As a move action on 
his turn, the soulknife can attempt a new Will save to rematerialize his mind blade 
while he remains within the psionics negating effect.

\textbf{Weapon Focus (Mind Blade): }A soulknife gains Weapon Focus (mind blade) 
as a bonus feat.

\textbf{Wild Talent: }A soulknife gains Wild Talent as a bonus feat. (This class 
feature provides the character with the psionic power he needs to materialize his 
mind blade, if he has no power points otherwise.)

\textbf{Throw Mind Blade (Ex):} A soul knife of 2nd level or higher can throw his 
mind blade as a ranged weapon with a range increment of 30 feet.

Whether or not the attack hits, a thrown mind blade then dissipates. A soulknife 
of 3rd level or higher can make a psychic strike (see below) with a thrown mind 
blade and can use the blade in conjunction with other special abilities (such as 
Knife to the Soul; see below).

\textbf{Psychic Strike (Su):} As a move action, a soulknife of 3rd level or higher 
can imbue his mind blade with destructive psychic energy. This effect deals an 
extra 1d8 points of damage to the next living, nonmindless target he successfully 
hits with a melee attack (or ranged attack, if he is using the throw mind blade 
ability). Creatures immune to mind-affecting effects are immune to psychic strike 
damage. (Unlike the rogue's sneak attack, the psychic strike is not precision damage 
and can affect creatures otherwise immune to extra damage from critical hits or 
more than 30 feet away, provided they are living, nonmindless creatures not immune 
to mind-affecting effects.)

A mind blade deals this extra damage only once when this ability is called upon, 
but a soulknife can imbue his mind blade with psychic energy again by taking another 
move action.

Once a soulknife has prepared his blade for a psychic strike, it holds the extra 
energy until it is used. Even if the soulknife drops the mind blade (or it otherwise 
dissipates, such as when it is thrown and misses), it is still imbued with psychic 
energy when the soulknife next materializes it.

At every four levels beyond 3rd (7th, 11th, 15th, and 19th), the extra damage from 
a soulknife's psychic strike increases as shown on the Table above.

\textbf{Free Draw (Su):} At 5th level, a soulknife becomes able to materialize 
his mind blade as a free action instead of a move action. He can make only one 
attempt to materialize the mind blade per round, however.

\textbf{Shape Mind Blade (Su):} At 5th level, a soulknife gains the ability to 
change the form of his mind blade. As a fullround action, he can change his mind 
blade to replicate a longsword (damage 1d8 for a Medium weapon wielded as a one-handed 
weapon) or a bastard sword (damage 1d10 for a Medium weapon, but he must wield 
it as a two-handed weapon unless he knows the Exotic Weapon Proficiency (bastard 
sword) feat). If a soulknife shapes his mind blade into the form of a bastard sword 
and wields it two-handed, he adds 1-1/2 times his Strength bonus to his damage 
rolls, just like when using any other two-handed weapon.

Alternatively, a soulknife can split his mind blade into two identical short swords, 
suitable for fighting with a weapon in each hand. (The normal penalties for fighting 
with two weapons apply.) However, both mind blades have an enhancement bonus 1 
lower than the soulknife would otherwise create with a single mind blade.

\textbf{Mind Blade Enhancement (Su): }At 6th level, a soulknife gains the ability 
to enhance his mind blade. He can add any one of the weapon special abilities on 
the table below that has an enhancement bonus value of +1.

At every four levels beyond 6th (10th, 14th, and 18th), the value of the enhancement 
a soulknife can add to his weapon improves to +2, +3, and +4, respectively. A soulknife 
can choose any combination of weapon special abilities that does not exceed the 
total allowed by the soulknife's level.

The weapon ability or abilities remain the same every time the soulknife materializes 
his mind blade (unless he decides to reassign its abilities; see below). The ability 
or abilities apply to any form the mind blade takes, including the use of the shape 
mind blade or bladewind class abilities.

\vspace{12pt}
\begin{tabular}{|>{\raggedright}p{99pt}|>{\raggedright}p{114pt}|}
\hline
W\textbf{eapon Special Ability} & E\textbf{nhancement Bonus Value}\tabularnewline
\hline
Defending & +1\tabularnewline
\hline
Keen & +1\tabularnewline
\hline
Lucky* & +1\tabularnewline
\hline
Mighty cleaving & +1\tabularnewline
\hline
Psychokinetic* & +1\tabularnewline
\hline
Sundering* & +1\tabularnewline
\hline
Vicious & +1\tabularnewline
\hline
Collision* & +2\tabularnewline
\hline
Mindcrusher* & +2\tabularnewline
\hline
Psychokinetic burst* & +2\tabularnewline
\hline
Suppression* & +2\tabularnewline
\hline
Wounding & +2\tabularnewline
\hline
Bodyfeeder* & +3\tabularnewline
\hline
Mindfeeder* & +3\tabularnewline
\hline
Soulbreaker* & +3\tabularnewline
\hline
\multicolumn{2}{|p{213pt}|}{*New special abilities}\tabularnewline
\hline
\end{tabular}

\vspace{12pt}
A soulknife can reassign the ability or abilities he has added to his mind blade. 
To do so, he must first spend 8 hours in concentration. After that period, the 
mind blade materializes with the new ability or abilities selected by the soulknife.

\textbf{Speed of Thought: }A soulknife gains Speed of Thought as a bonus feat at 
6th level.

\textbf{Bladewind (Su):} At 9th level, a soulknife gains the ability to momentarily 
fragment his mind blade into numerous identical blades, each of which strikes at 
a nearby opponent.

As a full attack, when wielding his mind blade, a soulknife can give up his regular 
attacks and instead fragment his mind blade to make one melee attack at his full 
base attack bonus against each opponent within reach. Each fragment functions identically 
to the soulknife's regular mind blade.

When using bladewind, a soulknife forfeits any bonus or extra attacks granted by 
other feats or abilities (such as the Cleave feat or the \textit{haste }spell).

The mind blade immediately reverts to its previous form after the bladewind attack.

\textbf{Greater Weapon Focus (Mind Blade):} A soulknife gains Greater Weapon Focus 
(mind blade) as a bonus feat at 9th level.

\textbf{Knife to the Soul (Su):} Beginning at 13th level, when a soulknife executes 
a psychic strike, he can choose to substitute Intelligence, Wisdom, or Charisma 
damage (his choice) for extra dice of damage. For each die of extra damage he gives 
up, he deals 1 point of damage to the ability score he chooses. A soulknife can 
combine extra dice of damage and ability damage in any combination.

The soulknife decides which ability score his psychic strike damages and the division 
of ability damage and extra dice of damage when he imbues his mind blade with the 
psychic strike energy.

\textbf{Multiple Throw (Ex): }At 17th level and higher, a soulknife can throw a 
number of mind blades per round equal to the number of melee attacks he could make.

\vspace{12pt}
{\LARGE{}WILDER}

\textbf{Alignment:} Any.

\textbf{Hit Die:} d6.

\vspace{12pt}
\section*{\textbf{Class Skills}}

The wilder's class skills (and the key ability for each skill) are Autohypnosis* 
(Wis), Balance (Dex), Bluff (Cha), Climb (Str), Concentration* (Con), Craft (Int), 
Escape Artist (Dex), Intimidate (Cha), Jump (Str), Knowledge (psionics)* (Int), 
Listen (Wis), Profession (Wis), Psicraft* (Int), Sense Motive (Wis), Spot (Wis), 
Swim (Str), and Tumble (Dex).

*New skill or expanded use of existing skill.

\textbf{Skill Points at 1st Level: }(4 + Int modifier) x4.

\textbf{Skill Points at Each Additional Level:} 4 + Int modifier.

\vspace{12pt}
\begin{tabular}{|>{\raggedright}p{14pt}|>{\raggedright}p{28pt}|>{\raggedright}p{14pt}|>{\raggedright}p{12pt}|>{\raggedright}p{12pt}|>{\raggedright}p{74pt}|>{\raggedright}p{24pt}|>{\raggedright}p{17pt}|>{\raggedright}p{44pt}|}
\hline
\multicolumn{9}{|p{242pt}|}{T\textbf{able: The Wilder}}\tabularnewline
\hline
L\textbf{evel } & B\textbf{ase Attack Bonus} & F\textbf{ort Save} & R\textbf{ef 
Save} & W\textbf{ill Save} & S\textbf{pecial} & P\textbf{ower Points/Day} & P\textbf{owers 
Known} & M\textbf{aximum Power Level Known}\tabularnewline
\hline
1st & +0 & +0 & +0 & +2 & Wild surge +1, psychic enervation & 2 & 1 & 1st\tabularnewline
\hline
2nd & +1 & +0 & +0 & +3 & Elude touch & 6 & 2 & 1st\tabularnewline
\hline
3rd & +2 & +1 & +1 & +3 & Wild surge +2 & 11 & 2 & 1st\tabularnewline
\hline
4th & +3 & +1 & +1 & +4 & Surging euphoria +1 & 17 & 3 & 2nd\tabularnewline
\hline
5th & +3 & +1 & +1 & +4 & Volatile mind (1 power point) & 25 & 3 & 2nd\tabularnewline
\hline
6th & +4 & +2 & +2 & +5 &  & 35 & 4 & 3rd\tabularnewline
\hline
7th & +5 & +2 & +2 & +5 & Wild surge +3 & 46 & 4 & 3rd\tabularnewline
\hline
8th & +6/+1 & +2 & +2 & +6 &  & 58 & 5 & 4th\tabularnewline
\hline
9th & +6/+1 & +3 & +3 & +6 & Volatile mind (2 power points) & 72 & 5 & 4th\tabularnewline
\hline
10th & +7/+2 & +3 & +3 & +7 &  & 88 & 6 & 5th\tabularnewline
\hline
11th & +8/+3 & +3 & +3 & +7 & Wild surge +4 & 106 & 6 & 5th\tabularnewline
\hline
12th & +9/+4 & +4 & +4 & +8 & Surging euphoria +2 & 126 & 7 & 6th\tabularnewline
\hline
13th & +9/+4 & +4 & +4 & +8 & Volatile mind (3 power points) & 147 & 7 & 6th\tabularnewline
\hline
14th & +10/+5 & +4 & +4 & +9 &  & 170 & 8 & 7th\tabularnewline
\hline
15th & +11/+6/+1 & +5 & +5 & +9 & Wild surge +5 & 195 & 8 & 7th\tabularnewline
\hline
16th & +12/+7/+2 & +5 & +5 & +10 &  & 221 & 9 & 8th\tabularnewline
\hline
17th & +12/+7/+2 & +5 & +5 & +10 & Volatile mind (4 power points) & 250 & 9 & 8th\tabularnewline
\hline
18th & +13/+8/+3 & +6 & +6 & +11 &  & 280 & 10 & 9th\tabularnewline
\hline
19th & +14/+9/+4 & +6 & +6 & +11 & Wild surge +6 & 311 & 10 & 9th\tabularnewline
\hline
20th & +15/+10/+5 & +6 & +6 & +12 & Surging euphoria +3 & 343 & 11 & 9th\tabularnewline
\hline
\end{tabular}

\vspace{12pt}
\section*{\textbf{Class Features}}

All the following are class features of the wilder.

\textbf{Weapon and Armor Proficiency: }Wilders are proficient with all simple weapons, 
with light armor, and with shields (except tower shields).

\textbf{Power Points/Day:} A wilder's ability to manifest powers is limited by 
the power points she has available. Her base daily allotment of power points is 
given on Table: The Wilder. In addition, she receives bonus power points per day 
if she has a high Charisma score (see Table: Ability Modifiers and Bonus Power 
Points). Her race may also provide bonus power points per day, as may certain feats 
and items.

\textbf{Powers Known:} A wilder begins play knowing one wilder power of your choice. 
At every even-numbered class level after 1st, she unlocks the knowledge of new 
powers.

Choose the powers known from the wilder power list. (\textit{Exception: }The feats 
Expanded Knowledge and Epic Expanded Knowledge do allow a wilder to learn powers 
from the lists of other classes.) A wilder can manifest any power that has a power 
point cost equal to or lower than her manifester level.

The total number of powers a wilder can manifest in a day is limited only by her 
daily power points.

A wilder simply knows her powers; they are ingrained in her mind. She does not 
need to prepare them (in the way that some spellcasters prepare their spells), 
though she must get a good night's sleep each day to regain all her spent power 
points.

The Difficulty Class for saving throws against wilder powers is 10 + the power's 
level + the wilder's Charisma modifier.

\textbf{Maximum Power Level Known:} A wilder begins play with the ability to learn 
1st-level powers. As she attains higher levels, she may gain the ability to master 
more complex powers.

To learn or manifest a power, a wilder must have a Charisma score of at least 10 
+ the power's level.

\textbf{Wild Surge (Su)}: A wilder can let her passion and emotion rise to the 
surface in a wild surge when she manifests a power. During a wild surge, a wilder 
gains phenomenal psionic strength, but may harm herself by the reckless use of 
her power (see Psychic Enervation, below). 

A wilder can choose to invoke a wild surge whenever she manifests a power. When 
she does so, she gains +1 to her manifester level with that manifestation of the 
power. The manifester level boost gives her the ability to augment her powers to 
a higher degree than she otherwise could; however, she pays no extra power point 
for this wild surge. Instead, the additional 1 power point that would normally 
be required to augment the power is effectively supplied by the wild surge.

Level-dependent power effects are also improved, depending on the power a wilder 
manifests with her wild surge.

This improvement in manifester level does not grant her any other benefits (psicrystal 
abilities do not advance, she does not gain higher-level class abilities, and so 
on).

She cannot use the Overchannel psionic feat and invoke her wild surge at the same 
time.

At 3rd level, a wilder can choose to boost her manifester level by two instead 
of one. At 7th level, she can boost her manifester level by up to three; at 11th 
level, by up to four; at 15th level, by up to five; and at 19th level, by up to 
six.

In all cases, the wild surge effectively pays the extra power point cost that is 
normally required to augment the power; only the unaugmented power point cost is 
subtracted from the wilder's power point reserve.

\textbf{Psychic Enervation (Ex):} Pushing oneself by invoking a wild surge is dangerous. 
Immediately following each wild surge, a wilder may be overcome by the strain of 
her effort. The chance of suffering psychic enervation is equal to 5\% per manifester 
level added with the wild surge.

A wilder who is overcome by psychic enervation is dazed until the end of her next 
turn and loses a number of power points equal to her wilder level.

\textbf{Elude Touch (Ex):} Starting at 2nd level, a wilder's intuition supersedes 
her intellect, alerting her to danger from touch attacks (including rays). She 
gains a bonus to Armor Class against all touch attacks equal to her Charisma bonus; 
however, her touch AC can never exceed her Armor Class against normal attacks.

\textbf{Surging Euphoria (Ex):} Starting at 4th level, when a wilder uses her wild 
surge ability, she gains a +1 morale bonus on attack rolls, damage rolls, and saving 
throws for a number of rounds equal to the intensity of her wild surge. 

If a wilder is overcome by psychic enervation following her wild surge, she does 
not gain the morale bonus for this use of her wild surge ability.

At 12th level, the morale bonus on a wilder's attack rolls, damage rolls, and saving 
throws increases to +2. At 20th level, the bonus increases to +3.

\textbf{Volatile Mind (Ex): }A wilder's temperamental mind is hard to encompass 
with the discipline of telepathy. When any telepathy power is manifested on a wilder 
of 5th level or higher, the manifester of the power must pay 1 power point more 
than he otherwise would have spent.

The extra cost is not a natural part of that power's cost. It does not augment 
the power; it is simply a wasted power point. The wilder's volatile mind can force 
the manifester of the telepathy power to exceed the normal power point limit of 
1 point per manifester level. If the extra cost raises the telepathy power's cost 
to more points than the manifester has remaining in his reserve, the power simply 
fails, and the manifester exhausts the rest of his power points.

At 9th level, the penalty assessed against telepathy powers manifested on a wilder 
is increased to 2 power points. At 13th level, the penalty increases to 3 power 
points, and at 17th level it increases to 4 power points.

As a standard action, a wilder can choose to lower this effect for 1 round.

\vspace{12pt}
{\LARGE{}PSIONIC PRESTIGE CLASSES}

\vspace{12pt}
{\LARGE{}CEREBREMANCER}

\textbf{Hit Die:} d4.

\vspace{12pt}
\section*{\textbf{Requirements}}

To qualify to become a cerebremancer, a character must fulfill all the following 
criteria.

\textbf{Skills:} Knowledge (arcana) 6 ranks, Knowledge (psionics) 6 ranks.

\textbf{Spells:} Able to cast 2nd-level arcane spells.

\textbf{Psionics:} Able to manifest 2nd-level powers.

\vspace{12pt}
\section*{\textbf{Class Skills}}

The cerebremancer's class skills (and the key ability for each skill) are Concentration 
(Con), Craft (Int), Decipher Script (Int), Knowledge (arcana) Int, Knowledge (psionics) 
(Int), Profession (Wis), Psicraft (Int), and Spellcraft (Int).

\textbf{Skill Points at Each Level:} 2 + Int modifier.

\vspace{12pt}
\begin{tabular}{|>{\raggedright}p{15pt}|>{\raggedright}p{19pt}|>{\raggedright}p{13pt}|>{\raggedright}p{14pt}|>{\raggedright}p{13pt}|>{\raggedright}p{202pt}|}
\hline
\multicolumn{6}{|p{278pt}|}{T\textbf{able: The Cerebremancer}}\tabularnewline
\hline
\section*{L\textbf{evel}} & \section*{B\textbf{ase Attack Bonus}} & \section*{F\textbf{ort 
Save}} & \section*{R\textbf{ef Save}} & \section*{W\textbf{ill Save}} & \section*{S\textbf{pells 
per Day/Powers Known}}\tabularnewline
\hline
1st & +0 & +0 & +0 & +2 & +1 level of existing arcane spellcasting class/+1 level 
of existing manifesting class\tabularnewline
\hline
2nd & +1 & +0 & +0 & +3 & +1 level of existing arcane spellcasting class/+1 level 
of existing manifesting class\tabularnewline
\hline
3rd & +1 & +1 & +1 & +3 & +1 level of existing arcane spellcasting class/+1 level 
of existing manifesting class\tabularnewline
\hline
4th & +2 & +1 & +1 & +4 & +1 level of existing arcane spellcasting class/+1 level 
of existing manifesting class\tabularnewline
\hline
5th & +2 & +1 & +1 & +4 & +1 level of existing arcane spellcasting class/+1 level 
of existing manifesting class\tabularnewline
\hline
6th & +3 & +2 & +2 & +5 & +1 level of existing arcane spellcasting class/+1 level 
of existing manifesting class\tabularnewline
\hline
7th & +3 & +2 & +2 & +5 & +1 level of existing arcane spellcasting class/+1 level 
of existing manifesting class\tabularnewline
\hline
8th & +4 & +2 & +2 & +6 & +1 level of existing arcane spellcasting class/+1 level 
of existing manifesting class\tabularnewline
\hline
9th & +4 & +3 & +3 & +6 & +1 level of existing arcane spellcasting class/+1 level 
of existing manifesting class\tabularnewline
\hline
10th & +5 & +3 & +3 & +7 & +1 level of existing arcane spellcasting class/+1 level 
of existing manifesting class\tabularnewline
\hline
\end{tabular}

\vspace{12pt}
\textbf{Class Features}

All the following are class features of the cerebremancer prestige class.

\textbf{Weapon and Armor Proficiency:} Cerebremancers gain no proficiency with 
any weapon or armor.

\textbf{Spells per Day/Powers Known:} When a new cerebremancer level is attained, 
the character gains new spells per day as if he had also attained a level in any 
one arcane spellcasting class he belonged to before he added the prestige class. 
He gains additional power points per day and access to new powers as if he had 
also gained a level in any one manifesting class he belonged to previously. He 
does not, however, gain any other benefit a character of either class would have 
gained (bonus metamagic, metapsionic, or item creation feats, psicrystal special 
abilities, and so on). This essentially means that he adds the level of cerebremancer 
to the level of whatever other arcane spellcasting class and manifesting class 
the character has, then determines spells per day, caster level, power points per 
day, powers known, and manifester level accordingly.

If a character had more than one arcane spellcasting class or more than one manifesting 
class before he became a cerebremancer, he must decide to which class he adds each 
level of cerebremancer for purpose of determining spells per day, caster level, 
power points per day, powers known, and manifester level.

\vspace{12pt}
{\LARGE{}ELOCATER}

\textbf{Hit Die:} d6.

\vspace{12pt}
\section*{\textbf{Requirements}}

To qualify to become an elocater, a character must fulfill all the following criteria.

\textbf{Base Attack Bonus:} +3.

\textbf{Skills:} Concentration 8 ranks.

\textbf{Feats:} Mobility, Spring Attack.

\textbf{Psionics:} Able to manifest 1st-level powers.

\vspace{12pt}
\section*{\textbf{Class Skills}}

The elocater's class skills (and the key ability for each skill) are Autohypnosis 
(Wis), Balance (Dex), Climb (Str), Concentration (Con), Craft (Int), Disable Device 
(Int), Escape Artist (Dex), Gather Information (Cha), Hide (Dex), Jump (Str), Knowledge 
(local) (Int), Knowledge (psionics), Listen (Wis), Move Silently (Dex), Open Lock 
(Dex), Perform (Cha), Profession (Wis), Psicraft (Int), Search (Int), Sense Motive 
(Wis), Sleight of Hand (Dex), Spot (Wis), Swim (Str), Tumble (Dex), Use Psionic 
Device (Cha), and Use Rope (Dex).

\textbf{Skill Points at Each Level: }6 + Int modifier.

\vspace{12pt}
\begin{tabular}{|>{\raggedright}p{16pt}|>{\raggedright}p{19pt}|>{\raggedright}p{17pt}|>{\raggedright}p{15pt}|>{\raggedright}p{17pt}|>{\raggedright}p{79pt}|>{\raggedright}p{100pt}|}
\hline
\multicolumn{7}{|p{266pt}|}{\section*{T\textbf{able: The Elocater}}}\tabularnewline
\hline
L\textbf{evel} & B\textbf{ase Attack Bonus} & F\textbf{ort Save} & R\textbf{ef 
Save} & W\textbf{ill Save} & S\textbf{pecial} & P\textbf{owers Known}\tabularnewline
\hline
1st & +0 & +0 & +2 & +2 & Scorn earth, Sidestep Charge & +1 level of existing manifesting 
class\tabularnewline
\hline
2nd & +1 & +0 & +3 & +3 & Opportunistic strike +2--- & \tabularnewline
\hline
3rd & +2 & +1 & +3 & +3 & Dimension step  & +1 level of existing manifesting class\tabularnewline
\hline
4th & +3 & +1 & +4 & +4 & Flanker & +1 level of existing manifesting class\tabularnewline
\hline
5th & +3 & +1 & +4 & +4 & Opportunistic strike +4 --- & \tabularnewline
\hline
6th & +4 & +2 & +5 & +5 & Transporter & +1 level of existing manifesting class\tabularnewline
\hline
7th & +5 & +2 & +5 & +5 & Capricious step & +1 level of existing manifesting class\tabularnewline
\hline
8th & +6 & +2 & +6 & +6 & Opportunistic strike +6--- & \tabularnewline
\hline
9th & +6 & +3 & +6 & +6 & Dimension spring attack & +1 level of existing manifesting 
class\tabularnewline
\hline
10th & +7 & +3 & +7 & +7 & Accelerated action & +1 level of existing manifesting 
class\tabularnewline
\hline
\end{tabular}

\vspace{12pt}
\section*{\textbf{Class Features}}

All the following are class features of the elocater prestige class.

\textbf{Weapon and Armor Proficiency:} Elocaters are proficient with all simple 
and martial weapons and with light armor.

\textbf{Powers Known:} At every level indicated on the table, the character gains 
additional power points per day and access to new powers as if she had also gained 
a level in whatever manifesting class she belonged to before she added the prestige 
class. She does not, however, gain any other benefit a character of that class 
would have gained (bonus feats, metapsionic or item creation feats, psicrystal 
special abilities, and so on). This essentially means that she adds the level of 
elocater to the level of whatever manifesting class the character has, then determines 
power points per day, powers known, and manifester level accordingly.

If a character had more than one manifesting class before she became an elocater, 
she must decide to which class she adds the new level of elocater for the purpose 
of determining power points per day, powers known, and manifester level.

\textbf{Scorn Earth (Su):} At 1st level, an elocater's feet lift from the ground. 
From now on, she can float a foot above the ground. Instead of walking she glides 
along, unconcerned with the hard earth or difficult terrain. While she remains 
within 1 foot of a flat surface of any solid or liquid, she can take normal actions 
and make normal attacks, and can move at her normal speed (she can even ``run'' 
at four times her normal speed). However, at distances higher than 1 foot above 
any surface, her speed diminishes to 10 feet per round.

While she remains within 1 foot of a surface, she can make melee and ranged attacks 
normally, but if she moves any higher, she incurs the penalties on melee and ranged 
attack rolls as if she were the subject of the \textit{psionic levitate }power.

\textbf{Sidestep Charge (Ex): }At 1st level, an elocater gains Sidestep Charge 
as a bonus feat, even if she does not meet the prerequisites. If the character 
already has this feat, she gains no benefit.

\textbf{Opportunistic Strike (Ex):} Beginning at 2nd level, an elocater's hyperawareness 
of spatial relations gives her an instinctive view of the battlefield, which allows 
her to make a cunning attack against distracted opponents. The elocater gains a 
+2 insight bonus on her attack roll and her damage roll (if the attack hits) for 
the first attack she makes against an opponent that has been dealt damage in melee 
by another character since the elocater's last action. At 5th level the insight 
bonus increases to +4, and at 8th level the insight bonus increases to +6.

\textbf{Dimension Step (Su): }An elocater of 3rd level or higher can slip psionically 
between spaces as if using the \textit{psionic dimension door }power, once per 
day. The elocater cannot bring any other creatures with her. Her manifester level 
for this effect is equal to her elocater level.

\textbf{Flanker (Ex):} An elocater of 4th level or higher can flank enemies from 
seemingly impossible angles. She can designate any adjacent square as the square 
from which flanking against an ally is determined (including the square where she 
stands, as normal). She can designate the square at the beginning of her turn or 
at any time during her turn. The designated square remains her effective square 
for flanking until she is no longer adjacent to it or until she chooses a different 
square (at the start of one of her turns). The character can even choose a square 
that is impassable or occupied.

\textbf{Transporter (Ex): }At 6th level, an elocater learns both \textit{psionic 
teleport }and \textit{psionic plane shift}. These powers are in addition to any 
powers the elocater normally learns by advancing a level.

The elocater treats these powers as if they were 3rd-level powers on her class 
list. This means, among other things, that manifesting these powers costs 5 power 
points. (If the character does not have a high enough manifester level to manifest 
3rd-level powers the character cannot manifest these powers until she has attained 
the required manifester level.)

\textbf{Capricious Step (Ex):} At 7th level, an elocater can take an extra 5-foot 
step in any round when she doesn't perform any other movement (except for the first 
5-foot step). Like the first, the second 5-foot step does not provoke attacks of 
opportunity. The character can take the extra 5-foot step immediately after taking 
the first, or wait until the end of her other actions for the round. In all other 
ways, the rules for taking a 5-foot step apply\textit{.}

\textbf{Dimension Spring Attack (Su):} An elocater of 9th level or higher can use 
her dimension step ability in conjunction with her Spring Attack feat once per 
day. This ability can be used only against opponents within 60 feet to which the 
elocater has line of sight. She can dimension step up to the target, use Spring 
Attack, and then use dimension step to return to her starting point. (When she 
uses this ability, the total distance she can travel before and after the attack 
is not limited by her speed.) The use of this ability counts as her use of the 
dimension step ability on that day (and this ability is not available during a 
day when she has already used dimension step).

\textbf{Accelerated Action (Su):} When she attains 10th level, an elocater can 
accelerate herself and thereby take more actions than normal. An elocater can accelerate 
herself for a total of 5 rounds per day. She can choose to parcel out her accelerated 
actions in 1-round increments. (This effect is not cumulative with similar effects 
that provide additional actions, such as \textit{schism }or a \textit{haste }spell---and 
in fact an elocater can't take an accelerated action if affected by these or similar 
effects.)

If she makes a full attack while accelerated, an elocater can make one extra attack 
with any weapon she is holding. The attack is made using her full base attack bonus, 
plus any modifiers appropriate to the situation. If the elocater uses her accelerated 
action to move, she gains an enhancement bonus to her speed of +30 feet. The elocater 
can use her accelerated action to manifest a power, as long as she has not already 
manifested a power in the current round and the one she wants to manifest has a 
manifesting time of 1 standard action or shorter. While accelerated, she gains 
a +2 dodge bonus on attack rolls and Reflex saves and a +2 dodge bonus to Armor 
Class. Any condition that makes her lose her Dexterity bonus to Armor Class (if 
any) also makes her lose these dodge bonuses. 

\vspace{12pt}
{\LARGE{}PSIONIC FIST}

\textbf{Hit Die:} d6.

\vspace{12pt}
\section*{\textbf{Requirements}}

To qualify to become a Psionic Fist, a character must fulfill all the following 
criteria.

\textbf{Base Attack Bonus:} +4.

\textbf{Skill:} Concentration 9 ranks.

\textbf{Feat:} Wild Talent.

\textbf{Special:} Still mind class feature.

\vspace{12pt}
\section*{\textbf{Class Skills}}

The Psionic Fist's class skills (and the key ability for each skill) are Autohypnosis 
(Wis), Concentration (Con), Craft (Int), Escape Artist (Dex), Hide (Dex), Jump 
(Str), Knowledge (psionics) (Int), Knowledge (religion) (Int), Listen (Wis), Move 
Silently (Dex), Psicraft (Int), Sense Motive (Wis), Spot (Wis), Tumble (Dex).

\textbf{Skill Points at Each Level:} 4 + Int modifier.

\vspace{12pt}
\begin{tabular}{|>{\raggedright}p{21pt}|>{\raggedright}p{38pt}|>{\raggedright}p{14pt}|>{\raggedright}p{17pt}|>{\raggedright}p{17pt}|>{\raggedright}p{58pt}|>{\raggedright}p{22pt}|>{\raggedright}p{26pt}|>{\raggedright}p{26pt}|}
\hline
\multicolumn{9}{|p{242pt}|}{\section*{T\textbf{able: The Psionic Fist}}}\tabularnewline
\hline
L\textbf{evel} & B\textbf{ase Attack Bonus} & F\textbf{ort Save} & R\textbf{ef 
Save} & W\textbf{ill Save} & \section*{S\textbf{pecial}} & P\textbf{oints/Day} & P\textbf{owers 
Known} & L\textbf{evel Known}\tabularnewline
\hline
1st & +0 & +0 & +2 & +2 & Monk abilities & 1 & 1 & 1st\tabularnewline
\hline
2nd & +1 & +0 & +3 & +3--- &  & 3 & 2 & 1st\tabularnewline
\hline
3rd & +2 & +1 & +3 & +3--- &  & 6 & 3 & 2nd\tabularnewline
\hline
4th & +3 & +1 & +4 & +4--- &  & 10 & 4 & 2nd\tabularnewline
\hline
5th & +3 & +1 & +4 & +4 & Bonus psionic feat & 15 & 5 & 3rd\tabularnewline
\hline
6th & +4 & +2 & +5 & +5--- &  & 23 & 6 & 3rd\tabularnewline
\hline
7th & +5 & +2 & +5 & +5--- &  & 31 & 7 & 4th\tabularnewline
\hline
8th & +6 & +2 & +6 & +6--- &  & 43 & 8 & 4th\tabularnewline
\hline
9th & +6 & +3 & +6 & +6--- &  & 55 & 9 & 5th\tabularnewline
\hline
10th & +7 & +3 & +7 & +7 & Bonus psionic feat & 71 & 10 & 5th\tabularnewline
\hline
\end{tabular}

\vspace{12pt}
\textbf{Class Features}

All the following are class features of the Psionic Fist prestige class.

\textbf{Weapon and Armor Proficiency:} Psionic Fists gain no proficiency with any 
weapon or armor.

\textbf{Monk Abilities:} A Psionic Fist's class levels stack with his monk levels 
for the purpose of determining his unarmed damage and bonuses to Armor Class and 
unarmored speed. His class levels do not apply to other monk abilities such as 
flurry of blows, slow fall, and so on.

\textbf{Power Points/Day:} A Psionic Fist can manifest powers. His ability to manifest 
powers is limited by the power points he has available. His base daily allotment 
of power points is given on Table: The Psionic Fist. In addition, he receives bonus 
power points per day if he has a high Wisdom score (see Table: Ability Modifiers 
and Bonus Power Points). His race may also provide bonus power points per day, 
as may certain feats and items. If a Psionic Fist has power points from a different 
class, those points are pooled together and usable to manifest powers from either 
class. Bonus power points from having a high ability score can be gained only for 
the character's highest psionic class.

\textbf{Powers Known: }A Psionic Fist chooses his powers from the psychic warrior 
power list. At 1st level, a Psionic Fist knows one psychic warrior power of your 
choice. Each time he attains a new level, he learns one new power. A Psionic Fist 
can manifest any power that has a power point cost equal to or lower than his manifester 
level. The total number of powers a Psionic Fist can manifest per day is limited 
only by his daily power points.

A Psionic Fist simply knows his powers; they are ingrained in his mind. He does 
not need to prepare them (in the way that some spellcasters pre pare their spells), 
though he must get a good night's sleep each day to regain all his spent power 
points.

The Difficulty Class for saving throws against Psionic Fist powers is 10 + the 
power's level + the Psionic Fist's Wisdom modifier.

\textbf{Maximum Power Level Known: }A Psionic Fist gains the ability to learn one 
1st-level power when he takes his first level in the prestige class. As he attains 
each new odd-numbered level, a Psionic Fist gains the ability to master more complex 
powers.

To learn or manifest a power, a Psionic Fist must have a Wisdom score of at least 
10 + the power's level.

\textbf{Bonus Psionic Feat:} At 5th and 10th level, a Psionic Fist can take any 
psionic feat as a bonus feat. He must still meet the prerequisites for the feat.

\textbf{Multiclass Note:} A monk who becomes a Psionic Fist may continue advancing 
as a monk.

\vspace{12pt}
{\LARGE{}SLAYER}

\textbf{Hit Die:} d8.

\vspace{12pt}
\section*{\textbf{Requirements}}

To qualify to become a slayer, a character must fulfill all the following criteria.

\textbf{Base Attack Bonus:} +4.

\textbf{Skill:} Knowledge (dungeoneering) 4 ranks.

\textbf{Feat:} Track.

\textbf{Psionics:} Must have a power point reserve of at least 1 power point.

\vspace{12pt}
\textbf{Class Skills}

The slayer's class skills (and the key ability for each skill) are Bluff (Cha), 
Concentration (Con), Knowledge (dungeoneering) (Int), Listen (Wis), Psicraft (Int), 
Sense Motive (Wis), Spot (Wis), and Survival (Wis). 

\textbf{Skill Points at Each Level:} 4 + Int modifier.

\vspace{12pt}
\begin{tabular}{|>{\raggedright}p{16pt}|>{\raggedright}p{36pt}|>{\raggedright}p{14pt}|>{\raggedright}p{17pt}|>{\raggedright}p{19pt}|>{\raggedright}p{64pt}|>{\raggedright}p{97pt}|}
\hline
\multicolumn{7}{|p{266pt}|}{T\textbf{able: The Slayer}}\tabularnewline
\hline
L\textbf{evel} & B\textbf{ase Attack Bonus} & F\textbf{ort Save} & R\textbf{ef 
 Save} & W\textbf{ill Save} & S\textbf{pecial} & P\textbf{owers Known}\tabularnewline
\hline
1st & +1 & +0 & +0 & +2 & Favored enemy +2, enemy sense --- & \tabularnewline
\hline
2nd & +2 & +0 & +0 & +3 & Brain nausea & +1 level of existing manifesting class\tabularnewline
\hline
3rd & +3 & +1 & +1 & +3 & Lucid buffer & +1 level of existing manifesting class\tabularnewline
\hline
4th & +4 & +1 & +1 & +4 & Favored enemy +4 & +1 level of existing manifesting class\tabularnewline
\hline
5th & +5 & +1 & +1 & +4--- &  & +1 level of existing manifesting class\tabularnewline
\hline
6th & +6 & +2 & +2 & +5 & Cerebral blind & +1 level of existing manifesting class\tabularnewline
\hline
7th & +7 & +2 & +2 & +5 & Favored enemy +6 & +1 level of existing manifesting class\tabularnewline
\hline
8th & +8 & +2 & +2 & +6 & Breach power resistance & +1 level of existing manifesting 
class\tabularnewline
\hline
9th & +9 & +3 & +3 & +6 & Cerebral immunity & +1 level of existing manifesting 
class\tabularnewline
\hline
10th & +10 & +3 & +3 & +7 & Blast feedback, favored enemy +8 & +1 level of existing 
manifesting class\tabularnewline
\hline
\end{tabular}

\vspace{12pt}
\textbf{Class Features}

All the following are class features of the slayer prestige class.

\textbf{Weapon and Armor Proficiency: }Slayers are proficient with all simple and 
martial weapons and with all types of armor. 

\textbf{Powers Known:} Beginning at 2nd level, a slayer gains additional power 
points per day and access to new powers as if she had also gained a level in whatever 
manifesting class she belonged to before she added the prestige class. She does 
not, however, gain any other benefit a character of that class would have gained 
(bonus feats, metapsionic or item creation feats, psicrystal special abilities, 
and so on). This essentially means that she adds the level of slayer to the level 
of whatever manifesting class the character has, then determines power points per 
day, powers known, and manifester level accordingly.

If a character had more than one manifesting class before she became a slayer, 
she must decide to which class she adds the new level of slayer for the purpose 
of determining power points per day, powers known, and manifester level.

If the character did not belong to a manifesting class before taking this prestige 
class, she does not gain manifesting levels.

\textbf{Favored Enemy (Ex):} When she enters the class, a slayer formally declares 
a type of psionic creature as the enemy she detests above all others. Due to her 
extensive study of her foes and training in the proper techniques for combating 
them, she gains a +2 bonus on Bluff, Listen, Sense Motive, Spot, and Survival checks 
when using these skills against her favored enemy.

Likewise, she gets a +2 bonus on weapon damage rolls against creatures of this 
kind. At 4th level the bonuses increase to +4, at 7th level to +6, and at 10th 
level to +8.

\textbf{Enemy Sense (Su):} A slayer can sense the presence of her favored enemy 
within 60 feet of herself, even if they are hidden by darkness or walls, but she 
cannot discern their exact location.

\textbf{Brain Nausea (Su):} At 2nd level, a slayer gains a constant defensive ability 
somewhat similar to the \textit{aversion }power. Any creature that attempts to 
eat the slayer's brain must succeed on a Will save (DC 15 + slayer's class level) 
or become disinclined to do so for 24 hours thereafter. A creature that fails this 
save may take any action it desires except extracting the slayer's brain (but does 
not realize it is being so affected). This ability is active even if the slayer 
is unconscious, stunned, or otherwise helpless.

\textbf{Lucid Buffer (Ex):} At 3rd level, a slayer becomes especially skilled at 
resisting mental attacks. She gains a +4 competence bonus on saving throws against 
all compulsions and mind-affecting effects. This ability is active even if the 
slayer is unconscious, stunned, or otherwise helpless.

\textbf{Cerebral Blind (Su): }After reaching 6th level, a slayer is protected from 
all devices, powers, and spells that reveal location. This ability protects against 
information gathering by clairsentience powers or effects that reveal location.

The ability even foils \textit{bend reality, limited wish, miracle, reality revision, 
}and \textit{wish }when they are used to gain information about the slayer's location 
(however, \textit{metafaculty }can pierce this protective barrier). In the case 
of \textit{remote viewing }or \textit{scrying }that scans an area a slayer is in, 
the effect works, but the slayer simply isn't detected. \textit{Remote viewing 
}or \textit{scrying }attempts that are targeted specifically at a slayer do not 
work. This ability is active as long as the slayer is psionically focused.

\textbf{Breach Power Resistance (Su):} A slayer of 8th level or higher can enhance 
her weapon with psionic might. Each successful melee attack (or ranged attack if 
the target is within 30 feet) she makes with her weapon against a creature with 
power resistance temporarily reduces its power resistance by 1. Unless the creature 
is slain, its lost power resistance returns all at once 12 hours later. This ability 
is active as long as the slayer is psionically focused.

\textbf{Cerebral Immunity (Su):} On reaching 9th level, a slayer gains protection 
from all devices, powers, and spells that influence the mind. This ability shields 
her against almost all mind-affecting powers and effects (though the slayer can 
selectively allow powers or spells to affect her). The ability even foils \textit{bend 
reality}, \textit{limited wish}, \textit{miracle, reality revision}, and \textit{wish 
}when they are used to mentally influence a slayer. This ability is active as long 
as the slayer is psionically focused.

\textbf{Blast Feedback (Su):} At 10th level, if a slayer makes her saving throw 
when attacked with \textit{mind blast, }the \textit{mind blast }rebounds upon the 
attacker. Only the original attacker is targeted in the rebound effect. If the 
original attacker fails a Will saving throw (DC equal to that of the original attack), 
the attacker is affected normally by the \textit{mind blast.}

\vspace{12pt}
{\LARGE{}METAMIND}

\textbf{Hit Die:} d4.

\vspace{12pt}
\section*{\textbf{Requirements}}

To qualify to become a metamind, a character must fulfill all the following criteria.

\textbf{Skills:} Knowledge (psionics) 8 ranks, Psicraft 4 ranks.

\textbf{Feat:} Psicrystal Affinity.

\textbf{Psionics: }Manifester level 4th.

\vspace{12pt}
\subsection*{\textbf{Class Skills}}

The metamind's class skills are Autohypnosis (Wis), Concentration* (Con), Craft 
(any) (Int), Knowledge (psionics), and Psicraft (Int).

\textbf{Skill Points at Each Level:} 2 + Int modifier.

\vspace{12pt}
\begin{tabular}{|>{\raggedright}p{15pt}|>{\raggedright}p{31pt}|>{\raggedright}p{16pt}|>{\raggedright}p{16pt}|>{\raggedright}p{18pt}|>{\raggedright}p{78pt}|>{\raggedright}p{90pt}|}
\hline
\multicolumn{7}{|p{266pt}|}{\section*{T\textbf{able: The Metamind}}}\tabularnewline
\hline
L\textbf{evel} & B\textbf{ase Attack Bonus} & F\textbf{ort Save} & R\textbf{ef 
Save} & W\textbf{ill Save} & S\textbf{pecial} & P\textbf{owers Known}\tabularnewline
\hline
1st & +0 & +0 & +0 & +2 & Free manifesting 1st, 3/day--- & \tabularnewline
\hline
2nd & +1 & +0 & +0 & +3 & Cognizance psicrystal 5 points & +1 level of existing 
manifesting class\tabularnewline
\hline
3rd & +1 & +1 & +1 & +3 & Free manifesting 2nd, 3/day--- & \tabularnewline
\hline
4th & +2 & +1 & +1 & +4 & Cognizance psicrystal 7 points & +1 level of existing 
manifesting class\tabularnewline
\hline
5th & +2 & +1 & +1 & +4 & Free manifesting 3rd, 1/day--- & \tabularnewline
\hline
6th & +3 & +2 & +2 & +5 & Cognizance psicrystal 9 points & +1 level of existing 
manifesting class\tabularnewline
\hline
7th & +3 & +2 & +2 & +5 & Free manifesting 4th, 1/day--- & \tabularnewline
\hline
8th & +4 & +2 & +2 & +6 & Cognizance psicrystal 11 points & +1 level of existing 
manifesting class\tabularnewline
\hline
9th & +4 & +3 & +3 & +6 & Free manifesting 5th, 1/day--- & \tabularnewline
\hline
10th & +5 & +3 & +3 & +7 & Font of power & +1 level of existing manifesting class\tabularnewline
\hline
\end{tabular}

\vspace{12pt}
\section*{\textbf{Class Features}}

All the following are class features of the metamind prestige class.

\textbf{Weapon and Armor Proficiency:} Metaminds gain no proficiency with any weapon 
or armor.

\textbf{Powers Known: }At every even-numbered level, a metamind gains additional 
power points per day and access to new powers as if he had also gained a level 
in whatever manifesting class he belonged to before he added the prestige class. 
He does not, however, gain any other benefit a character of that class would have 
gained (bonus feats, metapsionic or item creation feats, and so on). This essentially 
means that he adds the level of metamind to the level of whatever manifesting class 
the character has, then determines power points per day, powers known, and manifester 
level accordingly.

If a character had more than one manifesting class before he became a metamind, 
he must decide to which class he adds the new level of metamind for the purpose 
of determining power points per day, powers known, and manifester level.

\textbf{Free Manifesting (Ps):} At 1st level, a metamind can manifest any 1st-level 
power he knows for free (without spending power points) three times per day.

At higher levels, a metamind gains the ability to freely manifest additional higher-level 
powers: three 2nd-level powers per day at 3rd level; one 3rd-level power per day 
at 5th level; one 4th-level power per day at 7th level; and one 5th-level power 
per day at 9th level.

This benefit applies only to the power point cost of an unaugmented power. Points 
spent to augment a power and an experience point cost (if any) must be paid as 
normal.

\textbf{Cognizance Psicrystal (Ex):} At 2nd level, a metamind masters the trick 
of storing excess power points in a psicrystal. The psicrystal is now treated as 
a \textit{cognizance crystal} capable of storing 5 power points, in addition to 
its psicrystal abilities. At every even-numbered level, a metamind becomes able 
to store an additional 2 power points in his psicrystal, to a maximum of 11 points 
at 8th level.

\textbf{Font of Power (Ps): }A 10th-level metamind can act as a living \textit{cognizance 
crystal, }producing seemingly endless power points once per day, for up to 1 minute. 
His eyes shine like tiny stars, and faint illumination seems to beam out of his 
mouth and the end of each of his fingers. While so empowered, he can manifest any 
of his powers without drawing from his power point reserve. He finds the power 
points he needs welling up within his own body.

If a metamind using this ability enters a metaconcert, his power point reserve 
is accessed normally for the purpose of his contributing to the pool.

\vspace{12pt}
{\LARGE{}PSION UNCARNATE}

\textbf{Hit Die:} d4.

\vspace{12pt}
\section*{\textbf{Requirements}}

To qualify to become a psion uncarnate, a character must fulfill all the following 
criteria.

\textbf{Skills:} Knowledge (psionics) 8 ranks, Psicraft 8 ranks.

\textbf{Feat:} Psionic Body.

\textbf{Psionics:} Able to manifest 3rd-level powers.

\textbf{Special:} Must have had some instruction by another psion uncarnate.

\vspace{12pt}
\section*{\textbf{Class Skills}}

The psion uncarnate's class skills are Autohypnosis (Wis), Bluff (Cha), Concentration 
(Con), Craft (any) (Int), Disguise (Cha), Knowledge (the planes) (Int), Knowledge 
(psionics) (Int), Psicraft (Int), and Sense Motive (Wis).

\textbf{Skill Points at Each Level:} 2 + Int modifier.

\vspace{12pt}
\begin{tabular}{|>{\raggedright}p{16pt}|>{\raggedright}p{33pt}|>{\raggedright}p{17pt}|>{\raggedright}p{17pt}|>{\raggedright}p{19pt}|>{\raggedright}p{64pt}|>{\raggedright}p{97pt}|}
\hline
\multicolumn{7}{|p{266pt}|}{T\textbf{able: The Psion Uncarnate}}\tabularnewline
\hline
L\textbf{evel} & B\textbf{ase Attack Bonus} & F\textbf{ort Save} & R\textbf{ef 
Save} & W\textbf{ill Save} & S\textbf{pecial} & P\textbf{owers Known}\tabularnewline
\hline
1st & +0 & +0 & +0 & +2 & Incorporeal touch 1d6, uncarnate armor--- & \tabularnewline
\hline
2nd & +1 & +0 & +0 & +3 & Shed body 1/day & +1 level of existing manifesting class\tabularnewline
\hline
3rd & +1 & +1 & +1 & +3 & Assume equipment & +1 level of existing manifesting class\tabularnewline
\hline
4th & +2 & +1 & +1 & +4 & Assume likeness--- & \tabularnewline
\hline
5th & +2 & +1 & +1 & +4 & Incorporeal touch 2d6 & +1 level of existing manifesting 
class\tabularnewline
\hline
6th & +3 & +2 & +2 & +5 & Shed body 2/day & +1 level of existing manifesting class\tabularnewline
\hline
7th & +3 & +2 & +2 & +5 & Telekinetic force--- & \tabularnewline
\hline
8th & +4 & +2 & +2 & +6 & Uncarnate bridge & +1 level of existing manifesting class\tabularnewline
\hline
9th & +4 & +3 & +3 & +6 & Incorporeal touch 3d6 & +1 level of existing manifesting 
class\tabularnewline
\hline
10th & +5 & +3 & +3 & +7 & Uncarnate--- & \tabularnewline
\hline
\end{tabular}

\vspace{12pt}
\section*{\textbf{Class Features}}

All the following are class features of the psion uncarnate prestige class.

\textbf{Weapon and Armor Proficiency:} Psion uncarnates gain no proficiency with 
any weapon or armor.

\textbf{Powers Known:} At every level indicated on Table: The Psion Uncarnate, 
a psion uncarnate gains additional power points per day and access to new powers 
as if he had also gained a level in whatever manifesting class he belonged to before 
he added the prestige class. He does not, however, gain any other benefit a character 
of that class would have gained (bonus feats, metapsionic or item creation feats, 
psicrystal special abilities, and so on). This essentially means that he adds the 
level of psion uncarnate to the level of whatever manifesting class the character 
has, then determines power points per day, powers known, and manifester level accordingly.

If a character had more than one manifesting class before he became a psion uncarnate, 
he must decide to which class he adds the new level of psion uncarnate for the 
purpose of determining power points per day, powers known, and manifester level.

\textbf{Incorporeal Touch (Su):} Beginning at 1st level, a psion uncarnate can 
make up to three melee touch attacks per day that each deal 1d6 points of damage 
if they hit. The character's Strength modifier is not applied to this attack, but 
it is effective against incorporeal creatures (and against corporeal creatures 
while the psion uncarnate is incorporeal) The character's hand and arm seem to 
become slightly translucent when he makes these attacks. A miss still counts as 
a use of the ability.

While uncarnate (see below), a psion uncarnate can make melee touch attacks at 
will that do not count against his uses of this ability. 

For every four levels higher than 1st the psion uncarnate attains, the damage on 
these attacks increases by 1d6 points.

\textbf{Uncarnate Armor (Su):} A psion uncarnate wearing armor (or using \textit{inertial 
armor }or a similar effect) gets his armor bonus to AC even when he becomes incorporeal 
(see Shed Body, below). However, unlike other incorporeal creatures, a psion uncarnate 
does not gain a deflection bonus to Armor Class from his Charisma modifier. This 
ability works even if the armor being worn becomes incorporeal (such as through 
the use of the assume equipment ability described below).

\textbf{Shed Body (Su):} Starting at 2nd level, a psion uncarnate can become incorporeal 
(or ``uncarnate'') once per day as a standard action. The character can remain 
uncarnate for up to 1 minute. During this time, the character's body fades into 
an immaterial form that retains the character's basic likeness. While uncarnate, 
the character gains the incorporeal subtype (see below). He gains a fly speed equal 
to his land speed (perfect maneuverability). His material armor remains in place 
and continues to provide its armor bonus to AC (see Uncarnate Armor, above). His 
material weapons also remain corporeal. Losing his physical form allows the character 
to more easily access his mental abilities, and he gains a +1 bonus on all save 
DCs for powers he manifests while uncarnate.

He can use equipment normally, deriving benefits from items that enhance his capabilities; 
however, all his equipment remains material even when the character is uncarnate 
(but see the assume equipment ability, described below). 

Often, a psion uncarnate appears almost like a ghost wearing items of the material 
world. This doesn't make his equipment more susceptible to attack (the normal rules 
for attended objects apply), but it does make it impossible for the character to 
enter or pass through solid objects while wearing solid equipment. If he drops 
his material equipment, he can pass through solid objects at will as described 
below.

At 6th level and higher, a psion uncarnate can shed his body twice per day for 
up to 1 minute each time. 

\textit{Incorporeal Subtype: }An incorporeal psion uncarnate has no physical body. 
He can be harmed only by other incorporeal creatures, magic weapons or creatures 
that strike as magic weapons, and spells, spell-like abilities, or supernatural 
abilities. He is immune to all nonmagical attack forms. Even when hit by spells 
or magic weapons, he has a 50\% chance to ignore any damage from a corporeal source 
(except for positive energy, negative energy, force effects\textit{, }or attacks 
made with \textit{ghost touch }weapons).

An incorporeal psion uncarnate has no natural armor bonus---and, unlike other incorporeal 
creatures, does not gain a deflection bonus from his Charisma modifier. An incorporeal 
psion uncarnate can enter or pass through solid objects (subject to the restrictions 
described in the shed body and assume equipment abilities), but must remain adjacent 
to the object's exterior, and so cannot pass entirely through an object whose space 
is larger than his own. He can sense the presence of creatures or objects within 
a square adjacent to his current location, but enemies have total concealment (50\% 
miss chance) from an incorporeal psion uncarnate that is inside an object. To see 
farther from the object he is in and attack normally, the incorporeal psion uncarnate 
must emerge. An incorporeal psion uncarnate inside an object has total cover, but 
when he attacks a creature outside the object he only has cover, so a creature 
outside with a readied action could strike at him as he attacks. An incorporeal 
psion uncarnate cannot pass through a force effect.

An incorporeal psion uncarnate's attacks pass through (ignore) natural armor, armor, 
and shields, although deflection bonuses and force effects work normally against 
him. He can pass through and operate in water as easily as he does in air. An incorporeal 
psion uncarnate cannot fall or take falling damage. He cannot make trip or grapple 
attacks, nor can he be tripped or grappled. In fact, he cannot take any physical 
action that would move or manipulate an opponent or its equipment, nor is he subject 
to such actions.

Incorporeal creatures have no weight and do not set off traps that are triggered 
by weight. An incorporeal creature moves silently and cannot be heard with Listen 
checks if it doesn't wish to be. It has no Strength score, so its Dexterity modifier 
applies to both its melee attack rolls and its ranged attack rolls. Nonvisual senses, 
such as scent and blindsight, are either ineffective or only partly effective with 
regard to incorporeal creatures. Incorporeal creatures have an innate sense of 
direction and can move at full speed even when they cannot see.

\textbf{Assume Equipment (Su):} Beginning at 3rd level, a psion

uncarnate can designate a number of pieces of his worn equipment (including armor 
and weapons) equal to his psion uncarnate level to become incorporeal when he uses 
his shed body ability. This has no effect on the equipment's function, but now 
when the psion uncarnate is incorporeal, he can enter or pass through solid objects 
while wearing nothing other than the designated equipment. Once designated, the 
equipment automatically changes to incorporeal when the character sheds his body, 
and it returns to corporeality when the character does. The character can change 
his designations as he desires.

\textbf{Assume Likeness (Su):} At 4th level and higher, while incorporeal, a psion 
uncarnate can assume the likeness of any Small, Medium, or Large creature as a 
standard action that does not provoke attacks of opportunity. The character's abilities 
do not change, but he appears to be the creature that he assumes the likeness of, 
allowing him the ability to effectively disguise himself and bluff those who might 
wonder at his true nature. Each physical interaction with a creature requires a 
successful Bluff check (opposed by the creature's Sense Motive check) to convince 
the creature of the psion uncarnate's new appearance. The psion uncarnate must 
not do anything to give away his true (incorporeal) nature in order for the bluff 
to be successful; for instance, if he accepts an item from another creature only 
to have it fall through his immaterial hands, the Bluff check automatically fails. 
However, a Bluff check would be allowed if the psion uncarnate uses his telekinetic 
force ability (see below) to hold the received item.

When using his assume likeness ability, a psion uncarnate has an additional +10 
circumstance bonus on Disguise checks. If he can read an opponent's mind, he gets 
a further +4 circumstance bonus on Bluff and Disguise checks.

\textbf{Telekinetic Force (Su):} Beginning at 7th level, while incorporeal, a psion 
uncarnate can use a telekinetic force effect as a standard action that does not 
provoke attacks of opportunity. The save DC is equal to 14 + the psion uncarnate's 
key ability modifier (either Int, Wis, or Cha). The character's manifester level 
is the manifester level of the effect.

Even while corporeal, a psion uncarnate can use this ability, but only three times 
per day (uses while he is uncarnate do not count against this use limit).

\textbf{Uncarnate Bridge (Su): }At 8th level, as a creature of almost pure mind, 
a psion uncarnate becomes more closely attuned to the minds of other creatures. 
He gains the ability to transport himself via the minds of living creatures. Once 
per day as a standard action while incorporeal, he can seamlessly enter any living 
creature with an Intelligence score and pass to another living creature with an 
Intelligence score that is within line of sight of the first creature.

The psion uncarnate must be in a space adjacent to the entry creature before transporting, 
and he appears in a space adjacent to the destination creature after transporting. 
The entry and destination creatures need not be familiar to the character. A psion 
uncarnate cannot use himself as the entry or destination creature. Neither creature 
need be a willing participant.

When exiting the destination creature, the psion uncarnate chooses an adjacent 
square in which to appear. Entering and leaving a creature is painless, unless 
the psion uncarnate wishes otherwise (see below). In most cases, though, the destination 
creature finds being the endpoint of a mental bridge surprising and quite unsettling.

If he desires, a psion uncarnate can destructively exit the destination creature. 
If the creature fails a Will save (DC 15 + psion uncarnate's key ability modifier), 
the exiting psion uncarnate tunes his mental form to destructively interfere with 
the target's mind. He bursts forth explosively from the creature's body, dealing 
it 10d6 points of damage.

\textbf{Uncarnate (Ex):} At 10th level, a psion uncarnate becomes a being of pure 
psionic consciousness. This ability is similar to shed body, except the character 
is permanently incorporeal (and gains that subtype). If the character desires, 
he can become corporeal once per day for up to 1 minute, but he spends the rest 
of his time as an entity of mind untethered by the physical world.

\vspace{12pt}
{\LARGE{}PYROKINETICIST}

\textbf{Hit Die:} d8.

\vspace{12pt}
\section*{\textbf{Requirements}}

To qualify to become a pyrokineticist, a character must fulfill all the following 
criteria.

\textbf{Alignment:} Any chaotic.

\textbf{Skills:} Concentration 8 ranks, Craft (alchemy) 1 rank, Knowledge (psionics) 
2 ranks.

\textbf{Psionics:} Must have a power point reserve of at least 1 power point.

\textbf{Special:} Must have set fire to a structure of any size just to watch it 
burn.

\vspace{12pt}
\section*{\textbf{Class Skills}}

The pyrokineticist's class skills (and the key ability for each skill) are Climb 
(Str), Concentration (Con), Craft (any) (Int), Intimidate (Cha), Jump (Str), and 
Psicraft (Int). 

\textbf{Skill Points at Each Level:} 2 + Int modifier.

\vspace{12pt}
\begin{tabular}{|>{\raggedright}p{21pt}|>{\raggedright}p{52pt}|>{\raggedright}p{34pt}|>{\raggedright}p{34pt}|>{\raggedright}p{26pt}|>{\raggedright}p{109pt}|}
\hline
\multicolumn{6}{|p{278pt}|}{T\textbf{able: The Pyrokineticist}}\tabularnewline
\hline
\section*{L\textbf{evel}} & B\textbf{ase Attack Bonus} & F\textbf{ort Save} & R\textbf{ef 
Save} & W\textbf{ill Save} & \section*{S\textbf{pecial}}\tabularnewline
\hline
1st & +0 & +2 & +2 & +0 & F\textit{ire lash}\tabularnewline
\hline
2nd & +1 & +3 & +3 & +0 & Fire adaptation, \textit{hand afire}\tabularnewline
\hline
3rd & +2 & +3 & +3 & +1 & B\textit{olt of fire}\tabularnewline
\hline
4th & +3 & +4 & +4 & +1 & W\textit{eapon afire}\tabularnewline
\hline
5th & +3 & +4 & +4 & +1 & N\textit{imbus}\tabularnewline
\hline
6th & +4 & +5 & +5 & +2 & Firewalk\tabularnewline
\hline
7th & +5 & +5 & +5 & +2 & Fear no fire\tabularnewline
\hline
8th & +6 & +6 & +6 & +2 & G\textit{reater weapon afire}\tabularnewline
\hline
9th & +6 & +6 & +6 & +3 & H\textit{eat death}\tabularnewline
\hline
10th & +7 & +7 & +7 & +3 & C\textit{onflagration}\tabularnewline
\hline
\end{tabular}

\vspace{12pt}
\section*{\textbf{Class Features}}

All the following are class features of the pyrokineticist prestige class.

\textbf{Weapon and Armor Proficiency: }Pyrokineticists gain no proficiency with 
any weapon or armor.

\textit{\textbf{Fire Lash }}\textbf{(Ps):} A pyrokineticist gains the ability to 
fashion a 15-foot-long whip of fire from unstable ectoplasm as a move-equivalent 
action. She takes no damage from a fire lash she creates, and if she releases her 
hold, it immediately dissipates. The lash deals 1d8 points of fire damage to a 
target within 15 feet on a successful ranged touch attack. A pyro can take Weapon 
Focus and Weapon Specialization (if she otherwise meets the prerequisites) in conjunction 
with the fire lash, as well as any feats that apply to the use of a standard whip. 
The whip remains in existence as long as the pyrokineticist holds it.

\textbf{Fire Adaptation (Ex):} At 2nd level, a pyrokineticist becomes resistant 
to fire, gaining a +4 bonus on all saving throws against fire and heat spells and 
effects. In addition, she gains resistance to fire 10.

\textit{\textbf{Hand Afire }}\textbf{(Ps):} A pyrokineticist of 2nd level or higher 
can activate this ability as a move-equivalent action. Flames engulf one of the 
pyrokineticist's hands (but do her no harm). Her unarmed attacks with that hand 
are treated as armed. Such attacks deal an extra 2d6 points of fire damage. 

\textit{\textbf{Bolt of Fire }}\textbf{(Ps):} Starting at 3rd level, as a standard 
action, a pyrokineticist can launch a bolt of psionically manifested fire at any 
target in line of sight within 60 feet. This effect is treated as a ranged touch 
attack and deals 1d6 points of fire damage for each class level the pyro has.

\textit{\textbf{Weapon Afire }}\textbf{(Ps):} At 4th level and higher, a pyrokineticist 
can activate this ability as a move-equivalent action. Flames that harm neither 
her nor the weapon engulf one weapon she holds (which can be a projectile such 
as a stone, bullet, arrow, or bolt). The weapon deals an extra 2d6 points of fire 
damage on a successful hit. The weapon retains this effect for as long as the pyrokineticist 
wields it.

\textit{\textbf{Nimbus }}\textbf{(Ps):} Beginning at 5th level, a pyrokineticist 
can activate this ability as a move-equivalent action. Flames that harm neither 
the pyrokineticist nor her equipment engulf her entire body. While she is aflame, 
the character's Charisma score increases by 4, she can make a melee touch attack 
for 2d6 points of fire damage, and she gains damage reduction 5/magic. If she is 
struck in melee, the attacker takes 2d6 points of fire damage. This ability lasts 
for up to 1 minute per pyrokineticist level and is usable once per day.

\textbf{Firewalk (Su):} Beginning at 6th level, as a free action a pyrokineticist 
can expend her psionic focus (see the Concentration skill description) to literally 
walk on air. She moves at her normal speed in all directions, including vertically, 
but cannot move more than double her speed in a round. A firewalking pyro leaves 
footprints of flame in the air that disperse in 2 rounds, but her tread does not 
deal damage. She must pay 1 power point per round spent traveling in this fashion.

\textbf{Fear No Fire (Ex):} At 7th level, a pyrokineticist becomes highly resistant 
to fire, gaining a +8 bonus on all saving throws against fire and heat spells and 
effects and also gaining resistance to fire 20.

\textit{\textbf{Greater Weapon Afire }}\textbf{(Ps):} At 8th level, when a pyrokineticist 
activates her hand afire ability or her weapon afire ability, her unarmed attack 
or weapon deals an extra 4d6 points of fire damage instead of 2d6. Touch attacks 
made while she uses the nimbus ability likewise deal 4d6 points of damage instead 
of 2d6.

\textit{\textbf{Heat Death }}\textbf{(Ps):} A pyrokineticist who reaches 9th level 
can expend her psionic focus and take a full attack action to raise the internal 
temperature of one living creature within 30 feet to lethal levels. The target 
must succeed on a Fortitude saving throw (DC 14 + pyro's Cha modifier) or die horrifically 
as its blood (or other internal fluid) boils. Even on a successful save, the target 
takes 4d8 points of fire damage from the heat.

\textit{\textbf{Conflagration }}\textbf{(Ps):} At 10th level, a pyrokineticist 
gains the ability to create a massive burst of raging flames around herself, burning 
everything in the area. Once per day, as a standard action, she can use this ability 
to deal 15d6 points of fire damage in a 30-foot-radius burst emanating from herself. 
Any creature or object caught in the burst can make a Reflex saving throw (DC 15 
+ pyro's Cha modifier) for half damage. Anyone failing the Reflex save against 
the \textit{conflagration }must also make a Fortitude saving throw (same DC) or 
die due to extreme shock from the intense heat.

\vspace{12pt}
{\LARGE{}THRALLHERD}

\textbf{Hit Die:} d4.

\vspace{12pt}
\section*{\textbf{Requirements}}

To qualify to become a thrallherd, a character must fulfill all the following criteria.

\textbf{Skills:} Diplomacy 4 ranks, Knowledge (psionics) 8 ranks. 

\textbf{Feat:} Inquisitor.

\textbf{Psionics:} Manifester level 5th and able to manifest \textit{mindlink}.

\vspace{12pt}
\section*{\textbf{Class Skills}}

The thrallherd's class skills are Autohypnosis (Wis), Bluff (Cha), Concentration 
(Con), Craft (any) (Int), Diplomacy (Cha), Knowledge (psionics), Psicraft (Int), 
and Sense Motive (Wis).

\textbf{Skill Points at Each Level:} 2 + Int modifier.

\vspace{12pt}
\begin{tabular}{|>{\raggedright}p{16pt}|>{\raggedright}p{35pt}|>{\raggedright}p{20pt}|>{\raggedright}p{18pt}|>{\raggedright}p{23pt}|>{\raggedright}p{52pt}|>{\raggedright}p{99pt}|}
\hline
\multicolumn{7}{|p{266pt}|}{\section*{T\textbf{able: The Thrallherd}}}\tabularnewline
\hline
L\textbf{evel} & B\textbf{ase Attack Bonus} & F\textbf{ort Save} & R\textbf{ef 
Save} & W\textbf{ill Save} & S\textbf{pecial} & P\textbf{owers Known}\tabularnewline
\hline
1st & +0 & +0 & +0 & +2 & Thrallherd--- & \tabularnewline
\hline
2nd & +1 & +0 & +0 & +3--- &  & +1 level of existing manifesting class\tabularnewline
\hline
3rd & +1 & +1 & +1 & +3 & Psionic charm & +1 level of existing manifesting class\tabularnewline
\hline
4th & +2 & +1 & +1 & +4--- &  & +1 level of existing manifesting class\tabularnewline
\hline
5th & +2 & +1 & +1 & +4 & Psionic dominate & +1 level of existing manifesting class\tabularnewline
\hline
6th & +3 & +2 & +2 & +5--- &  & +1 level of existing manifesting class\tabularnewline
\hline
7th & +3 & +2 & +2 & +5 & Greater dominate & +1 level of existing manifesting class\tabularnewline
\hline
8th & +4 & +2 & +2 & +6--- &  & +1 level of existing manifesting class\tabularnewline
\hline
9th & +4 & +3 & +3 & +6 & Superior dominate & +1 level of existing manifesting 
class\tabularnewline
\hline
10th & +5 & +3 & +3 & +7 & Twofold master--- & \tabularnewline
\hline
\end{tabular}

\vspace{12pt}
\section*{\textbf{Class Features}}

All the following are class features of the thrallherd prestige class.

\textbf{Weapon and Armor Proficiency:} Thrallherds gain no proficiency with any 
weapon or armor.

\textbf{Powers Known:} At every level from 2nd through 9th, a thrallherd gains 
additional power points per day and access to new powers as if she had also gained 
a level in whatever manifesting class she belonged to before she added the prestige 
class. She does not, however, gain any other benefit a character of that class 
would have gained (bonus feats, metapsionic or item creation feats, psicrystal 
special abilities, and so on). This essentially means that she adds the level of 
thrallherd to the level of whatever manifesting class the character has, then determines 
power points per day, powers known, and manifester level accordingly.

If a character had more than one manifesting class before she became a thrallherd, 
she must decide to which class she adds the new level of thrallherd for the purpose 
of determining power points per day, powers known, and manifester level.

\textbf{Thrallherd (Ex):} A thrallherd who has just entered the class sends out 
a subtle psychic call for servants, and that call is answered. Essentially, the 
character gains something akin to the Leadership feat, but with some important 
differences.

Those who answer a thrallherd's call are not referred to as cohorts and followers, 
but rather as thralls and believers, respectively. They do not appear because they 
admire the character and want to serve her, but because a hidden psychic resonance 
connects the thrallherd and her servants.

As with the Leadership feat, a thrallherd has a Leadership score that determines 
the highest-level thrall and believers she can attract. A thrallherd's Leadership 
score is equal to her character level + her Cha modifier + her thrallherd level. 
(Note that her thrallherd level is counted twice.) This score is not affected by 
any of the modifiers mentioned in the Leadership feat\textit{.}

As with the Leadership feat, the called thrall's level is limited, even if the 
character's Leadership score would indicate a higher-level thrall. Unlike with 
the Leadership feat, the level limit of a thrall is equal to the character's total 
level minus 1 (not level minus 2, as is true for cohorts).

A thrallherd cannot take the Leadership feat; if the character already has it, 
the feat is lost and replaced by this ability; those who were previously cohorts 
and followers go their separate ways, and those who are more mentally pliable show 
up later to take up roles as thralls and believers. A thrallherd's first thrall 
and believers arrive within 24 hours of her entry into this class; likewise, lost 
thralls and believers are replaced within 24 hours.

Use the following table instead of the one with the Leadership feat to determine 
the level of a thrallherd's thrall and the number of believers of various levels 
a thrallherd can attract.

\vspace{12pt}
\begin{tabular}{|>{\raggedright}p{48pt}|>{\raggedright}p{34pt}|>{\raggedright}p{15pt}|>{\raggedright}p{15pt}|>{\raggedright}p{15pt}|>{\raggedright}p{15pt}|>{\raggedright}p{15pt}|>{\raggedright}p{16pt}|}
\hline
L\textbf{eadership Score } & T\textbf{hrall Level}--- & \multicolumn{6}{p{95pt}|}{\section*{ 
\textbf{Number of Believers by Level ---}}}\tabularnewline
\hline
 &  & 1\textbf{st} & 2\textbf{nd} & 3\textbf{rd} & 4\textbf{th} & 5\textbf{th} & 6\textbf{th}\tabularnewline
\hline
1 or lower--- & --- & --- & --- & --- & --- & --- & \tabularnewline
\hline
2 & 1st--- & --- & --- & --- & --- & --- & \tabularnewline
\hline
3 & 2nd--- & --- & --- & --- & --- & --- & \tabularnewline
\hline
4 & 3rd--- & --- & --- & --- & --- & --- & \tabularnewline
\hline
5 & 3rd--- & --- & --- & --- & --- & --- & \tabularnewline
\hline
6 & 4th--- & --- & --- & --- & --- & --- & \tabularnewline
\hline
7 & 5th--- & --- & --- & --- & --- & --- & \tabularnewline
\hline
8 & 5th--- & --- & --- & --- & --- & --- & \tabularnewline
\hline
9 & 6th--- & --- & --- & --- & --- & --- & \tabularnewline
\hline
10 & 7th & 5--- & --- & --- & --- & --- & \tabularnewline
\hline
11 & 7th & 6--- & --- & --- & --- & --- & \tabularnewline
\hline
12 & 8th & 8--- & --- & --- & --- & --- & \tabularnewline
\hline
13 & 9th & 10 & 1--- & --- & --- & --- & \tabularnewline
\hline
14 & 10th & 15 & 1--- & --- & --- & --- & \tabularnewline
\hline
15 & 10th & 20 & 2 & 1--- & --- & --- & \tabularnewline
\hline
16 & 11th & 25 & 2 & 1--- & --- & --- & \tabularnewline
\hline
17 & 12th & 30 & 3 & 1 & 1--- & --- & \tabularnewline
\hline
18 & 12th & 35 & 3 & 1 & 1--- & --- & \tabularnewline
\hline
19 & 13th & 40 & 4 & 2 & 1 & 1--- & \tabularnewline
\hline
20 & 14th & 50 & 5 & 3 & 2 & 1--- & \tabularnewline
\hline
21 & 15th & 60 & 6 & 3 & 2 & 1 & 1\tabularnewline
\hline
22 & 15th & 75 & 7 & 4 & 2 & 2 & 1\tabularnewline
\hline
23 & 16th & 90 & 9 & 5 & 3 & 2 & 1\tabularnewline
\hline
24 & 17th & 110 & 11 & 6 & 3 & 2 & 1\tabularnewline
\hline
25 or higher & 17th & 135 & 13 & 7 & 4 & 2 & 2\tabularnewline
\hline
\end{tabular}

\vspace{12pt}
\textbf{Psionic Charm (Ex):} At 3rd level, a thrallherd adds \textit{psionic charm 
}to her powers known (if she doesn't already know it). Once per day, she can manifest 
\textit{psionic charm }at a reduced power point cost. The cost of \textit{psionic 
charm }is reduced by the thrallherd's level, to a minimum of 1 power point. The 
effect of this power is still restricted by the thrallherd's manifester level.

\textbf{Psionic Dominate (Ex):} At 5th level, a thrallherd adds \textit{psionic 
dominate }to her powers known (if she doesn't already know it) Once per day, she 
can manifest \textit{psionic dominate }at a reduced power point cost. The cost 
of \textit{psionic dominate }is reduced by the thrallherd's level, to a minimum 
of 1 power point. The effect of this power is still restricted by the thrallherd's 
manifester level.

\textbf{Greater Dominate (Ex):} At 7th level and higher, a thrallherd does not 
have to pay 2 additional power points when she augments \textit{psionic dominate 
}to affect animals, fey, giants, magical beasts, and monstrous humanoids. This 
reduced point cost does not increase the save DC of the power as if she had spent 
the additional power points.

\textbf{Superior Dominate (Ex): }At 9th level, a thrallherd does not have to pay 
4 additional power points when she augments \textit{psionic dominate }to affect 
aberrations, dragons, elementals, and outsiders (in addition to the creature types 
mentioned in the greater dominate ability). This reduced point cost does not increase 
the save DC of the power as if she had spent the additional power points.

\textbf{Twofold Master (Ex):} At 10th level, a thrallherd can add a second thrall 
to her herd. This second thrall's maximum level is equal to the thrallherd's level 
minus 2, even if her Leadership score would indicate a higher-level thrall.

\vspace{12pt}
{\LARGE{}WAR MIND}

\textbf{Hit Die:} d10.

\vspace{12pt}
\section*{\textbf{Requirements}}

To qualify to become a war mind, a character must fulfill all the following criteria.

\textbf{Alignment:} Any nonchaotic.

\textbf{Base Attack Bonus: }+3.

\textbf{Skills:} Knowledge (history) 2 ranks, Knowledge (psionics) 8 ranks.

\textbf{Psionics:} Must have a power point reserve of at least 1 power point.

\textbf{Special:} Must have had some instruction by another war mind.

\vspace{12pt}
\section*{\textbf{Class Skills}}

The war mind's class skills (and the key ability for each skill) are Auto hypnosis 
(Wis), Concentration (Con), Intimidate (Cha), Knowledge (history) (Int), Knowledge 
(psionics) (Int), and Psicraft (Int).

\textbf{Skill Points at Each Level:} 2 + Int modifier.

\vspace{12pt}
\begin{tabular}{|>{\raggedright}p{15pt}|>{\raggedright}p{18pt}|>{\raggedright}p{11pt}|>{\raggedright}p{14pt}|>{\raggedright}p{11pt}|>{\raggedright}p{80pt}|>{\raggedright}p{30pt}|>{\raggedright}p{21pt}|>{\raggedright}p{38pt}|}
\hline
\multicolumn{9}{|p{242pt}|}{T\textbf{able: The War Mind}}\tabularnewline
\hline
L\textbf{evel} & B\textbf{ase Attack Bonus} & F\textbf{ort Save} & R\textbf{ef 
Save} & W\textbf{ill Save} & S\textbf{pecial} & P\textbf{ower Points/Day} & P\textbf{owers 
Known} & M\textbf{aximum Power Level Known}\tabularnewline
\hline
1st & +1 & +2 & +2 & +0 & Chain of personal superiority +2 & 2 & 1 & 1st\tabularnewline
\hline
2nd & +2 & +3 & +3 & +0 & Chain of defensive posture +2 & 5 & 2 & 1st\tabularnewline
\hline
3rd & +3 & +3 & +3 & +1 & Enduring body (DR 1/---) & 9 & 2 & 1st\tabularnewline
\hline
4th & +4 & +4 & +4 & +1--- &  & 14 & 3 & 2nd\tabularnewline
\hline
5th & +5 & +4 & +4 & +1 & Sweeping strike & 20 & 3 & 2nd\tabularnewline
\hline
6th & +6 & +5 & +5 & +2 & Enduring body (DR 2/---) & 28 & 4 & 3rd\tabularnewline
\hline
7th & +7 & +5 & +5 & +2 & Chain of personal superiority +4 & 37 & 4 & 3rd\tabularnewline
\hline
8th & +8 & +6 & +6 & +2 & Chain of defensive posture +4 & 47 & 5 & 4th\tabularnewline
\hline
9th & +9 & +6 & +6 & +3 & Enduring body (DR 3/---) & 58 & 5 & 4th\tabularnewline
\hline
10th & +10 & +7 & +7 & +3 & Chain of overwhelming force & 70 & 6 & 5th\tabularnewline
\hline
\end{tabular}

\vspace{12pt}
\section*{\textbf{Class Features}}

All the following are class features of the war mind prestige class.

\textbf{Weapon and Armor Proficiency:} War minds gain no proficiency with any weapon 
or armor.

\textbf{Power Points/Day:} A war mind can manifest powers. His ability to manifest 
powers is limited by the power points he has available. His base daily allotment 
of power points is given on Table: The War Mind. In addition, he receives bonus 
power points per day if he has a high Wisdom score. His race may also provide bonus 
power points per day, as may certain feats and items. If a war mind has power points 
from a different class, those points are pooled together and usable to manifest 
powers from either class. Bonus power points from having a high ability score can 
be gained only for the character's highest psionic class.

\textbf{Powers Known:} A war mind chooses his powers from the psychic warrior power 
list. At 1st level, a war mind knows one psychic warrior power of your choice. 
At every even-numbered level higher than 1st, he learns one new power. A war mind 
can manifest any power that has a power point cost equal to or lower than his manifester 
level. The total number of powers a war mind can manifest per day is limited only 
by his daily power points.

A war mind simply knows his powers; they are ingrained in his mind. He does not 
need to prepare them (in the way that some spellcasters prepare their spells), 
though he must get a good night's sleep each day to regain all his spent power 
points.

The Difficulty Class for saving throws against war mind powers is 10 + the power's 
level + the war mind's Wisdom modifier.

\textbf{Maximum Power Level Known: }A war mind gains the ability to learn one 1st-level 
power when he takes his first level in the prestige class. As he attains each even-numbered 
level beyond 2nd, a war mind gains the ability to master more complex powers.

To learn or manifest a power, a war mind must have a Wisdom score of at least 10 
+ the power's level.

\textbf{Chain of Personal Superiority (Ex):} At 1st level, a war mind learns the 
first principle of warfare for the individual combatant: the ability to both deal 
punishment and take it. Calling upon inner reserves of knowledge and dedication, 
a war mind can provide himself with a +2 insight bonus to Strength and Constitution 
for up to 1 minute. A war mind can use this power three times per day. Activating 
this power is a free action. At 7th level, the insight bonus to Strength and Constitution 
improves to +4.

\textbf{Chain of Defensive Posture (Ex):} At 2nd level, a war mind learns the second 
principle of warfare for the individual combatant: the ability to avoid the enemy's 
counterattacks if that enemy is not immediately overwhelmed. Calling upon inner 
reserves of knowledge and dedication, a war mind can provide himself with a +2 
insight bonus to Armor Class for up to 1 minute. A war mind can use this power 
three times per day. Activating this power is a free action. At 8th level, the 
insight bonus to Armor Class improves to +4.

\textbf{Enduring Body (Ex):} At 3rd level, a war mind learns the third principle 
of warfare for the individual combatant: to unleash in oneself the spirit of the 
enduring body. The spirit of the ideal body transforms a war mind, granting him 
damage reduction 1/-. At 6th level, his damage reduction improves to 2/-. At 9th 
level, his damage reduction improves to 3/-.

\textbf{Sweeping Strike (Ex):} At 5th level, a war mind gains the ability to make 
great, sweeping swings with a melee weapon. On each melee attack a war mind makes, 
he can choose squares he threatens that are adjacent to each other, and his attacks 
apply to creatures in those two squares equally. A war mind can use this ability 
on any attack, even an attack of opportunity or a cleave attempt.

A war mind cannot use this ability if he has moved more than 10 feet since the 
end of his last turn. If a war mind drops one or both of his foes with a sweeping 
strike, he can attempt a cleave normally; however, he makes only one cleave attempt 
per sweeping strike, even if he drops more than one foe.

\textbf{Chain of Overwhelming Force (Su):} At 10th level, a war mind learns the 
fourth principle of warfare for the individual combatant: to discover the underlying 
violence of the world and deliver it in a perfectly executed attack.

The war mind taps into this underlying energy and apply it to a single attack, 
dealing an extra 10d6 points of damage. A war mind can use this power once per 
day. Activating this power is a free action. If the attack misses, the power is 
wasted.

\newpage

\end{document}
