%&pdfLaTeX
% !TEX encoding = UTF-8 Unicode
\documentclass{article}
\usepackage{ifxetex}
\ifxetex
\usepackage{fontspec}
\setmainfont[Mapping=tex-text]{STIXGeneral}
\else
\usepackage[T1]{fontenc}
\usepackage[utf8]{inputenc}
\fi
\usepackage{textcomp}

\usepackage{array}
\usepackage{amssymb}
\usepackage{fancyhdr}
\renewcommand{\headrulewidth}{0pt}
\renewcommand{\footrulewidth}{0pt}

\begin{document}

This material is Open Game Content, and is licensed for public use under the terms 
of the Open Game License v1.0a.

\subsubsection*{{\LARGE{}PSIONIC SKILLS}}

\vspace{12pt}
{\LARGE{}SKILL DESCRIPTIONS}

The skills below relate to the use of psionics. In addition to three new skills 
(Autohypnosis, Psicraft, and Use Psionic Device), a new category is provided for 
the Knowledge skill, and new uses are given for Concentration.

\vspace{12pt}
\begin{tabular}{|>{\raggedright}p{90pt}|>{\raggedright}p{202pt}|}
\hline
\multicolumn{2}{|p{293pt}|}{T\textbf{able: Skill Synergies}}\tabularnewline
\hline
5\textbf{ or more ranks in. . . } & G\textbf{ives a +2 bonus on. . .}\tabularnewline
\hline
Autohypnosis & Knowledge (psionics) checks\tabularnewline
\hline
Concentration & Autohypnosis checks\tabularnewline
\hline
Knowledge (psionics) & Psicraft\tabularnewline
\hline
Psicraft & Use Psionic Device checks involving power stones\tabularnewline
\hline
Use Psionic Device & Psicraft checks to address power stones\tabularnewline
\hline
\end{tabular}

\vspace{12pt}
AUTOHYPNOSIS (WIS; TRAINED ONLY)

You have trained your mind to gain mastery over your body and the mind's own deepest 
capabilities.

\textbf{Check:} The DC and the effect of a successful check depend on the task 
you attempt.

\begin{tabular}{|>{\raggedright}p{98pt}|>{\raggedright}p{13pt}|}
\hline
\section*{T\textbf{ask}} & \section*{D\textbf{C}}\tabularnewline
\hline
Ignore caltrop wound & 18\tabularnewline
\hline
Memorize & 15\tabularnewline
\hline
Resist dying & 20\tabularnewline
\hline
Resist fear Fear effect & DC\tabularnewline
\hline
Tolerate poison Poison's & DC\tabularnewline
\hline
Willpower & 20\tabularnewline
\hline
\end{tabular}

\textit{Ignore Caltrop Wound: }If you are wounded by stepping on a caltrop, your 
speed is reduced to one-half normal. A successful Autohypnosis check removes this 
movement penalty. The wound doesn't go away---it is just ignored through self-persuasion.

\textit{Memorize: }You can attempt to memorize a long string of numbers, a long 
passage of verse, or some other particularly difficult piece of information (but 
you can't memorize magical writing or similarly exotic scripts). Each successful 
check allows you to memorize a single page of text (up to 800 words), numbers, 
diagrams, or sigils (even if you don't recognize their meaning). If a document 
is longer than one page, you can make additional checks for each additional page. 
You always retain this information; however, you can recall it only with another 
successful Autohypnosis check.

\textit{Resist Dying: }You can attempt to subconsciously prevent yourself from 
dying. If you have negative hit points and are losing hit points (at 1 per round, 
1 per hour), you can substitute a DC 15 Autohypnosis check for your d\% roll to 
see if you become stable. If the check is successful, you stop losing hit points 
(you do not gain any hit points, however, as a result of the check). You can substitute 
this check for the d\% roll in later rounds if you are initially unsuccessful.

\textit{Resist Fear: }In response to any fear effect, you make a saving throw normally. 
If you fail the saving throw, you can make an Autohypnosis check on your next round 
even while overcome by fear. If your autohypnosis check meets or beats the DC for 
the fear effect, you shrug off the fear. On a failed check, the fear affects you 
normally, and you gain no further attempts to shrug off that particular fear effect.

\textit{Tolerate Poison: }You can choose to substitute an Autohypnosis check for 
a saving throw against any standard poison's secondary damage or effect. This skill 
has no effect on the initial saving throw against poison.

\textit{Willpower: }If reduced to 0 hit points (disabled), you can make an Autohypnosis 
check. If successful, you can take a normal action while at 0 hit points without 
taking 1 point of damage. You must make a check for each strenuous action you want 
to take. A failed Autohypnosis check in this circumstance carries no direct penalty---you 
can choose not to take the strenuous action and thus avoid the hit point loss. 
If you do so anyway, you drop to -1 hit points, as normal when disabled.

\textbf{Action:} None. Making an Autohypnosis check doesn't require an action; 
it is either a free action (when attempted reactively) or part of another action 
(when attempted actively).

\textbf{Try Again: }Yes, for memorize and willpower uses, though a success doesn't 
cancel the effects of a previous failure. No for the other uses.

\textbf{Synergy: }If you have 5 or more ranks in Autohypnosis, you get a +2 bonus 
on Knowledge (psionics) checks.

If you have 5 or more ranks in Concentration, you get a +2 bonus on Autohypnosis 
checks.

\vspace{12pt}
CONCENTRATION{\large{} }(CON)

You are particularly good at focusing your mind. 

The following are additional uses for the concentration skill.

\textbf{Check:} You must make a Concentration check whenever you might potentially 
be distracted (by taking damage, by harsh weather, and so on) while engaged in 
some action that requires your full attention. Such actions include manifesting 
a power, concentrating on an active power, directing a power, or using a psi-like 
ability.

If the Concentration check succeeds, you can continue with the action as normal. 
If the check fails, the action automatically fails and is wasted. If you were in 
the process of manifesting a power, the power points are lost. If you were concentrating 
on an active power, the power ends as if you had ceased concentrating on it. If 
you were directing a power, the direction fails but the power remains active. If 
you were using a psi-like ability, that use of the ability is lost.

The table below summarizes various types of distractions that cause you to make 
a Concentration check. If the distraction occurs while you are trying to manifest 
a power, you must add the level of the power you are trying to manifest to the 
appropriate Concentration DC.

\vspace{12pt}
\begin{tabular}{|>{\raggedright}p{144pt}|>{\raggedright}p{182pt}|}
\hline
C\textbf{oncentration DC}\textsuperscript{\textbf{1}} & D\textbf{istraction}\tabularnewline
\hline
10 + damage dealt & Damaged during the action.\textsuperscript{\textbf{2}}\tabularnewline
\hline
10 + half of continuous damage last dealt & Taking continuous damage during the 
action.\textsuperscript{\textbf{3}}\tabularnewline
\hline
15 + power level  & Attempting to manifest a power without its display.\tabularnewline
\hline
15 & Entangled.\tabularnewline
\hline
Distracting power's save DC & Distracted by nondamaging power.\textsuperscript{\textbf{4}}\tabularnewline
\hline
20 & Gain psionic focus.\tabularnewline
\hline
20 & Grappling or pinned. (You can manifest powers normally unless you fail your 
Concentration check.)\tabularnewline
\hline
Distracting power's save DC & Weather caused by power\textsuperscript{\textbf{4}}\tabularnewline
\hline
\multicolumn{2}{|p{326pt}|}{1 If you are trying to manifest, concentrate on, or 
direct a power when the distraction occurs, add the level of the power to the indicated 
DC.}\tabularnewline
\hline
\multicolumn{2}{|p{326pt}|}{2 Such as during the manifestation of a power with 
a manifesting time of 1 round or more. Also from an attack of opportunity or readied 
attack made in response to the power being manifested (for powers with a manifesting 
time of 1 action) or the action being taken (for activities requiring no more than 
a full-round action).}\tabularnewline
\hline
\multicolumn{2}{|p{326pt}|}{3 Such as from standing in natural fire or lava.}\tabularnewline
\hline
\multicolumn{2}{|p{326pt}|}{4 If the power allows no save, use the save DC it would 
have if it did allow a save.}\tabularnewline
\hline
\end{tabular}

\vspace{12pt}
\textit{Gain Psionic Focus: }Merely holding a reservoir of psionic power points 
in mind gives psionic characters a special energy. Psionic characters can put that 
energy to work without actually paying a power point cost---they can become psionically 
focused as a special use of the Concentration skill.

If you have 1 or more power points available, you can meditate to attempt to become 
psionically focused. The DC to become psionically focused is 20. Meditating is 
a full-round action that provokes attacks of opportunity. When you are psionically 
focused, you can expend your focus on any single Concentration check you make thereafter. 
When you expend your focus in this manner, your Concentration check is treated 
as if you rolled a 15. It's like taking 10, except that the number you add to your 
Concentration modifier is 15. You can also expend your focus to gain the benefit 
of a psionic feat---many psionic feats are activated in this way.

Once you are psionically focused, you remain focused until you expend your focus, 
become unconscious, or go to sleep (or enter a meditative trance, in the case of 
elans), or until your power point reserve drops to 0.

\textbf{Action:} Usually none. In most cases, making a Concentration check doesn't 
require an action; it is either a free action (when attempted reactively) or part 
of another action (when attempted actively). Meditating to gain psionic focus is 
a full-round action.

\textbf{Try Again:} Yes, though a success doesn't cancel the effects of a previous 
failure, such as the loss of the power points for a power being manifested or the 
disruption of a power being concentrated on.

\textbf{Special:} You can use Concentration to manifest a power or use a psi-like 
ability defensively, so as to avoid attacks of opportunity altogether. The DC of 
the check is 15 + the power's level. If the Concentration check succeeds, you can 
manifest normally without provoking any attacks of opportunity. If the Concentration 
check fails, the power also automatically fails and the power points are wasted, 
just as if your concentration had been disrupted by a distraction.

A character with the Combat Manifestation feat gets a +4 bonus on Concentration 
checks made to manifest a power or use a psi-like ability while on the defensive 
or while grappling or pinned.

\textbf{Synergy:} If you have 5 or more ranks in Concentration, you get a +2 bonus 
on Autohypnosis checks.

\vspace{12pt}
KNOWLEDGE (PSIONICS){\large{} }(INT)

Like the Craft and Profession skills, Knowledge actually encompasses a number of 
unrelated skills. This entry specifically relates to the body of lore dealing with 
the phenomena of psionics in all its many manifestations.

Knowledge (psionics) covers ancient mysteries, psionic traditions, psychic symbols, 
cryptic phrases, astral constructs, and psionic races. You can use this skill to 
identify psionic monsters and their special powers or vulnerabilities.

\textbf{Synergy:} If you have 5 or more ranks in Knowledge (psionics), you get 
a +2 bonus on Psicraft checks.

If you have 5 or more ranks in Autohypnosis, you get a +2 bonus on Knowledge (psionics) 
checks.

\textbf{Untrained:} An untrained Knowledge (psionics) check is simply an Intelligence 
check. Without actual training, you know only common knowledge (DC 10 or lower).

\vspace{12pt}
PSICRAFT (INT; TRAINED ONLY)

Use this skill to identify powers as they are manifest or powers already in place.

\textbf{Check:} You can identify powers and psionic effects. The DCs for Psicraft 
checks relating to various tasks are summarized on the table below.

\vspace{12pt}
\begin{tabular}{|>{\raggedright}p{50pt}|>{\raggedright}p{276pt}|}
\hline
P\textbf{sicraft DC } & T\textbf{ask}\tabularnewline
\hline
15 + power level  & Identify a power being manifested. (You must sense the power's 
display, or see some visible effect, to identify a power.) No action required. 
No retry.\tabularnewline
\hline
15 + power level & When manifesting detect psionics, determine the discipline involved 
in the aura of a single item or creature you can see. (If the aura is not a power 
effect, the DC is 15 + 1/2 manifester level.) No action required.\tabularnewline
\hline
15 + power level & Address a power stone to figure out what power or powers it 
contains.\tabularnewline
\hline
20 + power level & Identify a power that's already in place and in effect. You 
must be able to see or detect the effects of the power. No action required. No 
retry.\tabularnewline
\hline
20 + power level & Identify materials created or shaped by psionics, such as noting 
that a particular object was created using a metacreativity power. No action required. 
No retry.\tabularnewline
\hline
25 + power level & After rolling a saving throw against a power targeted on you, 
determine what that power was. No action required. No retry.\tabularnewline
\hline
25 & Identify a psionic tattoo. Requires 1 minute. No retry.\tabularnewline
\hline
20 & Draw a diagram to enhance manifestation of psionic dimensional anchor on a 
summoned creature. Requires 10 minutes. No retry. The player does not see the result 
of this check.\tabularnewline
\hline
30 or higher & Understand a strange or unique psionic effect, such as the effects 
of an outcrop of psionically resonant crystal. Time required varies. No retry.\tabularnewline
\hline
\end{tabular}

\vspace{12pt}
Additionally, certain powers allow you to gain information about psionic effects, 
provided that you make a successful Psicraft check as detailed in the power description.

\textbf{Action: }Varies, as noted above. 

\textbf{Try Again:} See above.

\textbf{Special:} A psion gains a +2 bonus on Psicraft checks when dealing with 
a power or effect from his discipline.

If you have the Psionic Affinity feat, you get a +2 bonus on Psicraft checks.

\textbf{Synergy: }If you have 5 or more ranks in Psicraft, you get a +2 bonus on 
Use Psionic Device checks related to power stones.

If you have 5 or more ranks in Use Psionic Device, you get a +2 bonus on Psicraft 
checks to address power stones.

\vspace{12pt}
USE PSIONIC DEVICE (CHA; TRAINED ONLY)

Use this skill to activate psionic devices, including power stones (chunks of crystal 
that store specific powers) and  dorjes (slender crystal wands charged with several 
uses of the same power), that otherwise you could not activate.

\textbf{Check:} You can use this skill to address a power stone (to learn what 
powers are encoded on it) or to activate a psionic item. This skill lets you use 
a psionic item as if you had the manifesting ability or class features of another 
class, as if you were a different race, or as if you were a different alignment.

You make Use Psionic Device checks each time you activate a device such as a dorje. 
If you are using the check to emulate an alignment or some other quality in an 
ongoing manner, you need to make the relevant emulation checks once per hour.

You must consciously choose what to emulate. That is, you must know what you are 
trying to emulate when you make an emulation check. The DCs for various tasks involving 
Use Psionic Device are summarized on the table below.

\vspace{12pt}
\begin{tabular}{|>{\raggedright}p{95pt}|>{\raggedright}p{95pt}|}
\hline
U\textbf{se Psionic Device DC} & T\textbf{ask}\tabularnewline
\hline
25 & Activate blindly\tabularnewline
\hline
25 + power level & Address a power stone\tabularnewline
\hline
See text & Emulate an ability score\tabularnewline
\hline
30 & Emulate an alignment\tabularnewline
\hline
20 & Emulate a class feature\tabularnewline
\hline
25 & Emulate a race\tabularnewline
\hline
20 & Use a dorje\tabularnewline
\hline
20 + manifester level & Use a power stone\tabularnewline
\hline
\end{tabular}

\vspace{12pt}
\textit{Activate Blindly: }Some psionic items are activated by special specific 
thoughts or conceptions. You can activate such items as if you were using the activation 
method, even if you're not and even if you don't know it. You do have to use something 
equivalent. You have to wave the item around or otherwise attempt to get it to 
activate. You get a special +2 bonus if you've activated the item at least once 
before.

If you fail the check by 10 or more, you suffer brainburn. This brainburn affects 
you in the same way as brainburn that can occur when you attempt to manifest a 
power from a power stone, except that the damage is 1d4 points per power level 
instead of 1d6. Brainburn damage from activating blindly is in addition to brainburn 
damage from manifesting a power from a power stone.

\textit{Address a Power Stone: }Successfully addressing a power stone allows you 
to find out what power or powers it contains. Doing this requires 1 minute of concentration.

\textit{Emulate an Ability Score: }To manifest a power from a power stone, you 
need a high ability score in the appropriate ability. Your effective ability score 
(appropriate to the class you're emulating when you try to manifest the power from 
the power stone) is your check result minus 15. If you already have a high enough 
score in the appropriate ability, you don't need to make this check.

\textit{Emulate an Alignment: }Some psionic items have positive or negative effects 
based on your alignment. Use Psionic Device lets you use these items as if you 
were of an alignment of your choice. You can emulate only one alignment at a time. 

\textit{Emulate a Class Feature: }Sometimes you need to use a class feature to 
activate a psionic item. Your effective level in the emulated class equals your 
check result minus 20. This skill does not let you use the class feature of another 
class. It just lets you activate items as if you had the class feature.

If the class whose feature you are emulating has an alignment requirement, you 
must meet it, either honestly or by emulating an appropriate alignment as a separate 
check (see above).

\textit{Emulate a Race: }Some psionic items work only for certain races, or work 
better for those of certain races. You can use such an item as if you were a race 
of your choice. You can emulate only one race at a time.

\textit{Use a Dorje: }Normally, to use a dorje, you must have the dorje's power 
on your class power list. This use of the skill allows you to use a dorje as if 
you had a particular power on your class power list. This use of the skill applies 
to other power trigger psionic items, if applicable.

\textit{Use a Power Stone: }Normally, to manifest a power from a power stone, you 
must have the power stone's power on your class power list. This use of the skill 
allows you to use a power stone as if you had a particular power on your class 
power list. The DC is equal to 20 + the manifester level of the power you are trying 
to manifest from the power stone. \textit{Note: }Before you use

a power stone, you must first have addressed it to determine what powers it contains. 
In addition, manifesting a power from a power stone requires a minimum score (10 
+ power level) in the appropriate ability. If you don't have a high enough score, 
you must emulate the ability score with a separate check (see above). This use 
of the skill applies to other power completion

psionic items.

\textbf{Action:} None. The Use Psionic Device check is made as part of the action 
(if any) required to activate the psionic item.

\textbf{Try Again:} Yes, but if you ever roll a natural 1 while attempting to activate 
an item and you fail, you can't try to activate it again for a day.

\textbf{Special:} You cannot take 10 with this skill.

You can't aid another on Use Psionic Device checks. Only the user of the item can 
attempt such a check.

A character with the Psionic Affinity feat gets a +2 bonus on Use Psionic Device 
checks.

\textbf{Synergy: }If you have 5 or more ranks in Psicraft, you get a +2 bonus on 
Use Magic Device checks related to power stones.

If you have 5 or more ranks in Use Psionic Device, you get a +2 bonus on Psicraft 
checks to address power stones.

\newpage

\end{document}
