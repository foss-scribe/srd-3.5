%&pdfLaTeX
% !TEX encoding = UTF-8 Unicode
\documentclass{article}
\usepackage{ifxetex}
\ifxetex
\usepackage{fontspec}
\setmainfont[Mapping=tex-text]{STIXGeneral}
\else
\usepackage[T1]{fontenc}
\usepackage[utf8]{inputenc}
\fi
\usepackage{textcomp}

\usepackage{array}
\usepackage{amssymb}
\usepackage{fancyhdr}
\renewcommand{\headrulewidth}{0pt}
\renewcommand{\footrulewidth}{0pt}

\begin{document}

This material is Open Game Content, and is licensed for public use under the terms 
of the Open Game License v1.0a.

\subsubsection*{{\LARGE{}PSIONIC POWERS OVERVIEW}}

\vspace{12pt}
Psionic powers spring from sentient minds. Even an undead creature or a being that 
has no physical form can create a reserve of inner strength necessary to manifest 
powers, as long as it has an Intelligence score of at least 1. Vermin possessed 
of a hive mind ability are an exception to this rule.

A psionic power is a one-time psionic effect. Psionic characters and creatures 
need not prepare their powers for use ahead of time. They either have sufficient 
power points to manifest a power or they do not.

A power is manifested when a psionic character pays its power point cost. Some 
psionic creatures automatically manifest powers, called psi-like abilities, without 
paying a power point cost. Other creatures pay power points to manifest their powers, 
just as characters do.

\vspace{12pt}
Each power has a specific effect. A power known to a psionic character can be used 
whenever he or she has power points to pay for it.

\vspace{12pt}
MANIFESTING POWERS

Psionic characters and creatures manifest powers. Whether they cost power points 
when manifest by a psionic character, or are manifested as psi-like abilities, 
powers' effects remain the same. The process of manifesting a power is akin to 
casting a spell, but with significant differences.

\vspace{12pt}
CHOOSING A POWER

First you must choose which power to manifest. You can select any power you know, 
provided you are capable of manifesting powers of that level or higher. To manifest 
a power, you must pay power points, which count against your daily total. You can 
manifest the same power multiple times if you have points left to pay for it.

\vspace{12pt}
CONCENTRATION

To manifest a power, you must concentrate. If something threatens to interrupt 
your concentration while you're manifesting a power, you must succeed on a Concentration 
check or lose the power points without manifesting the power. The more distracting 
the interruption and the higher the level of the power that you are trying to manifest, 
the higher the DC. (Higher-level powers require more mental effort.)

\textbf{Injury:} Getting hurt or being affected by hostile psionics while trying 
to manifest a power can break your concentration and ruin a power. If you take 
damage while trying to manifest a power, you must make a Concentration check (DC 
10 + points of damage taken + the level of the power you're manifesting). The interrupting 
event strikes during manifestation if it occurs between when you start and when 
you complete manifesting a power (for a power with a manifesting time of 1 round 
or longer) or if it comes in response to your manifesting the power (such as an 
attack of opportunity provoked by the manifesting of the power or a contingent 
attack from a readied action).

If you are taking continuous damage half the damage is considered to take place 
while you are manifesting a power. You must make a Concentration check (DC 10 + 
1/2 the damage that the continuous source last dealt + the level of the power you're 
manifesting).

If the last damage dealt was the last damage that the effect could deal then the 
damage is over, and it does not distract you.

Repeated damage does not count as continuous damage.

\textbf{Power:} If you are affected by a power while attempting to manifest a power 
of your own, you must make a Concentration check or lose the power you are manifesting. 
If the power affecting you deals damage, the Concentration DC is 10 + points of 
damage + the level of the power you're manifesting. If the power interferes with 
you or distracts you in some other way, the Concentration DC is the power's save 
DC + the level of the power you're manifesting. For a power with no saving throw, 
it's the DC that the power's saving throw would have if a save were allowed.

\textbf{Grappling or Pinned:} To manifest a power while grappling or pinned, you 
must make a Concentration check (DC 20 + the level of the power you're manifesting) 
or lose the power.

\textbf{Vigorous Motion: }If you are riding on a moving mount, taking a bouncy 
ride in a wagon, on a small boat in rough water, belowdecks in a storm-tossed ship, 
or simply being jostled in a similar fashion, you must make a Concentration check 
(DC 10 + the level of the power you're manifesting) or lose the power.

\textbf{Violent Motion: }If you are on a galloping horse, taking a very rough ride 
in a wagon, on a small boat in rapids or in a storm, on deck in a storm-tossed 
ship, or being tossed roughly about in a similar fashion, you must make a Concentration 
check (DC 15 + the level of the power you're manifesting) or lose the power.

\textbf{Violent Weather: }If you are in a high wind carrying blinding rain or sleet, 
the DC is 5 + the level of the power you're manifesting. If you are in wind-driven 
hail, dust, or debris, the DC is 10 + the level of the power you're manifesting. 
In either case, you lose the power if you fail the Concentration check. If the 
weather is caused by a power, use the rules in the Power subsection above.

\textbf{Manifesting Powers on the Defensive:} If you want to manifest a power without 
provoking attacks of opportunity, you need to dodge and weave. You must make a 
Concentration check (DC 15 + the level of the power you're manifesting) to succeed. 
You lose the power points without successful manifestation if you fail.

\textbf{Entangled:} If you want to manifest a power while entangled in a net or 
while affected by a power with similar effects you must make a DC 15 Concentration 
check to manifest the power. You lose the power if you fail.

\vspace{12pt}
MANIFESTER LEVEL

The variables of a power's effect often depend on its manifester level, which is 
equal to your psionic class level. A power that can be augmented for additional 
effect is also limited by your manifester level (you can't spend more power points 
on a power than your manifester level). See Augment under Descriptive Text, below.

You can manifest a power at a lower manifester level than normal, but the manifester 
level must be high enough for you to manifest the power in question, and all level-dependent 
features must be based on the same manifester level.

In the event that a class feature or other special ability provides an adjustment 
to your manifester level, this adjustment applies not only to all effects based 
on manifester level (such as range, duration, and augmentation potential) but also 
to your manifester level check to overcome your target's power resistance and to 
the manifester level used in dispel checks (both the dispel check and the DC of 
the check).

\vspace{12pt}
POWER FAILURE

If you try to manifest a power in conditions where the characteristics of the power 
(range, area, and so on) cannot be made to conform, the manifestation fails and 
the power points are wasted. 

Powers also fail if your concentration is broken (see Concentration, above).

\vspace{12pt}
THE POWER'S RESULT

Once you know which creatures (or objects or areas) are affected, and whether those 
creatures have made successful saving throws (if any were allowed), you can apply 
whatever results a power entails.

\vspace{12pt}
SPECIAL POWER EFFECTS

Certain special features apply to all powers.

\textbf{Attacks:} Some powers refer to attacking. All offensive combat actions, 
even those that don't damage opponents, such as disarm and bull rush, are considered 
attacks. All powers that opponents can resist with saving throws, that deal damage, 
or that otherwise harm or hamper subjects are considered attacks. \textit{Astral 
construct }and similar powers are not considered attacks because the powers themselves 
don't harm anyone.

\textbf{Bonus Types:} Many powers give creatures bonuses to ability scores, Armor 
Class, attacks, and other attributes. Each bonus has a type that indicates how 
the power grants the bonus. The important aspect of bonus types is that two bonuses 
of the same type don't generally stack. With the exception of dodge bonuses, most 
circumstance bonuses, and racial bonuses, only the better bonus works (see Combining 
Psionic and Magical Effects, below). The same principle applies to penalties---a 
character taking two or more penalties of the same type applies only the worst 
one.

\textbf{Bringing Back the Dead:} Of all the psionic powers, only \textit{reality 
revision }has the ability to restore slain characters to life. When a living creature 
dies, its soul departs the body, leaves the Material Plane, travels through the 
Astral Plane, and goes to abide on the plane where the creature's deity resides. 
If the creature did not worship a deity, its soul departs to the plane corresponding 
to its alignment. Bringing someone back from the dead means retrieving his or her 
soul and returning it to his or her body.

\textit{Level Loss: }The passage from life to death and back again is a wrenching 
journey for a being's soul. Consequently, any creature brought back to life usually 
loses one level of experience. The character's new experience point total is midway 
between the minimum needed for his or her new (reduced) level and the minimum needed 
for the next one. If the character was 1st level at the time of death, he or she 
loses 2 points of Constitution instead of losing a level. This level loss or Constitution 
loss cannot be repaired by any mortal means, even the spells \textit{wish }or \textit{miracle}. 
A revived character can regain a lost level by earning XP through further adventuring. 
A revived character who was 1st level at the time of death can regain lost points 
of Constitution by improving his or her Constitution score when he or she attains 
a level that allows an ability score increase.

\textit{Preventing Revivif cation: }Enemies can take steps to make it more difficult 
for a character to be returned from the dead. Keeping the body prevents others 
from using a single manifestation of \textit{reality revision }to restore the slain 
character to life.

\textit{Revivif cation Against One's Will: }A soul cannot be returned to life if 
it does not wish to be. A soul knows the name, alignment, and patron deity (if 
any) of the character attempting to revive it and may refuse to return on that 
basis.

\vspace{12pt}
COMBINING PSIONIC AND MAGICAL EFFECTS

The default rule for the interaction of psionics and magic is simple: Powers interact 
with spells and spells interact with powers in the same way a spell or normal spell-like 
ability interacts with another spell or spell-like ability. This is known as psionics-magic 
transparency.

\textbf{Psionics-Magic Transparency:} Though not explicitly called out in the spell 
descriptions or magic item descriptions\textit{, }spells, spell-like abilities, 
and magic items that could potentially affect psionics do affect psionics. 

When the rule about psionics-magic transparency is in effect, it has the following 
ramifications.

Spell resistance is effective against powers, using the same mechanics. Likewise, 
power resistance is effective against spells, using the same mechanics as spell 
resistance. If a creature has one kind of resistance, it is assumed to have the 
other. (The effects have similar ends despite having been brought about by different 
means.)

All spells that dispel magic have equal effect against powers of the same level 
using the same mechanics, and vice versa.

The spell \textit{detect magic }detects powers, their number, and their strength 
and location within 3 rounds (though a Psicraft check is necessary to identify 
the discipline of the psionic aura).

Dead magic areas are also dead psionics areas.

\textbf{Multiple Effects:} Powers or psionic effects usually work as described 
no matter how many other powers, psionic effects, spells, or magical effects happen 
to be operating in the same area or on the same recipient. Except in special cases, 
a power does not affect the way another power or spell operates. Whenever a power 
has a specific effect on other powers or spells, the power description explains 
the effect (and vice versa for spells that affect powers). Several other general 
rules apply when powers, spells, magical effects, or psionic effects operate in 
the same place.

\textbf{Stacking Effects:} Powers that provide bonuses or penalties on attack rolls, 
damage rolls, saving throws, and other attributes usually do not stack with themselves. 
More generally, two bonuses of the same type don't stack even if they come from 
different powers, or one from a power and one from a spell. You use whichever bonus 
gives you the better result. 

\textit{Different Bonus Types}: The bonuses or penalties from two different powers, 
or a power and a spell, stack if the effects are of different types. A bonus that 
isn't named (just a ``+2 bonus'' rather than a ``+2 insight bonus'') stacks with 
any bonus.

\textit{Same Effect More than Once in Different Strengths}: In cases when two or 
more similar or identical effects are operating in the same area or on the same 
target, but at different strengths, only the best one applies. If one power or 
spell is dispelled or its duration runs out, the other power or spell remains in 
effect (assuming its duration has not yet expired).

\textit{Same Effect with Differing Results}: The same power or spell can sometimes 
produce varying effects if applied to the same recipient more than once. The last 
effect in a series trumps the others. None of the previous spells or powers are 
actually removed or dispelled, but their effects become irrelevant while the final 
spell or power in the series lasts.

\textit{One Effect Makes Another Irrelevant: }Sometimes, a power can render another 
power irrelevant.

\textit{Multiple Mental Control Effects}: Sometimes psionic or magical effects 
that establish mental control render one another irrelevant. Mental controls that 
don't remove the recipient's ability to act usually do not interfere with one another, 
though one may modify another. If a creature is under the control of two or more 
creatures, it tends to obey each to the best of its ability, and to the extent 
of the control each effect allows. If the controlled creature receives conflicting 
orders simultaneously, the competing controllers must make opposed Charisma checks 
to determine which one the creature obeys.

\textbf{Powers and Spells with Opposite Effects:} Powers and spells with opposite 
effects apply normally, with all bonuses, penalties, or changes accruing in the 
order that they apply. Some powers and spells negate or counter each other. This 
is a special effect that is noted in a power's or spell's description.

\textbf{Instantaneous Effects:} Two or more magical or psionic effects with instantaneous 
durations work cumulatively when they affect the same object, place, or creature. 

\vspace{12pt}
{\LARGE{}POWERS AND POWER POINTS}

Psionic characters manifest powers, which involve the direct manipulation of personal 
mental energy. These manipulations require natural talent and personal meditation. 
A psionic character's level limits the number of power points available to manifest 
powers. A psionic character's relevant high score might allow him to gain extra 
power points. He can manifest the same power more than once, but each manifestation 
subtracts power points from his daily limit. Manifesting a power is an arduous 
mental task. To do so, a psionic character must have a key ability score of at 
least 10 + the power's level.

\textbf{Daily Power Point Acquisition:} To regain used daily power points, a psionic 
character must have a clear mind. To clear his mind, he must first sleep for 8 
hours. The character does not have to slumber for every minute of the time, but 
he must refrain from movement, combat, manifesting powers, skill use, conversation, 
or any other demanding physical or mental task during the rest period. If his rest 
is interrupted, each interruption adds 1 hour to the total amount of time he has 
to rest to clear his mind, and he must have at least 1 hour of rest immediately 
prior to regaining lost power points. If the character does not need to sleep for 
some reason, he still must have 8 hours of restful calm before regaining power 
points.

\textbf{Recent Manifesting Limit/Rest Interruptions:} If a psionic character has 
manifested powers recently, the drain on his resources reduces his capacity to 
regain power points. When he regains power points for the coming day, all power 
points he has used within the last 8 hours count against his daily limit.

\textbf{Peaceful Environment: }To regain power points, a psionic character must 
have enough peace, quiet, and comfort to allow for proper concentration. The psionic 
character's surroundings need not be luxurious, but they must be free from overt 
distractions, such as combat raging nearby or other loud noises. Exposure to inclement 
weather prevents the necessary concentration, as does any injury or failed saving 
throw the character might incur while concentrating on regaining power points.

\textbf{Regaining Power Points:} Once the character has rested in a suitable environment, 
it takes only an act of concentration spanning 1 full round to regain all power 
points of the psionic character's daily limit. 

\textbf{Death and Power Points:} If a character dies, all daily power points stored 
in his mind are wiped away. A potent effect (such as \textit{reality revision}) 
can recover the lost power points when it recovers the character.

\vspace{12pt}
ADDING POWERS

Psionic characters can learn new powers when they attain a new level. A psion can 
learn any power from the psion/wilder list and powers from his chosen discipline's 
list. A wilder can learn any power from the psion/wilder list. A psychic warrior 
can learn any power from the psychic warrior list. 

\textbf{Powers Gained at a New Level: }Psions and other psionic characters perform 
a certain amount of personal meditation between adventures in an attempt to unlock 
latent mental abilities. Each time a psionic character attains a new level, he 
or she learns additional powers according to his class description. Psions, psychic 
warriors, and wilders learn new powers of their choice in this fashion. These powers 
represent abilities unlocked from latency. The powers must be of levels the characters 
can manifest.

\textbf{Independent Research:} A psion also can research a power independently, 
duplicating an existing power or creating an entirely new one. If characters are 
allowed to develop new powers, use these guidelines to handle the situation.

Any kind of manifester can create a new power. The research involved requires access 
to a retreat conducive to uninterrupted meditation. Research involves an expenditure 
of 200 XP per week and takes one week per level of the power. At the end of that 
time, the character makes a Psicraft check (DC 10 + spell level). If that check 
succeeds, the character learns the new power if her research produced a viable 
power. If the check fails, the character must go through the research process again 
if she wants to keep trying.

\section*{\textbf{Manifest an Unknown Power from Another's Powers Known}}

A psionic character can attempt to manifest a power from a source other than his 
own knowledge (usually a power stone or another willing psionic character). To 
do so, the character must first make contact (a process similar to addressing a 
power stone, requiring a Psicraft check against a DC of 15+ the level of the power 
to be manifested). A psionic character can make contact with only a willing psionic 
character or creature (unconscious creatures are considered willing, but not psionic 
characters under the effects of other immobilizing conditions). Characters who 
can't use power stones for any reason are also banned from attempting to manifest 
powers from the knowledge of other psionic characters. Mental contact requires 
1 full round of physical contact, which can provoke attacks of opportunity. Once 
contact is achieved, the character becomes aware of all the powers stored in the 
power stone or all the powers the other character knows up to the highest level 
of power the contactor knows himself. 

Next, the psionic character must choose one of the powers and make a second Psicraft 
check (DC 15 + the power's level) to see if he understands it. If the power is 
not on his class list, he automatically fails this check.

Upon successfully making contact with another willing psionic character or creature 
and learning what he can of one power in particular, the character can immediately 
attempt to manifest that power even if he doesn't know it (and assuming he has 
power points left for the day). He can attempt to manifest the power normally on 
his next turn, and he succeeds if he makes one additional Psicraft check (DC 15 
+ the power's level). He retains the ability to manifest the selected power for 
only 1 round. If he doesn't manifest the power, fails the Psicraft check, or manifests 
a different power, he loses his chance to manifest that power for the day.

\vspace{12pt}
USING STORED POWER POINTS

A variety of psionic items exist to store power points for later use, in particular 
a storage device called a \textit{cognizance crystal}. Regardless of what sort 
of item stores the power points, all psionic characters must follow strict rules 
when tapping stored power points.

\textbf{A Single Source: }When using power points from a storage item to manifest 
a power, a psionic character may not pay the power's cost with power points from 
more than one source. He must either use an item, his own power point reserve, 
or some other discrete power point source to pay the manifestation cost. 

\textbf{Recharging: }Most power point storage devices allow psionic characters 
to ``recharge'' the item with their own power points. Doing this depletes the character's 
power point reserve on a 1-for-1 basis as if he had manifested a power; however, 
those power points remain indefinitely stored. The opposite is not true---psionic 
characters may not use power points stored in a storage item to replenish their 
own power point reserves.

\vspace{12pt}
{\LARGE{}SPECIAL ABILITIES}

Psionic creatures can create psionic effects without having levels in a psionic 
class (although they can take a psionic class to further enhance their abilities); 
such creatures have the psionic subtype.

Characters using dorjes, \textit{cognizance crystals, }and other psionic items 
can also create psionic effects. In addition to existing spell-like and supernatural 
abilities, creatures can also have psi-like abilities. (Psionic creatures may also 
have extraordinary and natural abilities.)

\textbf{Psi-Like Abilities (Ps):} The manifestation of powers by a psionic character 
is considered a psi-like ability, as is the manifestation of powers by creatures 
without a psionic class (creatures with the psionic subtype, also simply called 
psionic creatures). Usually, a psionic creature's psi-like ability works just like 
the power of that name. A few psi-like abilities are unique; these are explained 
in the text where they are described.

Psi-like abilities have no verbal, somatic, or material components, nor do they 
require a focus or have an XP cost (even if the equivalent power has an XP cost). 
The user activates them mentally. Armor never affects a psi-like ability's use. 
A psi-like ability has a manifesting time of 1 standard action unless noted otherwise 
in the ability description. In all other ways, a psi-like ability functions just 
like a power. However, a psionic creature does not have to pay a psi-like ability's 
power point cost.

Psi-like abilities are subject to power resistance and to being dispelled by \textit{dispel 
psionics}. They do not function in areas where psionics is suppressed or negated.

\textbf{Supernatural Abilities:} Some creatures have psionic abilities that are 
considered supernatural. Psionic feats are also supernatural abilities. These abilities 
cannot be disrupted in combat, as powers can be, and do not provoke attacks of 
opportunity (except as noted in their descriptions). Supernatural abilities are 
not subject to power resistance and cannot be negated or dispelled; however, they 
do not function in areas where psionics is suppressed\textit{.}

\vspace{12pt}
\section*{{\LARGE{}PSIONIC MALADIES}}

\textbf{Ability Burn:} This is a special form of ability damage that cannot be 
magically or psionically healed. It is caused by the use of certain psionic feats 
and powers. It returns only through natural healing.

\textbf{Disease, Cascade Flu:} Spread by brain moles and other vermin; injury; 
DC 13; incubation one day; damage psionic cascade.

A psionic cascade is a loss of control over psionic abilities. Using power points 
becomes dangerous for a character infected by cascade flu, once the incubation 
period has run its course. Every time an afflicted character manifests a power, 
she must make a DC 16 Concentration check. On a failed check, a psionic cascade 
is triggered. The power operates normally, but during the following round, without 
the character's volition, two additional powers she knows manifest randomly, and 
their power cost is deducted from the character's reserve. During the next round, 
three additional powers manifest, and so on, until all the psionic character's 
power points are drained. Powers with a range of personal or touch always affect 
the diseased character. For other powers that affect targets, roll d\%: On a 01-50 
result, the power affects the diseased character, and 51-00 indicates that the 
power targets other creatures in the vicinity. Psionic creatures (those that manifest 
their powers without paying points) cascade until all the powers they know have 
manifested at least twice.

As with any disease, a psionic character who is injured or attacked by a creature 
carrying a disease or parasite, or who otherwise has contact with contaminated 
material, must make an immediate Fortitude save. On a success, the disease fails 
to gain a foothold. On a failure, the character takes damage (or incurs the specified 
effect) after the incubation period. Once per day afterward, the afflicted character 
must make a successful Fortitude save to avoid repeating the damage. Two successful 
saving throws in a row indicate she has fought off the disease.

\textbf{Disease, Cerebral Parasites:} Spread by contact with infected psionic creatures; 
contact; DC 15; incubation 1d4 days; damage 1d8 power points. 

Cerebral parasites are tiny organisms, undetectable to normal sight. An afflicted 
character may not even know he carries the parasites---until he discovers he has 
fewer power points for the day than expected. Psionic creatures with cerebral parasites 
are limited to using each of their known powers only once per day (instead of freely 
manifesting them). See the note about diseases under Cascade Flu, above.

\textbf{Negative Levels:} Psionic characters can gain negative levels just like 
members of other character classes. A psionic character loses access to one power 
per negative level from the highest level of power he can manifest; he also loses 
a number of power points equal to the cost of that power. If two or more powers 
fit these criteria, the manifester decides which one becomes inaccessible. The 
lost power becomes available again as soon the negative level is removed, providing 
the manifester is capable of using it at that time. Lost power points also return.

\vspace{36pt}
{\LARGE{}POWER DESCRIPTIONS}

The description of each power is presented in a standard format. Each category 
of information is explained and defined below.

\vspace{12pt}
NAME

The first line of every power description gives the name by which the power is 
generally known. A power might be known by other names in some locales, and specific 
manifesters might have names of their own for their powers.

\vspace{12pt}
DISCIPLINE (SUBDISCIPLINE)

Beneath the power name is a line giving the discipline (and the subdiscipline in 
parentheses, if appropriate) that the power belongs to.

Every power is associated with one of six disciplines. A discipline is a group 
of related powers that work in similar ways. Each of the disciplines is discussed 
below.

\vspace{12pt}
Clairsentience

Clairsentience powers enable you to learn secrets long forgotten, to glimpse the 
immediate future and predict the far future, to find hidden objects, and to know 
what is normally unknowable.

For the purpose of psionics-magic transparency, clairsentience powers are equivalent 
to powers of the divination school (thus, creatures immune to divination spells 
are also immune to clairsentience powers).

Many clairsentience powers have cone-shaped areas. These move with you and extend 
in the direction you look. The cone defines the area that you can sweep each round. 
If you study the same area for multiple rounds, you can often gain additional information, 
as noted in the descriptive text for the power.

\textbf{Scrying:} A power of the scrying subdiscipline creates an invisible sensor 
that sends you information. Unless noted otherwise, the sensor has the same powers 
of sensory acuity that you possess. This includes any powers or effects that target 
you, but not powers or effects that emanate from you. However, the sensor is treated 
as a separate, independent sensory organ of yours, and thus functions normally 
even if you have been blinded, deafened, or otherwise suffered sensory impairment. 
Any creature with an Intelligence score of 12 or higher can notice the sensor by 
making a DC 20 Intelligence check. The sensor can be dispelled as if it were an 
active power. Lead sheeting or psionic protection blocks scrying powers, and you 
sense that the power is so blocked.

\vspace{12pt}
Metacreativity

Metacreativity powers create objects, creatures, or some form of matter. Creatures 
you create usually, but not always, obey your commands.

A metacreativity power draws raw ectoplasm from the Astral Plane to create an object 
or creature in the place the psionic character designates (subject to the limits 
noted above). Objects created in this fashion are as solid and durable as normal 
objects, despite their originally diaphanous substance. Psionic creatures created 
with metacreativity powers are considered constructs, not outsiders.

A creature or object brought into being cannot appear inside another creature or 
object, nor can it appear floating in an empty space. It must arrive in an open 
location on a surface capable of supporting it. The creature or object must appear 
within the power's range, but it does not have to remain within the range.

For the purpose of psionics-magic transparency, metacreativity powers are equivalent 
to powers of the conjuration school (thus, creatures immune to conjuration spells 
are also immune to metacreativity powers).

\textbf{Creation:} A power of the creation subdiscipline creates an object or creature 
in the place the manifester designates (subject to the limits noted above). If 
the power has a duration other than instantaneous, psionic energy holds the creation 
together, and when the power ends, the created creature or object vanishes without 
a trace, except for a thin film of glistening ectoplasm that quickly evaporates. 
If the power has an instantaneous duration, the created object or creature is merely 
assembled through psionics. It lasts indefinitely and does not depend on psionics 
for its existence.

\vspace{12pt}
Psychokinesis

Psychokinesis powers manipulate energy or tap the power of the mind to produce 
a desired end. Many of these powers produce spectacular effects above and beyond 
the power's standard display (see Display, below), such as moving, melting, transforming, 
or blasting a target. Psychokinesis powers can deal large amounts of damage.

For the purpose of psionics-magic transparency, psychokinesis powers are equivalent 
to powers of the evocation school (thus, creatures immune to evocation spells are 
also immune to psychokinesis powers).

\vspace{12pt}
Psychometabolism

Psychometabolism powers change the physical properties of some creature, thing, 
or condition. 

For the purpose of psionics- magic transparency, psychometabolism powers are equivalent 
to powers of the transmutation school (thus, creatures immune to transmutation 
spells are also immune to psychometabolism powers).

\textbf{Healing:} Psychometabolism powers of the healing subdiscipline can remove 
damage from creatures. However, psionic healing usually falls short of divine magical 
healing, in direct comparisons. 

\vspace{12pt}
Psychoportation

Psychoportation powers move the manifester, an object, or another creature through 
space and time.

For the purpose of psionics-magic transparency, psychoportation powers do not have 
an equivalent school.

\textbf{Teleportation:} A power of the teleportation subdiscipline transports one 
or more creatures or objects a great distance. The most potent of these powers 
can cross planar boundaries. Usually the transportation is one-way (unless otherwise 
noted) and not dispellable. Teleportation is instantaneous travel through the Astral 
Plane. Anything that blocks astral travel also blocks teleportation.

\vspace{12pt}
Telepathy

Telepathy powers can spy on and affect the minds of others, influencing or controlling 
their behavior.

Most telepathy powers are mind-affecting.

For the purpose of psionics-magic transparency, telepathy powers are equivalent 
to powers of the enchantment school (thus, creatures resistant to enchantment spells 
are equally resistant to telepathy powers).

\textbf{Charm:} A power of the charm subdiscipline changes the way the subject 
views you, typically making it see you as a good friend.

\textbf{Compulsion:} A power of the compulsion subdiscipline forces the subject 
to act in some manner or changes the way her mind works. Some compulsion powers 
determine the subject's actions or the effects on the subject, some allow you to 
determine the subject's actions when you manifest them, and others give you ongoing 
control over the subject.

\vspace{12pt}
[DESCRIPTOR]

Appearing on the same line as the discipline and subdiscipline (when applicable) 
is a descriptor that further categorizes the power in some way. Some powers have 
more than one descriptor.

The descriptors that apply to powers are acid, cold, death, electricity, evil, 
fire, force, good, language-dependent, light, mind-affecting, and sonic.

Most of these descriptors have no game effect by themselves, but they govern how 
the power interacts with other powers, with spells, with special abilities, with 
unusual creatures, with alignment, and so on.

A language-dependent power uses intelligible language as a medium. 

A mind-affecting power works only against creatures with an Intelligence score 
of 1 or higher.

\vspace{12pt}
LEVEL

The next line of the power description gives a power's level, a number between 
1 and 9 that defines the power's relative strength. This number is preceded by 
the name of the class whose members can manifest the power. If a power is part 
of a discipline's list instead of the psion's general power list, this will be 
indicated by the name of the discipline's specialist. The specialists a power can 
be associated with include Egoist (psychometabolism), Kineticist (psychokinesis), 
Nomad (psychoportation), Seer (clairsentience), Shaper (metacreativity), and Telepath 
(telepathy). 

\vspace{12pt}
DISPLAY

When a power is manifested, a display may accompany the primary effect. This secondary 
effect may be auditory, material, mental, olfactory, or visual. No power's display 
is significant enough to create consequences for the psionic creatures, allies, 
or opponents during combat. The secondary effect for a power occurs only if the 
power's description indicates it. If multiple powers with similar displays are 
in effect simultaneously, the displays do not necessary become more intense. Instead, 
the overall display remains much the same, though with minute spikes in intensity. 
A Psicraft check (DC 10 + 1 per additional power in use) reveals the exact number 
of simultaneous powers in play.

\textit{Dispense with Displays: }Despite the fact that almost every power has a 
display, a psionic character can always choose to manifest the power without the 
flashy accompaniment. To manifest a power without any display (no matter how many 
displays it might have), a manifester must make a Concentration check (DC 15 + 
the level of the power). This check is part of the action of manifesting the power. 
If the check is unsuccessful, the power manifests normally with its display.

Even if a manifester manifests a power without a display, he is still subject to 
attacks of opportunity in appropriate circumstances. (Of course, another Concentration 
check can be made as normal to either manifest defensively or maintain the power 
if attacked.)

\textbf{Auditory: }A bass-pitched hum issues from the manifester's vicinity or 
in the vicinity of the power's subject (manifester's choice), eerily akin to many 
deep-pitched voices. The sound grows in a second from hardly noticeable to as loud 
as a shout strident enough to be heard within 100 feet. At the manifester's option, 
the instantaneous sound can be so soft that it can be heard only within 15 feet 
with a successful DC 10 Listen check. Some powers describe unique auditory displays.

\textbf{Material:} The subject or the area is briefly slicked with a translucent, 
shimmering substance. The glistening substance evaporates after 1 round regardless 
of the power's duration. Sophisticated psions recognize the material as ectoplasmic 
seepage from the Astral Plane; this substance is completely inert.

\textbf{Mental:} A subtle chime rings once in the minds of creatures within 15 
feet of either the manifester or the subject (at the manifester's option). At the 
manifester's option, the chime can ring continuously for the power's duration. 
Some powers describe unique mental displays. 

\textbf{Olfactory:} An odd but familiar odor brings to mind a brief mental flash 
of a long-buried memory. The scent is difficult to pin down, and no two individuals 
ever describe it the same way. The odor originates from the manifester and spreads 
to a distance of 20 feet, then fades in less than a second (or lasts for the duration, 
at the manifester's option).

\textbf{Visual: }The manifester's eyes burn like points of silver fire while the 
power remains in effect. A rainbow-flash of light sweeps away from the manifester 
to a distance of 5 feet and then dissipates, unless a unique visual display is 
described. This is the case when the Display entry includes ``see text,'' which 
means that a visual effect is described somewhere in the text of the power. 

\vspace{12pt}
MANIFESTING TIME

Most powers have a manifesing time of 1 standard action. Others take 1 round or 
more, while a few require only a free action.

A power that takes 1 round to manifest requires a full-round action. It comes into 
effect just before the beginning of your turn in the round after you began manifesting 
the power. You then act normally after the power is completed. 

A power that takes 1 minute to manifest comes into effect just before your turn 
1 minute later (and for each of those 10 rounds, you are manifesting a power as 
a full-round action, as noted above for 1-round manifesting times). These actions 
must be consecutive and uninterrupted, or the power points are lost and the power 
fails.

When you use a power that takes 1 round or longer to manifest, you must continue 
the concentration from the current round to just before your turn in the next round 
(at least). If you lose concentration before the manifesting time is complete, 
the power points are lost and the power fails.

You make all pertinent decisions about a power (range, target, area, effect, version, 
and so forth) when the power comes into effect. 

\vspace{12pt}
New Action Types

\textbf{Swift Action: }A swift action consumes a very small amount of time, but 
represents a larger expenditure of effort and energy than a free action. You can 
perform one swift action per turn without affecting your ability to perform other 
actions. In that regard, a swift action is like a free action. However, you can 
perform only a single swift action per turn, regardless of what other actions you 
take. You can take a swift action any time you would normally be allowed to take 
a free action. Swift actions usually involve psionics or the activation of psionic 
items; many characters (especially those who don't use psionics) never have an 
opportunity to take a swift action.

Manifesting a quickened power is a swift action. In addition, manifesting any power 
with a casting time of 1 swift action is a swift action.

Manifesting a power with a manifesting time of 1 swift action does not provoke 
attacks of opportunity. 

\textbf{Immediate Action:} Much like a swift action, an immediate action consumes 
a very small amount of time, but represents a larger expenditure of effort and 
energy than a free action. However, unlike a swift action, an immediate action 
can be performed at any time---even if it's not your turn. Using an immediate action 
on your turn is the same as using a swift action, and counts as your swift action 
for that turn. You cannot use another immediate action or a swift action until 
after your next turn if you have used an immediate action when it is not currently 
your turn (effectively, using an immediate action before your turn is equivalent 
to using your swift action for the coming turn). You also cannot use an immediate 
action if you are currently flat-footed.

\vspace{12pt}
RANGE

A power's range indicates how far from you it can reach, as defined in the Range 
entry of the power description. A power's range is the maximum distance from you 
that the power's effect can occur, as well as the maximum distance at which you 
can designate the power's point of origin. If any portion of the area would extend 
beyond the range, that area is wasted. Standard ranges include the following.

\textbf{Personal: }The power affects only you.

\textbf{Touch:} You must touch a creature or object to affect it. A touch power 
that deals damage can score a critical hit just as a weapon can. A touch power 
threatens a critical hit on a natural roll of 20 and deals double damage on a successful 
critical hit. Some touch powers\textit{ }allow you to touch multiple targets. You 
can touch as many willing targets as you can reach, but all targets of the spell 
must be touched in the same round that you manifest the power.

\textbf{Close: }The power reaches as far as 25 feet away from you. The maximum 
range increases 5 feet for every two manifester levels you have.

\textbf{Medium: }The power reaches as far as 100 feet + 10 feet per manifester 
level.

\textbf{Long: }The power reaches as far as 400 feet + 40 feet per manifester level.

\textbf{Range Expressed in Feet:} Some powers have no standard range category, 
just a range expressed in feet.

\vspace{12pt}
AIMING A POWER

You must make some choice about whom the power is to affect or where the power's 
effect is to originate, depending on the type of power. The next entry in a power 
description defines the power's target (or targets), its effect, or its area, as 
appropriate.

\textbf{Target or Targets: }Some powers\textit{ }have a target or targets. You 
manifest these powers on creatures or objects, as defined by the power itself. 
You must be able to see or touch the target, and you must specifically choose that 
target. However, you do not have to select your target until you finish manifesting 
the power.

If you manifest a targeted power on the wrong type of target the power has no effect. 
If the target of a power is yourself (the power description has a line that reads 
``Target: You''), you do not receive a saving throw and power resistance does not 
apply. The Saving Throw and Power Resistance lines are omitted from such powers.

Some powers can be manifested only on willing targets. Declaring yourself as a 
willing target is something that can be done at any time (even if you're flat-footed 
or it isn't your turn). Unconscious creatures are automatically considered willing, 
but a character who is conscious but immobile or helpless (such as one who is bound, 
cowering, grappling, paralyzed, pinned, or stunned) is not automatically willing. 
The Saving Throw and Power Resistance lines are usually omitted from such powers, 
since only willing subjects can be targeted.

\textbf{Effect:} Some powers, such as most metacreativity powers, create things 
rather than affect things that are already present. Unless otherwise noted in the 
power description, you must designate the location where these things are to appear, 
either by seeing it or defining it. Range determines how far away an effect can 
appear, but if the effect is mobile, it can move regardless of the power's range 
once created.

\textit{Ray: }Some effects are rays. You aim a ray as if using a ranged weapon, 
though typically you make a ranged touch attack rather than a normal ranged attack. 
As with a ranged weapon, you can fire into the dark or at an invisible creature 
and hope you hit something. You don't have to see the creature you're trying to 
hit, as you do with a targeted power. Intervening creatures and obstacles, however, 
can block your line of sight or provide cover for the creature you're aiming at.

If a ray power has a duration, it's the duration of the effect that the ray causes, 
not the length of time the ray itself persists.

If a ray power deals damage, you can score a critical hit just as if it were a 
weapon. A ray power threatens a critical hit on a natural roll of 20 and deals 
double damage on a successful critical hit.

\textit{Spread: }Some effects\textit{ }spread out from a point of origin (which 
may be a grid intersection, or may be the manifester) to a distance described in 
the power. The effect can extend around corners and into areas that you can't see. 
Figure distance by actual distance traveled, taking into account turns the effect 
may take. When determining distance for spread effects, count around walls, not 
through them. As with movement, do not trace diagonals across corners. You must 
designate the point of origin for such an effect (unless the effect is centered 
on you), but you need not have line of effect (see below) to all portions of the 
effect.

\textit{(S) Shapeable: }If an Effect line ends with ``(S)'' you can shape the power. 
A shaped effect can have no dimension smaller than 10 feet.

\textbf{Area:} Some powers affect an area. Sometimes a power description specifies 
a specially defined area, but usually an area falls into one of the categories 
defined below.

Regardless of the shape of the area, you select the point where the power originates, 
but otherwise you usually don't control which creatures or objects the power affects. 
The point of origin of a power that affects an area is always a grid intersection. 
When determining whether a given creature is within the area of a power, count 
out the distance from the point of origin in squares just as you do when moving 
a character or when determining the range for a ranged attack. The only difference 
is that instead of counting from the center of one square to the center of the 
next, you count from intersection to intersection.

You can count diagonally across a square, but every second diagonal counts as 2 
squares of distance. If the far edge of a square is within the power's area, anything 
within that square is within the power's area. If the power's area touches only 
the near edge of a square, however, anything within that square is unaffected by 
the power.

\textit{Burst, Emanation, or Spread: }Most powers that affect an area function 
as a burst, an emanation, or a spread. In each case, you select the power's point 
of origin and measure its effect from that point. A burst power affects whatever 
it catches in its area, even including creatures that you can't see. It can't affect 
creatures with total cover from its point of origin (in other words, its effects 
don't extend around corners). The default shape for a burst effect is a sphere, 
but some burst powers are specifically described as cone-shaped.

A burst's area defines how far from the point of origin the power's effect extends.

An emanation power functions like a burst power, except that the effect continues 
to radiate from the point of origin for the duration of the power.

A spread power spreads out like a burst but can turn corners. You select the point 
of origin, and the power spreads out a given distance in all directions. Figure 
the area the power effect fills by taking into account any turns the effect takes.

\textit{Cone, Line, or Sphere: }Most powers that affect an area have a particular 
shape, such as a cone, line, or sphere. A cone-shaped power shoots away from you 
in a quarter-circle in the direction you designate. It starts from any corner of 
your square and widens out as it goes. Most cones are either bursts or emanations 
(see above), and thus won't go around corners.

A line-shaped power shoots away from you in a line in the direction you designate. 
It starts from any corner of your square and extends to the limit of its range 
or until it strikes a barrier that blocks line of effect. A line-shaped power affects 
all creatures in squares that the line passes through or touches.

A sphere-shaped power expands from its point of origin to fill a spherical area. 
Spheres may be bursts, emanations, or spreads.

\textit{Other: }A power can have a unique area, as defined in its description.

\textbf{Line of Effect:} A line of effect is a straight, unblocked path that indicates 
what a power can affect. A solid barrier cancels a line of effect, but it is not 
blocked by fog, darkness, and other factors that limit normal sight. You must have 
a clear line of effect to any target that you manifest a power on or to any space 
in which you wish to create an effect. You must have a clear line of effect to 
the point of origin of any power you manifest.

A burst, cone, or emanation power affects only an area, creatures, or objects to 
which it has line of effect from its origin (a spherical burst's center point, 
a cone-shaped burst's starting point, or an emanation's point of origin). An otherwise 
solid barrier with a hole of at least 1 square foot through it does not block a 
power's line of effect. Such an opening means that the 5-foot length of wall containing 
the hole is no longer considered a barrier for the purpose of determining a power's 
line of effect.

\vspace{12pt}
DURATION

A power's Duration line tells you how long the psionic energy of the power lasts.

\textbf{Timed Durations:} Many durations are measured in rounds, minutes, hours, 
or some other increment. When the time is up, the psionic energy sustaining the 
effect fades, and the power ends. If a power's duration is variable it is rolled 
secretly.

\textbf{Instantaneous: }The psionic energy comes and goes the instant the power 
is manifested, though the consequences might be long-lasting.

\textbf{Permanent: }The energy remains as long as the effect does. This means the 
power is vulnerable to \textit{dispel psionics}.

\textbf{Concentration:} The power lasts as long as you concentrate on it. Concentrating 
to maintain a power is a standard action that does not provoke attacks of opportunity. 
Anything that could break your concentration when manifesting a power can also 
break your concentration while you're maintaining one, causing the power to end. 
You can't manifest a power while concentrating on another one. Some powers may 
last for a short time after you cease concentrating. In such a case, the power 
keeps going for the given length of time after you stop concentrating, but no longer. 
Otherwise, you must concentrate to maintain the power, but you can't maintain it 
for more than a stated duration in any event. If a target moves out of range, the 
power reacts as if your concentration had been broken.

\textbf{Subjects, Effects, and Areas:} If the power affects creatures directly 
the result travels with the subjects for the power's duration. If the power creates 
an effect, the effect lasts for the duration. The effect might move or remain still. 
Such an effect can be destroyed prior to when its duration ends. If the power affects 
an area then the power stays with that area for its duration. Creatures become 
subject to the power when they enter the area and are no longer subject to it when 
they leave.

\textbf{Touch Powers and Holding the Charge:} In most cases, if you don't discharge 
a touch power on the round you manifest it, you can hold the charge (postpone the 
discharge of the power) indefinitely. You can make touch attacks round after round. 
If you touch anything with your hand while holding a charge, the power discharges. 
If you manifest another power, the touch power dissipates.

Some touch powers\textit{ }allow you to touch multiple targets as part of the power. 
You can't hold the charge of such a power; you must touch all the targets of the 
power in the same round that you finish manifesting the power. You can touch one 
friend (or yourself) as a standard action or as many as six friends as a full round 
action.

\textbf{Discharge: }Occasionally a power lasts for a set duration or until triggered 
or discharged.

\textbf{(D) Dismissible:} If the Duration line ends with ``(D),'' you can dismiss 
the power at will. You must be within range of the power's effect and must mentally 
will the dismissal, which causes the same display as when you first manifested 
the power. Dismissing a power is a standard action that does not provoke attacks 
of opportunity. A power that depends on concentration is dismissible by its very 
nature, and dismissing it does not take an action or cause a display, since all 
you have to do to end the power is to stop concentrating on your turn.

\vspace{12pt}
SAVING THROW

Usually a harmful power allows a target to make a saving throw to avoid some or 
all of the effect. The Saving Throw line in a power description defines which type 
of saving throw the power allows and describes how saving throws against the power 
work.

\textbf{Negates:} The power has no effect on a subject that makes a successful 
saving throw.

\textbf{Partial:} The power causes an effect on its subject, such as death. A successful 
saving throw means that some lesser effect occurs (such as being dealt damage rather 
than being killed).

\textbf{Half:} The power deals damage, and a successful saving throw halves the 
damage taken (round down). 

\textbf{None: }No saving throw is allowed.

\textbf{(object):} The power can be manifested on objects, which receive saving 
throws only if they are psionic or if they are attended (held, worn, grasped, or 
the like) by a creature resisting the power, in which case the object uses the 
creature's saving throw bonus unless its own bonus is greater. (This notation does 
not mean that a power can be manifested only on objects. Some powers of this sort 
can be manifested on creatures or objects.) A psionic item's saving throw bonuses 
are each equal to 2 + one-half the item's manifester level.

\textbf{(harmless):} The power is usually beneficial, not harmful, but a targeted 
creature can attempt a saving throw if it desires.

\textbf{Saving Throw Difficulty Class:} A saving throw against your power has a 
DC 10 + the level of the power + your key ability modifier (Intelligence for a 
psion, Wisdom for a psychic warrior, or Charisma for a wilder). A power's level 
can vary depending on your class. Always use the power level applicable to your 
class.

\textbf{Succeeding on a Saving Throw:} A creature that successfully saves against 
a power that has no obvious physical effects feels a hostile force or a tingle, 
but cannot deduce the exact nature of the attack. Likewise, if a creature's saving 
throw succeeds against a targeted power you sense that the power has failed. You 
do not sense when creatures succeed on saves against effect and area powers.

\textbf{Failing a Saving Throw against Mind-Affecting Powers: }If you fail your 
save, you are unaware that you have been affected by a power.

\textbf{Automatic Failures and Successes:} A natural 1 (the d20 comes up 1) on 
a saving throw is always a failure, and the power may deal damage to exposed items 
(see Items Surviving after a Saving Throw, below). A natural 20 (the d20 comes 
up 20) is always a success.

\textbf{Voluntarily Giving up a Saving Throw:} A creature can voluntarily forego 
a saving throw and willingly accept a power's result. Even a character with a special 
resistance to psionics can suppress this quality.

\textbf{Items Surviving after a Saving Throw:} Unless the descriptive text for 
the power specifies otherwise, all items carried or worn by a creature are assumed 
to survive a psionic attack. If a creature rolls a natural 1 on its saving throw 
against the effect, however, an exposed item is harmed (if the attack can harm 
objects). Refer to Table: Items Affected by Psionic Attacks.

Determine which four objects carried or worn by the creature are most likely to 
be affected and roll randomly among them. The randomly determined item must make 
a saving throw against the attack form or take whatever damage the attack deals.

\vspace{12pt}
\begin{tabular}{|>{\raggedright}p{28pt}|>{\raggedright}p{201pt}|}
\hline
\multicolumn{2}{|p{229pt}|}{\section*{T\textbf{able: Items Affected by Psionic 
Attacks}}}\tabularnewline
\hline
O\textbf{rder}\textsuperscript{\textbf{1}} & I\textbf{tem}\tabularnewline
\hline
1st & Shield\tabularnewline
\hline
2nd  & Armor\tabularnewline
\hline
3rd  & Psionic or magic helmet, or psicrown\tabularnewline
\hline
4th  & Item in hand (including weapon, dorje, or the like)\tabularnewline
\hline
5th  & Psionic or magic cloak\tabularnewline
\hline
6th  & Stowed or sheathed weapon\tabularnewline
\hline
7th  & Psionic or magic bracers\tabularnewline
\hline
8th  & Psionic or magic clothing\tabularnewline
\hline
9th  & Psionic or magic jewelry (including rings)\tabularnewline
\hline
10th  & Anything else\tabularnewline
\hline
\multicolumn{2}{|p{229pt}|}{1 In order of most likely to least likely to be affected.}\tabularnewline
\hline
\end{tabular}

\vspace{12pt}
POWER RESISTANCE

Power resistance is a special defensive ability. If your power is being resisted 
by a creature with power resistance, you must make a manifester level check (d20 
+ manifester level) at least equal to the creature's power resistance for the power 
to affect that creature. The defender's power resistance functions like an Armor 
Class against psionic attacks. Spell resistance is equivalent to power resistance 
unless the Psionics Is Different option is in use. Include any adjustments to your 
manifester level on this manifester level check.

The Power Resistance line and the descriptive text of a power description tell 
you whether power resistance protects creatures from the power. In many cases, 
power resistance applies only when a resistant creature is targeted by the power, 
not when a resistant creature encounters a power that is already in place.

The terms ``object'' and ``harmless'' mean the same thing for power resistance 
as they do for saving throws. A creature with power resistance must voluntarily 
lower the resistance (a standard action) to be affected by a power noted as harm 
less. In such a case, you do not need to make the manifester level check described 
above.

\vspace{12pt}
POWER POINTS

All powers have a Power Points line, indicating the power's cost.

The psionic character class tables show how many power points a character has access 
to each day, depending on level.

A power's cost is determined by its level, as shown below. Every power's cost is 
noted in its description for ease of reference.

\vspace{12pt}
\begin{tabular}{|>{\raggedright}p{73pt}|>{\raggedright}p{3pt}|>{\raggedright}p{3pt}|>{\raggedright}p{3pt}|>{\raggedright}p{3pt}|>{\raggedright}p{3pt}|>{\raggedright}p{8pt}|>{\raggedright}p{8pt}|>{\raggedright}p{8pt}|>{\raggedright}p{8pt}|}
\hline
\multicolumn{10}{|p{127pt}|}{\section*{T\textbf{able: Power Points by Power Level}}}\tabularnewline
\hline
P\textbf{ower Level} & 1 & 2 & 3 & 4 & 5 & 6 & 7 & 8 & 9\tabularnewline
\hline
P\textbf{ower Point Cost} & 1 & 3 & 5 & 7 & 9 & 11 & 13 & 15 & 17\tabularnewline
\hline
\end{tabular}

\vspace{12pt}
\textbf{Power Point Limit:} Some powers allow you to spend more than their base 
cost to achieve an improved effect, or augment the power. The maximum number of 
points you can spend on a power (for any reason) is equal to your manifester level.

\textbf{XP Cost (XP):} On the same line that the power point cost of a power is 
indicated, the power's experience point cost, if any, is noted. Particularly powerful 
effects entail an experience point cost to you. No spell or power can restore XP 
lost in this manner. You cannot spend so much XP that you lose a level, so you 
cannot manifest a power with an XP cost unless you have enough XP to spare. However, 
you can, on gaining enough XP to attain a new level, use those XP for manifesting 
a power rather than keeping them and advancing a level. The XP are expended when 
you manifest the power, whether or not the manifestation succeeds.

\vspace{12pt}
DESCRIPTIVE TEXT

This portion of a power description details what the power does and how it works. 
If one of the previous lines in the description included ``see text,'' this is 
where the explanation is found. If the power you're reading about is based on another 
power you might have to refer to a different power for the ``see text'' information. 
If a power is the equivalent of a spell an entry of ``see spell text'' directs 
you to the appropriate spell description\textit{.}

\textbf{Augment:} Many powers have variable effects based on the number of power 
points you spend when you manifest them. The more points spent, the more powerful 
the manifestation. How this extra expenditure affects a power is specific to the 
power. Some augmentations allow you to increase the number of damage dice, while 
others extend a power's duration or modify a power in unique ways. Each power that 
can be augmented includes an entry giving how many power points it costs to augment 
and the effects of doing so. However, you can spend only a total number of points 
on a power equal to your manifester level.

Augmenting a power takes place as part of another action (manifesting a power). 
Unless otherwise noted in the Augment section of an individual power description, 
you can augment a power only at the time you manifest it.

\newpage

\end{document}
