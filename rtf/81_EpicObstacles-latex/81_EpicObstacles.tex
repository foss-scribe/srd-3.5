%&pdfLaTeX
% !TEX encoding = UTF-8 Unicode
\documentclass{article}
\usepackage{ifxetex}
\ifxetex
\usepackage{fontspec}
\setmainfont[Mapping=tex-text]{STIXGeneral}
\else
\usepackage[T1]{fontenc}
\usepackage[utf8]{inputenc}
\fi
\usepackage{textcomp}

\usepackage{array}
\usepackage{amssymb}
\usepackage{fancyhdr}
\renewcommand{\headrulewidth}{0pt}
\renewcommand{\footrulewidth}{0pt}

\begin{document}

This material is Open Game Content, and is licensed for public use under the terms 
of the Open Game License v1.0a.

\subsubsection*{{\LARGE{}EPIC OBSTACLES}}

\vspace{12pt}
\subsection*{Walls}

In addition to the standard types of wall\textit{, }walls in dungeons can be made 
of mithral, adamantine, or even pure force.

\begin{tabular}{|>{\raggedright}p{94pt}|>{\raggedright}p{59pt}|>{\raggedright}p{29pt}|>{\raggedright}p{26pt}|>{\raggedright}p{33pt}|>{\raggedright}p{29pt}|}
\hline
W\textbf{all Type} & T\textbf{ypical Thickness} & B\textbf{reak DC} & H\textbf{ardness} & H\textbf{it 
Points*} & C\textbf{limb DC}\tabularnewline
\hline
Paper & Paper-thin & 1--- &  & 1hp & 30\tabularnewline
\hline
Wood & 6in. & 20 & 5 & 60hp & 21\tabularnewline
\hline
Masonry & 1ft. & 35 & 8 & 90hp & 15\tabularnewline
\hline
Masonry, superior & 1ft. & 35 & 8 & 90hp & 20\tabularnewline
\hline
Masonry, reinforced & 1ft. & 45 & 8 & 180hp & 15\tabularnewline
\hline
Stone, hewn & 3ft. & 50 & 8 & 540hp & 22\tabularnewline
\hline
Stone, unworked & 5ft. & 65 & 8 & 900hp & 20\tabularnewline
\hline
Iron & 3in. & 30 & 10 & 90hp & 25\tabularnewline
\hline
Mithral & 3in. & 46 & 15 & 90hp & 70\tabularnewline
\hline
Adamantine & 3in. & 66 & 20 & 120hp & 70\tabularnewline
\hline
Magically treated**--- &  & 20 & x2 & x2†--- & \tabularnewline
\hline
\section*{W\textit{all of force}} & 1in. & n/a & n/a & n/a & 70\tabularnewline
\hline
W\textit{all of ice} & 1in./lvl & 15+1/in. & 0 & 3hp/in. & 25\tabularnewline
\hline
W\textit{all of iron} & 1in./4lvls & 25+2/in. & 10 & 30hp/in. & 25\tabularnewline
\hline
W\textit{all of stone} & 1in./4lvls & 20+2/in. & 8 & 15hp/in. & 22\tabularnewline
\hline
*Per 10-ft.-by-10-ft. section.  &  &  &  &  & \tabularnewline
\hline
\multicolumn{6}{|p{273pt}|}{**These modifiers can be applied to any of the other 
categories and types. }\tabularnewline
\hline
†Or 50, whichever is greater. &  &  &  &  & \tabularnewline
\hline
\end{tabular} 

\vspace{12pt}
\section*{Doors}

\section*{\begin{tabular}{|>{\raggedright}p{77pt}|>{\raggedright}p{70pt}|>{\raggedright}p{31pt}|>{\raggedright}p{32pt}|>{\raggedright}p{14pt}|>{\raggedright}p{21pt}|}
\hline
} &  &  &  & \multicolumn{2}{p{35pt}|}{B\textbf{reak DC }}\tabularnewline
\hline
D\textbf{oor Type} & T\textbf{ypical Thickness} & H\textbf{ardness} & H\textbf{it 
Points} & S\textbf{tuck} & L\textbf{ocked }\tabularnewline
\hline
Simple wooden & 1 in. & 5 & 10 hp & 13 & 15\tabularnewline
\hline
Good wooden & 1 1/2 in. & 5 & 15 hp & 16 & 18\tabularnewline
\hline
Strong wooden & 2 in. & 5 & 20 hp & 23 & 25\tabularnewline
\hline
Stone & 4 in. & 8 & 60 hp & 28 & 28\tabularnewline
\hline
Iron & 2 in. & 10 & 60 hp & 28 & 28\tabularnewline
\hline
Mithral & 2 in. & 15 & 60 hp & 40 & 40\tabularnewline
\hline
Adamantine & 2 in. & 20 & 80 hp & 60 & 60\tabularnewline
\hline
Force & 1 in. & n/a & n/a & n/a & n/a \tabularnewline
\hline
Portcullis, wodden & 3 in. & 5 & 30 hp & 25* & 25* \tabularnewline
\hline
Portcullis, iron & 2 in. & 10 & 60 hp & 25* & 25* \tabularnewline
\hline
Portcullis, mithral & 2 in. & 15 & 60 hp & 30* & 30* \tabularnewline
\hline
Portcullis, adamatine & 2 in. & 20 & 80 hp & 40* & 40* \tabularnewline
\hline
Portcullis, force & 1 in. & 10 & n/a & n/a & 50* \tabularnewline
\hline
\multicolumn{3}{|p{178pt}|}{*DC to lift. Use appropriate door figure for breaking. 
} &  &  & \tabularnewline
\hline
\end{tabular}

\vspace{12pt}
\section*{Obstacles and Hazards}

\section*{\begin{tabular}{|>{\raggedright}p{73pt}|>{\raggedright}p{248pt}|}
\hline
O\textbf{bstacle/Hazard}} & E\textbf{ffect }\tabularnewline
\hline
Acid tank & 1d6 damage per round, or 10d6 per round for total immersion; plus poison 
fumes. \tabularnewline
\hline
\section*{A\textit{ntimagic field}} & Negates all spells or magical effects. \tabularnewline
\hline
D\textit{imensional anchor} trap & Blocks bodily extradimensional travel. \tabularnewline
\hline
Hurricane-force winds & Ranged attacks impossible, flight virtually impossible. 
\tabularnewline
\hline
Lava pit & 2d6 damage per round, or 20d6 per round for total immersion; plus continuing 
damage.\tabularnewline
\hline
Permanent \textit{prismatic sphere} & Requires seven different spells to bypass. 
\tabularnewline
\hline
Permanent \textit{solid fog} & Move at one-tenth normal speed, -2 penalty on attack 
and damage (good when coupled with incorporeal monsters). \tabularnewline
\hline
Permanent \textit{wall of force} & Blocks most spells and ethereal travel, can't 
be \textit{dispelled}.\tabularnewline
\hline
Three-dimensional dungeons & Levitation/flying required to move between areas. 
\tabularnewline
\hline
Unconnected rooms & Teleportation required to move between areas. \tabularnewline
\hline
Variable gravity & As \textit{reverse gravity}, but direction random each round. 
\tabularnewline
\hline
\end{tabular}

\vspace{12pt}
\section*{Slimes, Molds, and Fungi}

For purposes of spells and other special effects, all slimes, molds, and fungi 
are treated as plants. Like traps, dangerous slimes and molds have Challenge Ratings, 
and characters earn experience points for encountering them.

\vspace{12pt}
\textbf{Flux Slime (CR 21):}

Flux slime appears as a clear, viscous liquid that seeps from some unseen origin 
point. This origin point is extradimensional, so the slime may even appear in midair. 
As the slime flows, it settles and fills the area around the origin point. 

Flux slime seems to be an inert substance, devoid of sentience. It is not caustic 
or toxic, but it radiates an \textit{antimagic field }within a radius of 10 feet. 
This \textit{antimagic field} has a caster level of 21. Any quantity of slime that 
is removed from the main mass yellows and hardens in a matter of minutes, turning 
into a flaky material that will not adhere to anything. 

In reality, flux slime is a growth with a ravenous appetite for magical forces. 
It is a natural draining phenomenon: Magical energy drains through the origin point 
in one direction in exchange for the residue on the far side. The \textit{antimagic 
field }a flux slime generates is actually the byproduct of the consumption of magical 
energy. 

In addition to the \textit{antimagic fiel}d's effects, magic items that come into 
contact with flux slime permanently lose their magical abilities; creatures with 
spell-like or super-natural abilities that come into contact with it take 2d6 points 
of temporary Constitution damage per round while it devours flesh; creatures without 
such abilities are immune to this effect. 

On the first round of contact, the slime can be scraped off a creature, but after 
that it must be frozen, burned, or cut away (dealing damage to the victim as well). 
Extreme cold, heat, or sunlight destroys a patch of flux slime. 

When destroyed, a patch of slime releases the byproducts of its magical digestion 
in a dangerous burst that radiates out 50 feet. All creatures caught in this burst 
are subject to some random and permanent transmutation effect, as generated on 
the table below. Each burst generates one of these effects. Creatures may resist 
this effect with a Fortitude saving throw (DC 29).

\begin{tabular}{|>{\raggedright}p{20pt}|>{\raggedright}p{301pt}|}
\hline
d\textbf{\%} & R\textbf{esult }\tabularnewline
\hline
01-10 & Blindness (as blindness/deafness spell) \tabularnewline
\hline
11-16 & Cursed (as bestow curse spell; -4 enhancement penalty on attack rolls, 
saving throws, ability checks, and skill checks)\tabularnewline
\hline
17-26 & Deafness (as blindness/deafness spell) \tabularnewline
\hline
27-32 & Disintegrate (subject is destroyed by a disintegrate spell) \tabularnewline
\hline
33-40 & Etherealness (as etherealness spell) \tabularnewline
\hline
41-48 & Gaseous (as gaseous form spell) \tabularnewline
\hline
49-54 & Iron body (as iron body spell) \tabularnewline
\hline
55-60 & Petrification (as flesh to stone spell) \tabularnewline
\hline
61-68 & Plane shift (subject instantly transports to a random plane) \tabularnewline
\hline
69-74 & Polymorph (as polymorph other spell; choose form randomly) \tabularnewline
\hline
75-80 & Reverse gravity (flux slime becomes the center of a reverse gravity spell). 
\tabularnewline
\hline
81-88 & Teleport (each subject teleports to a different,  random location)\tabularnewline
\hline
89-94 & Temporal stasis (as temporal stasis spell) \tabularnewline
\hline
95-00 & Reverse aging (subject gets younger each year, disappearing at moment of 
``birth'')\tabularnewline
\hline
\end{tabular}

After the burst, the extradimensional origin point is sealed.

\newpage

\end{document}
