%&pdfLaTeX
% !TEX encoding = UTF-8 Unicode
\documentclass{article}
\usepackage{ifxetex}
\ifxetex
\usepackage{fontspec}
\setmainfont[Mapping=tex-text]{STIXGeneral}
\else
\usepackage[T1]{fontenc}
\usepackage[utf8]{inputenc}
\fi
\usepackage{textcomp}

\usepackage{array}
\usepackage{amssymb}
\usepackage{fancyhdr}
\renewcommand{\headrulewidth}{0pt}
\renewcommand{\footrulewidth}{0pt}
\usepackage{color}

\definecolor{color08}{rgb}{1.00,1.00,1.00}

\begin{document}

\section*{This material is Open Game Content, and is licensed for public use under 
the terms of the Open Game License v1.0a.}

\section*{{\LARGE{}DIVINE RANKS AND POWERS}}

\vspace{12pt}
\section*{Divine Ranks}

Each deity has a divine rank. A deity's divine rank determines how much power the 
entity has.

\vspace{12pt}
\textbf{Rank 0:} Creatures of this rank are sometimes called quasi-deities or hero 
deities. Creatures that have a mortal and a deity as parents also fall into this 
category. These entities cannot grant spells, but are immortal and usually have 
one or more ability

scores that are far above the norm for their species. They may have some worshipers. 
Ordinary mortals do not have a divine rank of 0. They lack a divine rank altogether.

\vspace{12pt}
\textbf{Rank 1-5:} These entities, called demigods, are the weakest of the deities. 
A demigod can grant spells and perform a few deeds that are beyond mortal limits. 
A demigod has anywhere from a few hundred to a few thousand devoted mortal worshipers 
and may receive veneration or respect from many more. A demigod controls a small 
godly realm (usually on an Outer Plane) and has minor control over a portfolio 
that includes one or more aspects of mortal existence. A demigod might be very 
accomplished in a single skill or a group of related skills, gain combat advantages 
in special circumstances, or be able to bring about minor changes in reality itself 
related to the portfolio.

\vspace{12pt}
\textbf{Rank 6-10:} Called lesser deities, these entities grant spells and can 
perform more powerful deeds than demigods can. Lesser deities have anywhere from 
a few thousand to tens of thousands of worshipers and control larger godly realms 
than demigods. They also have keener senses where their portfolios are concerned.

\vspace{12pt}
\textbf{Rank 11-15:} These entities are called intermediate deities. They have 
hundreds of thousands of mortal worshipers and control larger godly realms than 
demigods or lesser deities. 

\vspace{12pt}
\textbf{Rank 16-20:} Called greater deities, these entities may have millions of 
mortal worshipers, and they command respect even among other deities. The most 
powerful of greater deities rule over other deities just as mortal sovereigns rule 
over commoners.

\vspace{12pt}
\textbf{Rank 21+:} These entities are beyond the ken of mortals and care nothing 
for worshipers. They do not grant spells, do not answer prayers, and do not respond 
to queries. If they are known at all, it is to a handful of scholars on the Material 
Plane. They are called overdeities. In some pantheistic systems, the consent of 
an overdeity is required to become a god.{\color{color08} 25}

\vspace{12pt}
\section*{Divine Characteristics}

Most deities are creatures of the outsider type (usually with 20 outsider Hit Dice). 
All deities that are outsiders have all alignment subtypes that correspond with 
their alignment.  Unlike other outsiders, they have no darkvision unless noted 
in the deity description. Deities' physical characteristics vary from deity to 
deity. A deity's outsider type, along with its class or classes, determines its 
weapon proficiencies, feats, and skills. Deities have some or all of the following 
additional qualities, depending on their divine rank.

\vspace{12pt}
\textbf{Levels:} Most deities are 20 HD outsiders with 30 to 50 character levels 
as well. These additional character levels beyond an effective character level 
of 20th follow the rules for epic levels.

Character levels above 20th confer some, but not all, of the benefits of normal 
levels. Deities gain all the class features for all their levels. The deity also 
gains the normal Hit Die for that class, plus additional skill points as if the 
deity had a level in that class normally. The deity gains an ability bonus every 
four levels, and a feat every three levels.

Beyond character level 20th, deities' attack and saving throw bonuses increase 
at new rates. Deities gain a +1 epic attack bonus at 21st level and every other 
level thereafter. No deity can have more than four attacks derived strictly from 
its base attack bonus.  Deities also gain a +1 bonus on saving throws at 22nd level 
and every other level thereafter.

\vspace{12pt}
\textbf{Hit Points:} Deities receive maximum hit points for each Hit Die.

\vspace{12pt}
\textbf{Speed:} Deities can move much more quickly than mortals. A deity's base 
land speed depends on its form (biped or quadruped) and its size, as shown on the 
following table. Some deities are exceptions, with speeds faster or slower than 
the norm.

\begin{tabular}{|>{\raggedright}p{45pt}|>{\raggedright}p{28pt}|>{\raggedright}p{58pt}|}
\hline
\subsection*{S\textbf{ize }} & \subsection*{B\textbf{iped*}} & \subsection*{Q\textbf{uadruped**}}\tabularnewline
\hline
Fine  & 20 ft.  & 60 ft.\tabularnewline
\hline
Diminutive  & 30 ft.  & 70 ft.\tabularnewline
\hline
Tiny  & 40 ft. & 80 ft.\tabularnewline
\hline
Small  & 50 ft.  & 90 ft.\tabularnewline
\hline
Medium  & 60 ft.  & 100 ft.\tabularnewline
\hline
Large & 80 ft.  & 120 ft.\tabularnewline
\hline
Huge  & 100 ft. & 140 ft.\tabularnewline
\hline
Gargantuan  & 120 ft.  & 160 ft.\tabularnewline
\hline
Colossal  & 140 ft.  & 180 ft.\tabularnewline
\hline
\multicolumn{3}{|p{132pt}|}{*Or any form with two or fewer legs.}\tabularnewline
\hline
\multicolumn{3}{|p{132pt}|}{**Or any form with three or more legs.}\tabularnewline
\hline
\end{tabular}

Note: Use the Biped column for burrow and swim speeds for all deities regardless 
of form. Use half the value in the Biped column for climb speeds for all deities. 
Use twice the value in the Quadruped column for fly speeds for all deities capable 
of flying.

\vspace{12pt}
\textbf{Armor Class:} A field of divine energy encompasses a deity's body, granting 
it a divine Armor Class bonus equal to its divine rank. This bonus stacks with 
all other Armor Class bonuses and is effective against touch attacks and incorporeal 
touch attacks.

Most deities (all those with 20 outsider Hit Dice) have a natural armor bonus of 
their divine rank +13. All deities also have a deflection bonus to their AC equal 
to their Charisma bonus (if any).

Deities who aren't outsiders have their normal natural armor bonus + their divine 
rank.

Many deities have other Armor Class bonuses as noted in their individual descriptions.

\vspace{12pt}
\textbf{Attacks:} A deity's Hit Dice and type and character level determine its 
base attack bonus. In addition to the figures for weapon attacks, this section 
of the statistics block also includes melee touch attack and ranged touch attack 
bonuses, to be used when the deity casts a spell or uses a spell-like ability that 
requires a touch attack to affect its target. A deity gets its divine rank as a 
divine bonus on all attack rolls. Deities of rank 1 or higher do not automatically 
fail on a natural attack roll of 1.

\textit{Bypassing Damage Reduction:} As outsiders with alignment subtypes, the 
natural attacks of deities, as well as any weapons they wield, are considered aligned 
the same as the deity for the purposes of bypassing damage reduction.

\vspace{12pt}
\textbf{Always Maximize Roll:} Greater deities (rank 16-20) automatically get the 
best result possible on any check, saving throw, attack roll, or damage roll. Calculate 
success, failure, or other effects accordingly. When a greater deity makes a check, 
attack, or save assume a 20 was rolled and calculate success or failure from there. 
A d20 should still be rolled and used to check for a threat of a critical hit. 
 This quality means that greater deities never need the Maximize Spell feat, because 
their spells have maximum effect already.

\vspace{12pt}
\textbf{Saving Throws:} A deity's outsider Hit Dice and character level determine 
its base saving throw bonuses. A deity gets its divine rank as a divine bonus on 
all saving throws. Deities of rank 1 or higher do not automatically fail on a natural 
saving throw roll of 1.

\vspace{12pt}
\textbf{Checks:} A deity gets its divine rank as a divine bonus on all skill checks, 
ability checks, caster level checks, and turning checks. Lesser deities (rank 6-10) 
may take 10 on any check, provided they need to make a check at all. Intermediate 
and greater deities (rank 11-20) always get a result of 20 on any check, provided 
they need to make a check at all.

\section*{\textit{Deities and Synergy Bonuses}: For every 20 extra ranks a deity 
has in a skill, the deity's synergy bonus from the skill (if any) increases by 
+2.}

\vspace{12pt}
\textbf{Immunities:} Deities have the following immunities. Individual deities 
may have more immunities. Unless otherwise indicated, these immunities do not apply 
if the attacker is a deity of equal or higher rank.

\textit{Transmutation: }A deity is immune to polymorphing, petrification, or any 
other attack that alters its form. Any shape-altering powers the deity might have 
work normally on itself.

\textit{Energy Drain, Ability Drain, Ability Damage: }A deity is not subject to 
energy drain, ability drain, or ability damage.

\textit{Mind-Affecting Effects: }A deity is immune to mind-affecting effects (charms, 
compulsions, phantasms, patterns, and morale effects).

\textit{Energy Immunity: }Deities of rank 1 or higher are immune to electricity, 
cold, and acid, even if the attacker is a deity of higher divine rank. Some deities 
have additional energy immunities.

Deities of rank 1 or higher are immune to disease and poison, stunning, sleep, 
paralysis, and death effects, and disintegration.

Deities of rank 6 or higher are immune to effects that imprison or banish them. 
Such effects include \textit{banishment, binding, dimensional anchor, dismissal, 
imprisonment, repulsion, soul bind, temporal stasis, trap the soul, }and turning 
and rebuking.

\vspace{12pt}
\textbf{Damage Reduction:} A deity has damage reduction as shown below

\begin{tabular}{|>{\raggedright}p{104pt}|>{\raggedright}p{80pt}|}
\hline
D\textbf{ivine Rank} & D\textbf{amage Reduction}\tabularnewline
\hline
Quasi-deity (0) & 10/epic\tabularnewline
\hline
Demigod (1-5) & 15/epic\tabularnewline
\hline
Lesser deity (11-15) & 20/epic\tabularnewline
\hline
Intermediate deity (11-15) & 25/epic\tabularnewline
\hline
Greater deity (16-20) & 30/epic\tabularnewline
\hline
\end{tabular}

\vspace{12pt}
If the deity also has damage reduction from another source this damage reduction 
does not stack with the damage reduction granted by divine ranks. Instead, the 
deity gets the benefit of the best damage reduction in a given situation. Whenever 
a deity has a second kind of damage reduction that might apply to an attack, that 
damage reduction is listed in parentheses after the damage reduction entry in the 
deity's statistics block.

\vspace{12pt}
\textbf{Resistances:} All deities have at least the following resistances.  Individual 
deities may have additional resistances.

\textit{Energy Resistance: }A deity has fire resistance of 5 + its divine rank.

\textit{Spell Resistance: }A deity has spell resistance of 32 + its divine rank.

\vspace{12pt}
\textbf{Salient Divine Abilities}: Every deity of rank 1 or higher has at least 
one additional power, called a salient divine ability, per divine rank (see Salient 
Divine Abilities).

\vspace{12pt}
\textbf{Domain Powers:} A deity of rank 1 or higher can use any domain power it 
can grant a number of times per day equal to its divine rank (if the power normally 
can be used more often than that, the deity gets the greater number of uses). If 
a domain power is based on a cleric's level (or one-half a cleric's level), a deity 
with no cleric levels has an effective cleric level equal to the deity's divine 
rank (or one-half the deity's divine rank) for this purpose.

\vspace{12pt}
\textbf{Spell-Like Abilities:} A deity can use any domain spell it can grant as 
a spell-like ability at will. The deity's effective caster level for such abilities 
is 10 + the deity's divine rank. The saving throw DC for such abilities is 10 + 
the spell level + the deity's Charisma bonus (if any) + the deity's divine rank.

\vspace{12pt}
\textbf{Immortality:} All deities (even those of rank 0) are naturally immortal 
and cannot die from natural causes. Deities do not age, and they do not need to 
eat, sleep, or breathe. The only way for a deity to die is through special circumstances, 
usually by being slain in magical or physical combat. Deities of rank 1 or higher 
are not subject to death from massive damage.

\vspace{12pt}
\textbf{Senses: }Deities of rank 1 or higher have incredibly sensitive perception. 
Such a deity's senses (including darkvision and low-light vision, if the deity 
has them) extend out to a radius of one mile per divine rank. Perception is limited 
to the senses a deity possesses. A deity cannot see through solid objects without 
using its remote sensing ability or some sort of x-ray vision power.

\textit{Remote Sensing: }As a standard action, a deity of rank 1 or higher can 
perceive everything within a radius of one mile per rank around any of its worshipers, 
holy sites, or other objects or locales sacred to the deity. This supernatural 
effect can also be centered on any place where someone speaks the deity's name 
or title for up to 1 hour after the name is spoken, and at any location when an 
event related to the deity's portfolio occurs.The remote sensing power can cross 
planes and penetrate any barrier except a divine shield (described in Salient Divine 
Abilities) or an area otherwise blocked by a deity of equal or higher rank. Remote 
sensing is not fooled by \textit{misdirection }or \textit{nondetection }or similar 
spells, and it does not create a magical sensor that other creatures can detect. 
A deity can extend its senses to two or more remote locations at once (depending 
on divine rank) and still sense what's going on nearby.

\vspace{12pt}
\begin{tabular}{|>{\raggedright}p{52pt}|>{\raggedright}p{75pt}|}
\hline
D\textbf{ivine Rank} & \subsubsection*{R\textbf{emote Locations}}\tabularnewline
\hline
1-5 & 2\tabularnewline
\hline
6-10 & 5\tabularnewline
\hline
11-15 & 10\tabularnewline
\hline
16-20 & 20\tabularnewline
\hline
\end{tabular}

\vspace{12pt}
Once a deity chooses a remote location to sense, it automatically receives sensory 
information from that location until it chooses a new location to sense, or until 
it can't sense the location.

\textit{Block Sensing: }As a standard action, a deity of rank 1 or higher can block 
the sensing ability of other deities of its rank or lower. This power extends for 
a radius of one mile per rank of the deity, or within the same distance around 
a temple or other locale sacred to the deity, or the same distance around a portfolio-related 
event. The deity can block two remote locations at once, plus the area within one 
mile of itself. The blockage lasts 1 hour per divine rank.

\vspace{12pt}
\textbf{Portfolio: }Every deity of rank 1 or higher has at least limited knowledge 
and control over some aspect of mortal existence. A deity's connection to its portfolio 
gives it a number of powers.

\textit{Portfolio Sense: }Demigods have a limited ability to sense events involving 
their portfolios. They automatically sense any event that involves one thousand 
or more people. The ability is limited to the present. Lesser deities automatically 
sense any event that involves their portfolios and affects five hundred or more 
people. Intermediate deities automatically sense any event that involves their 
portfolios, regardless of the number of people involved. In addition, their senses 
extend one week into the past for every divine rank they have. Greater deities 
automatically sense any event that involves their portfolios, regardless of the 
number of people involved. In addition, their senses extend one week into the past 
and one week into the future for every divine rank they have. When a deity senses 
an event, it merely knows that the event is occurring and where it is. The deity 
receives no sensory information about the event. Once a deity notices an event, 
it can use its remote sensing power to perceive the event.

\vspace{12pt}
\textbf{Automatic Actions: }When performing an action within its portfolio, a deity 
can perform any action as a free action, as long as the check DC is equal to or 
less than the number on the table below. The number of free actions a deity can 
perform each round is also determined by the deity's divine rank.

\vspace{12pt}
\begin{tabular}{|>{\raggedright}p{56pt}|>{\raggedright}p{96pt}|>{\raggedright}p{69pt}|}
\hline
D\textbf{ivine Rank} & M\textbf{aximum DC for Automatic Action} & F\textbf{ree 
Actions per Round}\tabularnewline
\hline
1-5  & 15 & 2\tabularnewline
\hline
6-10  & 20 & 5\tabularnewline
\hline
11-15  & 25 & 10\tabularnewline
\hline
16-20  & 30 & 20\tabularnewline
\hline
\end{tabular}

\vspace{12pt}
\textbf{Create Magic Items:} A deity of rank 1 or higher can create magic items 
related to its portfolio without any requisite item creation feat, provided that 
the deity possesses all other prerequisites for the item. The maximum item value 
a deity can create is a function of its divine rank (see the table below). The 
item's cost and creation time remain unchanged, but the deity is free to undertake 
any activity when not laboring on the item.

\vspace{12pt}
\begin{tabular}{|>{\raggedright}p{52pt}|>{\raggedright}p{131pt}|}
\hline
\subsection*{D\textbf{ivine Rank}} & \subsection*{M\textbf{aximum Market Price}}\tabularnewline
\hline
1-5  & 4,500 gp\tabularnewline
\hline
6-10  & 30,000 gp\tabularnewline
\hline
11-15  & 200,000 gp (any nonartifact)\tabularnewline
\hline
16-20  & No maximum (including artifact)\tabularnewline
\hline
\end{tabular}

\vspace{12pt}
If a deity has the item creation feat pertaining to the item it wishes to create, 
the cost (in gold and XP) and creation times are halved.

\vspace{12pt}
\textbf{Divine Aura:} The mere presence of a deity of rank 1 or higher can deeply 
affect mortals and beings of lower divine rank. All divine aura effects are mind-affecting, 
extraordinary abilities. Mortals and other deities of lower rank can resist the 
aura's effects with successful Will saves; the DC is 10 + the deity's rank + the 
deity's Charisma modifier. Deities are immune to the auras of deities of equal 
or lower rank. Any being who makes a successful saving throw against a deity's 
aura power becomes immune to that deity's aura power for one day. Divine aura is 
an emanation that extends around the deity in a radius whose size is a function 
of divine rank. The deity chooses the size of the radius and can change it as a 
free action. If the deity chooses a radius of 0 feet, its aura power effectively 
becomes non-functional. When two or more deities' auras cover the same area, only 
the aura that belongs to the deity with the highest rank functions. If divine ranks 
are equal, the auras coexist.

\vspace{12pt}
\begin{tabular}{|>{\raggedright}p{52pt}|>{\raggedright}p{71pt}|}
\hline
\subsection*{D\textbf{ivine Rank }} & \subsection*{D\textbf{ivine Aura Size}}\tabularnewline
\hline
1-5  & 10 ft./rank\tabularnewline
\hline
6-10  & 100 ft./rank\tabularnewline
\hline
11-15  & 100 ft./rank\tabularnewline
\hline
16-20  & 1 mile/rank\tabularnewline
\hline
\end{tabular}

\vspace{12pt}
The deity can make its own worshipers, beings of its alignment, or both types of 
individuals immune to the effect as a free action. The immunity lasts one day or 
until the deity dismisses it. Once affected by an aura power, creatures remain 
affected as long as they remain within the aura's radius. The deity can choose 
from the following effects each round as a free action.

\textit{Daze: }Affected beings just stare at the deity in fascination. They can 
defend themselves normally but can take no actions.

\textit{Fright: }Affected beings become shaken and suffer a -2 morale penalty on 
attack rolls, saves, and checks. The merest glance or gesture from the deity makes 
them frightened, and they flee as quickly as they can, although they can choose 
the path of their flight.

\textit{Resolve: }The deity's allies receive a +4 morale bonus on attack rolls, 
saves, and checks, while the deity's foes receive a -4 morale penalty on attack 
rolls, saves, and checks.

\vspace{12pt}
\textbf{Grant Spells: }A deity automatically grants spells and domain powers to 
mortal divine spellcasters who pray to it. Most deities can grant spells from the 
cleric spell list, the ranger spell list, and from three or more domains. Deities 
with levels in the druid class can grant spells from the druid spell list, and 
deities with paladin levels can grant spells from the paladin spell list. A deity 
can withhold spells from any particular mortal as a free action; once a spell has 
been granted, it remains in the mortal's mind until expended.

\vspace{12pt}
\textbf{Spontaneous Casting:} A deity of rank 1 or higher who has levels in a divine 
spellcasting class can spontaneously cast any spell it can grant. 

\vspace{12pt}
\textbf{Communication:} A deity of rank 1 or higher can understand, speak, and 
read any language, including nonverbal languages. The deity can speak directly 
to any beings within one mile of itself per divine rank.

\textit{Remote Communication: }As a standard action, a deity of rank 1 or higher 
can send a communication to a remote location. The deity can speak to any of its 
own worshipers, and to anyone within one mile per rank away from a site dedicated 
to the deity, or within one mile per rank away from a statue or other likeness 
of the deity. The creature being contacted can receive a telepathic message that 
only it can hear. Alternatively, the deity's voice can seem to issue from the air, 
the ground, or from some object of the deity's choosing (but not an object or locale 
dedicated to another deity of equal or higher rank than the deity who is speaking). 
In the latter case, anyone within earshot of the sound can hear it. The deity can 
send a manifestation or omen instead of a spoken or telepathic message. The exact 
nature of this communication varies with the deity, but it usually is some visible 
phenomenon. A deity's communication power can cross planes and penetrate any barrier. 
Once communication is initiated, the deity can continue communicating as a free 
action until it decides to end the communication. A deity can carry on as many 
remote communications at one time as it can remote sense at one time (see Remote 
Sensing, above).

\vspace{12pt}
\textbf{Godly Realm:} Each deity of rank 1 or higher has a location that serves 
as a workplace, personal residence, audience chamber, and sometimes as a retreat 
or fortress. A deity is at its most powerful within its godly realm. A deity has 
at least modest control over the environment within its realm, controlling the 
temperature and minor elements of the environment. The radius of this control is 
a function of the deity's rank and whether the realm is located on an Outer Plane 
or some other plane (including the Material Plane).

\vspace{12pt}
\begin{tabular}{|>{\raggedright}p{52pt}|>{\raggedright}p{57pt}|>{\raggedright}p{70pt}|}
\hline
\subsection*{D\textbf{ivine Rank}}--------- & \multicolumn{2}{p{128pt}|}{\subsection*{ 
\textbf{Radius of Control ---------}}}\tabularnewline
\hline
  & O\textbf{uter Plane } & M\textbf{aterial Plane}\tabularnewline
\hline
1-5  & 100 ft./rank  & 100 ft./rank\tabularnewline
\hline
6-10  & 1 mile  & 100 ft./rank\tabularnewline
\hline
11-15  & 10 miles  & 100 ft./rank\tabularnewline
\hline
16-20  & 100 miles & 100 ft./rank\tabularnewline
\hline
\end{tabular}

\vspace{12pt}
Within this area, the deity can set any temperature that is normal for the plane 
where the realm is located (for the Material Plane, any temperature from -20ºF 
to 120ºF), and fill the area with scents and sounds as the deity sees fit. Sounds 
can be no louder than one hundred humans could make, but not intelligible speech 
or harmful sound. The deity's ability to create scents is similar. Deities of rank 
6 or higher can create the sounds of intelligible speech. A demigod or lesser deity 
can erect buildings and alter the landscape, but must do so through its own labor, 
through magic, or through its divine powers. A deity of rank 6 or higher not only 
has control over the environment, but also controls links to the Astral Plane. 
Manipulating a realm's astral links renders teleportation and similar effects useless 
within the realm. The deity can designate certain locales within the realm where 
astral links remain intact. Likewise, the deity can block off the realm from planar 
portals or designate locations where portals are possible. A deity of rank 11 or 
higher can also apply the enhanced magic or impeded magic trait to up to four groups 
of spells (schools, domains, or spells with the same descriptor). The enhanced 
magic trait enables a metamagic feat to be applied to a group of spells without 
requiring higher-level spell slots. Many deities apply the enhanced magic trait 
to their domain spells, making them maximized (as the Maximize Spell feat) within 
the boundaries of their realm. The impeded magic trait doesn't affect the deity's 
spells and spell-like abilities.

In addition, a deity of rank 11 or higher can erect buildings as desired and alter 
terrain within ten miles to become any terrain type found on the Material Plane. 
These buildings and alterations are manifestations of the deity's control over 
the realm.

A greater deity (rank 16 or higher) also can perform any one of the following acts:• 

\parindent=3pt
Change or apply a gravity trait within the realm.• 

Change or apply an elemental or energy trait within the realm.• 

\parindent=7pt
Change or apply a time trait within the realm.• 

\parindent=3pt
Apply the limited magic trait to a particular school, domain, or spell descriptor 
within the area, preventing such spells and spell-like abilities from functioning. 
The greater deity's own spells and spell-like abilities are not limited by these 
restrictions.

\parindent=0pt
Once a deity sets the conditions in its realm, they are permanent, though the deity 
can change them. As a standard action, the deity can specify a new environmental 
condition. The change gradually takes effect over the next 10 minutes. Changing 
astral links, planar traits, or terrain requires more effort, and the deity must 
labor for a year and a day to change them. During this time, the deity must spend 
8 hours a day on the project. During the remaining 16 hours of each day, the deity 
can perform any action it desires, so long as it remains within the realm. The 
astral links, planar traits, and terrain remain unchanged until the labor is complete. 

\vspace{12pt}
\textbf{Travel:} A deity of rank 1 or higher can use \textit{teleport without error 
}as a spell-like ability at will, as the spell cast by a 20th-level character, 
except that the deity can transport only itself and up to 100 pounds of objects 
per divine rank. A deity of rank 6 or higher also can use \textit{plane shift }as 
a spell-like ability at will, as the spell cast by a 20th-level character, except 
that the deity can only transport itself and up to 100 pounds of objects. If the 
deity has a familiar, personal mount, or personal intelligent weapon, the creature 
can accompany the deity in any mode of travel if the deity touches it. The creature's 
weight counts against the deity's weight limit.

\vspace{12pt}
\textbf{Familiar:} A deity of rank 1 or higher with levels as a sorcerer or a wizard 
has the ability to treat any creature of a given kind as a familiar, as long as 
that creature is within a distance of one mile per divine rank of the deity. This 
special familiar ability only applies to one creature at a time, but the deity 
can switch between one creature and another instantaneously, as long as the second 
creature is within range. This special familiar ability does not replace the deity's 
ability to have a normal familiar, which could be any kind of eligible creature.

\newpage

\end{document}
