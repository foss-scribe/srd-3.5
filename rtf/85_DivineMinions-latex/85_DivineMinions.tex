%&pdfLaTeX
% !TEX encoding = UTF-8 Unicode
\documentclass{article}
\usepackage{ifxetex}
\ifxetex
\usepackage{fontspec}
\setmainfont[Mapping=tex-text]{STIXGeneral}
\else
\usepackage[T1]{fontenc}
\usepackage[utf8]{inputenc}
\fi
\usepackage{textcomp}

\usepackage{amssymb}
\usepackage{fancyhdr}
\renewcommand{\headrulewidth}{0pt}
\renewcommand{\footrulewidth}{0pt}

\begin{document}

\subsection*{This material is Open Game Content, and is licensed for public use 
under the terms of the Open Game License v1.0a.}

\subsection*{{\LARGE{}DIVINE MINIONS}}

\vspace{12pt}
All types of beings may serve deities. In general, a deity only accepts minions 
who have accomplished some great deed in service to the deity. Such minions usually 
have the same alignment as the deity. No minion's alignment is opposed to the patron 
deity's alignment on either the law-chaos axis or the good-evil axis. 

\vspace{12pt}
\section*{Proxies}

A divine proxy speaks and acts on behalf of the divine being. When the demand for 
a deity's presence is too high, the deity may use proxies.

Proxies are divine minions invested with a small portion of the deity's power. 
A deity may invest 1 rank of its power (reducing its divine rank accordingly) in 
a single servant for as long as the deity chooses. The minion must be physically 
present for the deity to perform the investiture. While so invested, the proxy 
gains any salient divine abilities held by the patron deity as well as the powers 
and abilities of a rank 1 demigod. Without the requisite ability scores or divine 
ranks, the proxy may not be able to use all those powers and abilities. A deity 
may have more than one proxy, but it must lose 1 divine rank for each proxy it 
invests. A deity can retrieve a single divine rank as a standard action, and doing 
so it does not require the physical presence of the proxy. 

\vspace{12pt}
\section*{Petitioners}

Some spirits demonstrate their devotion to their deity by traveling to the deity's 
home plane. Those that survive the journey across the planes become servants of 
their deity. While a few may remain disembodied spirits, most become petitioners 
through the divine will of their patron deity. 

In general, petitioners appear in the form that they had when they died, though 
they may be remade by deities to fit the nature of their particular afterlife. 
In general, petitioners who become divine servants are creatures that originally 
had at least 1 Intelligence and 1 Wisdom.

The following creature types may become petitioners depending on the deity: aberrations, 
animals, dragons, fey, giants, humanoids, magical beasts, monstrous humanoids, 
and plants, oozes, and vermin with sufficient ability scores. Constructs and undead 
are not usually made into petitioners, though the spirits of their original forms 
may be. Elementals and outsiders tend to meld with their native planes, and as 
such do not become petitioners. Their spirits may still be called back from the 
dead, however.

The template presented below is for NPCs, not player characters. If dead characters 
who are petitioners are later restored to life (once again becoming player characters), 
they forget any of their experiences as petitioners.

\vspace{12pt}
\section*{Creating a Petitioner}

``Petitioner'' is a template that may be added to any creature as determined by 
the nature of the campaign (referred to hereafter as the base creature). The creature's 
type changes to outsider, and the creature uses all the base creature's statistics 
and special abilities except as noted here.

\textbf{Hit Dice: }Change to 2d8. Retain bonus hit points.

\textbf{AC:} Natural Armor Class, Dexterity, and size bonuses or penalties apply. 
Armor bonuses are not applicable.

\textbf{Attacks:} Base attack bonus is reduced to +2, subject to modifications 
for size and Strength.

\textbf{Special Attacks:} A petitioner loses all supernatural and spell-like attacks, 
but retains normal and exceptional attacks.

\textbf{Special Qualities:} A petitioner loses all supernatural and spell-like 
abilities, but retains exceptional abilities. In addition, it gains the following 
qualities.

\textit{Mental Immunity: }All petitioners are immune to mind-affecting effects. 

\textit{Other Immunities: }Depending on its nature, the petitioner is immune to 
two of the following effects: acid, cold, electricity, fire, poison, petrifaction, 
or polymorphing. These immunities are applied similarly to all petitioners of a 
particular plane or deity.

\textit{Resistance}s: Depending on the nature of the petitioner's plane, the petitioner 
gains resistance 20 against two of the following effects: acid, cold, electricity, 
or fire.

\textit{Planar Commitment: }Petitioners cannot leave the plane they inhabit. They 
are teleported one hundred miles in a random direction if an attempt is made to 
force them to leave.

\textit{Additional Special Qualities: }Particular planes may provide additional 
benefits for petitioners of those planes. Typical additional special qualities 
may include any one of the following.• 

\parindent=3pt
Damage reduction 5/silver and spell resistance 5.• 

Continuous magic circle against evil.• 

\parindent=7pt
Fast healing 1.• 

\parindent=3pt
Damage reduction 10/magic.• 

Spell resistance 10.• 

\parindent=7pt
Additional 2d8 Hit Dice.• 

\parindent=3pt
Remove all immunities and resistances except immunity to mind-affecting effects. 
Add acid, cold, electricity, fire, and poison resistance 5.

\parindent=0pt
Such modifications are the result of the nature of the plane or the powerful beings 
within it.

\textbf{Saves:} Base saving throw bonuses are +3.

\textbf{Abilities:} Same as the base creature. Some cosmologies or deities may 
set a maximum of 18 for petitioner ability scores. Abilities higher than that are 
reduced to the maximum.

\textbf{Skills: }Petitioners have no skills. Previous skills are lost.

\textbf{Feats: }Petitioners have no feats. Previous feats are lost.

\textbf{Climate/Terrain:} Any land and underground (within the same plane).

\textbf{Organization: }Same as the base creature.

\textbf{Challenge Rating:} 1.

\textbf{Treasure:} None.

\textbf{Alignment: }Same as the native plane.

\textbf{Advancement: }None.

\vspace{12pt}
\section*{Exceptional Petitioners}

The deities may choose particular servants for specific tasks that may retain the 
knowledge of their previous selves. These exceptional petitioners retain the feats 
and skills they had in life, but are otherwise limited as for the petitioners of 
their plane.

\newpage

\end{document}
